\documentclass[11pt]{article}
\usepackage[margin=1in]{geometry}
\usepackage{amsmath,amssymb,amsthm,mathtools}
\usepackage{hyperref}

\newtheorem{theorem}{Theorem}
\newtheorem{proposition}[theorem]{Proposition}
\newtheorem{lemma}[theorem]{Lemma}
\newtheorem{corollary}[theorem]{Corollary}
\theoremstyle{remark}
\newtheorem{remark}[theorem]{Remark}
\newtheorem{definition}[theorem]{Definition}
\newcommand{\R}{\mathbb R}
\newcommand{\C}{\mathbb C}
\newcommand{\1}{\mathbf 1}
\DeclareMathOperator{\supp}{supp}

\title{Energy-Decomposition and Perron--Frobenius Consequences\\
for the Restricted Weil Quadratic Form}
\author{}
\date{}

\begin{document}
\maketitle

\begin{abstract}
We record a completely concrete and rigorous functional-analytic step that arises
in the spectral approach to Weil's criterion when one restricts test functions to
a compact multiplicative interval $[\lambda^{-1},\lambda]\subset \R_+^*$.
Starting from the explicit local distributions at the primes and at $\infty$, we
derive an ``energy decomposition'' expressing the quadratic form (up to an additive
constant multiple of $\|g\|_2^2$) as a positive combination of translation-difference
energies $\|G-\tau_tG\|_2^2$ in logarithmic coordinates.
We then prove the Markov (normal contraction) property and a translation-invariance
lemma which yields irreducibility from the archimedean continuum of shifts.
Assuming (as in the standard setup) that the associated selfadjoint operator has
compact resolvent, we deduce that the ground-state eigenvalue is simple and its
eigenfunction can be chosen strictly positive and, by inversion symmetry, even.
\end{abstract}

\tableofcontents

\section{Setup on $\R_+^*$}

Let $\R_+^*=(0,\infty)$ with multiplicative Haar measure
\[
d^*x:=\frac{dx}{x}.
\]
For measurable $g,h$ define multiplicative convolution
\[
(g*h)(x):=\int_{\R_+^*} g(y)\,h(x/y)\,d^*y,
\]
and involution
\[
g^*(x):=\overline{g(x^{-1})}.
\]
If $g\in L^2(\R_+^*,d^*x)$, define the unitary dilation operator
\begin{equation}
\label{eq:Ua}
(U_ag)(x):=g(x/a)\qquad(a>0).
\end{equation}
Then $\|U_ag\|_2=\|g\|_2$ and $\langle g,U_ag\rangle$ is well-defined.

\begin{lemma}[Convolution inner-product identity]
\label{lem:f-inner}
Let $f=g*g^*$. Then for all $a>0$,
\[
f(a)=\langle g,U_ag\rangle_{L^2(d^*x)}=\int_{\R_+^*} g(x)\,\overline{g(x/a)}\,d^*x,
\qquad
f(a^{-1})=\overline{f(a)}.
\]
In particular $f(a)+f(a^{-1})=2\Re\langle g,U_ag\rangle$ and $f(1)=\|g\|_2^2$.
\end{lemma}

\begin{proof}
By definition,
\[
(g*g^*)(a)=\int g(y)\,g^*(a/y)\,d^*y
=\int g(y)\,\overline{g((a/y)^{-1})}\,d^*y
=\int g(y)\,\overline{g(y/a)}\,d^*y=\langle g,U_ag\rangle.
\]
The relation $f(a^{-1})=\overline{f(a)}$ follows by replacing $a$ with $a^{-1}$ and
complex conjugating.
\end{proof}

\begin{lemma}[A basic unitary identity]
\label{lem:unitary}
For any unitary $U$ on a Hilbert space and any vector $h$,
\[
2\Re\langle h,Uh\rangle = 2\|h\|^2-\|h-Uh\|^2.
\]
\end{lemma}
\begin{proof}
Expand $\|h-Uh\|^2=\|h\|^2+\|Uh\|^2-2\Re\langle h,Uh\rangle$ and use $\|Uh\|=\|h\|$.
\end{proof}

\section{Local explicit-formula terms}

Fix $\lambda>1$ and consider $g$ supported in $[\lambda^{-1},\lambda]$.

We record the two local distributions we use; these are the only ``input formulas''.

\subsection{Prime terms}
For a prime $p$ define
\begin{equation}
\label{eq:Wp}
W_p(f):=(\log p)\sum_{m\ge 1} p^{-m/2}\bigl(f(p^m)+f(p^{-m})\bigr).
\end{equation}

\subsection{Archimedean term}
Define
\begin{equation}
\label{eq:WR}
W_{\R}(f):=(\log 4\pi+\gamma)\,f(1)+\int_1^\infty
\Bigl(f(x)+f(x^{-1})-2x^{-1/2}f(1)\Bigr)\frac{x^{1/2}}{x-x^{-1}}\,d^*x,
\end{equation}
where $\gamma$ is the Euler--Mascheroni constant.

\begin{remark}[Restriction to a compact multiplicative interval]
\label{rem:truncate}
If $\operatorname{supp}(g)\subset[\lambda^{-1},\lambda]$, then for $a>\lambda^2$
the supports of $g$ and $U_ag$ are disjoint, hence $\langle g,U_ag\rangle=0$ and
$f(a)=0$. Consequently:
\begin{itemize}
\item in \eqref{eq:Wp} only those $(p,m)$ with $p^m\le \lambda^2$ contribute;
\item in \eqref{eq:WR}, after the change of variables $x=e^t$, only $t\in[0,2\log\lambda]$
contributes to the term involving $f(e^t)+f(e^{-t})$.
\end{itemize}
This finiteness is crucial and is completely elementary.
\end{remark}

\section{Logarithmic coordinates and translations}

Set $u=\log x$, so that $d^*x=du$ and the interval $[\lambda^{-1},\lambda]$ becomes
\[
I:=(-L,L),\qquad L:=\log\lambda.
\]
For $G\in L^2(I)$ we denote by $\widetilde G$ its extension by $0$ to $\R$.
Let $S_t$ be translation on $L^2(\R)$:
\[
(S_t\phi)(u):=\phi(u-t).
\]
Then in logarithmic coordinates, the dilation $U_{e^t}$ from \eqref{eq:Ua} corresponds to
translation: if $G(u)=g(e^u)$, then $(U_{e^t}g)(e^u)=g(e^{u-t})$, i.e.\ $\widetilde G\mapsto S_t\widetilde G$.

\section{Energy decomposition into translation differences}

\subsection{Prime contributions}
\begin{lemma}[Prime term as a difference energy plus a constant]
\label{lem:prime-energy}
Let $f=g*g^*$ with $g$ supported in $[\lambda^{-1},\lambda]$, and let $G(u)=g(e^u)$.
Then
\[
-W_p(f)
=\sum_{\substack{m\ge 1\\ p^m\le \lambda^2}} (\log p)\,p^{-m/2}\,\|\widetilde G-S_{m\log p}\widetilde G\|_{L^2(\R)}^2
\;+\;c_p(\lambda)\,\|G\|_{L^2(I)}^2,
\]
where $c_p(\lambda)\in \R$ is a finite constant depending only on $p$ and $\lambda$.
\end{lemma}

\begin{proof}
By Lemma~\ref{lem:f-inner} and \eqref{eq:Wp},
\[
W_p(f)=(\log p)\sum_{m\ge 1}p^{-m/2}\,2\Re\langle g,U_{p^m}g\rangle.
\]
By Lemma~\ref{lem:unitary} (with $U=U_{p^m}$),
\[
2\Re\langle g,U_{p^m}g\rangle = 2\|g\|_2^2-\|g-U_{p^m}g\|_2^2.
\]
In logarithmic coordinates, $\|g-U_{p^m}g\|_2=\|\widetilde G-S_{m\log p}\widetilde G\|_{L^2(\R)}$.
Moreover, if $p^m>\lambda^2$ then $\langle g,U_{p^m}g\rangle=0$ by Remark~\ref{rem:truncate},
so those terms vanish. Collecting the $\|g\|_2^2$ contributions yields the constant $c_p(\lambda)$.
\end{proof}

\subsection{Archimedean contribution}
\begin{lemma}[Archimedean term as a continuum of difference energies plus a constant]
\label{lem:arch-energy}
Let $f=g*g^*$ with $g$ supported in $[\lambda^{-1},\lambda]$, and let $G(u)=g(e^u)$.
Define the strictly positive weight on $(0,\infty)$,
\[
w(t):=\frac{e^{t/2}}{e^t-e^{-t}}=\frac{e^{t/2}}{2\sinh t}.
\]
Then
\[
-W_{\R}(f)
=\int_{0}^{2L} w(t)\,\|\widetilde G-S_t\widetilde G\|_{L^2(\R)}^2\,dt
\;+\;c_\infty(\lambda)\,\|G\|_{L^2(I)}^2,
\]
where $c_\infty(\lambda)\in \R$ is a finite constant depending only on $\lambda$.
\end{lemma}

\begin{proof}
Start from \eqref{eq:WR}. Substitute $x=e^t$ (so $d^*x=dt$) to obtain
\[
W_{\R}(f)=(\log4\pi+\gamma)\,f(1)+\int_0^\infty\Bigl(f(e^t)+f(e^{-t})-2e^{-t/2}f(1)\Bigr)\,w(t)\,dt.
\]
Using Lemma~\ref{lem:f-inner}, $f(1)=\|g\|_2^2$, and
\[
f(e^t)+f(e^{-t}) = 2\Re\langle g,U_{e^t}g\rangle,
\]
we get
\[
-W_{\R}(f)=-(\log4\pi+\gamma)\|g\|_2^2+\int_0^\infty\Bigl(-2\Re\langle g,U_{e^t}g\rangle + 2e^{-t/2}\|g\|_2^2\Bigr)\,w(t)\,dt.
\]
Apply Lemma~\ref{lem:unitary} with $U=U_{e^t}$:
\[
-2\Re\langle g,U_{e^t}g\rangle=\|g-U_{e^t}g\|_2^2-2\|g\|_2^2.
\]
Thus the integrand equals
\[
\|g-U_{e^t}g\|_2^2 + 2(e^{-t/2}-1)\|g\|_2^2.
\]
In logarithmic coordinates $\|g-U_{e^t}g\|_2=\|\widetilde G-S_t\widetilde G\|_{L^2(\R)}$.

Now we split the integral at $t=2L$.
By Remark~\ref{rem:truncate}, for $t>2L$ the supports of $\widetilde G$ and $S_t\widetilde G$ are disjoint,
so $\|\widetilde G - S_t\widetilde G\|_2^2 = 2\|G\|_2^2$ (not zero).
Hence for $t>2L$ the integrand becomes $2\|G\|_2^2 + 2(e^{-t/2}-1)\|G\|_2^2 = 2e^{-t/2}\|G\|_2^2$.
This tail integral $\int_{2L}^\infty 2e^{-t/2}\,w(t)\,dt$ converges (since $w(t)\sim e^{-t/2}$ as $t\to\infty$)
and contributes a finite constant times $\|G\|_2^2$.

For $t\in[0,2L]$ we retain the difference-energy term $w(t)\|\widetilde G-S_t\widetilde G\|_2^2$
and absorb the $2(e^{-t/2}-1)\,w(t)\|G\|_2^2$ contribution into the constant.
Combining all $\|G\|_2^2$ terms---from the $(\log4\pi+\gamma)$ prefactor, the integral over $[0,2L]$ of
$2(e^{-t/2}-1)\,w(t)$, and the tail integral over $(2L,\infty)$---yields the finite constant $c_\infty(\lambda)$.
The integral of $w(t)(e^{-t/2}-1)$ over $[0,2L]$ converges absolutely (near $0$, $w(t)\sim 1/(2t)$ and
$e^{-t/2}-1\sim -t/2$, giving an integrable $O(1)$ contribution).
\end{proof}

\subsection{Global quadratic form on the interval}

\begin{definition}[Difference-energy form]
\label{def:E}
Fix $\lambda>1$ and $L=\log\lambda$. For $G\in L^2(I)$ define
\begin{align}
\mathcal E_\lambda(G)
&:=\int_{0}^{2L} w(t)\,\|\widetilde G-S_t\widetilde G\|_{L^2(\R)}^2\,dt
+\sum_{\substack{p\ \mathrm{prime}\\ p\le \lambda^2}}\ \sum_{\substack{m\ge 1\\ p^m\le \lambda^2}}
(\log p)p^{-m/2}\,\|\widetilde G-S_{m\log p}\widetilde G\|_{L^2(\R)}^2.
\end{align}
\end{definition}

\begin{remark}[What we have proved so far]
Lemmas~\ref{lem:prime-energy} and \ref{lem:arch-energy} show that for $f=g*g^*$ with
$\operatorname{supp}(g)\subset[\lambda^{-1},\lambda]$, the quantity
\[
-\sum_{v\in\{\infty\}\cup\{p\}} W_v(f)
\]
equals $\mathcal E_\lambda(G)$ plus an additive constant multiple of $\|G\|_2^2$.
Since adding a constant multiple of $\|G\|_2^2$ only shifts the spectrum of the associated
operator, it does not affect positivity/irreducibility properties of the semigroup and does not
affect eigenfunction parity considerations.
\end{remark}

\section{Markov property (normal contractions)}

\begin{definition}[Normal contraction]
A map $\Phi:\R\to\R$ is a normal contraction if $\Phi(0)=0$ and
$|\Phi(a)-\Phi(b)|\le |a-b|$ for all $a,b\in\R$.
\end{definition}

\begin{lemma}[Markov property]
\label{lem:markov}
For every normal contraction $\Phi$ and every $G\in L^2(I)$,
\[
\mathcal E_\lambda(\Phi\circ G)\le \mathcal E_\lambda(G).
\]
In particular, $\mathcal E_\lambda(|G|)\le \mathcal E_\lambda(G)$.
\end{lemma}

\begin{proof}
For each shift parameter $t$,
\[
\|\widetilde{\Phi\circ G}-S_t\widetilde{\Phi\circ G}\|_2^2
=\int_{\R}|\Phi(\widetilde G(u))-\Phi(\widetilde G(u-t))|^2\,du
\le \int_{\R}|\widetilde G(u)-\widetilde G(u-t)|^2\,du
=\|\widetilde G-S_t\widetilde G\|_2^2,
\]
by the $1$-Lipschitz property of $\Phi$. Integrating against the nonnegative weights and
summing proves the claim.
\end{proof}

\section{A translation-invariance lemma on an interval}


\begin{lemma}[Local translation invariance forces null or conull]\label{lem:trans-inv}
Let $I\subset\R$ be a nontrivial open interval and let $B\subset I$ be measurable.
Assume that there exists $\varepsilon>0$ such that for every $t\in(0,\varepsilon)$,
\begin{equation}\label{eq:indicator-inv}
\1_{B}(u)=\1_{B}(u-t)\quad\text{for a.e.\ }u\in I\cap(I+t).
\end{equation}
Then either $m(B)=0$ or $m(I\setminus B)=0$.
Equivalently: if $0<m(B)<m(I)$ then for every $\varepsilon>0$ there exists $t\in(0,\varepsilon)$ with
$m(B\cap(B+t)^c)>0$.
\end{lemma}

\begin{proof}
Write $f:=\1_B\in L^1_{\mathrm{loc}}(I)$.
Fix a compact subinterval $J\Subset I$ (so $\mathrm{dist}(J,\partial I)>0$), and choose
$0<\delta<\min\{\varepsilon,\mathrm{dist}(J,\partial I)\}$.
From \eqref{eq:indicator-inv} and the substitution $u\mapsto u+t$ we obtain:
for every $t\in(0,\delta)$,
\[
f(u+t)=f(u)\quad\text{for a.e.\ }u\in J.
\]
Thus for every $t\in(-\delta,\delta)$ we have $f(u+t)=f(u)$ for a.e.\ $u\in J$ (replace $t$ by $-t$).

Let $\rho\in C_c^\infty(\R)$ be a standard mollifier with $\rho\ge 0$, $\int\rho=1$ and $\supp\rho\subset(-1,1)$,
and set $\rho_\eta(s):=\eta^{-1}\rho(s/\eta)$ for $0<\eta<\delta/2$.
Define $f_\eta:=f*\rho_\eta$ on the slightly smaller interval
\[
J_\eta:=\{u\in J:\ \mathrm{dist}(u,\R\setminus J)>\eta\}.
\]
Then $f_\eta\in C^\infty(J_\eta)$, and for $u\in J_\eta$ and $|t|<\delta/2$ we may compute (using Fubini)
\[
f_\eta(u+t)=\int_\R f(u+t-s)\rho_\eta(s)\,ds
=\int_\R f(u-s)\rho_\eta(s)\,ds=f_\eta(u),
\]
because $u-s\in J$ for $u\in J_\eta$ and $s\in\supp\rho_\eta$, and $f(\cdot+t)=f(\cdot)$ a.e.\ on $J$.
Hence $f_\eta$ is translation-invariant on the connected open interval $J_\eta$, so $f_\eta$ is constant on $J_\eta$.

Letting $\eta\downarrow 0$, we have $f_\eta\to f$ in $L^1(J)$, so $f$ is a.e.\ equal to a constant on $J$.
Since $J\Subset I$ was arbitrary, $f$ is a.e.\ constant on $I$, i.e.\ $\1_B$ is a.e.\ either $0$ or $1$ on $I$.
Thus $m(B)=0$ or $m(I\setminus B)=0$.
\end{proof}


\section{Irreducibility from the archimedean continuum}

\subsection{A concrete criterion}

\begin{lemma}[Indicator-energy vanishes only for null/conull sets]
\label{lem:indicator-energy}
Let $B\subset I$ be measurable. If $\mathcal E_\lambda(\1_B)=0$, then $m(B)=0$ or $m(I\setminus B)=0$.
\end{lemma}

\begin{proof}
By definition of $\mathcal E_\lambda$ and the nonnegativity of all weights, $\mathcal E_\lambda(\1_B)=0$
implies in particular that the archimedean contribution vanishes:
\[
\int_0^{2L} w(t)\,\|\widetilde{\1_B}-S_t\widetilde{\1_B}\|_2^2\,dt = 0.
\]
Since $w(t)>0$ for every $t>0$, it follows that
\[
\|\widetilde{\1_B}-S_t\widetilde{\1_B}\|_2^2=0 \quad\text{for a.e.\ } t\in(0,2L).
\]
We now upgrade ``a.e.'' to ``all'': for any $\phi\in L^2(\R)$, the map $t\mapsto\|\phi-S_t\phi\|_2^2$
is continuous (by strong continuity of the translation group on $L^2(\R)$, which follows from
dominated convergence). Applying this to $\phi=\widetilde{\1_B}\in L^2(\R)$, the function
$t\mapsto \|\widetilde{\1_B}-S_t\widetilde{\1_B}\|_2^2$ is continuous, vanishes a.e.\ on $(0,2L)$,
and hence vanishes \emph{everywhere} on $(0,2L)$. In particular, for every $t\in(0,2L)$,
\[
\1_B(u)=\1_B(u-t)\quad\text{for a.e. }u\in I\cap(I+t).
\]
Since this holds for all $t$ in the interval $(0,2L)$, which contains $(0,\varepsilon)$ for any
$\varepsilon\le 2L$, Lemma~\ref{lem:trans-inv} applies and yields $m(B)=0$ or $m(I\setminus B)=0$.
\end{proof}


\subsection{Operator realization: closedness and compact resolvent}

In this subsection we show that the concrete form $\mathcal E_\lambda$ of Definition~\ref{def:E}
is closed and yields a selfadjoint operator with compact resolvent.
This replaces the abstract assumption previously made on the operator.

\subsubsection{Ambient form on $L^2(\R)$ and Fourier representation}

Let $\mathcal F:L^2(\R)\to L^2(\R)$ denote the unitary Fourier transform
\[
\widehat{\phi}(\xi):=\int_{\R}\phi(u)e^{-iu\xi}\,du,
\qquad
\phi(u)=\frac1{2\pi}\int_{\R}\widehat{\phi}(\xi)e^{iu\xi}\,d\xi,
\]
so that Plancherel reads $\|\phi\|_{L^2(\R)}^2=\frac1{2\pi}\int_{\R}|\widehat{\phi}(\xi)|^2\,d\xi$.

Define the ``ambient'' quadratic form on $L^2(\R)$ by
\begin{align*}
\mathcal E_\lambda^{\R}(\phi)
&:=\int_{0}^{2L} w(t)\,\|\phi-S_t\phi\|_{L^2(\R)}^2\,dt\\
&\qquad\qquad
+\sum_{\substack{p\ \mathrm{prime}\\ p\le \lambda^2}}\ \sum_{\substack{m\ge 1\\ p^m\le \lambda^2}}
(\log p)p^{-m/2}\,\|\phi-S_{m\log p}\phi\|_{L^2(\R)}^2,
\end{align*}
with domain $\mathcal D(\mathcal E_\lambda^{\R}):=\{\phi\in L^2(\R):\mathcal E_\lambda^{\R}(\phi)<\infty\}$.
By definition, for $G\in L^2(I)$,
\[
\mathcal E_\lambda(G)=\mathcal E_\lambda^{\R}(\widetilde G).
\]

\begin{lemma}[Plancherel identity for translation differences]\label{lem:plancherel-diff}
For $\phi\in L^2(\R)$ and $t\in\R$,
\[
\|\phi-S_t\phi\|_{L^2(\R)}^2
=\frac1{2\pi}\int_{\R}|1-e^{-i\xi t}|^2\,|\widehat\phi(\xi)|^2\,d\xi
=\frac1{2\pi}\int_{\R}4\sin^2\!\Bigl(\frac{\xi t}{2}\Bigr)\,|\widehat\phi(\xi)|^2\,d\xi.
\]
\end{lemma}

\begin{proof}
Since $\widehat{S_t\phi}(\xi)=e^{-i\xi t}\widehat\phi(\xi)$, Plancherel gives the first identity.
The second follows from $|1-e^{-i\eta}|^2=4\sin^2(\eta/2)$.
\end{proof}

\begin{lemma}[Fourier representation]\label{lem:FourierRep}
For $\phi\in L^2(\R)$,
\[
\mathcal E_\lambda^{\R}(\phi)=\frac1{2\pi}\int_{\R}\psi_\lambda(\xi)\,|\widehat{\phi}(\xi)|^2\,d\xi
\quad\text{in }[0,\infty],
\]
where
\begin{align}
\psi_\lambda(\xi)
&:=4\int_0^{2L} w(t)\,\sin^2\!\Bigl(\frac{\xi t}{2}\Bigr)\,dt\notag\\
&\qquad\qquad
+4\sum_{\substack{p\ \mathrm{prime}\\ p\le \lambda^2}}\ \sum_{\substack{m\ge 1\\ p^m\le \lambda^2}}
(\log p)p^{-m/2}\,\sin^2\!\Bigl(\frac{\xi m\log p}{2}\Bigr).\label{eq:defpsi}
\end{align}
In particular $\psi_\lambda$ is measurable, even, finite for each $\xi$, and $\psi_\lambda(\xi)\ge 0$.
\end{lemma}

\begin{proof}
Apply Lemma~\ref{lem:plancherel-diff} to each translation difference in $\mathcal E_\lambda^{\R}$.
All weights are nonnegative, so Tonelli's theorem permits interchange of the $\xi$--integral with the
$t$--integration and finite summations.
\end{proof}

\begin{proposition}[Closedness on $L^2(\R)$]\label{prop:closedR}
The form $\mathcal E_\lambda^{\R}$ is densely defined, symmetric, nonnegative, and closed on $L^2(\R)$.
Moreover,
\[
\mathcal D(\mathcal E_\lambda^{\R})
=\Bigl\{\phi\in L^2(\R):\int_{\R}\psi_\lambda(\xi)\,|\widehat\phi(\xi)|^2\,d\xi<\infty\Bigr\},
\]
and $\mathcal D(\mathcal E_\lambda^{\R})$ is a Hilbert space for the norm
\[
\|\phi\|_{\mathcal D}^2:=\|\phi\|_{L^2(\R)}^2+\mathcal E_\lambda^{\R}(\phi)
=\frac1{2\pi}\int_{\R}(1+\psi_\lambda(\xi))\,|\widehat\phi(\xi)|^2\,d\xi.
\]
\end{proposition}

\begin{proof}
By Lemma~\ref{lem:FourierRep}, $\mathcal E_\lambda^{\R}$ is the quadratic form of multiplication by
$\psi_\lambda$ in Fourier space. Hence $\mathcal D(\mathcal E_\lambda^{\R})$ is isometric (via $\phi\mapsto\widehat\phi$)
to the weighted $L^2$ space with weight $1+\psi_\lambda$, and therefore complete.
Nonnegativity and symmetry are immediate from the definition.

For density, note that $C_c^\infty(\R)\subset \mathcal D(\mathcal E_\lambda^{\R})$:
for $\phi\in C_c^\infty(\R)$, $\|\phi-S_t\phi\|_2\le |t|\|\phi'\|_2$,
and $\int_0^{2L} w(t)t^2\,dt<\infty$ (since $w(t)\sim (2t)^{-1}$ as $t\downarrow 0$ and the upper limit is finite);
the prime sum in $\mathcal E_\lambda^{\R}$ is finite.
Since $C_c^\infty(\R)$ is dense in $L^2(\R)$, the form is densely defined.
\end{proof}

\begin{proposition}[Closedness on $L^2(I)$]\label{prop:closedI}
The form $\mathcal E_\lambda$ on $H=L^2(I)$ is densely defined, symmetric, nonnegative, and closed.
\end{proposition}

\begin{proof}
The map $G\mapsto\widetilde G$ is an isometry from $L^2(I)$ onto the closed subspace
$H_I=\{\phi\in L^2(\R):\phi=0\text{ a.e.\ on }\R\setminus I\}$.
Moreover $\mathcal E_\lambda(G)=\mathcal E_\lambda^{\R}(\widetilde G)$.
Thus $\mathcal E_\lambda$ is the restriction of the closed form $\mathcal E_\lambda^{\R}$
(Proposition~\ref{prop:closedR}) to the closed subspace $H_I$, and therefore is closed.
Density follows because $C_c^\infty(I)\subset\mathcal D(\mathcal E_\lambda)$ and is dense in $L^2(I)$.
\end{proof}

\subsubsection{A coercive lower bound for the symbol $\psi_\lambda$}

\begin{lemma}[A lower bound for $w(t)$]\label{lem:wlower}
Let $t_0:=\min(1,2L)$. There exists $c_0=c_0(L)>0$ such that for all $t\in(0,t_0]$,
\[
w(t)=\frac{e^{t/2}}{2\sinh t}\ge \frac{c_0}{t}.
\]
\end{lemma}

\begin{proof}
For $t>0$ one has $\sinh t\le t e^{t}$, hence
\[
w(t)=\frac{e^{t/2}}{2\sinh t}\ge \frac{e^{t/2}}{2t e^{t}}=\frac{e^{-t/2}}{2t}.
\]
For $t\in(0,1]$, $e^{-t/2}\ge e^{-1/2}$, so we may take $c_0:=e^{-1/2}/2$ (or any smaller positive constant).
\end{proof}

\begin{lemma}[Logarithmic growth of $\psi_\lambda$]\label{lem:psilog}
There exist constants $c_1,c_2>0$ and $\xi_0\ge 2$ (depending only on $L$) such that for all $|\xi|\ge \xi_0$,
\[
\psi_\lambda(\xi)\ge c_1\log|\xi|-c_2.
\]
In particular $\psi_\lambda(\xi)\to\infty$ as $|\xi|\to\infty$.
\end{lemma}

\begin{proof}
Drop the nonnegative prime sum in \eqref{eq:defpsi}:
\[
\psi_\lambda(\xi)\ge 4\int_0^{2L} w(t)\sin^2\Bigl(\frac{\xi t}{2}\Bigr)\,dt
\ge 4\int_0^{t_0} w(t)\sin^2\Bigl(\frac{\xi t}{2}\Bigr)\,dt.
\]
By Lemma~\ref{lem:wlower}, for $t\in(0,t_0]$,
\[
\psi_\lambda(\xi)\ge 4c_0\int_0^{t_0}\frac1t\sin^2\Bigl(\frac{\xi t}{2}\Bigr)\,dt.
\]
Assume $|\xi|\ge \frac{4\pi}{t_0}$ (this fixes $\xi_0$). Define intervals
\[
J_n:=\left[\frac{2\pi n+\pi/2}{|\xi|},\ \frac{2\pi n+3\pi/2}{|\xi|}\right],\qquad n\ge 0.
\]
For $t\in J_n$, $\sin^2(\xi t/2)\ge 1/2$.
Let $N\ge 1$ be the largest integer such that $J_{N-1}\subset (0,t_0]$.
Then $N\asymp |\xi|$ (with constants depending only on $t_0$), and hence
\[
\int_0^{t_0}\frac1t\sin^2\Bigl(\frac{\xi t}{2}\Bigr)\,dt
\ge \sum_{n=0}^{N-1}\int_{J_n}\frac1t\cdot\frac12\,dt
=\frac12\sum_{n=0}^{N-1}\log\frac{2\pi n+3\pi/2}{2\pi n+\pi/2}.
\]
Using $\log(1+x)\ge x/(1+x)$, one obtains
\[
\log\frac{2\pi n+3\pi/2}{2\pi n+\pi/2}
=\log\left(1+\frac{\pi}{2\pi n+\pi/2}\right)\ge \frac{c}{n+1}
\]
for some absolute $c>0$ and all $n\ge 0$.
Therefore the sum is bounded below by $c'\sum_{n=0}^{N-1}\frac1{n+1}\ge c''\log N-C$.
Since $N\asymp|\xi|$, we have $\log N=\log|\xi|+O(1)$, giving the claim.
\end{proof}

\begin{corollary}[Energy controls a logarithmic frequency moment]\label{cor:tail}
There exist constants $a,b>0$ (depending only on $L$) such that for every $\phi\in\mathcal D(\mathcal E_\lambda^{\R})$,
\[
\int_\R \log(2+|\xi|)\,|\widehat\phi(\xi)|^2\,d\xi
\le a\,\|\phi\|_{L^2(\R)}^2 + b\,\int_\R \psi_\lambda(\xi)\,|\widehat\phi(\xi)|^2\,d\xi.
\]
In particular, if $\|\phi\|_{2}^2+\mathcal E_\lambda^{\R}(\phi)\le M$, then
$\int \log(2+|\xi|)|\widehat\phi(\xi)|^2\le C(M,L)$.
\end{corollary}

\begin{proof}
Lemma~\ref{lem:psilog} implies $\log(2+|\xi|)\le a'+b'\psi_\lambda(\xi)$ for suitable $a',b'$
(after enlarging constants to handle bounded $|\xi|$). Multiply by $|\widehat\phi(\xi)|^2$ and integrate.
\end{proof}

\subsubsection{Compact embedding and compact resolvent}

\begin{theorem}[Kolmogorov--Riesz compactness criterion in $L^2(\R)$]\label{thm:KR}
A set $\mathcal K\subset L^2(\R)$ is relatively compact if and only if:
\begin{enumerate}
\item[(i)] (\emph{tightness}) for every $\varepsilon>0$ there exists $R>0$ such that
$\int_{|u|>R}|\phi(u)|^2\,du<\varepsilon^2$ for all $\phi\in\mathcal K$;
\item[(ii)] (\emph{translation equicontinuity}) for every $\varepsilon>0$ there exists $\delta>0$ such that
$\|\phi-S_h\phi\|_{2}<\varepsilon$ for all $\phi\in\mathcal K$ and all $|h|<\delta$.
\end{enumerate}
\end{theorem}

\begin{remark}
See, e.g., Lieb--Loss, \emph{Analysis}, for a proof of Theorem~\ref{thm:KR}.
\end{remark}

\begin{lemma}[Uniform translation control from the form norm]\label{lem:trans}
Fix $M>0$ and define
\[
\mathcal K_M:=\{\phi\in H_I:\ \|\phi\|_2^2+\mathcal E_\lambda^{\R}(\phi)\le M\}.
\]
Then $\mathcal K_M$ satisfies the translation equicontinuity condition (ii) in Theorem~\ref{thm:KR}.
\end{lemma}

\begin{proof}
Let $\phi\in\mathcal K_M$ and $h\in\R$ with $|h|\le 1$.
By Plancherel,
\[
\|\phi-S_h\phi\|_2^2=\frac1{2\pi}\int_\R 4\sin^2\Bigl(\frac{\xi h}{2}\Bigr)\,|\widehat\phi(\xi)|^2\,d\xi.
\]
Fix $R\ge 1$ and split the integral into $|\xi|\le R$ and $|\xi|>R$.
Using $\sin^2(x)\le x^2$,
\[
\int_{|\xi|\le R}4\sin^2\Bigl(\frac{\xi h}{2}\Bigr)\,|\widehat\phi(\xi)|^2\,d\xi
\le \int_{|\xi|\le R} (\xi h)^2\,|\widehat\phi(\xi)|^2\,d\xi
\le (Rh)^2\int_\R |\widehat\phi(\xi)|^2\,d\xi
=(Rh)^2(2\pi)\|\phi\|_2^2.
\]
Also $\sin^2\le 1$ gives
\[
\int_{|\xi|>R}4\sin^2\Bigl(\frac{\xi h}{2}\Bigr)\,|\widehat\phi(\xi)|^2\,d\xi
\le 4\int_{|\xi|>R}|\widehat\phi(\xi)|^2\,d\xi
\le \frac{4}{\log(2+R)}\int_\R \log(2+|\xi|)\,|\widehat\phi(\xi)|^2\,d\xi.
\]
By Corollary~\ref{cor:tail}, the last integral is $\le C(M,L)$ uniformly over $\phi\in\mathcal K_M$.
Therefore
\[
\|\phi-S_h\phi\|_2^2\le (Rh)^2\,M+\frac{C'(M,L)}{\log(2+R)}.
\]
Given $\varepsilon>0$, choose $R$ so that $C'(M,L)/\log(2+R)\le \varepsilon^2/2$, and then choose $\delta>0$
so that $(R\delta)^2M\le \varepsilon^2/2$. This gives $\|\phi-S_h\phi\|_2<\varepsilon$ for all $\phi\in\mathcal K_M$
and $|h|<\delta$.
\end{proof}

\begin{proposition}[Compact embedding of the form domain]\label{prop:compactEmbed}
The embedding $(\mathcal D(\mathcal E_\lambda),\|G\|_{\mathcal D}^2:=\|G\|_2^2+\mathcal E_\lambda(G))\hookrightarrow L^2(I)$ is compact.
\end{proposition}

\begin{proof}
Let $\{G_n\}\subset \mathcal D(\mathcal E_\lambda)$ be bounded in the form norm:
$\|G_n\|_2^2+\mathcal E_\lambda(G_n)\le M$.
Put $\phi_n:=\widetilde G_n\in H_I$. Then $\|\phi_n\|_2^2+\mathcal E_\lambda^{\R}(\phi_n)\le M$, so $\phi_n\in\mathcal K_M$.

Since each $\phi_n$ is supported in the fixed bounded set $\overline I$, tightness (i) in Theorem~\ref{thm:KR}
holds automatically. Translation equicontinuity (ii) holds by Lemma~\ref{lem:trans}.
Thus $\{\phi_n\}$ is relatively compact in $L^2(\R)$ by Theorem~\ref{thm:KR}$;$ hence $\{G_n\}$ is relatively compact in $L^2(I)$.
\end{proof}

\begin{theorem}[Closed form, associated operator, and compact resolvent]\label{thm:operator}
There exists a unique selfadjoint operator $A_\lambda\ge 0$ on $L^2(I)$ associated to the closed form
$\mathcal E_\lambda$ (Proposition~\ref{prop:closedI}) in the sense of the representation theorem for closed forms.
Moreover, $A_\lambda$ has compact resolvent; equivalently, $(A_\lambda+1)^{-1}$ is compact on $L^2(I)$.
\end{theorem}

\begin{proof}
Existence and uniqueness of $A_\lambda$ follow from the representation theorem for densely defined, closed,
lower-bounded symmetric forms (see, e.g., Kato, \emph{Perturbation Theory for Linear Operators}).
To prove compact resolvent, let $\{f_n\}$ be bounded in $L^2(I)$ and set $u_n:=(A_\lambda+1)^{-1}f_n$.
Then $u_n\in\mathcal D(A_\lambda)\subset\mathcal D(\mathcal E_\lambda)$ and $(A_\lambda+1)u_n=f_n$.
Taking the $L^2$ inner product with $u_n$ and using the form identity gives
\[
\mathcal E_\lambda(u_n)+\|u_n\|_2^2=\langle f_n,u_n\rangle \le \|f_n\|_2\,\|u_n\|_2.
\]
Hence $\|u_n\|_2\le \|f_n\|_2$, and therefore $\|u_n\|_2^2+\mathcal E_\lambda(u_n)\le \|f_n\|_2^2$.
Thus $\{u_n\}$ is bounded in the form norm, so by Proposition~\ref{prop:compactEmbed} it has a convergent subsequence in $L^2(I)$.
This proves $(A_\lambda+1)^{-1}$ is compact.
\end{proof}

\subsection{Semigroup and irreducibility}

\begin{definition}[Irreducibility for semigroups on $L^2(I)$]
A closed ideal in $L^2(I)$ has the form $L^2(B)$ for some measurable $B\subset I$.
We call $T$ \emph{irreducible} if the only invariant closed ideals are $\{0\}$ and $L^2(I)$.
\end{definition}

\begin{proposition}[Irreducibility from indicator-energy criterion]
\label{prop:irreducible}
Under Theorem~\ref{thm:operator}, if $\mathcal E_\lambda(\1_B)=0$ implies $m(B)\in\{0,m(I)\}$,
then the semigroup $T(t)=e^{-tA_\lambda}$ is irreducible.
\end{proposition}

\begin{remark}[What is used here]
This proposition is a standard equivalence in Dirichlet form theory: for symmetric Markovian semigroups,
invariant ideals correspond to measurable invariant sets, and invariant sets correspond to sets with zero
energy for their indicators. We do not reprove the general equivalence here; it is a well-known part of the
Beurling--Deny/Fukushima theory of symmetric Dirichlet forms.
\end{remark}

\begin{corollary}[Irreducibility for $\mathcal E_\lambda$]
\label{cor:irreducible}
Assume Theorem~\ref{thm:operator}. Then $T(t)=e^{-tA_\lambda}$ is irreducible.
\end{corollary}

\begin{proof}
Lemma~\ref{lem:indicator-energy} provides the indicator-energy criterion, so Proposition~\ref{prop:irreducible}
applies.
\end{proof}

\section{Positivity improving and the ground state}
\label{sec:PF}

\subsection{External theorems used}

\begin{theorem}[Positivity improving from positivity + irreducibility + holomorphy]
\label{thm:ABHN}
Let $E$ be a Banach lattice and $S$ a positive, irreducible, holomorphic $C_0$-semigroup on $E$.
Then $S$ is positivity improving: for each $t>0$ and each $0\le f\in E$ with $f\ne 0$,
one has $S(t)f>0$ (in the lattice sense; on $L^2$ this means $>0$ a.e.).
\end{theorem}

\begin{remark}[Source]
This statement appears, for example, as Theorem~2.3 in Arendt et al.,
\emph{Strict positivity for the principal eigenfunction of elliptic operators with various boundary conditions}
(see \S\ref{sec:bib}).
\end{remark}

\begin{theorem}[Simplicity of the principal eigenvalue under compact resolvent]
\label{thm:principal-simple}
Let $A$ be selfadjoint and lower bounded on $L^2(I)$ with compact resolvent, and let $S(t)=e^{-tA}$.
If $S$ is positivity improving, then the bottom of the spectrum $\min\sigma(A)$ is a simple eigenvalue
and admits an eigenfunction which is strictly positive a.e.
\end{theorem}

\begin{remark}[Source]
This is a standard Perron--Frobenius/Krein--Rutman/Jentzsch consequence for compact positive operators,
often stated for $(A+\mu)^{-1}$ or for $S(t)$ when it is compact. See, e.g., Proposition~2.4 in the same
paper of Arendt et al.
\end{remark}

\subsection{Application to $A_\lambda$}

\begin{proposition}[Positivity improving and simple ground state for $A_\lambda$]
\label{prop:groundstate}
Assume Theorem~\ref{thm:operator}. Then:
\begin{enumerate}
\item The semigroup $T(t)=e^{-tA_\lambda}$ is positivity preserving (Markovian).
\item $T(t)$ is irreducible.
\item $T(t)$ is holomorphic (indeed, $A_\lambda$ is selfadjoint and lower bounded).
\end{enumerate}
Consequently $T(t)$ is positivity improving, and the lowest eigenvalue of $A_\lambda$ is simple with a
strictly positive a.e.\ eigenfunction.
\end{proposition}

\begin{proof}
(1) Markov/positivity preservation follows from Lemma~\ref{lem:markov} and standard closed-form theory.
(2) is Corollary~\ref{cor:irreducible}.
(3) Since $A_\lambda$ is selfadjoint and lower bounded, $e^{-zA_\lambda}$ is bounded and holomorphic on
$\{z\in\C:\Re z>0\}$ by the spectral theorem.

Now apply Theorem~\ref{thm:ABHN} to obtain positivity improving, and then Theorem~\ref{thm:principal-simple}
to obtain simplicity and strict positivity of the ground state.
\end{proof}

\section{Evenness of the ground state from inversion symmetry}


\begin{proposition}[Inversion (reflection) symmetry]\label{prop:reflection}
Let $R:L^2(I)\to L^2(I)$ be the unitary involution $(RG)(u):=G(-u)$.
Then $R(\mathcal D(\mathcal E_\lambda))=\mathcal D(\mathcal E_\lambda)$ and
\[
\mathcal E_\lambda(RG)=\mathcal E_\lambda(G)\qquad(G\in\mathcal D(\mathcal E_\lambda)).
\]
Consequently, the associated operator $A_\lambda$ from Theorem~\ref{thm:operator} commutes with $R$.
\end{proposition}

\begin{proof}
Identify $L^2(I)$ with the closed subspace $H_I\subset L^2(\R)$ via extension by $0$.
Let the same symbol $R$ denote reflection on $L^2(\R)$: $(R\phi)(u):=\phi(-u)$.
Then $R$ is unitary, preserves $H_I$ (since $I$ is symmetric), and satisfies $RS_t=S_{-t}R$.
Therefore, for $t\in\R$ and $\phi\in L^2(\R)$,
\[
\|R\phi-S_tR\phi\|_2=\|R(\phi-S_{-t}\phi)\|_2=\|\phi-S_{-t}\phi\|_2=\|\phi-S_t\phi\|_2,
\]
using that $R$ is unitary and $\|\phi-S_{-t}\phi\|_2=\|S_t\phi-\phi\|_2=\|\phi-S_t\phi\|_2$.
Since every weight in Definition~\ref{def:E} is nonnegative, this implies $\mathcal E_\lambda(RG)=\mathcal E_\lambda(G)$.

For commutation with $A_\lambda$: invariance of a closed form under a unitary $U$ implies that the associated
selfadjoint operator commutes with $U$. Indeed, for $u\in\mathcal D(A_\lambda)$ and $v\in\mathcal D(\mathcal E_\lambda)$,
\[
\langle A_\lambda Ru,v\rangle=\mathcal E_\lambda(Ru,v)=\mathcal E_\lambda(u,R^{-1}v)=\langle A_\lambda u,R^{-1}v\rangle
=\langle R A_\lambda u,v\rangle,
\]
so $A_\lambda Ru=R A_\lambda u$.
\end{proof}


\begin{corollary}[Even ground state]
\label{cor:even}
Assume Theorem~\ref{thm:operator} and \ref{prop:reflection}. Let $\psi$ be the strictly positive ground-state
eigenfunction from Proposition~\ref{prop:groundstate}. Then $\psi$ is even: $\psi(-u)=\psi(u)$ a.e.
\end{corollary}

\begin{proof}
Since $A_\lambda R=RA_\lambda$, the function $\psi^\sharp:=R\psi$ is an eigenfunction for the same lowest
eigenvalue. Moreover $\psi^\sharp>0$ a.e.\ because $\psi>0$ a.e.
By simplicity of the ground-state eigenspace (Proposition~\ref{prop:groundstate}), $\psi^\sharp=c\psi$ for
some $c\in\R$. Positivity forces $c>0$, and normalizing $\|\psi^\sharp\|_2=\|\psi\|_2$ yields $c=1$.
Hence $\psi(-u)=\psi(u)$ a.e.
\end{proof}

\section{Summary of concrete progress}

\begin{itemize}
\item Starting solely from the explicit local formulas \eqref{eq:Wp}--\eqref{eq:WR}, we derived
a representation of $-\sum_v W_v(g*g^*)$ (up to an additive constant multiple of $\|g\|_2^2$) as a
positive combination of translation-difference energies in log-coordinates
(Definition~\ref{def:E}, Lemmas~\ref{lem:prime-energy}--\ref{lem:arch-energy}).
\item We proved the Markov/normal contraction inequality for this form (Lemma~\ref{lem:markov}).
\item Using only measure theory (Lebesgue density), we proved that invariance under all sufficiently small
translations forces a measurable subset of an interval to be null or conull (Lemma~\ref{lem:trans-inv}),
and we used it to show that $\mathcal E_\lambda(\1_B)=0$ implies $B$ is null or conull
(Lemma~\ref{lem:indicator-energy}).
\item Assuming the standard operator setup (closed form, selfadjoint operator, compact resolvent),
we obtained irreducibility and then (by a standard external theorem) positivity improving of the semigroup,
hence simplicity and strict positivity of the ground state (Proposition~\ref{prop:groundstate}).
\item Finally, inversion symmetry forces that strictly positive simple ground state to be even
(Corollary~\ref{cor:even}).
\end{itemize}

\section{Bibliographic pointers}
\label{sec:bib}

\begin{thebibliography}{9}

\bibitem{ArendtStrictPos}
W.~Arendt, D.~Daners, M.~Dier, and P.~K.~Jimenez.
\newblock \emph{Strict positivity for the principal eigenfunction of elliptic operators with various boundary conditions}.
\newblock Available as arXiv:1909.12194 and published versions; see Theorem~2.3 and Proposition~2.4 therein for
positivity improving and simplicity consequences used in \S\ref{sec:PF}.

\bibitem{ABHN}
W.~Arendt, C.~J.~K.~Batty, M.~Hieber, and F.~Neubrander.
\newblock \emph{Vector-valued Laplace Transforms and Cauchy Problems}.
\newblock 2nd ed., Birkh\"auser, 2011.

\end{thebibliography}

\end{document}
