\documentclass[11pt]{article}

\usepackage[margin=1in,top=0.85in,bottom=0.85in]{geometry}
\usepackage{amsmath,amssymb}
\usepackage{hyperref}
\usepackage{xcolor}
\usepackage{enumitem}
\usepackage[T1]{fontenc}

\hypersetup{colorlinks=true, linkcolor=black, citecolor=black, urlcolor=blue}

\definecolor{alertred}{RGB}{180,40,40}
\newcommand{\verify}{\textcolor{alertred}{\textbf{[VERIFY]}}\;}

\title{\textbf{Additional References}\\[0.4em]
  \large\normalfont\itshape
  Energy Decomposition, Compact Resolvent, and Perron--Frobenius Properties\\
  of the Restricted Weil Quadratic Form}
\author{}
\date{}

\begin{document}
\maketitle
\thispagestyle{empty}

\begin{abstract}\noindent
This document collects the 13 additional references recommended for inclusion in the
paper, grouped by priority.  Entries marked \textcolor{alertred}{\textbf{[VERIFY]}} require
confirmation of publication details against the original source before use; all other
entries have been verified against independent sources or primary documents seen during
the audit process.
\end{abstract}

\bigskip

%%% ================================================================
\section*{Group~1: Essential --- direct technical ancestors}
%%% ================================================================

\begin{enumerate}[label={\textbf{[\arabic*]}}, leftmargin=2.5em, itemsep=0.9em]

\item \label{Weil1952}
A.~Weil,
``Sur les formules explicites de la th\'eorie des nombres premiers,''
\textit{Comm.\ S\'em.\ Math.\ Univ.\ Lund} (1952), 252--265.
(Volume dedicated to Marcel Riesz.)\\
\textit{Note}: Primary source for the Weil explicit formula and positivity
criterion in the form used in the paper; the distributions $W_p$ and $W_{\mathbb{R}}$
descend from this work.

\item \label{Connes1999}
A.~Connes,
``Trace formula in noncommutative geometry and the zeros of the Riemann zeta function,''
\textit{Selecta Math.\ (N.S.)}~\textbf{5} (1999), no.~1, 29--106.\\
DOI:~\href{https://doi.org/10.1007/s000290050042}{10.1007/s000290050042}.\\
\textit{Note}: Establishes the semilocal trace formula and the explicit distributions
$W_p$, $W_{\mathbb{R}}$ whose negatives form the quadratic form studied in the paper;
also introduces the adele class space framework.

\item \label{ConnesConsani2023}
A.~Connes and C.~Consani,
``Spectral triples and $\zeta$-cycles,''
\textit{Enseign.\ Math.}~\textbf{69} (2023), no.~1--2, 93--148.\\
DOI:~\href{https://doi.org/10.4171/LEM/1042}{10.4171/LEM/1042}.\\
\textit{Note}: Asserts for each $\lambda>1$ the existence of a selfadjoint operator
$A_\lambda$ on $L^2([\lambda^{-1},\lambda], d^*x)$ with compact resolvent (see
\S6.4 of Connes~2026 [5] and its reference~[25]); the present paper supplies the
self-contained proof of this property.

\item \label{ConnesVanSuijlekom2025}
A.~Connes and W.~van Suijlekom,
``Quadratic forms, real zeros and echoes of the spectral action,''
\textit{Commun.\ Math.\ Phys.}~\textbf{406} (2025), Paper no.~312.\\
DOI:~\href{https://doi.org/10.1007/s00220-025-05240-6}{10.1007/s00220-025-05240-6}.
(Volume dedicated to H.~Araki.)\\
\textit{Note}: Proves (Theorem~6.1 of Connes~2026 [5]) that if the minimum eigenvalue
of $A_\lambda$ is simple with even eigenfunction, then all zeros of the Mellin transform
of the minimiser lie on the critical line; the present paper establishes precisely those
hypotheses.

\end{enumerate}

\bigskip
%%% ================================================================
\section*{Group~2: Highly recommended --- direct context and payoff}
%%% ================================================================

\begin{enumerate}[resume, label={\textbf{[\arabic*]}}, leftmargin=2.5em, itemsep=0.9em]

\item \label{Connes2026}
A.~Connes,
``The Riemann Hypothesis: Past, Present and a Letter Through Time,''
arXiv:2602.04022v1 [math.NT], February~2026.\\
\textit{Note}: Survey and original contribution explaining why the compact resolvent
and Perron--Frobenius properties of $A_\lambda$ (proved in the present paper) are
needed for the programme outlined in \S6.4 thereof; also provides the most current
bibliographic guide to the full Connes programme.

\item \label{ConnesConsani2021}
A.~Connes and C.~Consani,
``Weil positivity and trace formula, the archimedean place,''
\textit{Selecta Math.\ (N.S.)}~\textbf{27} (2021), no.~4, Paper no.~77, 70~pp.\\
DOI:~\href{https://doi.org/10.1007/s00029-021-00672-x}{10.1007/s00029-021-00672-x}.\\
\textit{Note}: Studies archimedean Weil positivity and the Sonin-space lower bound;
provides essential geometric context for the semilocal operator $A_\lambda$.

\end{enumerate}

\bigskip
%%% ================================================================
\section*{Group~3: Worthwhile --- complementary approaches and broader context}
%%% ================================================================

\begin{enumerate}[resume, label={\textbf{[\arabic*]}}, leftmargin=2.5em, itemsep=0.9em]

\item \label{BurnolPropagator}
\verify
J.-F.~Burnol,
``The explicit formula and a propagator,''\\
\textit{Note}: Complementary treatment of the Weil explicit formula through a
propagator analysis; the precise publication venue and year require verification.

\item \label{BurnolConductor}
\verify
J.-F.~Burnol,
``Spectral analysis of the local conductor operator,''\\
\textit{Note}: Analyses the spectral theory of a local operator complementary to the
approach taken in the present paper; publication details require verification.

\item \label{BurnolHilbert2001}
J.-F.~Burnol,
``Sur certains espaces de Hilbert de fonctions enti\`eres, li\'es \`a la transformation
de Fourier et aux fonctions $L$ de Dirichlet et de Riemann,''
\textit{C.\ R.\ Acad.\ Sci.\ Paris S\'er.\ I}~\textbf{333} (2001), 201--206.\\
DOI:~\href{https://doi.org/10.1016/S0764-4442(01)02049-3}{10.1016/S0764-4442(01)02049-3}.\\
\textit{Note}: Introduces Hilbert spaces of entire functions in the $L^2$ analysis
of Fourier/Dirichlet/Riemann $L$-functions; foundational for the Sonin space approach
credited in Connes~2026 [5], \S7.2.

\item \label{BurnolSonine2002}
J.-F.~Burnol,
``Sur les espaces de Sonine associ\'es par de~Branges \`a la transformation de Fourier,''
\textit{C.\ R.\ Acad.\ Sci.\ Paris S\'er.\ I}~\textbf{335} (2002), 689--692.\\
DOI:~\href{https://doi.org/10.1016/S1631-073X(02)02569-5}{10.1016/S1631-073X(02)02569-5}.\\
\textit{Note}: Studies Sonin spaces via de~Branges' theory of Fourier transforms;
introduced into the RH context in Connes~2026 [5], \S7.2.

\item \label{BurnolComplete2004}
J.-F.~Burnol,
``Two complete and minimal systems associated with the zeros of the Riemann zeta
function,''
\textit{J.\ Th\'eor.\ Nombres Bordeaux}~\textbf{16} (2004), no.~1, 65--94.\\
URL:~\href{http://jtnb.cedram.org/item?id=JTNB_2004__16_1_65_0}{jtnb.cedram.org}.\\
\textit{Note}: Constructs complete and minimal systems linked to zeta zeros; provides
$L^2$ functional-analytic background complementary to the present paper's approach.

\item \label{Poitou1977}
G.~Poitou,
``Sur les petits discriminants,''
\textit{S\'eminaire Delange--Pisot--Poitou. Th\'eorie des Nombres},
18\`eme ann\'ee, 1976--77, Expos\'e no.~6, 18~pp.,
Secr\'etariat Math\'ematique, Paris, 1977.\\
\textit{Note}: Classical application of Weil positivity to lower bounds on
discriminants of number fields; exemplifies the arithmetic power of the
positivity criterion that motivates the present paper.

\item \label{BattistoniMolteni}
\verify
F.~Battistoni and G.~Molteni,
``Explicit formul\ae\ for $L$-functions and generators of the class group,''\\
\textit{Note}: Applies explicit formula techniques to class group structure;
publication details (journal, volume, year) require verification before inclusion.

\end{enumerate}

\bigskip\bigskip
\noindent\textbf{Notes on \textcolor{alertred}{[VERIFY]} entries.}
Three entries above carry a \verify flag:
\begin{itemize}[leftmargin=2em, itemsep=3pt]
  \item \textbf{[7]} Burnol, ``The explicit formula and a propagator'': the title is
    known from literature but the precise journal, volume, and year could not be
    confirmed without access to a database.
  \item \textbf{[8]} Burnol, ``Spectral analysis of the local conductor operator'':
    same situation.
  \item \textbf{[13]} Battistoni--Molteni: full bibliographic data unavailable at
    time of drafting.
\end{itemize}
All three should be resolved via MathSciNet, zbMATH, or arXiv before the paper
is submitted.

\end{document}
