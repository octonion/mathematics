\documentclass[11pt]{article}

\usepackage[margin=0.85in,top=0.75in,bottom=0.75in]{geometry}
\usepackage{amsmath,amssymb}
\usepackage{booktabs}
\usepackage{array}
\usepackage{longtable}
\usepackage{xcolor}
\usepackage{colortbl}
\usepackage{xspace}
\usepackage{hyperref}
\usepackage{enumitem}
\usepackage[T1]{fontenc}
\usepackage{pdflscape}

\hypersetup{colorlinks=true, linkcolor=black, citecolor=black, urlcolor=blue}

%% Colours
\definecolor{headbg}{RGB}{44,62,80}
\definecolor{flagbg}{RGB}{255,243,205}
\definecolor{flagborder}{RGB}{230,126,34}
\definecolor{sectionbg}{RGB}{236,240,241}
\definecolor{sectionfg}{RGB}{85,85,85}

%% Custom column types
\newcolumntype{L}[1]{>{\raggedright\arraybackslash}p{#1}}
\newcolumntype{C}[1]{>{\centering\arraybackslash}p{#1}}

%% Flag cell macro
\newcommand{\flagrow}{\rowcolor{flagbg}}
\newcommand{\okrow}{\rowcolor{green!8}}
\newcommand{\sectionrow}{\rowcolor{sectionbg}}
\newcommand{\warn}{\textcolor{flagborder}{\textbf{[!]}}\xspace}
\newcommand{\ok}{\textcolor{green!60!black}{\textbf{[\checkmark]}}\xspace}

\newcommand{\bibkey}[1]{\texttt{\small #1}}
\newcommand{\locn}[1]{\textit{#1}}
\newcommand{\id}[1]{\textbf{#1}}
\newcommand{\internal}[1]{\texttt{\small #1}}

\setlist[itemize]{leftmargin=1.2em, topsep=1pt, itemsep=0pt, parsep=0pt}

%%% ----------------------------------------------------------------
\title{\textbf{External Theorem Audit}\\[0.4em]
  \large\normalfont\itshape
  Energy Decomposition, Compact Resolvent, and Perron--Frobenius Properties\\
  of the Restricted Weil Quadratic Form}
\author{}
\date{}
%%% ----------------------------------------------------------------

\begin{document}
\maketitle
\thispagestyle{empty}

\begin{abstract}\noindent
This document lists every external mathematical result invoked in the paper, together with
(i)~the precise statement assumed, (ii)~the exact bibliographic reference cited, and
(iii)~notes for the human verifier.  Its purpose is to make cross-checking against the
original sources as mechanical as possible.
The audit has been conducted in three rounds: initial identification; a first literature
search locating specific theorem numbers; and a second deep literature search with
extended source analysis, including direct quotation from secondary sources.
Items marked \textcolor{flagborder}{\textbf{[!]}} still require physical confirmation
against the book: the theorem number has been located by search but not verified against
the actual page.
Items marked \textcolor{green!60!black}{\textbf{[\checkmark]}} have been confirmed by
multiple independent sources (some with direct quotation); a physical spot-check is still
advisable but the result is expected correct.
\end{abstract}

\bigskip
\noindent\textbf{Bibliography keys used below.}
\begin{itemize}[leftmargin=2em]
  \item \bibkey{KatoPerturbation}: T.~Kato, \textit{Perturbation Theory for Linear Operators},
    Classics in Mathematics, Springer, 1995. DOI:~10.1007/978-3-642-66282-9.\newline
    {\footnotesize Note: the 1995 printing is a reprint of the 1980 second edition;
    theorem numbers VI.2.1, VI.2.23, IX.1.24, and Ex.\,IX.1.25 are consistent with this edition.
    \textbf{Correction}: ``IX.1.4'' cited in earlier audit versions is Remark~1.4
    (unrelated); the correct holomorphic semigroup references are Thm.\,IX.1.24 and Ex.\,IX.1.25.
    Numbering differs from the 1966 first edition.}
  \item \bibkey{FOT}: M.~Fukushima, Y.~Oshima, M.~Takeda, \textit{Dirichlet Forms and
    Symmetric Markov Processes}, 2nd revised ed., De~Gruyter Studies in Math.\ vol.~19,
    De~Gruyter, 2011. DOI:~10.1515/9783110218091.\newline
    {\footnotesize Note: publication year is 2011 per De~Gruyter catalogue; some sources cite 2010.}
  \item \bibkey{ATGStrictPos2020}: W.~Arendt, A.\,F.\,M.~ter~Elst, J.~Gl\"{u}ck,
    ``Strict positivity for the principal eigenfunction\ldots'',
    \textit{Adv.\ Nonlinear Stud.}\ \textbf{20} (2020), no.\,3, 633--650.
    DOI:~10.1515/ans-2020-2091.\newline
    {\footnotesize Use the published DOI (not the arXiv preprint 1909.12194).}
  \item \bibkey{OuhabazHeatEq}: E.-M.~Ouhabaz, \textit{Analysis of Heat Equations on Domains},
    LMS Monographs vol.~31, Princeton UP, 2005. ISBN:~978-0-691-12016-4.
  \item \bibkey{BrezisFA}: H.~Brezis, \textit{Functional Analysis, Sobolev Spaces and Partial
    Differential Equations}, Universitext, Springer, 2011.
    ISBN:~978-0-387-70913-0. DOI:~10.1007/978-0-387-70914-7.\newline
    {\footnotesize Note: cover reads ``2010'' but copyright/print year is 2011.
    This is the correct source for T2. \textit{(New entry; add to bibliography.)}}
  \item \bibkey{LiebLossAnalysis}: E.\,H.~Lieb, M.~Loss, \textit{Analysis}, 2nd ed.,
    Grad.\ Studies in Math.\ vol.~14, AMS, 2001.
    ISBN:~978-0-8218-2783-3. DOI:~10.1090/gsm/014.\newline
    {\footnotesize \textbf{Does not contain the Kolmogorov--Riesz theorem.}
    The ``Thm.\,2.13'' number was a citation error: that number belongs to
    \textit{Ground States of Quantum Field Models} (Hiroshima 2019), a secondary source
    which internally numbers a version of the result as 2.13.
    Remove \texttt{LiebLossAnalysis} from the T2 citation.}
  \item \bibkey{SchaeferBanachLattices}: H.\,H.~Schaefer, \textit{Banach Lattices and Positive
    Operators}, Grundlehren vol.~215, Springer, 1974.
    ISBN:~978-3-540-06936-2. DOI:~10.1007/978-3-642-65970-6.
    \textit{(New entry; add to bibliography.)}
\end{itemize}

%%% ================================================================
\clearpage
\begin{landscape}

\renewcommand{\arraystretch}{1.35}
\small

\begin{longtable}{
    C{0.55cm}
    L{4.2cm}
    L{7.2cm}
    L{4.5cm}
    L{4.6cm}
  }

\toprule
\rowcolor{headbg}
\textcolor{white}{\textbf{\#}} &
\textcolor{white}{\textbf{Internal label / where used}} &
\textcolor{white}{\textbf{Statement assumed in this paper}} &
\textcolor{white}{\textbf{Cited reference}} &
\textcolor{white}{\textbf{Notes for verifier}} \\
\midrule
\endfirsthead

\toprule
\rowcolor{headbg}
\textcolor{white}{\textbf{\#}} &
\textcolor{white}{\textbf{Internal label / where used}} &
\textcolor{white}{\textbf{Statement assumed in this paper}} &
\textcolor{white}{\textbf{Cited reference}} &
\textcolor{white}{\textbf{Notes for verifier}} \\
\midrule
\endhead

\bottomrule
\multicolumn{5}{r}{\small\itshape Continued on next page\ldots}\\
\endfoot

\bottomrule
\endlastfoot

%% ─── Section: Form / operator correspondence ───────────────────
\sectionrow
\multicolumn{5}{l}{\textcolor{sectionfg}{\small\textsc{Form\,/\,operator correspondence}}} \\
\midrule

\okrow
\id{T1} \ok &
\textbf{Representation theorem for closed forms} \newline
\textit{Used in:} \internal{thm:operator} Step~1 (first occurrence) &
If $\mathcal{E}$ is a densely defined, closed, lower-bounded, symmetric form on a Hilbert
space $H$, then there exists a unique selfadjoint operator $A\ge 0$ such that
$\mathcal{D}(\mathcal{E})=\mathcal{D}(A^{1/2})$ and $\mathcal{E}(u,v)=\langle A^{1/2}u,A^{1/2}v\rangle$
for all $u,v\in\mathcal{D}(\mathcal{E})$. &
\bibkey{KatoPerturbation} \locn{Thm.\,VI.2.1}\newline
Also:\newline
\bibkey{KatoPerturbation} \locn{Thm.\,VI.2.23}\newline
Also:\newline
\bibkey{FOT} \locn{Thm.\,1.3.1} &
\ok \textbf{Physically confirmed} from uploaded Kato and FOT.
Kato VI.2.1 (\textit{First} Representation Theorem): existence of unique m-sectorial $T$
with $t[u,v]=(Tu,v)$ for $u\in D(T)$, $v\in D(t)$.
Kato VI.2.23 (\textit{Second} Representation Theorem): $D(H^{1/2})=D(\mathfrak{h})$
and $\mathfrak{h}[u,v]=(H^{1/2}u,H^{1/2}v)$.
Both theorems together give the full statement.
FOT Thm.\,1.3.1 (physically confirmed): one-to-one correspondence between closed
symmetric forms and non-positive definite selfadjoint operators on $H$;
Corollary~1.3.1 gives the domain characterisation $\mathcal{E}(u,v)=(-Au,v)$
for $u\in D(A)$. \\[3pt]

\okrow
\id{T1$'$} \ok &
\textbf{Representation theorem (domain characterisation)} \newline
\textit{Used in:} \internal{prop:reflection} Step~5.4 &
$w\in\mathcal{D}(A)$ if and only if there exists $h\in H$ such that
$\mathcal{E}(w,v)=\langle h,v\rangle$ for all $v\in\mathcal{D}(\mathcal{E})$;
in that case $Aw=h$.
(Applied to conclude $Ru\in\mathcal{D}(A_\lambda)$ and
$A_\lambda(Ru)=R(A_\lambda u)$.) &
\bibkey{KatoPerturbation} \locn{Thm.\,VI.2.1 (iii)}\newline
(Same reference as T1) &
\ok \textbf{Physically confirmed} from uploaded Kato.
Kato Thm.\,VI.2.1 condition~(iii) states explicitly: ``if $u\in D(t)$, $w\in H$,
and $t[u,v]=(w,v)$ for every $v$ in a core of $t$, then $u\in D(T)$ and $Tu=w$.''
This is the ``if'' direction.
The ``only if'' direction follows from condition~(i): $t[u,v]=(Tu,v)$ for $u\in D(T)$.
Together conditions (i) and (iii) give the full ``iff'', validating Step~5.4. \\[3pt]

%% ─── Section: Compactness ───────────────────────────────────────
\sectionrow
\multicolumn{5}{l}{\textcolor{sectionfg}{\small\textsc{Compactness}}} \\
\midrule

\okrow
\id{T2} \ok &
\textbf{Kolmogorov--Riesz--Fr\'{e}chet compactness criterion} \newline
\textit{Stated as:} \internal{thm:KR}\newline
\textit{Applied in:} \internal{prop:compactEmbed} Step~4 &
\textbf{Brezis Thm.\,4.26} (bounded $\Omega$, no tightness needed):
$F$ bounded in $L^p(\mathbb{R}^N)$ + translation equicontinuity
$\Rightarrow$ $F|_\Omega$ has compact closure in $L^p(\Omega)$ for any measurable
$\Omega$ with finite measure.\newline\newline
\textbf{Brezis Cor.\,4.27} (full $\mathbb{R}^N$, tightness required):
Adds condition $\forall\varepsilon>0\;\exists\Omega\subset\mathbb{R}^N$ bounded such that
$\|f\|_{L^p(\mathbb{R}^N\setminus\Omega)}<\varepsilon$ $\forall f\in F$
$\Rightarrow$ $F$ compact in $L^p(\mathbb{R}^N)$.
(Remark~13 confirms the converse holds; this is the complete characterization.) &
\bibkey{BrezisFA}\newline
\locn{Thm.\,4.26 \&}\newline
\locn{Cor.\,4.27, p.\,111--113}\newline
{\footnotesize(physically confirmed from uploaded book pages)} &
\ok \textbf{Physically confirmed} from uploaded book pages.
\textbf{Critical distinction:}
If the compactness argument is on a bounded domain (finite measure),
Thm.\,4.26 alone suffices.
If on $L^2(\mathbb{R})$ (full line), Cor.\,4.27 is required;
the proof must separately establish tightness
(uniform $L^2$-tail decay).\newline
\textbf{Citation correction:} \bibkey{LiebLossAnalysis} does not
contain this theorem.  ``Thm.\,2.13'' originated in
\textit{Ground States of Quantum Field Models} (Hiroshima 2019)
with its own internal numbering.
Replace with:\newline
{\footnotesize\texttt{\textbackslash cite[Cor.\textasciitilde 4.27]\{BrezisFA\}}}. \\[3pt]

%% ─── Section: Semigroup / Markov ────────────────────────────────
\sectionrow
\multicolumn{5}{l}{\textcolor{sectionfg}{\small\textsc{Markov property and positivity}}} \\
\midrule

\okrow
\id{T3} \ok &
\textbf{Markov form $\Rightarrow$ positivity-preserving semigroup} \newline
\textit{Used in:} \internal{prop:groundstate} Step~1 &
If a closed symmetric form $\mathcal{E}$ satisfies the normal contraction (Markov) property
$\mathcal{E}(\Phi\circ G)\le\mathcal{E}(G)$ for every normal contraction $\Phi$,
then the associated $C_0$-semigroup $T(t)=e^{-tA}$ is positivity preserving
(i.e.\ $G\ge 0$ a.e.\ implies $T(t)G\ge 0$ a.e.). &
\bibkey{FOT} \locn{Thm.\,1.4.1}\newline
Also:\newline
\bibkey{OuhabazHeatEq} \locn{Thm.\,1.4.1} &
\ok \textbf{Physically confirmed} from uploaded FOT.
FOT Thm.\,1.4.1 (physically verified): lists five equivalent conditions (a)--(e)
for the Markovian property, including
(a)~$T_t$ Markovian $\forall t>0$; (c)~$\mathcal{E}$ Markovian; (e)~every normal
contraction operates on $\mathcal{E}$.
The five conditions are mutually equivalent; the implication used in the
manuscript is (e)$\Rightarrow$(a) [or (c)$\Rightarrow$(a)], which is direct.
Ouhabaz Thm.\,1.4.1 treats the more general sectorial setting. \\[3pt]

\okrow
\id{T4} \ok &
\textbf{Positivity improving from positivity + irreducibility + holomorphy} \newline
\textit{Stated as:} \internal{thm:ABHN}\newline
\textit{Applied in:} \internal{prop:groundstate} Step~4 &
Let $E$ be a Banach lattice and $S$ a positive, irreducible, holomorphic $C_0$-semigroup
on $E$.  Then $S$ is \textit{positivity improving\/}: for each $t>0$ and each
$0\le f\in E$, $f\ne 0$, one has $S(t)f>0$ (on $L^2$ this means~$>0$ a.e.). &
\bibkey{ATGStrictPos2020} \locn{Thm.\,2.3} &
\ok \textbf{Physically confirmed} from uploaded ATG PDF (arXiv:1909.12194v3,
p.\,5).
ATG Thm.\,2.3 is stated for general Banach lattice $C_0$-semigroups (not restricted to
elliptic generators); proof defers to Majewski--Robinson [23, Thm.\,3].
\textbf{Holomorphy dependency chain}: selfadjoint $A_\lambda\ge0$
$\Rightarrow$ m-sectorial (T6, Kato \textbf{Ex.\,IX.1.25}) $\Rightarrow$ holomorphic semigroup
$\Rightarrow$ hypothesis of T4 satisfied.
(\textbf{Correction}: previous note wrongly cited ``IX.1.4'' here; corrected to match T6.)
Without T6, this application would be unjustified. \\[3pt]

\okrow
\id{T5} \ok &
\textbf{Simplicity of principal eigenvalue under positivity improving\,+\,compact resolvent} \newline
\textit{Stated as:} \internal{thm:principal-simple}\newline
\textit{Applied in:} \internal{prop:groundstate} Step~5 &
Let $A$ be selfadjoint, lower bounded, with compact resolvent on $L^2(I)$.
If $e^{-tA}$ is positivity improving, then $\min\sigma(A)$ is a simple eigenvalue
admitting an eigenfunction strictly positive a.e. &
\bibkey{ATGStrictPos2020} \locn{Prop.\,2.4} &
\ok \textbf{Physically confirmed} from uploaded ATG PDF (arXiv:1909.12194v3,
p.\,5).
ATG Prop.\,2.4 requires exactly positivity-improving + compact resolvent
(no additional hypotheses).  The \textbf{four} explicit conclusions are:
(a)~$\sigma(A)\ne\emptyset$;
(b)~$\lambda_1:=\inf\{\operatorname{Re}\lambda:\lambda\in\sigma(A)\}$ is an eigenvalue
(so the infimum is a minimum);
(c)~the associated eigenfunction satisfies $u\gg 0$ (strictly positive quasi-interior point);
(d)~the algebraic multiplicity of $\lambda_1$ is one.
\textbf{Audit correction}: previous version collapsed (a) and (b) and reported
only three conclusions.
Compact resolvent $\Rightarrow$ $e^{-tA}$ compact for $t>0$ (spectral theorem), which
supplies the compactness needed for the Perron--Frobenius/Krein--Rutman argument. \\[3pt]

%% ─── Section: Standard results without citation ─────────────────
\sectionrow
\multicolumn{5}{l}{\textcolor{sectionfg}{\small\textsc{Standard results invoked without explicit citation}}} \\
\midrule

\okrow
\id{T6} \ok &
\textbf{Selfadjoint $A\ge0$ generates holomorphic semigroup} \newline
\textit{Used in:} \internal{prop:groundstate} Step~3 &
If $A$ is selfadjoint with $A\ge0$, then $e^{-zA}$ is a bounded operator for all
$z$ with $\operatorname{Re}z>0$, $\|e^{-zA}\|\le1$, and
$z\mapsto e^{-zA}$ is holomorphic on $\{\operatorname{Re}z>0\}$. &
\bibkey{KatoPerturbation}\newline
\locn{Thm.\,IX.1.24}\newline
\locn{Ex.\,IX.1.25}\newline
{\footnotesize(\textbf{correction}: previous\newline citation ``IX.1.4'' was wrong)} &
\ok \textbf{Physically confirmed} from uploaded Kato.
\textbf{Critical correction}: ``Thm.\,IX.1.4'' does \textbf{not exist} in Kato.
Label IX.1.4 is \textit{Remark~1.4} (about recovering a generator from a contraction
semigroup --- unrelated to holomorphic semigroups).
\textbf{Correct references (both physically confirmed):}
\begin{itemize}[noitemsep,topsep=2pt,leftmargin=1.2em]
  \item \textbf{Kato Thm.\,IX.1.24}: Let $T$ be m-sectorial in a Hilbert space
    with vertex~$0$ and half-angle $\omega\in(0,\tfrac\pi2]$.
    Then $e^{-tT}$ is holomorphic for $|\arg t|<\omega$ and $\|e^{-tT}\|\le 1$.
  \item \textbf{Kato Ex.\,IX.1.25} (the directly applicable result):
    ``If $H$ is a nonneg.\ selfadjoint operator in a Hilbert space,
    $e^{-tH}$ is holomorphic for $\operatorname{Re}t>0$ and $\|e^{-tH}\|\le 1$.''
\end{itemize}
Logical chain: selfadjoint $A\ge0$ has numerical range in $[0,\infty)$,
hence is m-sectorial with any $\omega<\tfrac\pi2$; Ex.\,1.25 gives holomorphy
directly.
\textbf{Action}: in the manuscript, replace citation \texttt{IX.1.4} with
Kato \textbf{Ex.\,IX.1.25} (or Thm.\,IX.1.24 for the general m-sectorial version). \\[3pt]

\okrow
\id{T7} \ok &
\textbf{Closed ideals in $L^2(I)$ have the form $L^2(B)$} \newline
\textit{Used in:} \internal{cor:irreducible} Step~1 &
Every closed ideal in $L^2(I)$ (a closed subspace $J$ such that $|f|\le|g|$ a.e.\
and $g\in J$ implies $f\in J$) has the form $L^2(B)$ for some measurable $B\subset I$.
(Called ``standard lattice-theory fact'' with no citation.) &
\bibkey{SchaeferBanachLattices}\newline
\locn{Sect.\,III.1, Ex.\,1}\newline
{\footnotesize(\textbf{correction}: previous\newline locator ``Ex.\,1.1.6''\newline
was wrong; see notes)} &
\ok \textbf{Confirmed} via ATG's own citation, physically verified from uploaded ATG PDF.
ATG p.\,3 characterises closed ideals in $L_p(\Omega)$ and explicitly cites
``[28, Section~III.1, Example~1]''; ATG's [28] is Schaefer \textit{Banach Lattices and
Positive Operators}, i.e.\ \texttt{SchaeferBanachLattices} in our bibliography.
The correct Schaefer locator is therefore \textbf{Section~III.1, Example~1}
(not ``Ex.\,1.1.6'' as previously conjectured).
Schaefer's ``Examples'' sections contain theorem-grade content; no edition discrepancy.
\textbf{Locator correction}: replace ``Ex.\,1.1.6'' with
{\footnotesize\texttt{\textbackslash cite[Sect.\textasciitilde III.1,\textasciitilde Ex.\textasciitilde 1]\{SchaeferBanachLattices\}}}. \\[3pt]

%% ─── Section: Bibliographic anomaly (resolved) ─────────────────
\sectionrow
\multicolumn{5}{l}{\textcolor{sectionfg}{\small\textsc{Bibliographic anomaly (resolved)}}} \\
\midrule

\okrow
--- \ok &
\textbf{Orphaned bibliography entry} \newline
\texttt{OuhabazConvex1996} appears in the bibliography but is never \texttt{\textbackslash cite}d in the body. &
E.-M.~Ouhabaz, ``Invariance of closed convex sets and domination criteria for
semigroups,'' \textit{Potential Anal.}\ \textbf{5} (1996), 611--625.
DOI:~10.1007/BF00275797. &
\textit{Bibliography only; never cited in body.} &
\ok Confirmed by literature search: the relevant results are fully covered by
\texttt{OuhabazHeatEq} (the book).
\textbf{Action: remove \texttt{OuhabazConvex1996}} from the bibliography. \\[3pt]

\end{longtable}

\end{landscape}

%%% ================================================================
\clearpage
\section*{Summary}

\begin{center}
\renewcommand{\arraystretch}{1.3}
\begin{tabular}{L{12cm}c}
\toprule
Item & Count \\
\midrule
Total distinct external results invoked (T1--T7;\@ T1 and T1$'$ are two uses of one theorem) & 7 \\[2pt]
\midrule
\multicolumn{2}{l}{\textit{Citation status after four rounds of audit (ATG PDF physically confirmed):}} \\[2pt]
Fully confirmed \textcolor{green!60!black}{\textbf{[\checkmark]}} (no further action needed) & 7 \\
\quad T1 (Kato VI.2.1 + VI.2.23 + FOT 1.3.1; \textbf{physically confirmed}) & \\
\quad T1$'$ (Kato VI.2.1(iii); iff confirmed from conditions (i) and (iii); \textbf{physically confirmed}) & \\
\quad T2 (Brezis Thm.\,4.26 + Cor.\,4.27; \textbf{physically confirmed from uploaded pages}) & \\
\quad T3 (FOT Thm.\,1.4.1; 5 equivalent conditions confirmed; \textbf{physically confirmed}) & \\
\quad T4 (ATG Thm.\,2.3; \textbf{physically confirmed from uploaded ATG PDF}; T4 notes corrected: ``IX.1.4''$\to$``Ex.\,IX.1.25'') & \\
\quad T5 (ATG Prop.\,2.4; \textbf{physically confirmed from uploaded ATG PDF}; \textbf{four} conclusions, not three) & \\
\quad T6 (Kato Ex.\,IX.1.25 + Thm.\,IX.1.24; \textbf{physically confirmed};
  previous citation ``IX.1.4'' was wrong --- Remark~1.4 is unrelated) & \\
\quad T7 (Schaefer Sect.\,III.1, Ex.\,1; \textbf{confirmed via ATG's own citation};
  locator ``Ex.\,1.1.6'' was wrong --- corrected to ``Sect.\,III.1, Ex.\,1'') & \\[2pt]
Requires physical book check \textcolor{flagborder}{\textbf{[!]}} & 0 \\
\midrule
Orphaned bibliography entries: confirmed for removal & 2 \\
\quad \texttt{OuhabazConvex1996} (results covered by \texttt{OuhabazHeatEq}) & \\
\quad \texttt{LiebLossAnalysis} (does \textbf{not} contain Kolmogorov--Riesz; citation was an error) & \\
New bibliography entries required & 2 \\
\quad \texttt{BrezisFA} (Brezis 2011, Thm.\,4.26\,+\,Cor.\,4.27; primary source for T2) & \\
\quad \texttt{SchaeferBanachLattices} (Schaefer 1974; for T7, if option~(a) chosen) & \\
Manuscript citations to correct or add & 5 \\
\quad T6: \texttt{\textbackslash cite[Ex.\textasciitilde IX.1.25]\{KatoPerturbation\}}
  (replaces wrong ``IX.1.4''; Ex.\,IX.1.25 \textbf{physically confirmed}) & \\
\quad T1: add \texttt{\textbackslash cite[Thm.\textasciitilde VI.2.23]\{KatoPerturbation\}}
  (Second Representation Theorem; \textbf{physically confirmed}) & \\
\quad T2: replace \texttt{\textbackslash cite\{LiebLossAnalysis\}} with
  \texttt{\textbackslash cite[Cor.\textasciitilde 4.27]\{BrezisFA\}} (wrong source) & \\
\quad T3: update Ouhabaz citation from ``Ch.\,1'' to
  \texttt{\textbackslash cite[Thm.\textasciitilde 1.4.1]\{OuhabazHeatEq\}} & \\
\quad T7: replace wrong locator with
  \texttt{\textbackslash cite[Sect.\textasciitilde III.1,\textasciitilde Ex.\textasciitilde 1]\{SchaeferBanachLattices\}}
  (confirmed via ATG's own citation) & \\
\bottomrule
\end{tabular}
\end{center}

\bigskip
\noindent\textbf{Remaining actions before submission} (in priority order).
\begin{enumerate}[leftmargin=2em]

  \item \textbf{T6 (holomorphic semigroup) --- citation correction.}\newline
    The previously proposed citation ``Kato Thm.\,IX.1.4'' is \textbf{wrong}:
    that label is Remark~1.4 (recovering a generator from a contraction semigroup),
    unrelated to holomorphy. \textbf{Correct references} (physically confirmed):\newline
    (a)~\textbf{Kato Ex.\,IX.1.25}: nonneg.\ selfadjoint $\Rightarrow$ $e^{-tH}$ holomorphic
    for $\operatorname{Re}t>0$, $\|e^{-tH}\|\le1$ --- the directly applicable result.\newline
    (b)~\textbf{Kato Thm.\,IX.1.24}: m-sectorial with vertex~$0$ $\Rightarrow$ holomorphic
    semigroup, $\|e^{-tT}\|\le1$ --- the general theorem backing Ex.\,1.25.\newline
    Update manuscript citation to
    {\small\texttt{\textbackslash cite[Ex.\textasciitilde IX.1.25]\{KatoPerturbation\}}}.

  \item \textbf{T2 (Kolmogorov--Riesz) --- citation correction + possible proof gap.}\newline
    The citation \texttt{LiebLossAnalysis} is wrong (Lieb--Loss does not contain this theorem;
    ``Thm.\,2.13'' originated in Hiroshima 2019 with its own internal numbering).
    \textbf{Correct citation:} Brezis Thm.\,4.26 (bounded $\Omega$) or Cor.\,4.27 (full
    $\mathbb{R}^N$, requires tightness).
    \textbf{Proof gap check:} if the compactness is invoked on $L^2(\mathbb{R})$,
    Cor.\,4.27 is needed and the proof must establish uniform $L^2$-tail decay.
    Update citation to
    {\small\texttt{\textbackslash cite[Cor.\textasciitilde 4.27]\{BrezisFA\}}}.

  \item \textbf{T1 (form-domain identity)}: the manuscript should cite both Kato theorems:
    VI.2.1 (existence, operator equality) and \textbf{VI.2.23} (domain identity
    $D(H^{1/2})=D(\mathfrak{h})$). Add:\newline
    {\small\texttt{\textbackslash cite[Thm.\textasciitilde VI.2.23]\{KatoPerturbation\}}}.

  \item \textbf{T3 (Ouhabaz precise number)}: update citation to:\newline
    {\small\texttt{\textbackslash cite[Thm.\textasciitilde 1.4.1]\{OuhabazHeatEq\}}}.

  \item \textbf{T4 (holomorphy chain)}: the internal note previously cited ``IX.1.4'';
    now corrected to Ex.\,IX.1.25 (matching T6).  No manuscript change needed beyond
    the T6 correction already required.

  \item \textbf{T5 (Prop.\,2.4 conclusions)}: the manuscript statement should reflect
    all \textbf{four} conclusions of ATG Prop.\,2.4 --- in particular conclusion~(a)
    ($\sigma(A)\ne\emptyset$) and conclusion~(b) ($\lambda_1$ is an eigenvalue) are
    separate and should not be collapsed.  Verify that the manuscript proof of
    \internal{prop:groundstate} Step~5 invokes all four.

  \item \textbf{T7 (closed ideals, Schaefer locator) --- citation correction.}\newline
    Previous locator ``Ex.\,1.1.6'' was incorrect.
    \textbf{Correct locator} (confirmed via ATG's own citation of Schaefer):
    \textbf{Schaefer Sect.\,III.1, Example~1}.
    Update manuscript citation to:\newline
    {\small\texttt{\textbackslash cite[Sect.\textasciitilde III.1,\textasciitilde
    Ex.\textasciitilde 1]\{SchaeferBanachLattices\}}}.

  \item \textbf{Bibliography cleanup:}
    remove \texttt{OuhabazConvex1996} (subsumed);
    remove \texttt{LiebLossAnalysis} from T2 cite;
    add \texttt{BrezisFA}; add \texttt{SchaeferBanachLattices}.

\end{enumerate}

\bigskip\bigskip
\noindent\textbf{Verified BibTeX entries} (ready for copy-paste into \texttt{.bib} file).
\begin{small}
\begin{verbatim}
@book{KatoPerturbation,
  author    = {Kato, Tosio},
  title     = {Perturbation Theory for Linear Operators},
  series    = {Classics in Mathematics},
  publisher = {Springer-Verlag},
  address   = {Berlin},
  year      = {1995},
  note      = {Reprint of the 1980 second edition},
  isbn      = {978-3-540-58661-6},
  doi       = {10.1007/978-3-642-66282-9}
}

% NEW ENTRY (add for T2; replaces LiebLossAnalysis for this citation):
@book{BrezisFA,
  author    = {Brezis, Ha{\"i}m},
  title     = {Functional Analysis, {S}obolev Spaces and Partial
               Differential Equations},
  series    = {Universitext},
  publisher = {Springer},
  address   = {New York},
  year      = {2011},
  isbn      = {978-0-387-70913-0},
  doi       = {10.1007/978-0-387-70914-7}
}

@book{FOT,
  author    = {Fukushima, Masatoshi and Oshima, Yoichi
               and Takeda, Masayoshi},
  title     = {Dirichlet Forms and Symmetric Markov Processes},
  edition   = {2},
  series    = {De Gruyter Studies in Mathematics},
  volume    = {19},
  publisher = {De Gruyter},
  address   = {Berlin},
  year      = {2011},
  doi       = {10.1515/9783110218091}
}

@book{OuhabazHeatEq,
  author    = {Ouhabaz, El-Maati},
  title     = {Analysis of Heat Equations on Domains},
  series    = {London Mathematical Society Monographs Series},
  volume    = {31},
  publisher = {Princeton University Press},
  address   = {Princeton, NJ},
  year      = {2005},
  isbn      = {978-0-691-12016-4}
}

@article{ATGStrictPos2020,
  author  = {Arendt, Wolfgang and ter Elst, A. F. M. and
             Gl{\"u}ck, Jochen},
  title   = {Strict positivity for the principal eigenfunction
             of elliptic operators with various boundary
             conditions},
  journal = {Advanced Nonlinear Studies},
  volume  = {20},
  number  = {3},
  pages   = {633--650},
  year    = {2020},
  doi     = {10.1515/ans-2020-2091}
}

% NEW ENTRY (add for T7):
@book{SchaeferBanachLattices,
  author    = {Schaefer, Helmut H.},
  title     = {Banach Lattices and Positive Operators},
  series    = {Die Grundlehren der mathematischen Wissenschaften},
  volume    = {215},
  publisher = {Springer-Verlag},
  address   = {Berlin, Heidelberg},
  year      = {1974},
  isbn      = {978-3-540-06936-2},
  doi       = {10.1007/978-3-642-65970-6}
}

% REMOVE these entries:
% OuhabazConvex1996  -- orphaned; subsumed by OuhabazHeatEq
% LiebLossAnalysis (from T2 cite only) -- does not contain this theorem;
%   keep entry if cited elsewhere in paper, but remove from T2 \cite
\end{verbatim}
\end{small}

\end{document}
