\documentclass[11pt]{article}

\usepackage[margin=1in]{geometry}
\usepackage{amsmath,amssymb,amsthm}
\usepackage{enumitem}
\usepackage{hyperref}

\title{Does the Writeup Discharge the First Requirement in \S 6.6?\\
Simplicity and Evenness of the Lowest Eigenvector of $QW_\lambda$}
\author{}
\date{\today}

\begin{document}
\maketitle

\begin{abstract}
Section 6.6 (Connes) states that to apply Theorem 6.1 one needs to show that the smallest eigenvalue
of the Weil quadratic form $QW_\lambda$ is simple with an even eigenvector, analogously to a known fact
for the prolate wave operator. This note explains how the combined writeup establishes that property
for the truncated/semi-local Weil form constructed from the local explicit-formula distributions,
and it highlights which parts rely on standard external results.
\end{abstract}

\section{Statement from \S 6.6}
The relevant requirement in \S 6.6 is:

\begin{quote}
``In order to apply Theorem 6.1 one needs to show that the smallest eigenvalue of the Weil quadratic form
$QW_\lambda$ is simple with even eigenvector. The analogue of this property is known for the prolate wave operator.''
\end{quote}

\noindent The combined writeup addresses precisely this \emph{spectral simplicity and symmetry} property
for the form it realizes (up to an additive constant shift).

\section{What the writeup proves}
Interpreting $QW_\lambda$ as the truncated (semi-local) Weil quadratic form attached to test functions
supported in $[\lambda^{-1},\lambda]$ and defined using the local explicit-formula distributions
$W_p$ and $W_{\mathbb R}$, the writeup establishes:

\begin{itemize}[leftmargin=2em]
\item \textbf{Simplicity of the bottom eigenvalue:}
the lowest eigenvalue is simple and has a strictly positive eigenfunction (a.e.).
\item \textbf{Evenness of a lowest eigenvector:}
the ground state can be chosen even (under the reflection symmetry in logarithmic coordinates).
\end{itemize}

\noindent Since adding a constant multiple of the identity to a selfadjoint operator shifts the spectrum
without changing eigenvectors or multiplicities, the same simplicity/eigenfunction conclusions hold
for the quadratic form before/after such a constant shift.

\section{How the writeup achieves ``simple + even''}
The argument decomposes into three conceptual steps.

\subsection{Step 1: Rewrite $QW_\lambda$ as a Dirichlet-form-type energy (up to a constant)}
The local terms are rewritten as nonnegative translation-difference energies plus constants.

\begin{itemize}[leftmargin=2em]
\item \textbf{Archimedean term:} after the change of variables $x=e^t$,
the term $-W_{\mathbb R}(f)$ is written in the form
\[
-W_{\mathbb R}(f)
\;=\;
\int w(t)\,\|\widetilde G - S_t \widetilde G\|^2\,dt
\;+\; c_\infty(\lambda)\,\|G\|^2,
\]
where $S_t$ denotes translation on the logarithmic variable, and the tail $t>2L$ collapses to a constant
multiple of $\|G\|^2$ by disjoint support.
\item \textbf{Prime terms:} similarly, each $-W_p(f)$ becomes a (finite, by truncation) weighted sum of squared differences
\[
-W_p(f)
\;=\;
\sum_{m} \alpha_{p,m}\,\|\widetilde G - S_{m\log p}\widetilde G\|^2
\;+\; c_p(\lambda)\,\|G\|^2,
\]
with only finitely many relevant $m$ because the support truncation forces $\langle g,U_a g\rangle=0$ for $a>\lambda^2$.
\end{itemize}

\noindent These formulas motivate the definition of the quadratic form $\mathcal E_\lambda$ as the sum of all
translation-difference energies (prime and Archimedean), so that $QW_\lambda$ agrees with $\mathcal E_\lambda$
up to an additive constant multiple of $\|G\|^2$.

\subsection{Step 2: Show the ground state is simple (via positivity improving + compact resolvent)}
The writeup then places $\mathcal E_\lambda$ in the framework of (symmetric) Dirichlet forms:

\begin{itemize}[leftmargin=2em]
\item \textbf{Markov property:} the form satisfies the normal contraction property
$\mathcal E_\lambda(\Phi(\phi))\le \mathcal E_\lambda(\phi)$ for standard contractions $\Phi$.
This yields a positive semigroup (standard Dirichlet form theory).

\item \textbf{Irreducibility criterion:} the writeup proves a key measurable-set lemma:
if $B\subset I$ and $\mathcal E_\lambda(\mathbf 1_B)=0$, then $B$ is null or conull in $I$.
Combined with standard equivalences in Dirichlet form theory, this gives \emph{irreducibility}
of the associated semigroup.

\item \textbf{Compact resolvent:} the writeup realizes the form as a closed form and uses translation-control
and the Kolmogorov--Riesz compactness criterion to show the embedding of the form domain into $L^2(I)$ is compact,
hence the selfadjoint operator $A_\lambda$ has compact resolvent.

\item \textbf{Positivity improving:} invoking a standard theorem of the form
\[
\text{positive} + \text{irreducible} + \text{holomorphic semigroup}
\;\Longrightarrow\;
\text{positivity improving},
\]
the semigroup maps nonzero $f\ge 0$ to strictly positive functions for $t>0$.

\item \textbf{Perron--Frobenius/Krein--Rutman:} positivity improving together with compact resolvent implies
the principal (lowest) eigenvalue is simple and admits a strictly positive eigenfunction.
\end{itemize}

\noindent This establishes the \emph{simplicity} portion required in \S 6.6 (for the realized/truncated form).

\subsection{Step 3: Force evenness by reflection symmetry}
Finally, the writeup checks that the form is invariant under reflection in the logarithmic variable:
\[
(RG)(u) := G(-u).
\]
Form invariance implies the associated operator commutes with $R$.
Since the lowest eigenspace is one-dimensional by Step 2, it is invariant under $R$ and $R$ acts by a scalar $\pm 1$
on that eigenspace. Because the ground state is strictly positive a.e., it cannot be odd; hence it is even.

\section{Does this resolve the first part of \S 6.6?}
\subsection{Answer}
\textbf{Yes}, \emph{for the truncated/semi-local Weil form $QW_\lambda$ as realized in the writeup}
(from local explicit-formula distributions and test functions supported in $[\lambda^{-1},\lambda]$):
the writeup proves the smallest eigenvalue is \textbf{simple} and its eigenvector can be chosen \textbf{even}.

\subsection{What is still ``input'' rather than proved in-line}
The conclusion relies on the following external or assumed items:
\begin{enumerate}[leftmargin=2.2em]
\item \textbf{Identification of $QW_\lambda$ with the realized form:}
one must match Connes's definition of $QW_\lambda$ with the writeup's explicit-formula-based construction
(up to an additive constant shift). The constant shift does not affect eigenvectors/multiplicity,
but the equality of the underlying quadratic form must be checked at the level of definitions.

\item \textbf{Standard Dirichlet-form equivalences:}
the step from the indicator-energy lemma to semigroup irreducibility uses standard theorems
(not reproved in full detail).

\item \textbf{Positivity-improving and Perron--Frobenius consequences:}
the implication ``positive + irreducible + holomorphic $\Rightarrow$ positivity improving''
and the resulting simplicity/positivity of the ground state with compact resolvent
are standard inputs (e.g.\ Arendt--Batty--Hieber--Neubrander and Jentzsch/Krein--Rutman type results).
\end{enumerate}

\section{Important: \S 6.6 mentions an additional remaining step}
Section 6.6 also states that it \emph{still remains} to show an approximation property
(e.g.\ that a certain $k_\lambda$ approximates $\theta_x$ with $\lambda=\sqrt{x}$).
The writeup under discussion addresses the \emph{spectral simplicity/evenness} requirement,
but it does \emph{not} by itself resolve that separate approximation problem.

\section{Conclusion}
The writeup does discharge the first prerequisite singled out in \S 6.6:
it establishes that the bottom of the spectrum of the realized/truncated Weil quadratic form
is \emph{simple} and its eigenfunction can be taken \emph{even}.
The remaining work in \S 6.6 (as stated there) concerns approximation of auxiliary kernels/functions,
which is logically separate from the Perron--Frobenius/symmetry analysis.
\end{document}
