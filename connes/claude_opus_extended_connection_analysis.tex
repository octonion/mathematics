\documentclass[11pt]{article}
\usepackage[margin=1in]{geometry}
\usepackage{amsmath,amssymb,amsthm,mathtools}
\usepackage{hyperref}
\usepackage{enumitem}
\usepackage{array}
\usepackage{booktabs}

\newtheorem{theorem}{Theorem}
\newtheorem{proposition}[theorem]{Proposition}
\newtheorem{lemma}[theorem]{Lemma}
\newtheorem{corollary}[theorem]{Corollary}
\theoremstyle{definition}
\newtheorem{definition}[theorem]{Definition}
\theoremstyle{remark}
\newtheorem{remark}[theorem]{Remark}

\newcommand{\R}{\mathbb{R}}
\newcommand{\C}{\mathbb{C}}
\newcommand{\Z}{\mathbb{Z}}
\newcommand{\1}{\mathbf{1}}
\DeclareMathOperator{\Tr}{Tr}
\DeclareMathOperator{\supp}{supp}

\title{Resolution of the Simplicity and Evenness Condition\\
for the Ground State of the Restricted Weil Quadratic Form:\\
Connecting the Energy-Decomposition Paper to Connes' Programme}
\author{}
\date{}

\begin{document}
\maketitle

\begin{abstract}
We analyse how the results of the paper \emph{Energy-Decomposition and
Perron--Frobenius Consequences for the Restricted Weil Quadratic Form}
(hereafter \textbf{[ED]}) resolve the first of the two open conditions
identified by Alain Connes in \S6.6 of his 2026 survey \emph{The Riemann
Hypothesis: Past, Present and a Letter Through Time} (hereafter
\textbf{[C26]}).  Specifically, \textbf{[ED]} proves that the self-adjoint operator
$A_\lambda$ associated with the restricted Weil quadratic form has compact
resolvent and that its lowest eigenvalue is simple with a strictly positive,
even eigenfunction---exactly the hypothesis needed for Connes--van Suijlekom's
Theorem~6.1 in \textbf{[C26]} to guarantee that the Fourier transform of the
ground-state minimiser has all its zeros on the critical line.  We give a
detailed term-by-term comparison, discuss what remains unresolved, and place
the contribution within the broader strategy towards the Riemann Hypothesis
outlined in~\textbf{[C26]}.
\end{abstract}

\tableofcontents
\newpage

%% ===================================================================
\section{Context: Connes' strategy and the role of Theorem~6.1}
\label{sec:context}
%% ===================================================================

\subsection{The Weil positivity approach}

Connes' survey \textbf{[C26]} describes a strategy towards the Riemann Hypothesis (RH) rooted
in the Weil explicit formula.  For $\lambda > 1$, one restricts test functions to the compact
multiplicative interval $[\lambda^{-1}, \lambda] \subset \R_+^*$ and studies the quadratic form
\begin{equation}\label{eq:QW}
  QW_\lambda(g) := \sum_{v} W_v(g * g^*),
\end{equation}
where $W_v$ are the local distributions at each place $v$ of~$\mathbb{Q}$, evaluated on
$f = g * g^*$ (multiplicative convolution with involution).  By Weil's criterion, the positivity of
$QW_\lambda$ for all $\lambda > 1$ (subject to the vanishing condition
$\hat{g}(\pm i/2) = 0$) is equivalent to~RH.

\subsection{Theorem~6.1 of Connes--van Suijlekom}

A central tool in \textbf{[C26]} is the following theorem from joint work with
van Suijlekom~[32]:

\begin{theorem}[{Connes--van Suijlekom, [32]}]\label{thm:CvS}
Let $L > 0$, let $D$ be a real distribution on $[0, L]$, and let $\widetilde{D}$ be the
associated even distribution on $[-L, L]$.  Assume that the quadratic form with Schwartz
kernel $\widetilde{D}(x - y)$ defines a lower-bounded selfadjoint operator on
$L^2([-L/2, L/2])$, and that the minimum of its spectrum is a simple, isolated eigenvalue
with even eigenfunction~$\eta$.  Then all zeros of the entire function $\hat{\eta}(z)$,
$z \in \C$, lie on the real line.
\end{theorem}

This theorem provides a mechanism for constructing entire functions whose zeros are
provably on the critical line: if the ground-state eigenfunction of an appropriate
operator is simple and even, its Fourier transform has only real zeros.

\subsection{The two remaining steps (\S6.6 of [C26])}

In \S6.6 of \textbf{[C26]}, Connes identifies two conditions that remain to be
established in order to apply Theorem~\ref{thm:CvS} to the Weil quadratic form
$QW_\lambda$ and ultimately connect the approximating zeros to those of the Riemann
zeta function:

\begin{enumerate}[label=(\Roman*)]
  \item\label{item:simple-even}
  \textbf{Simplicity and evenness:}
  Show that the smallest eigenvalue of $QW_\lambda$ is simple and that the
  corresponding eigenfunction is even.

  \item\label{item:approx}
  \textbf{Approximation quality:}
  Show that the prolate-function ansatz $k_\lambda$ (constructed via the summation
  map $\mathcal{E}$ from prolate spheroidal wave functions) is a sufficiently good
  approximation of the actual minimal eigenvector~$\theta_x$ of~$QW_\lambda$.
\end{enumerate}

\noindent
Connes notes that ``the analogue of this property is known for the prolate wave
operator'' but that the result for $QW_\lambda$ itself had not been established.

\medskip
\noindent
\textbf{The central claim of this note} is that condition~\ref{item:simple-even}
is fully resolved by~\textbf{[ED]}.

%% ===================================================================
\section{What the energy-decomposition paper proves}
\label{sec:ED}
%% ===================================================================

We summarise the relevant results of \textbf{[ED]}, using its numbering.

\subsection{The quadratic form $\mathcal{E}_\lambda$ and its relation to $QW_\lambda$}

Working in logarithmic coordinates $u = \log x$ on the interval $I = (-L, L)$ with
$L = \log\lambda$, and writing $G(u) = g(e^u)$ with $\widetilde{G}$ denoting extension by
zero to~$\R$, paper~\textbf{[ED]} defines
\begin{equation}\label{eq:Elambda}
  \mathcal{E}_\lambda(G)
  := \int_0^{2L} w(t)\,\|\widetilde{G} - S_t\widetilde{G}\|_{L^2(\R)}^2\,dt
  + \sum_{\substack{p\text{ prime}\\ p^m \le \lambda^2}}
  (\log p)\,p^{-m/2}\,\|\widetilde{G} - S_{m\log p}\widetilde{G}\|_{L^2(\R)}^2,
\end{equation}
where $w(t) = e^{t/2}/(2\sinh t)$ and $S_t$ is translation by~$t$.
Lemmas~3 and~4 of \textbf{[ED]} show that
\begin{equation}\label{eq:relation}
  -\sum_v W_v(g * g^*) = \mathcal{E}_\lambda(G) + c(\lambda)\,\|G\|_2^2,
\end{equation}
where $c(\lambda) \in \R$ is a finite constant.  The additive shift $c(\lambda)\|G\|_2^2$ merely
translates the spectrum of the associated operator by a scalar, leaving simplicity,
positivity of eigenfunctions, and evenness invariant.

\subsection{Selfadjoint operator with compact resolvent}

\begin{theorem}[{Theorem~18 of \textbf{[ED]}}]\label{thm:op}
There exists a unique selfadjoint operator $A_\lambda \ge 0$ on $L^2(I)$ associated with
the closed form $\mathcal{E}_\lambda$.  Moreover, $A_\lambda$ has compact resolvent;
equivalently, $(A_\lambda + 1)^{-1}$ is compact on $L^2(I)$.
\end{theorem}

The proof proceeds in several self-contained steps:
\begin{enumerate}[nosep]
  \item \emph{Fourier representation.}  The ambient form $\mathcal{E}_\lambda^{\R}$ on
  $L^2(\R)$ is identified via Plancherel as multiplication by a symbol
  $\psi_\lambda(\xi) \ge 0$ in Fourier space.
  \item \emph{Closedness.}  The form domain, equipped with the graph norm, is isometric to
  a weighted~$L^2$ space and hence complete.  Restriction to the closed subspace
  $H_I = \{\phi \in L^2(\R) : \phi = 0 \text{ a.e.\ on } \R \setminus I\}$ preserves
  closedness.
  \item \emph{Logarithmic coercivity.}  A lower bound $w(t) \ge c_0/t$ for small~$t$ yields
  $\psi_\lambda(\xi) \ge c_1\log|\xi| - c_2$ for large~$|\xi|$, via an interval-counting
  argument.
  \item \emph{Compact embedding.}  The logarithmic growth of $\psi_\lambda$ provides a
  log-frequency moment bound, which combined with the Kolmogorov--Riesz compactness
  criterion (tightness is automatic from bounded support) gives compactness of the
  embedding $\mathcal{D}(\mathcal{E}_\lambda) \hookrightarrow L^2(I)$.
  \item \emph{Compact resolvent.}  Standard resolvent estimates reduce to compact embedding.
\end{enumerate}

Compact resolvent implies the spectrum of $A_\lambda$ is discrete, consisting of isolated
eigenvalues of finite multiplicity accumulating only at~$+\infty$.

\subsection{Simplicity and strict positivity of the ground state}

\begin{proposition}[{Proposition~20 of \textbf{[ED]}}]\label{prop:gs}
The semigroup $T(t) = e^{-tA_\lambda}$ is positivity improving, and the lowest
eigenvalue of $A_\lambda$ is simple with a strictly positive a.e.\ eigenfunction~$\psi$.
\end{proposition}

This is established by verifying three properties:

\begin{enumerate}[nosep]
  \item \textbf{Positivity preservation (Markov property).}
  For every normal contraction~$\Phi$ (i.e.\ $\Phi(0) = 0$,
  $|\Phi(a) - \Phi(b)| \le |a - b|$), the difference-energy structure of
  $\mathcal{E}_\lambda$ gives
  $\mathcal{E}_\lambda(\Phi \circ G) \le \mathcal{E}_\lambda(G)$ pointwise under the
  integral.  This is the Beurling--Deny criterion for Markovian semigroups.

  \item \textbf{Irreducibility.}
  The archimedean \emph{continuum} of shifts---the integral over $t \in (0, 2L)$ with
  strictly positive weight $w(t) > 0$---is the decisive ingredient.  If
  $\mathcal{E}_\lambda(\1_B) = 0$ for a measurable $B \subset I$, then the non-negativity
  of all weights forces
  $\|\widetilde{\1_B} - S_t\widetilde{\1_B}\|_2 = 0$ for a.e.\ $t \in (0, 2L)$.
  Continuity of $t \mapsto \|\phi - S_t\phi\|_2^2$ upgrades this to \emph{all}
  $t \in (0, 2L)$, and a mollifier argument (Lemma~7 of \textbf{[ED]}) forces $B$ to be
  null or conull.  Via the Beurling--Deny/Fukushima equivalence from Dirichlet form theory,
  this yields irreducibility of~$T(t)$.

  \item \textbf{Holomorphy.}
  Since $A_\lambda$ is selfadjoint and lower bounded, $e^{-zA_\lambda}$ is bounded and
  holomorphic on $\{\Re z > 0\}$ by the spectral theorem.
\end{enumerate}

These three properties together invoke the theorem of Arendt et al.:\ positivity
$+$ irreducibility $+$ holomorphy $\Rightarrow$ positivity improving.  The
Krein--Rutman / Perron--Frobenius theorem for compact positive operators then gives
simplicity and strict positivity of the ground state.

\subsection{Evenness of the ground state}

\begin{corollary}[{Corollary~22 of \textbf{[ED]}}]\label{cor:even}
The ground-state eigenfunction $\psi$ satisfies $\psi(-u) = \psi(u)$ a.e.
\end{corollary}

\begin{proof}[Proof sketch]
The reflection operator $R\colon G(u) \mapsto G(-u)$ is unitary on $L^2(I)$ (since
$I = (-L, L)$ is symmetric) and satisfies $RS_t = S_{-t}R$.  Using
$\|\phi - S_{-t}\phi\| = \|\phi - S_t\phi\|$ and the symmetry of all weights in
$\mathcal{E}_\lambda$, one obtains
$\mathcal{E}_\lambda(RG) = \mathcal{E}_\lambda(G)$, hence $A_\lambda R = RA_\lambda$.

Since $R\psi$ is then a strictly positive eigenfunction for the same (simple) eigenvalue,
$R\psi = c\psi$ with $c > 0$.  Unitarity of~$R$ forces $c = 1$, so $\psi$ is even.
\end{proof}

%% ===================================================================
\section{Term-by-term comparison with Connes' requirements}
\label{sec:comparison}
%% ===================================================================

The following table makes the correspondence explicit.

\medskip
\begin{center}
\renewcommand{\arraystretch}{1.4}
\begin{tabular}{>{\raggedright\arraybackslash}p{6.2cm}
                >{\raggedright\arraybackslash}p{6.2cm}}
\toprule
\textbf{Requirement from \S6.6 of [C26]} &
\textbf{Result in [ED]} \\
\midrule
$QW_\lambda$ defines a lower-bounded selfadjoint operator &
Theorem~18: $A_\lambda \ge 0$ is selfadjoint, associated
to the closed form $\mathcal{E}_\lambda$ via Kato's
representation theorem \\
\addlinespace
Spectrum is discrete with isolated eigenvalues &
Theorem~18: $A_\lambda$ has compact resolvent, hence
discrete spectrum accumulating at $+\infty$ \\
\addlinespace
Smallest eigenvalue is simple &
Proposition~20: Markov property $+$ irreducibility $+$
holomorphy $\Rightarrow$ positivity improving; then
Krein--Rutman gives simplicity \\
\addlinespace
Ground-state eigenfunction is even &
Corollary~22: $A_\lambda$ commutes with $R\colon G(u) \mapsto G(-u)$;
simplicity $+$ strict positivity forces $R\psi = \psi$ \\
\bottomrule
\end{tabular}
\end{center}

\medskip

The relationship between $\mathcal{E}_\lambda$ in \textbf{[ED]} and $QW_\lambda$ in
\textbf{[C26]} is given by~\eqref{eq:relation}: they differ by an additive constant times
$\|G\|_2^2$.  Such a shift translates every eigenvalue by the same scalar $c(\lambda)$
without altering eigenvectors, simplicity, or eigenfunction symmetry.  Paper \textbf{[ED]}
notes this explicitly.

%% ===================================================================
\section{The role of the archimedean continuum}
\label{sec:arch}
%% ===================================================================

It is worth emphasising the structural reason why irreducibility holds for $A_\lambda$ but
would \emph{not} hold for a purely ``prime'' version of the quadratic form.

The prime contributions to $\mathcal{E}_\lambda$ involve differences $S_{m\log p}$ at
\emph{discrete} shift values $\{m\log p : p^m \le \lambda^2\}$.  A measurable set
$B \subset I$ could, in principle, be invariant under all these discrete shifts while having
$0 < m(B) < m(I)$.

The archimedean term, by contrast, involves a \emph{continuum} of shifts---an integral over
all $t \in (0, 2L)$ with the strictly positive weight $w(t) > 0$.  If
$\mathcal{E}_\lambda(\1_B) = 0$, then $\1_B$ must be invariant under translation by
\emph{every} $t$ in a full interval $(0, \varepsilon)$.  The mollifier argument of
Lemma~7 in \textbf{[ED]} then forces $\1_B$ to be a.e.\ constant---the set is null or
conull.

This mirrors an important theme in Connes' programme: the archimedean place carries
qualitatively different (and essential) information compared to the non-archimedean
places.  In the language of \textbf{[C26]}, the archimedean trace formula
(\S7.1) and the role of the Sonin space (\S7.2) are manifestations of the same
phenomenon.

%% ===================================================================
\section{What remains unresolved}
\label{sec:open}
%% ===================================================================

Paper~\textbf{[ED]} resolves condition~\ref{item:simple-even} of \S\ref{sec:context}
completely.  It does \textbf{not} address condition~\ref{item:approx}: proving that the
prolate-function ansatz $k_\lambda$ (equation~(17) of~\textbf{[C26]}) is a sufficiently
good approximation of the true ground-state eigenvector $\theta_x$ of~$QW_\lambda$.

The full chain of Connes' strategy requires the following additional steps, none of which
are provided by~\textbf{[ED]}:

\begin{enumerate}[label=(\alph*),nosep]
  \item Showing that the prolate-based approximation $k_\lambda = \mathcal{E}(h_\lambda)$
  (where $h_\lambda$ is the appropriate linear combination of $h_{0,\lambda}$ and
  $h_{4,\lambda}$ with vanishing integral) converges to the true minimiser $\theta_x$ in a
  suitable norm.

  \item Establishing that the Fourier transforms of the true minimisers $\theta_x$ converge,
  as $\lambda \to \infty$, to Riemann's $\Xi$-function uniformly on compact subsets of the
  open strip $|\!\operatorname{Im}(z)| < 1/2$.

  \item Applying Hurwitz's theorem to conclude that the zeros of the limit function (which
  are the nontrivial zeros of~$\zeta$) must lie on the real line, since the approximating
  functions have all their zeros there by Theorem~\ref{thm:CvS}.
\end{enumerate}

Step (a) amounts to comparing two objects: the exact minimiser of $QW_\lambda$ (whose
existence, simplicity, and evenness are now established by~\textbf{[ED]}) and the
``near-radical'' function $k_\lambda$ constructed from prolate wave functions.
As Connes discusses in \S6.4 of~\textbf{[C26]}, the conceptual justification is that the
range of the summation map $\mathcal{E}$ lies in the radical of the \emph{global} Weil
form, so elements of this range restricted to $[\lambda^{-1}, \lambda]$ should be close
to the radical of the \emph{restricted} form $QW_\lambda$.  The exponential smallness of
the eigenvalue $\varepsilon(\lambda)$ (and its similarity to $1 - \chi_2(\lambda)$, as
shown in Figure~1 of~\textbf{[C26]}) provides strong numerical evidence for this, but a
rigorous proof has not yet been given.

Step (b) is partially addressed by Fact~6.4 of \textbf{[C26]}, which establishes the
convergence of the Fourier transforms of the prolate-based approximations $\hat{k}_\lambda$
towards the $\Xi$-function.  The gap is in transferring this convergence from the ansatz
$k_\lambda$ to the true minimiser~$\theta_x$.

%% ===================================================================
\section{Summary}
\label{sec:summary}
%% ===================================================================

The results of \textbf{[ED]} supply the missing operator-theoretic foundation for
condition~\ref{item:simple-even} in Connes' strategy.  The logical chain can be depicted
as follows:

\bigskip
\begin{center}
\fbox{\parbox{0.9\textwidth}{\centering
\textbf{Energy decomposition} (Lemmas 3--4 of [ED])\\[3pt]
$\Downarrow$\\[3pt]
\textbf{Closed form $\mathcal{E}_\lambda$, compact resolvent}
(Props.\ 11--12, 17, Thm.\ 18 of [ED])\\[3pt]
$\Downarrow$\\[3pt]
\textbf{Markov $+$ Irreducibility $+$ Holomorphy}
(Lemma 6, Cor.\ 19, spectral thm.)\\[3pt]
$\Downarrow$\\[3pt]
\textbf{Positivity improving} (Arendt et al.)
$\;\Rightarrow\;$ \textbf{Simple ground state, $\psi > 0$}
(Krein--Rutman)\\[3pt]
$\Downarrow$\\[3pt]
\textbf{Reflection symmetry} (Prop.\ 21 of [ED])
$\;\Rightarrow\;$ \textbf{$\psi$ is even} (Cor.\ 22 of [ED])\\[3pt]
$\Downarrow$\\[3pt]
\textbf{Theorem 6.1 of [C26] applies:}\\
\emph{All zeros of $\hat{\psi}(z)$ lie on the real line}
}}
\end{center}

\bigskip

The remaining challenge---condition~\ref{item:approx}---is to show that the
prolate-based approximation $k_\lambda$ adequately represents the true ground state, and
that the convergence $\hat{k}_\lambda \to \Xi$ can be transferred to the actual minimisers.
This is the content of the semilocal trace formula approach described in~\S7.4
of~\textbf{[C26]} and the subject of ongoing work by Connes, Consani, and Moscovici.

\bigskip
\begin{center}
\renewcommand{\arraystretch}{1.3}
\begin{tabular}{>{\raggedright\arraybackslash}p{5.5cm}c
                >{\raggedright\arraybackslash}p{5.5cm}}
\toprule
\textbf{Step in Connes' programme} & \textbf{Status} & \textbf{Source} \\
\midrule
$QW_\lambda$ has simple, even ground state &
\checkmark &
[ED]: Prop.\ 20, Cor.\ 22 \\
\addlinespace
Zeros of $\hat\psi$ lie on the real line &
\checkmark &
Follows from above $+$ Thm.\ 6.1 of [C26] \\
\addlinespace
$k_\lambda \approx \theta_x$ (prolate approximation) &
Open &
Numerical evidence in [C26] \S6.4;
rigorous proof pending \\
\addlinespace
$\hat{k}_\lambda \to \Xi$ uniformly on compact sets &
\checkmark &
Fact 6.4 of [C26] (for the ansatz) \\
\addlinespace
Transfer convergence from ansatz to true minimiser &
Open &
Requires comparing $k_\lambda$ and $\theta_x$ \\
\addlinespace
Hurwitz's theorem $\Rightarrow$ zeros of $\Xi$ on the line &
Conditional &
Follows if all preceding steps are completed \\
\bottomrule
\end{tabular}
\end{center}

%% ===================================================================
\section*{References}
%% ===================================================================

\begin{description}[labelwidth=2.5em, leftmargin=3em, labelsep=0.5em, font=\normalfont]

\item[\textbf{[C26]}]
A.~Connes,
\emph{The Riemann Hypothesis: Past, Present and a Letter Through Time},
arXiv:2602.04022v1, February 2026.

\item[\textbf{[ED]}]
\emph{Energy-Decomposition and Perron--Frobenius Consequences for the Restricted
Weil Quadratic Form} (the paper analysed in this note).

\item[\textbf{[32]}]
A.~Connes and W.~van Suijlekom,
\emph{Quadratic Forms, Real Zeros and Echoes of the Spectral Action},
Commun.\ Math.\ Phys.\ (2025) 406:312.

\item[\textbf{[24]}]
A.~Connes and C.~Consani,
\emph{Weil positivity and trace formula, the archimedean place},
Selecta Math.\ (N.S.) \textbf{27} (2021), no.~4, Paper no.~77.

\item[\textbf{[25]}]
A.~Connes and C.~Consani,
\emph{Spectral triples and $\zeta$-cycles},
Enseign.\ Math.\ \textbf{69} (2023), no.~1--2, 93--148.

\end{description}

\end{document}
