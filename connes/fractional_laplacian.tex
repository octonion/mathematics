\documentclass[11pt]{article}
\usepackage[margin=1in]{geometry}
\usepackage{amsmath,amssymb,amsthm}
\usepackage{mathrsfs}
\usepackage{hyperref}
\usepackage{enumitem}
\usepackage{booktabs}
\usepackage{xcolor}

% --- Tags to mark standard vs. expository/new ingredients
\newcommand{\tagstd}{\textcolor{gray}{\scriptsize\bfseries[standard]}}
\newcommand{\tagexp}{\textcolor{gray}{\scriptsize\bfseries[expository]}}


\newtheorem{theorem}{Theorem}[section]
\newtheorem{lemma}[theorem]{Lemma}
\newtheorem{proposition}[theorem]{Proposition}
\newtheorem{corollary}[theorem]{Corollary}
\theoremstyle{definition}
\newtheorem{definition}[theorem]{Definition}
\newtheorem{example}[theorem]{Example}
\theoremstyle{remark}
\newtheorem{remark}[theorem]{Remark}

\newcommand{\R}{\mathbb{R}}
\newcommand{\C}{\mathbb{C}}
\newcommand{\N}{\mathbb{N}}
\newcommand{\cE}{\mathcal{E}}
\newcommand{\cD}{\mathcal{D}}
\newcommand{\cF}{\mathcal{F}}
\newcommand{\cH}{\mathcal{H}}

\title{Ground State Simplicity via Energy Decomposition:\\
A Dirichlet Form Proof for the Fractional Laplacian\\[6pt]
\large With a View Toward the Weil Quadratic Form}
\author{}
\date{February 2026}

\begin{document}
\maketitle

\begin{abstract}
We give a self-contained proof that the fractional Laplacian
$(-\Delta)^s$ ($0 < s < 1$) on a bounded domain $\Omega \subset \R^d$
with Dirichlet boundary conditions has a simple lowest eigenvalue
whose eigenfunction is strictly positive. Our proof avoids both the
maximum principle for non-local operators and probabilistic arguments
involving stable L\'evy processes. Instead, it proceeds by a purely
analytic route: we decompose the associated quadratic form as a
continuous superposition of translation-difference energies, derive
the Markov property and irreducibility from this decomposition, prove
compact resolvent via Kolmogorov--Riesz, and conclude ground state
simplicity from the Krein--Rutman theorem for positivity-improving
semigroups. We develop this argument in parallel with the analogous
(and much less familiar) application to the Weil quadratic form from
analytic number theory, where the same structural pattern---energy
decomposition, Markov property, irreducibility, compact resolvent,
Perron--Frobenius---resolves an open problem in the Connes program
toward the Riemann Hypothesis.
\end{abstract}

\tableofcontents
\newpage

%% ===================================================================
\section{Introduction}
\label{sec:intro}
%% ===================================================================

\subsection{The main result}

Let $\Omega \subset \R^d$ be a bounded open domain and let
$0 < s < 1$. The \emph{fractional Laplacian with Dirichlet boundary
conditions}, denoted $(-\Delta)^s_\Omega$, is the non-negative
self-adjoint operator on $L^2(\Omega)$ associated with the quadratic
form
\begin{equation}
\label{eq:frac_form}
\cE(u,u) = \frac{C_{d,s}}{2} \iint_{\R^d \times \R^d}
\frac{|\tilde u(x) - \tilde u(y)|^2}{|x-y|^{d+2s}} \, dx \, dy,
\end{equation}
where $\tilde u$ denotes the extension of $u$ by zero outside
$\Omega$, and the normalizing constant is
\[
C_{d,s} = \frac{2^{2s} s \, \Gamma(d/2+s)}{\pi^{d/2} \Gamma(1-s)}.
\]
The form domain is the fractional Sobolev space
$\cD(\cE) = H^s_0(\Omega) = \{ u \in H^s(\R^d) : u = 0 \text{ a.e.\
on } \R^d \setminus \Omega \}$.

The following result is well known (see, e.g.,
\cite{ServadeiValdinoci2014,RosOtonSerra2014}):

\begin{theorem}[Ground state simplicity \tagstd]
\label{thm:main}
The operator $(-\Delta)^s_\Omega$ has compact resolvent and hence
purely discrete spectrum $0 < \lambda_1 < \lambda_2 \le \lambda_3 \le
\cdots \to \infty$. The first eigenvalue $\lambda_1$ is simple, and
the corresponding eigenfunction $\varphi_1$ can be chosen to satisfy
$\varphi_1(x) > 0$ for a.e.\ $x \in \Omega$.

If, moreover, $\Omega$ is symmetric under a reflection
$R: x \mapsto \bar x$ (i.e., $R\Omega = \Omega$), then $\varphi_1$
is symmetric: $\varphi_1(\bar x) = \varphi_1(x)$ for a.e.\ $x$.
\end{theorem}

Standard proofs of Theorem~\ref{thm:main} rely on either the
\emph{maximum principle} for non-local operators (extending the
classical argument for $-\Delta$ via Hopf's lemma) or on
\emph{probabilistic} arguments showing that the symmetric
$2s$-stable L\'evy process killed on exiting $\Omega$ has an
irreducible transition semigroup. Both approaches, while natural,
obscure the purely analytic mechanism at work.

In this paper we give a proof that uses neither. Instead, we follow a
five-step pipeline:
\begin{center}
\fbox{\parbox{0.85\textwidth}{\centering
Energy Decomposition $\longrightarrow$ Markov Property
$\longrightarrow$ Irreducibility \\[3pt]
$\longrightarrow$ Compact Resolvent $\longrightarrow$
Perron--Frobenius / Krein--Rutman}}
\end{center}
Each step is elementary, and the chain as a whole constitutes a
proof of ground state simplicity that generalizes to a much broader
class of operators---including, as we explain in
Section~\ref{sec:parallel}, the operator associated with the
\emph{Weil quadratic form} in analytic number theory.

\subsection{Motivation: the Weil quadratic form}
\label{sec:motivation}

The impetus for writing this paper comes from a recent development in
the Connes program toward the Riemann Hypothesis. The \emph{Weil
quadratic form} $QW_\lambda$, obtained by restricting Weil's explicit
formula to test functions supported on an interval
$[\lambda^{-1},\lambda]$, defines a self-adjoint operator $A_\lambda$
on an appropriate $L^2$ space. Three papers by Connes and
collaborators published in late 2025 and early 2026
\cite{ConnesvanSuijlekom2025,ConnesConsaniMoscovici2025,Connes2026}
establish that \emph{if} the lowest eigenvalue of $A_\lambda$ is
simple with an even eigenfunction, \emph{then} far-reaching
consequences follow---including the fact that all zeros of certain
approximating functions lie on the critical line.

These papers explicitly identify the verification of simplicity and
evenness as ``the key difficulty'' and list it among ``the missing
steps.'' In concurrent work, this verification is carried out by an
\emph{energy-decomposition method} that rewrites the Weil quadratic
form as a continuous superposition of translation-difference energies,
recognizes this as a Dirichlet form, proves irreducibility via the
archimedean continuum of shifts, establishes compact resolvent via
logarithmic coercivity, and applies Krein--Rutman.

The purpose of the present paper is to make this method accessible by
demonstrating it in a familiar setting---the fractional
Laplacian---where every step can be verified independently and the
result is already known by other means.

\subsection{Notation and conventions}

Throughout, $\Omega \subset \R^d$ is a bounded open set. We write
$L^2 = L^2(\Omega)$ with norm $\|\cdot\|_2$. For $h \in \R^d$, the
translation operator is $(\tau_h f)(x) = f(x+h)$. For a measurable
function $u$ on $\Omega$, we write $\tilde u$ for the extension of
$u$ by zero to $\R^d$. The symmetric difference of sets is
$A \triangle B = (A \setminus B) \cup (B \setminus A)$. Constants
$C, c > 0$ may change from line to line.

%% ===================================================================
\section{Preliminaries}
\label{sec:prelim}
%% ===================================================================

We collect the background results used in the proof. Each is standard
and can be found in the references cited.

\subsection{Symmetric Dirichlet forms}

\begin{definition}[Symmetric Dirichlet form]
A \emph{symmetric Dirichlet form} on $L^2(X,\mu)$ is a closed,
densely defined, non-negative symmetric bilinear form
$(\cE, \cD(\cE))$ satisfying the \emph{Markov property}: for every
$u \in \cD(\cE)$, the function $\hat u = (0 \vee u) \wedge 1$
belongs to $\cD(\cE)$ and $\cE(\hat u, \hat u) \le \cE(u,u)$.
\end{definition}

More generally, the Markov property can be stated for
\emph{normal contractions}: a function $\Phi: \R \to \R$ with
$\Phi(0) = 0$ and $|\Phi(t) - \Phi(s)| \le |t-s|$ for all
$t, s \in \R$. For a Dirichlet form, $\Phi \circ u \in \cD(\cE)$
and $\cE(\Phi \circ u, \Phi \circ u) \le \cE(u,u)$ for every normal
contraction $\Phi$ \cite{FukushimaOshimaTakeda2011}.

The fundamental link between Dirichlet forms and operator theory is
that every symmetric Dirichlet form $(\cE, \cD(\cE))$ on
$L^2(X,\mu)$ is associated, via the Kato representation theorem, with
a unique non-negative self-adjoint operator $A$ satisfying
$\cE(u,v) = \langle A^{1/2}u, A^{1/2}v \rangle$ for
$u, v \in \cD(\cE) = \cD(A^{1/2})$. The operator $-A$ generates a
strongly continuous contraction semigroup $(T_t)_{t \ge 0}$ on
$L^2(X,\mu)$. When $\cE$ satisfies the Markov property, the
semigroup is \emph{sub-Markovian}: $0 \le f \le 1$ implies
$0 \le T_t f \le 1$ \cite{FukushimaOshimaTakeda2011}.


\subsection{Irreducibility}

We use the standard ``invariant set'' formulation of irreducibility,
which is the natural one for the Perron--Frobenius/Krein--Rutman step.

\begin{definition}[Invariant sets and irreducibility \tagstd]
\label{def:irreducible}
Let $(\cE,\cD(\cE))$ be a densely defined, closed, non-negative
symmetric form on $L^2(\Omega)$. A measurable set $M\subset\Omega$ is
called \emph{(form-)invariant} if for every $u\in \cD(\cE)$ one has
$\mathbf 1_M u\in \cD(\cE)$ and
\begin{equation}
\label{eq:inv_decomp}
\cE(u,u)=\cE(\mathbf 1_M u,\mathbf 1_M u)+
\cE(\mathbf 1_{\Omega\setminus M} u,\mathbf 1_{\Omega\setminus M} u).
\end{equation}
The form is called \emph{irreducible} if the only invariant sets $M$
satisfy $|M|=0$ or $|\Omega\setminus M|=0$.
\end{definition}

\begin{remark}[\tagstd]
If $(\cE,\cD(\cE))$ is a symmetric Dirichlet form, then
\eqref{eq:inv_decomp} is equivalent to invariance for the associated
sub-Markovian semigroup $(T_t)_{t\ge 0}$: $M$ is invariant iff
$T_t(\mathbf 1_M f)=\mathbf 1_M T_t f$ for all $f\in L^2(\Omega)$ and
all $t>0$; see, for instance, \cite[Ch.~1]{FukushimaOshimaTakeda2011}
or \cite[Sec.~1]{LenzStollmannVeselic2009}.
\end{remark}
\subsection{Positivity-improving semigroups}

\begin{definition}
A bounded operator $T$ on $L^2(\Omega)$ is
\emph{positivity-preserving} if $f \ge 0$ implies $Tf \ge 0$, and
\emph{positivity-improving} if $f \ge 0$, $f \not\equiv 0$ implies
$Tf > 0$ a.e.
\end{definition}

The connection between irreducibility and positivity improvement is:

\begin{theorem}[Irreducibility and positivity improvement \tagstd\ {\cite[Thm.~1.4]{LenzStollmannVeselic2009}}]
\label{prop:pos_improving}
Let $(\cE,\cD(\cE))$ be a symmetric Dirichlet form on $L^2(\Omega)$
with associated non-negative self-adjoint operator $A$ and semigroup
$(T_t)_{t\ge 0}$. Then the following are equivalent:
\begin{enumerate}[label=\textup{(\roman*)},nosep]
\item $(\cE,\cD(\cE))$ is irreducible in the sense of
Definition~\ref{def:irreducible}.
\item For every $t>0$, the operator $T_t$ is positivity-improving.
\item For every $\alpha>-\inf\sigma(A)$, the resolvent
$(A+\alpha)^{-1}$ is positivity-improving.
\end{enumerate}
\end{theorem}

\subsection{The Krein--Rutman theorem}

The following is the infinite-dimensional analogue of the
Perron--Frobenius theorem for non-negative matrices:

\begin{theorem}[Krein--Rutman for positivity-improving compact operators \tagstd\ {\cite{KreinRutman1948,Schaefer1974}}]
\label{thm:KR}
Let $T$ be a compact, positivity-improving operator on $L^2(\Omega)$
with spectral radius $r(T) > 0$. Then:
\begin{enumerate}[nosep,label=(\roman*)]
\item $r(T)$ is an eigenvalue of $T$ with algebraic multiplicity one.
\item The corresponding eigenfunction $\varphi$ can be chosen to
  satisfy $\varphi(x) > 0$ for a.e.\ $x \in \Omega$.
\item No other eigenvalue of $T$ has a non-negative eigenfunction.
\end{enumerate}
\end{theorem}

\begin{corollary}
\label{cor:ground_state}
Let $A$ be a non-negative self-adjoint operator on $L^2(\Omega)$ with
compact resolvent, and suppose $e^{-tA}$ is positivity-improving for
all $t > 0$. Then the lowest eigenvalue $\lambda_1$ of $A$ is simple,
and the corresponding eigenfunction is strictly positive a.e.
\end{corollary}

\begin{proof}
The operator $T = e^{-A}$ is compact (since $A$ has compact
resolvent), self-adjoint, and positivity-improving (by hypothesis).
Its spectral radius is $r(T) = e^{-\lambda_1} > 0$, and
Theorem~\ref{thm:KR} gives that $e^{-\lambda_1}$ is a simple
eigenvalue of $T$ with a strictly positive eigenfunction. Since
the eigenspaces of $T$ and $A$ coincide, $\lambda_1$ is a simple
eigenvalue of $A$ with the same eigenfunction.
\end{proof}

\subsection{The Kolmogorov--Riesz compactness theorem}

\begin{theorem}[Kolmogorov--Riesz \tagstd\ {\cite{Hanche-OlsenHolden2010}}]
\label{thm:KR_compact}
A bounded subset $\cF \subset L^2(\R^d)$ is precompact if and only
if:
\begin{enumerate}[nosep,label=(\roman*)]
\item \textup{(Equi-continuity in mean)} $\sup_{f \in \cF}
  \|\tau_h f - f\|_2 \to 0$ as $|h| \to 0$.
\item \textup{(Tightness)} For every $\varepsilon > 0$, there exists
  $R > 0$ such that $\sup_{f \in \cF} \int_{|x|>R} |f(x)|^2 \, dx
  < \varepsilon$.
\end{enumerate}
\end{theorem}

%% ===================================================================
\section{The proof}
\label{sec:proof}
%% ===================================================================

We now prove Theorem~\ref{thm:main} in five steps.

\subsection{Step 1: Energy decomposition}
\label{sec:step1}

The starting point is the observation that the quadratic form
\eqref{eq:frac_form} can be written as a continuous superposition of
\emph{translation-difference energies}.

\begin{proposition}[Energy decomposition \tagstd]
\label{prop:energy_decomp}
For $u \in H^s_0(\Omega)$, with $\tilde u$ the zero extension to
$\R^d$:
\begin{equation}
\label{eq:energy_decomp}
\cE(u,u) = \frac{C_{d,s}}{2} \int_{\R^d}
\|\tau_h \tilde u - \tilde u\|_{L^2(\R^d)}^2 \;
|h|^{-(d+2s)} \, dh.
\end{equation}
\end{proposition}

\begin{proof}
Substitute $y = x + h$ in \eqref{eq:frac_form}:
\begin{align*}
\cE(u,u)
&= \frac{C_{d,s}}{2} \iint_{\R^d \times \R^d}
  \frac{|\tilde u(x) - \tilde u(y)|^2}{|x-y|^{d+2s}} \, dx \, dy \\
&= \frac{C_{d,s}}{2} \int_{\R^d} \left( \int_{\R^d}
  |\tilde u(x+h) - \tilde u(x)|^2 \, dx \right) |h|^{-(d+2s)} \, dh \\
&= \frac{C_{d,s}}{2} \int_{\R^d}
  \|\tau_h \tilde u - \tilde u\|_{L^2(\R^d)}^2 \;
  |h|^{-(d+2s)} \, dh.
\end{align*}
The exchange of integration order is justified by Tonelli's theorem,
since the integrand is non-negative.
\end{proof}

Formula \eqref{eq:energy_decomp} exhibits $\cE(u,u)$ as an integral
over all translations $h \in \R^d$ of the squared $L^2$ distance
between $\tilde u$ and its translate $\tau_h \tilde u$, weighted by
the \emph{strictly positive} density
$w(h) = \frac{C_{d,s}}{2} |h|^{-(d+2s)}$. Every feature of the
proof flows from this decomposition and the strict positivity of $w$.

\begin{remark}[Structural parallel with the Weil form]
\label{rem:weil_decomp}
In the setting of the Weil quadratic form, the analogous decomposition
takes the form
\[
\cE_\lambda(G,G) = \sum_{p \text{ prime}}
  \cE_p(G,G) + \cE_\R(G,G),
\]
where each summand $\cE_p$ is a difference energy associated with the
discrete scaling $x \mapsto x + \log p$, and $\cE_\R$ is an integral
of translation-difference energies over a continuum of shifts with a
non-negative weight determined by the archimedean distribution $W_\R$.
The fractional Laplacian's decomposition
\eqref{eq:energy_decomp} is a pure-continuum analogue of this mixed
discrete-plus-continuum structure.
\end{remark}

\subsection{Step 2: The Markov property}
\label{sec:step2}

\begin{proposition}[Markov property \tagstd]
\label{prop:markov}
Let $\Phi: \R \to \R$ be a normal contraction (i.e.,
$\Phi(0) = 0$ and $|\Phi(t) - \Phi(s)| \le |t-s|$ for all
$t,s$). If $u \in H^s_0(\Omega)$, then
$\Phi \circ u \in H^s_0(\Omega)$ and
$\cE(\Phi \circ u, \Phi \circ u) \le \cE(u,u)$.
\end{proposition}

\begin{proof}
Since $\Phi(0) = 0$ and $u$ vanishes outside $\Omega$, we have
$\Phi \circ u = 0$ on $\R^d \setminus \Omega$, so
$\widetilde{\Phi \circ u} = \Phi \circ \tilde u$.

For any $h \in \R^d$ and a.e.\ $x \in \R^d$:
\[
|(\Phi \circ \tilde u)(x+h) - (\Phi \circ \tilde u)(x)|
= |\Phi(\tilde u(x+h)) - \Phi(\tilde u(x))|
\le |\tilde u(x+h) - \tilde u(x)|,
\]
by the contraction property. Squaring and integrating over $x$:
\[
\|\tau_h(\Phi \circ \tilde u) - \Phi \circ \tilde u\|_{L^2}^2
\le \|\tau_h \tilde u - \tilde u\|_{L^2}^2.
\]
Integrating over $h$ against the non-negative weight
$w(h) = \frac{C_{d,s}}{2} |h|^{-(d+2s)}$:
\[
\cE(\Phi \circ u, \Phi \circ u) \le \cE(u,u).
\]
Since $|\Phi(t)| \le |t|$ (from $\Phi(0)=0$ and the Lipschitz
condition), $\|\Phi \circ u\|_{L^2} \le \|u\|_{L^2}$, and the
bound on $\cE$ shows
$\Phi \circ u \in H^s_0(\Omega)$.
\end{proof}

The key observation is that the Markov property is inherited
\emph{pointwise in $h$}: each translation-difference energy
$\|\tau_h \tilde u - \tilde u\|^2$ is individually contracted by
$\Phi$, and integration with non-negative weights preserves the
inequality. This is the structural reason why the energy
decomposition implies the Markov property.


\subsection{Step 3: Irreducibility}
\label{sec:step3}

\begin{proposition}[Irreducibility \tagstd]
\label{prop:irred}
The Dirichlet form $(\cE, H^s_0(\Omega))$ on $L^2(\Omega)$ is
irreducible in the sense of Definition~\ref{def:irreducible}.
\end{proposition}

\begin{proof}
Let $M\subset\Omega$ be an invariant set. We show that either $|M|=0$
or $|\Omega\setminus M|=0$.

Assume for contradiction that $0<|M|<|\Omega|$. Choose points
$x\in M$ and $y\in \Omega\setminus M$ that are Lebesgue density points
of their respective sets. Pick radii $r_x,r_y>0$ so small that the
closed balls $\overline{B(x,r_x)}$ and $\overline{B(y,r_y)}$ are
contained in $\Omega$ and are disjoint. In particular,
\[
|M\cap B(x,r_x)|>0,\qquad |(\Omega\setminus M)\cap B(y,r_y)|>0.
\]

Let $\eta\in C_c^\infty(\Omega)$ satisfy $\eta\ge 0$ and
$\eta\ge 1$ on $B(x,r_x)\cup B(y,r_y)$. Set
\[
u := \mathbf 1_M\,\eta,\qquad v := \mathbf 1_{\Omega\setminus M}\,\eta.
\]
By invariance, $u,v\in H^s_0(\Omega)$ and the decomposition
\eqref{eq:inv_decomp} applied to $\eta$ gives
$\cE(\eta,\eta)=\cE(u,u)+\cE(v,v)$. Expanding
$\eta=u+v$ and using bilinearity yields $\cE(u,v)=0$.

On the other hand, using the bilinear form associated with
\eqref{eq:frac_form}, for a.e.\ $(x',y')\in M\times(\Omega\setminus M)$
we have $u(x')=\eta(x')$, $u(y')=0$, $v(x')=0$, $v(y')=\eta(y')$, hence
\[
(u(x')-u(y'))(v(x')-v(y')) = -\eta(x')\,\eta(y').
\]
By symmetry of the integrand, this implies
\begin{equation}
\label{eq:cross_energy_negative}
\cE(u,v)
= - C_{d,s}\iint_{M\times(\Omega\setminus M)}
\frac{\eta(x')\,\eta(y')}{|x'-y'|^{d+2s}}\,dx'\,dy' \le 0.
\end{equation}
Moreover, restricting the integral in \eqref{eq:cross_energy_negative}
to the subset
$(M\cap B(x,r_x))\times((\Omega\setminus M)\cap B(y,r_y))$ and using
$\eta\ge 1$ there, we obtain
\[
\cE(u,v)
\le -C_{d,s}\,
\frac{|M\cap B(x,r_x)|\,|(\Omega\setminus M)\cap B(y,r_y)|}
{\mathrm{dist}(B(x,r_x),B(y,r_y))^{d+2s}}
<0,
\]
a contradiction to $\cE(u,v)=0$. Therefore no such invariant $M$
exists, and the form is irreducible.
\end{proof}

\begin{remark}[\tagexp]
This is the only place where ``mixing'' enters the argument.
Analytically, mixing is encoded by the strictly positive jump kernel
$|x-y|^{-d-2s}$: any two sets of positive measure interact through the
cross term in the energy. Conceptually, the energy decomposition
\eqref{eq:energy_decomp} makes this interaction transparent by writing
$\cE$ as a superposition of translation-difference energies at
\emph{all} shifts $h\in\R^d$.
\end{remark}
\subsection{Step 4: Compact resolvent}
\label{sec:step4}

\begin{proposition}[Compact resolvent \tagstd]
\label{prop:compact}
The operator $(-\Delta)^s_\Omega$ has compact resolvent. Equivalently,
the inclusion $H^s_0(\Omega) \hookrightarrow L^2(\Omega)$ is compact.
\end{proposition}

\begin{proof}
We verify the hypotheses of the Kolmogorov--Riesz theorem
(Theorem~\ref{thm:KR_compact}) for the unit ball
$\cF = \{ \tilde u : u \in H^s_0(\Omega), \; \cE(u,u) + \|u\|_2^2
\le 1 \}$ as a subset of $L^2(\R^d)$.

\medskip
\noindent\textbf{Tightness.} Every $\tilde u \in \cF$ is supported
in $\overline\Omega$, which is bounded. Hence for $R$ large enough
that $\overline\Omega \subset B_R(0)$, we have
$\int_{|x|>R} |\tilde u(x)|^2 \, dx = 0$. Tightness is immediate.

\medskip
\noindent\textbf{Equi-continuity in mean.} For any $h \in \R^d$ with
$|h| \le 1$, we estimate $\|\tau_h \tilde u - \tilde u\|_{L^2}^2$.
By the energy decomposition, for any $\delta > 0$:
\begin{align*}
\cE(u,u)
&= \frac{C_{d,s}}{2} \int_{\R^d}
  \|\tau_k \tilde u - \tilde u\|_{L^2}^2 \; |k|^{-(d+2s)} \, dk \\
&\ge \frac{C_{d,s}}{2} \int_{|k-h| < \delta}
  \|\tau_k \tilde u - \tilde u\|_{L^2}^2 \; |k|^{-(d+2s)} \, dk.
\end{align*}

We use a softer estimate. The Fourier symbol of $(-\Delta)^s$ is
$|\xi|^{2s}$, and the Gagliardo energy \eqref{eq:frac_form} is
(up to a constant depending on $d,s$) equivalent to
$\int_{\R^d} |\xi|^{2s}|\hat{\tilde u}(\xi)|^2\,d\xi$; see, e.g.,
\cite[Sec.~3]{DiNezzaPalatucciValdinoci2012}. By Plancherel's theorem:
\begin{align*}
\|\tau_h \tilde u - \tilde u\|_{L^2}^2
&= \int_{\R^d} |e^{2\pi i h \cdot \xi} - 1|^2
  |\hat{\tilde u}(\xi)|^2 \, d\xi \\
&= \int_{\R^d} |e^{2\pi i h \cdot \xi} - 1|^2
  \cdot |\xi|^{-2s} \cdot |\xi|^{2s}
  |\hat{\tilde u}(\xi)|^2 \, d\xi.
\end{align*}
Now $|e^{2\pi i h \cdot \xi} - 1|^2 \le \min(4, \; 4\pi^2
|h|^2 |\xi|^2)$. For a parameter $M > 0$ to be chosen:
\begin{align*}
\|\tau_h \tilde u - \tilde u\|_{L^2}^2
&\le 4\pi^2 |h|^2 \int_{|\xi| \le M} |\xi|^{2-2s}
  |\hat{\tilde u}(\xi)|^2 \cdot |\xi|^{2s} \, d\xi \;
  \\& \quad + \;
  4 \int_{|\xi| > M} |\xi|^{-2s}
  |\xi|^{2s} |\hat{\tilde u}(\xi)|^2 \, d\xi \\
&\le 4\pi^2 |h|^2 M^{2-2s} \int_{\R^d} |\xi|^{2s}
  |\hat{\tilde u}(\xi)|^2 \, d\xi
  + 4 M^{-2s} \int_{\R^d} |\xi|^{2s}
  |\hat{\tilde u}(\xi)|^2 \, d\xi.
\end{align*}
Recognizing that $\int |\xi|^{2s} |\hat{\tilde u}|^2 \, d\xi$ is
proportional to $\cE(u,u)$ (up to a factor depending on $C_{d,s}$
and $2\pi$), and that $\cE(u,u) \le 1$ for $u \in \cF$, we get
\[
\|\tau_h \tilde u - \tilde u\|_{L^2}^2
\le C \bigl( |h|^2 M^{2-2s} + M^{-2s} \bigr)
\]
for a constant $C$ depending on $d, s$. Choosing $M = |h|^{-1}$:
\[
\|\tau_h \tilde u - \tilde u\|_{L^2}^2
\le C \bigl( |h|^{2s} + |h|^{2s} \bigr)
= 2C |h|^{2s} \to 0 \quad \text{as } |h| \to 0,
\]
uniformly over $\cF$. This gives equi-continuity in mean.

\medskip
By the Kolmogorov--Riesz theorem, $\cF$ is precompact in
$L^2(\R^d)$, hence also in $L^2(\Omega)$. It follows that the
inclusion $H^s_0(\Omega) \hookrightarrow L^2(\Omega)$ is compact, and
therefore $(-\Delta)^s_\Omega$ has compact resolvent.
\end{proof}

\begin{remark}[Comparison with the Weil form]
\label{rem:compact_weil}
In the Weil form setting, the Fourier symbol $\psi_\lambda(\xi)$ of
the operator $A_\lambda$ grows like $\log|\xi|$ as $|\xi| \to \infty$
(logarithmic coercivity), rather than $|\xi|^{2s}$ (polynomial
coercivity). The slower growth means compact embedding is harder to
prove and requires the full force of the Kolmogorov--Riesz criterion
(polynomial growth would give compact embedding by Rellich--Kondrachov
directly). In both cases, however, the argument has the same shape:
\emph{growth of the Fourier symbol} $\Rightarrow$
\emph{equi-continuity in mean} $\Rightarrow$
\emph{compact inclusion} $\Rightarrow$ \emph{compact resolvent}.
\end{remark}

\subsection{Step 5: Perron--Frobenius and ground state simplicity}
\label{sec:step5}

We now assemble the pieces.

\begin{proof}[Proof of Theorem~\ref{thm:main}]
\textbf{Steps 1--2} (Propositions~\ref{prop:energy_decomp} and
\ref{prop:markov}): The quadratic form $\cE$ is a symmetric Dirichlet
form on $L^2(\Omega)$ with domain $H^s_0(\Omega)$.

\textbf{Step 3} (Proposition~\ref{prop:irred}): The Dirichlet form
$\cE$ is irreducible.

\textbf{Step 4} (Proposition~\ref{prop:compact}): The associated
operator $A = (-\Delta)^s_\Omega$ has compact resolvent.

By Proposition~\ref{prop:pos_improving}, irreducibility implies the
semigroup $T_t = e^{-tA}$ is positivity-improving for all $t > 0$.
Combined with compact resolvent, Corollary~\ref{cor:ground_state}
gives: the lowest eigenvalue $\lambda_1$ of $A$ is \textbf{simple},
with a corresponding eigenfunction $\varphi_1 > 0$ a.e.

Since $A$ has compact resolvent, the spectrum is purely discrete and
the eigenvalues accumulate at $+\infty$: $0 < \lambda_1 < \lambda_2
\le \lambda_3 \le \cdots \to \infty$. The strict inequality
$\lambda_1 > 0$ follows because $\cE(u,u) = 0$ with $u \in
H^s_0(\Omega)$ implies $u = 0$ (Step~3), so $A$ has trivial kernel.

\medskip
\textbf{Symmetry.} Suppose $\Omega$ is symmetric under a reflection
$R: x \mapsto \bar x$. The substitution $x \to \bar x$, $y \to \bar y$
in the double integral \eqref{eq:frac_form} shows that
$\cE(u \circ R, u \circ R) = \cE(u,u)$. It follows that $A$ commutes
with the reflection operator $(Rf)(x) = f(\bar x)$:
\[
A(f \circ R) = (Af) \circ R.
\]
Since $\lambda_1$ is simple, the eigenspace is one-dimensional,
spanned by $\varphi_1$. Both $\varphi_1$ and $\varphi_1 \circ R$ are
eigenfunctions for $\lambda_1$, so $\varphi_1 \circ R =
c \, \varphi_1$ for some constant $c$. Applying $R$ again:
$\varphi_1 = c^2 \varphi_1$, so $c = \pm 1$. Since $\varphi_1 > 0$
and $\varphi_1 \circ R > 0$, the case $c = -1$ is impossible.
Therefore $\varphi_1 \circ R = \varphi_1$: the ground state is
symmetric under $R$.
\end{proof}

%% ===================================================================
\section{The parallel with the Weil quadratic form}
\label{sec:parallel}
%% ===================================================================

We now make explicit the structural analogy between the fractional
Laplacian and the Weil quadratic form. The purpose is to show that
the five-step pipeline is not specific to either setting but is a
general strategy applicable to any operator whose quadratic form
admits an energy decomposition with strictly positive weights.

\subsection{The setup}

For a finite set $S$ of places including the archimedean place $\R$,
and a cutoff parameter $\lambda > 1$, the Weil quadratic form
restricted to test functions supported on $[\lambda^{-1},\lambda]$
(in multiplicative coordinates) defines a quadratic form
$\cE_\lambda$ on $L^2([-L,L])$ where $L = \log\lambda$. The Weil
explicit formula gives
\[
\cE_\lambda(G,G) = \sum_{p \in S \setminus \{\R\}}
\cE_p(G,G) \; + \; \cE_\R(G,G),
\]
where:

\begin{itemize}[nosep]
\item Each \emph{prime contribution} $\cE_p(G,G)$ is a
  discrete difference energy. Writing $\ell_p = \log p$, the
  contribution from each power $p^k$ involves terms
  $\|G - S_{\pm k\ell_p} G\|^2$ where $S_t$ is the shift operator
  $(S_t G)(u) = G(u-t)$ restricted to $[-L,L]$.

\item The \emph{archimedean contribution} $\cE_\R(G,G)$ involves
  a continuous integral of translation-difference energies:
  \[
  \cE_\R(G,G) = \int_0^{2L} \|S_t \tilde G - \tilde G\|_{L^2}^2 \;
  w_\R(t) \, dt,
  \]
  where $w_\R(t) \ge 0$ is determined by the archimedean local
  distribution $W_\R$.
\end{itemize}

\subsection{Step-by-step comparison}

\begin{center}
\renewcommand{\arraystretch}{1.4}
\begin{tabular}{@{}p{2.2cm}p{5.5cm}p{5.5cm}@{}}
\toprule
\textbf{Step} & \textbf{Fractional Laplacian} & \textbf{Weil form} \\
\midrule
Energy decomposition &
$\displaystyle\cE = \int_{\R^d} \|\tau_h \tilde u - \tilde u\|^2 \,
w(h) \, dh$
\newline
$w(h) = \frac{C_{d,s}}{2}|h|^{-(d+2s)}$ &
$\displaystyle\cE_\lambda = \sum_p \cE_p + \int_0^{2L} \|\cdot\|^2 \,
w_\R \, dt$
\newline
Discrete (primes) $+$ continuum (archim.) \\
\midrule
Markov property &
$|\Phi(a) - \Phi(b)| \le |a-b|$
\newline
applied pointwise in $h$ &
Same pointwise argument applied to each $\cE_p$ and
to $\cE_\R$ \\
\midrule
Irreducibility &
$w(h) > 0$ for all $h \ne 0$;
\newline
$\tilde u$ vanishes outside $\Omega$ &
$w_\R(t) > 0$ for $t \in (0,2L)$;
\newline
$\tilde G$ vanishes outside $[-L,L]$ \\
\midrule
Compact resolvent &
$|\xi|^{2s} \to \infty$ (polynomial)
\newline
$\Rightarrow$ Rellich--Kondrachov &
$\psi_\lambda(\xi) \gtrsim \log|\xi|$ (logarithmic)
\newline
$\Rightarrow$ Kolmogorov--Riesz \\
\midrule
Perron--\newline Frobenius &
Krein--Rutman $\Rightarrow$ $\lambda_1$ simple,
$\varphi_1 > 0$ &
Krein--Rutman $\Rightarrow$ $\lambda_1$ simple,
ground state $> 0$ \\
\midrule
Symmetry &
$\Omega$ symmetric $\Rightarrow$
$\varphi_1$ symmetric &
$[-L,L]$ symmetric $\Rightarrow$ ground state even \\
\bottomrule
\end{tabular}
\end{center}

\subsection{Where the proofs differ}

Despite the structural parallelism, there are genuine differences:

\begin{enumerate}[nosep,label=(\arabic*)]
\item \textbf{Coercivity.} The fractional Laplacian has polynomial
  growth $|\xi|^{2s}$ of its Fourier symbol, which gives compact
  embedding by the standard fractional Rellich--Kondrachov theorem.
  The Weil form has only logarithmic growth, making compact embedding
  more delicate and requiring explicit use of the Kolmogorov--Riesz
  criterion. (In our proof of Proposition~\ref{prop:compact}, we
  deliberately used Kolmogorov--Riesz to parallel the arithmetic
  setting, even though a simpler proof is available.)

\item \textbf{Structure of shifts.} The fractional Laplacian
  integrates over \emph{all} translations in $\R^d$, making
  irreducibility trivial. The Weil form has a discrete part (primes)
  and a continuous part (archimedean), and irreducibility depends
  on the archimedean contribution---without it, one would have
  only countably many shift directions, which is not enough.

\item \textbf{Boundary conditions.} The fractional Laplacian uses
  zero extension to $\R^d$ (Dirichlet conditions), which kills the
  constant function. The Weil form uses restriction to $[-L,L]$ with
  similar effect: functions in the domain vanish outside $[-L,L]$, so
  the only function with zero energy is zero.

\item \textbf{Context and significance.} Ground state simplicity
  for the fractional Laplacian is a well-known result with multiple
  proofs. For the Weil form, it resolves an open problem explicitly
  identified by Connes, Consani, Moscovici, and van Suijlekom as ``the
  key difficulty'' in their program.
\end{enumerate}

%% ===================================================================
\section{An abstract framework}
\label{sec:abstract}
%% ===================================================================

The argument naturally generalizes to the following class of operators.

\begin{theorem}[Abstract ground state simplicity \tagexp]
\label{thm:abstract}
Let $X \subset \R^d$ be a bounded measurable set with $|X| > 0$, and
let $w: \R^d \to [0,\infty)$ be a measurable weight satisfying:
\begin{enumerate}[nosep,label=\textup{(H\arabic*)}]
\item \label{H1} \textup{(Local integrability)}
  $\int_{|h|<1} |h|^2 \, w(h) \, dh < \infty$.
\item \label{H2} \textup{(Coercivity)} The Fourier symbol
  $\psi(\xi) = \int_{\R^d} |e^{2\pi i h \cdot \xi} - 1|^2 w(h) \, dh$
  satisfies $\psi(\xi) \to \infty$ as $|\xi| \to \infty$.
\item \label{H3} \textup{(Irreducibility)} For every $h_0 \in \R^d
  \setminus \{0\}$ and every $\varepsilon > 0$,
  $\int_{|h - h_0| < \varepsilon} w(h) \, dh > 0$.
\end{enumerate}
Define
\[
\cE_w(u,u) = \int_{\R^d} \|\tau_h \tilde u - \tilde u\|_{L^2}^2
\, w(h) \, dh
\]
on $\cD(\cE_w) = \{ u \in L^2(X) : \cE_w(u,u) < \infty \}$, where
$\tilde u$ is the zero extension to $\R^d$. Then $(\cE_w, \cD(\cE_w))$
is a symmetric Dirichlet form on $L^2(X)$ whose associated operator
has compact resolvent, simple lowest eigenvalue, and a strictly
positive eigenfunction.
\end{theorem}

\begin{proof}[Proof sketch]
The Markov property follows from the pointwise contraction argument
(Step~2). Hypothesis \ref{H3} guarantees irreducibility by the same
argument as Step~3: if $\cE_w(u,u) = 0$, then
$\|\tau_h \tilde u - \tilde u\| = 0$ for $h$ in the support of $w$,
and \ref{H3} ensures this support is dense in $\R^d$, forcing
$\tilde u$ to be constant and hence zero. Hypothesis \ref{H2} gives
equi-continuity in mean: $\|\tau_h \tilde u - \tilde u\|^2 \le
\int |\hat{\tilde u}|^2 |e^{2\pi i h \cdot \xi}-1|^2 \, d\xi$,
and the growth of $\psi(\xi)$ controls this. Tightness follows from
the bounded support of $X$. Kolmogorov--Riesz gives compact inclusion,
and Krein--Rutman gives simplicity.
\end{proof}

The fractional Laplacian corresponds to $w(h) = \frac{C_{d,s}}{2}
|h|^{-(d+2s)}$ (satisfying all three hypotheses for $0<s<1$). The
Weil form corresponds to a weight that is a superposition of discrete
masses at prime-logarithm multiples and a continuous archimedean
density, with the Fourier symbol growing logarithmically.

\begin{remark}[Scope and limitations \tagexp]
Hypothesis~\ref{H3} is a convenient \emph{sufficient} condition for the
irreducibility mechanism used in the proof of
Theorem~\ref{thm:abstract}: it forces the translation-difference energy
to see \emph{all} small shifts. When $w$ is supported on a discrete
subgroup (e.g.\ a lattice), Hypothesis~\ref{H3} fails and reducibility
can indeed occur.

A concrete example is $d=1$, $w=\delta_{2}+\delta_{-2}$ and
$X=(0,1)\cup(3,4)$. Then the energy $\cE_w(u,u)$ only couples values of
$\tilde u$ across shifts of size $2$, and the subspaces
$L^2((0,1))$ and $L^2((3,4))$ are invariant: the form is not
irreducible and the lowest eigenvalue of the associated operator has
multiplicity $2$ (one copy on each component). In contrast, for the
fractional Laplacian the continuum of shifts encoded by
\eqref{eq:energy_decomp} prevents such decoupling.
\end{remark}


%% ===================================================================
\section{Discussion}
\label{sec:discussion}
%% ===================================================================

\subsection{Comparison with other proofs}

The classical proof of ground state simplicity for the standard
Laplacian $-\Delta$ on a bounded domain (the Courant--Hilbert
argument) uses the variational characterization of $\lambda_1$
together with the observation that $|u|$ has the same Dirichlet
energy as $u$, so minimizers can be taken non-negative, and then
the strong maximum principle forces them to be strictly positive.
Simplicity follows because two orthogonal positive functions cannot
exist.

For the fractional Laplacian, this argument breaks down at the
maximum principle step: the non-local maximum principle for
$(-\Delta)^s$ is substantially harder to prove and requires
regularity theory \cite{RosOtonSerra2014, CaffarelliSilvestre2007}.

The probabilistic proof (via the symmetric stable process) replaces
the maximum principle with the observation that the L\'evy jump kernel
$|x-y|^{-(d+2s)}$ is strictly positive, so the process can jump from
any region to any other, making the semigroup positivity-improving.
This is elegant but requires probabilistic machinery.

Our Dirichlet form proof occupies a middle ground: it uses neither the
maximum principle nor probability theory, and it makes the mechanism
transparent. The ground state is simple because (a) the energy is a
superposition of non-negative pieces (Markov property), (b) the
pieces involve a rich enough family of shifts (irreducibility), and
(c) the Fourier symbol grows (compact resolvent). Each condition is
easy to verify, and together they are sufficient.

\subsection{Further directions}

The abstract framework of Theorem~\ref{thm:abstract} suggests several
natural extensions:

\begin{enumerate}[nosep]
\item \textbf{Other L-functions.} The Weil explicit formula for
  Dirichlet $L$-functions $L(s,\chi)$ with real characters has local
  distributions $W_p^\chi$ that differ from the Riemann zeta case only
  by character twists. The energy-decomposition method should extend
  to verify the Connes--van Suijlekom hypotheses for these
  $L$-functions, giving simplicity and evenness for an infinite family
  of restricted operators.

\item \textbf{Non-local operators with arithmetic kernels.} Integral
  operators whose kernels are defined by Euler products or Epstein
  zeta functions may admit energy decompositions of the form
  considered here. The abstract theorem provides a systematic criterion
  for when such operators have simple ground states.

\item \textbf{Operators on groups.} The translation-difference
  structure generalizes naturally to locally compact abelian groups.
  The Weil form on the id\`ele class group and the fractional Laplacian
  on $\R^d$ are both instances of convolution-type operators on groups,
  and the energy-decomposition method should extend to this level of
  generality.
\end{enumerate}

%% ===================================================================
\section*{Acknowledgments}
%% ===================================================================

This paper is intended as an expository companion to the
energy-decomposition approach to the Weil quadratic form. Its purpose
is to demonstrate the method in a setting where the result is already
known, thereby making the strategy accessible to readers from PDE and
spectral theory who may not be familiar with the number-theoretic
context.

\begin{thebibliography}{99}

\bibitem{CaffarelliSilvestre2007}
L.~Caffarelli, L.~Silvestre, An extension problem related to the
fractional Laplacian, \emph{Comm.\ Partial Differential Equations}
\textbf{32} (2007), 1245--1260.

\bibitem{DiNezzaPalatucciValdinoci2012}
E.~Di~Nezza, G.~Palatucci, E.~Valdinoci, Hitchhiker's guide to the
fractional Sobolev spaces, \emph{Bull.\ Sci.\ Math.}\ \textbf{136}
(2012), 521--573.

\bibitem{Connes2026}
A.~Connes, The Riemann Hypothesis: Past, Present and a Letter Through
Time, arXiv:2602.04022 (2026).

\bibitem{ConnesConsani2021}
A.~Connes, C.~Consani, Weil positivity and Trace formula, the
archimedean place, \emph{Selecta Math.}\ \textbf{27}, 77 (2021).

\bibitem{ConnesConsani2023}
A.~Connes, C.~Consani, Spectral Triples and Zeta-Cycles,
\emph{Enseign.\ Math.}\ \textbf{69} (2023), 93--148.

\bibitem{ConnesConsaniMoscovici2025}
A.~Connes, C.~Consani, H.~Moscovici, Zeta Spectral Triples,
arXiv:2511.22755 (2025).

\bibitem{ConnesvanSuijlekom2025}
A.~Connes, W.\,D.~van Suijlekom, Quadratic Forms, Real Zeros and
Echoes of the Spectral Action, arXiv:2511.23257 (2025).

\bibitem{FukushimaOshimaTakeda2011}
M.~Fukushima, Y.~\=Oshima, M.~Takeda, \emph{Dirichlet Forms and
Symmetric Markov Processes}, 2nd ed., de~Gruyter, 2011.

\bibitem{LenzStollmannVeselic2009}
D.~Lenz, P.~Stollmann, I.~Veseli\'c, The Allegretto--Piepenbrink theorem
for strongly local Dirichlet forms, \emph{Documenta Math.}\ \textbf{14}
(2009), 167--189.

\bibitem{Hanche-OlsenHolden2010}
H.~Hanche-Olsen, H.~Holden, The Kolmogorov--Riesz compactness
theorem, \emph{Expo.\ Math.}\ \textbf{28} (2010), 385--394.

\bibitem{KreinRutman1948}
M.\,G.~Kre\u\i n, M.\,A.~Rutman, Linear operators leaving invariant
a cone in a Banach space, \emph{Uspehi Mat.\ Nauk} \textbf{3}
(1948), no.~1(23), 3--95; English transl.\ in \emph{AMS Translations}
Ser.~1, Vol.~10 (1962), 199--325.

\bibitem{RosOtonSerra2014}
X.~Ros-Oton, J.~Serra, The Dirichlet problem for the fractional
Laplacian: regularity up to the boundary, \emph{J.\ Math.\ Pures
Appl.}\ \textbf{101} (2014), 275--302.

\bibitem{Schaefer1974}
H.\,H.~Schaefer, \emph{Banach Lattices and Positive Operators},
Springer, 1974.

\bibitem{ServadeiValdinoci2014}
R.~Servadei, E.~Valdinoci, Variational methods for non-local
operators of elliptic type, \emph{Discrete Contin.\ Dyn.\ Syst.}\
\textbf{33} (2013), 2105--2137.

\end{thebibliography}

\end{document}
