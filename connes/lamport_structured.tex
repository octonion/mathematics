\documentclass[11pt]{article}
\usepackage[margin=1in]{geometry}
\usepackage{amsmath,amssymb,amsthm,mathtools}
\usepackage{hyperref}
\usepackage{enumitem}

\newtheorem{theorem}{Theorem}
\newtheorem{proposition}[theorem]{Proposition}
\newtheorem{lemma}[theorem]{Lemma}
\newtheorem{corollary}[theorem]{Corollary}
\theoremstyle{remark}
\newtheorem{remark}[theorem]{Remark}
\newtheorem{definition}[theorem]{Definition}
\newcommand{\R}{\mathbb R}
\newcommand{\C}{\mathbb C}
\newcommand{\1}{\mathbf 1}
\DeclareMathOperator{\supp}{supp}

%%% ----------------------------------------------------------------
%%% Lamport structured-proof environment
%%% ----------------------------------------------------------------
\newcounter{proofstep}
\newcounter{proofsubstep}[proofstep]
\newcounter{proofsubsubstep}[proofsubstep]

\newcommand{\Step}[1]{%
  \refstepcounter{proofstep}%
  \medskip\noindent\textbf{Step~\theproofstep.}\quad #1%
}
\newcommand{\Substep}[1]{%
  \refstepcounter{proofsubstep}%
  \smallskip\noindent\hspace{1em}\textbf{Step~\theproofstep.\theproofsubstep.}\quad #1%
}
\newcommand{\Subsubstep}[1]{%
  \refstepcounter{proofsubsubstep}%
  \smallskip\noindent\hspace{2em}\textbf{Step~\theproofstep.\theproofsubstep.\theproofsubsubstep.}\quad #1%
}
\newcommand{\Stepjust}[1]{%
  \par\noindent\hspace{2em}\emph{Justification:}\ #1%
}
\newcommand{\Substepjust}[1]{%
  \par\noindent\hspace{3em}\emph{Justification:}\ #1%
}
\newcommand{\Subsubstepjust}[1]{%
  \par\noindent\hspace{4em}\emph{Justification:}\ #1%
}
\newcommand{\QEDstep}{%
  \medskip\noindent\textbf{Q.E.D.}%
}

\newenvironment{structuredproof}{%
  \begin{proof}[Structured Proof]%
  \setcounter{proofstep}{0}%
}{%
  \end{proof}%
}

%%% ----------------------------------------------------------------

\title{Energy Decomposition, Compact Resolvent, and Perron--Frobenius Properties\\
of the Restricted Weil Quadratic Form}
\author{}
\date{}

\begin{document}
\maketitle

\begin{abstract}
We record a completely concrete and rigorous functional-analytic step that arises
in the spectral approach to Weil's criterion when one restricts test functions to
a compact multiplicative interval $[\lambda^{-1},\lambda]\subset \R_+^*$.
Starting from the explicit local distributions at the primes and at $\infty$, we
derive an ``energy decomposition'' expressing the quadratic form (up to an additive
constant multiple of $\|g\|_2^2$) as a positive combination of translation-difference
energies $\|G-S_tG\|_2^2$ in logarithmic coordinates.
We then prove the Markov (normal contraction) property and a translation-invariance
lemma which yields irreducibility from the archimedean continuum of shifts.
Finally, we show that the quadratic form is closed and that its associated
selfadjoint operator has compact resolvent, using a logarithmic lower bound on the
Fourier symbol together with the Kolmogorov--Riesz compactness criterion.
From this we deduce that the ground-state eigenvalue is simple and its
eigenfunction can be chosen strictly positive and, by inversion symmetry, even.

\medskip\noindent
\emph{Note on proof style.}  Every proof in this document is presented in
L.~Lamport's hierarchical structured-proof format~\cite{LamportHowToWrite}.
Each step states a claim and its justification, and sub-steps may be expanded
for further detail.  The intent is that any single step can be verified
independently.
\end{abstract}

\tableofcontents

%% =================================================================
\section{Setup on $\R_+^*$}
%% =================================================================

Let $\R_+^*=(0,\infty)$ with multiplicative Haar measure
\[
d^*x:=\frac{dx}{x}.
\]
For measurable $g,h$ define multiplicative convolution
\[
(g*h)(x):=\int_{\R_+^*} g(y)\,h(x/y)\,d^*y,
\]
and involution
\[
g^*(x):=\overline{g(x^{-1})}.
\]
If $g\in L^2(\R_+^*,d^*x)$, define the dilation operator
\begin{equation}
\label{eq:Ua}
(U_ag)(x):=g(x/a)\qquad(a>0).
\end{equation}
Then $U_a$ is unitary on $L^2(\R_+^*,d^*x)$: it is isometric
($\|U_ag\|_2=\|g\|_2$ by Haar invariance $d^*(x/a)=d^*x$) and surjective
($U_a^{-1}=U_{a^{-1}}$).  In particular, $\langle g,U_ag\rangle$ is
well-defined and finite by Cauchy--Schwarz:
$|\langle g,U_ag\rangle|\le\|g\|_2\|U_ag\|_2=\|g\|_2^2$.

%% -----------------------------------------------------------------
\begin{lemma}[Convolution inner-product identity]
\label{lem:f-inner}
Let $f=g*g^*$. Then for all $a>0$,
\[
f(a)=\langle g,U_ag\rangle_{L^2(d^*x)}=\int_{\R_+^*} g(x)\,\overline{g(x/a)}\,d^*x,
\qquad
f(a^{-1})=\overline{f(a)}.
\]
In particular $f(a)+f(a^{-1})=2\Re\langle g,U_ag\rangle$ and $f(1)=\|g\|_2^2$.
\end{lemma}

\begin{structuredproof}
\Step{$f(a)=\langle g,U_ag\rangle$.}

\Substep{Expand the definition of multiplicative convolution applied to $g*g^*$ at the point~$a$.}
\Substepjust{By definition, $(g*g^*)(a)=\int_{\R_+^*} g(y)\,g^*(a/y)\,d^*y$.}

\Substep{Apply the definition of involution: $g^*(a/y)=\overline{g((a/y)^{-1})}=\overline{g(y/a)}$.}
\Substepjust{The involution is defined as $g^*(x)=\overline{g(x^{-1})}$.  Set $x=a/y$ to get $g^*(a/y)=\overline{g(y/a)}$.}

\Substep{Substitute into the integral: $(g*g^*)(a)=\int_{\R_+^*}g(y)\,\overline{g(y/a)}\,d^*y$.}
\Substepjust{Combine Steps~1.1 and 1.2.}

\Substep{Recognize this as $\langle g,U_ag\rangle_{L^2(d^*x)}$.}
\Substepjust{By definition of the dilation operator~\eqref{eq:Ua}, $(U_ag)(y)=g(y/a)$, and the $L^2(d^*x)$
inner product is $\langle g,h\rangle=\int g(y)\overline{h(y)}\,d^*y$.  Thus
$\int g(y)\,\overline{g(y/a)}\,d^*y=\langle g,U_ag\rangle$.}

\Step{$f(a^{-1})=\overline{f(a)}$.}
\Stepjust{Replace $a$ by $a^{-1}$ in the result of Step~1:
$f(a^{-1})=\langle g,U_{a^{-1}}g\rangle
=\int g(y)\,\overline{g(ya)}\,d^*y$.
Substituting $y'=ya$ (so $y=y'/a$ and $d^*y=d^*y'$ by Haar invariance) gives
$\int g(y/a)\,\overline{g(y)}\,d^*y = \overline{\int g(y)\,\overline{g(y/a)}\,d^*y}
=\overline{f(a)}$.}

\Step{$f(a)+f(a^{-1})=2\Re\langle g,U_ag\rangle$ and $f(1)=\|g\|_2^2$.}
\Stepjust{From Steps~1 and 2, $f(a)+f(a^{-1})=\langle g,U_ag\rangle+\overline{\langle g,U_ag\rangle}=2\Re\langle g,U_ag\rangle$.
Setting $a=1$: $U_1=\mathrm{Id}$, so $f(1)=\langle g,g\rangle=\|g\|_2^2$.}

\QEDstep
\end{structuredproof}

%% -----------------------------------------------------------------
\begin{lemma}[A basic unitary identity]
\label{lem:unitary}
For any unitary $U$ on a Hilbert space and any vector $h$,
\[
2\Re\langle h,Uh\rangle = 2\|h\|^2-\|h-Uh\|^2.
\]
\end{lemma}

\begin{structuredproof}
\Step{$\|h-Uh\|^2=\|h\|^2+\|Uh\|^2-2\Re\langle h,Uh\rangle$.}
\Stepjust{Expand the inner product:
$\|h-Uh\|^2=\langle h-Uh,h-Uh\rangle
=\langle h,h\rangle-\langle h,Uh\rangle-\langle Uh,h\rangle+\langle Uh,Uh\rangle
=\|h\|^2+\|Uh\|^2-2\Re\langle h,Uh\rangle$.}

\Step{$\|Uh\|^2=\|h\|^2$.}
\Stepjust{$U$ is unitary, hence isometric.}

\Step{Substituting Step~2 into Step~1: $\|h-Uh\|^2=2\|h\|^2-2\Re\langle h,Uh\rangle$.}
\Stepjust{Replace $\|Uh\|^2$ by $\|h\|^2$ in Step~1.}

\Step{Rearrange: $2\Re\langle h,Uh\rangle=2\|h\|^2-\|h-Uh\|^2$.}
\Stepjust{Solve Step~3 for $2\Re\langle h,Uh\rangle$.}

\QEDstep
\end{structuredproof}

%% =================================================================
\section{Local explicit-formula terms}
%% =================================================================

Fix $\lambda>1$ and consider $g\in C_c^\infty([\lambda^{-1},\lambda])$.
(This regularity ensures that $f=g*g^*$ is smooth and compactly supported in
$[\lambda^{-2},\lambda^2]$, so all integrals below converge absolutely.
The quadratic form $\mathcal E_\lambda$ defined in Section~\ref{sec:globalform}
makes sense for arbitrary $G\in L^2(I)$ as an extended-real-valued form,
and the subsequent functional-analytic results depend only on that abstract definition.)

We record the two local distributions we use; these are the only ``input formulas''.

\subsection{Prime terms}
For a prime $p$ define
\begin{equation}
\label{eq:Wp}
W_p(f):=(\log p)\sum_{m\ge 1} p^{-m/2}\bigl(f(p^m)+f(p^{-m})\bigr).
\end{equation}

\subsection{Archimedean term}
Define
\begin{equation}
\label{eq:WR}
W_{\R}(f):=(\log 4\pi+\gamma)\,f(1)+\int_1^\infty
\Bigl(f(x)+f(x^{-1})-2x^{-1/2}f(1)\Bigr)\frac{x^{1/2}}{x-x^{-1}}\,d^*x,
\end{equation}
where $\gamma$ is the Euler--Mascheroni constant.

\begin{remark}[Restriction to a compact multiplicative interval]
\label{rem:truncate}
If $\operatorname{supp}(g)\subset[\lambda^{-1},\lambda]$, then for $a>\lambda^2$
the supports of $g$ and $U_ag$ are disjoint, hence $\langle g,U_ag\rangle=0$ and
$f(a)=0$. Consequently:
\begin{itemize}
\item in \eqref{eq:Wp} only those $(p,m)$ with $p^m\le \lambda^2$ contribute;
\item in \eqref{eq:WR}, after the change of variables $x=e^t$, only $t\in[0,2\log\lambda]$
contributes to the term involving $f(e^t)+f(e^{-t})$.
\end{itemize}
This finiteness is crucial and is completely elementary.
\end{remark}

%% =================================================================
\section{Logarithmic coordinates and translations}
%% =================================================================

Set $u=\log x$, so that $d^*x=du$ and the interval $[\lambda^{-1},\lambda]$ becomes
\[
I:=(-L,L),\qquad L:=\log\lambda.
\]
For $G\in L^2(I)$ we denote by $\widetilde G$ its extension by $0$ to $\R$.
Let $S_t$ be translation on $L^2(\R)$:
\[
(S_t\phi)(u):=\phi(u-t).
\]
Then in logarithmic coordinates, the dilation $U_{e^t}$ from \eqref{eq:Ua} corresponds to
translation: if $G(u)=g(e^u)$, then $(U_{e^t}g)(e^u)=g(e^{u-t})$, i.e.\ $\widetilde G\mapsto S_t\widetilde G$.

%% =================================================================
\section{Energy decomposition into translation differences}
%% =================================================================

\subsection{Prime contributions}
\begin{lemma}[Prime term as a difference energy plus a constant]
\label{lem:prime-energy}
Let $f=g*g^*$ with $g\in C_c^\infty([\lambda^{-1},\lambda])$, and let $G(u)=g(e^u)$.
Then
\[
-W_p(f)
=\sum_{\substack{m\ge 1\\ p^m\le \lambda^2}} (\log p)\,p^{-m/2}\,\|\widetilde G-S_{m\log p}\widetilde G\|_{L^2(\R)}^2
\;+\;c_p(\lambda)\,\|G\|_{L^2(I)}^2,
\]
where $c_p(\lambda)\in \R$ is a finite constant depending only on $p$ and $\lambda$.
\end{lemma}

\begin{structuredproof}
\Step{$W_p(f)=(\log p)\sum_{m\ge 1}p^{-m/2}\,2\Re\langle g,U_{p^m}g\rangle$.}

\Substep{By~\eqref{eq:Wp}, $W_p(f)=(\log p)\sum_{m\ge 1}p^{-m/2}\bigl(f(p^m)+f(p^{-m})\bigr)$.}
\Substepjust{Definition of $W_p$.  (The sum is in fact finite: Step~3 below shows
$\langle g,U_{p^m}g\rangle=0$ for $p^m>\lambda^2$, so only finitely many terms
contribute.  Steps~1--2 are therefore a finite computation.)}

\Substep{$f(p^m)+f(p^{-m})=2\Re\langle g,U_{p^m}g\rangle$.}
\Substepjust{By Lemma~\ref{lem:f-inner}, $f(a)+f(a^{-1})=2\Re\langle g,U_ag\rangle$.  Set $a=p^m$.}

\Substep{Combine Steps~1.1 and~1.2.}
\Substepjust{Substitute the identity from Step~1.2 into the sum from Step~1.1.}

\Step{For each $m\ge 1$ with $p^m\le\lambda^2$:
$2\Re\langle g,U_{p^m}g\rangle = 2\|g\|_2^2-\|g-U_{p^m}g\|_2^2$.}
\Stepjust{Lemma~\ref{lem:unitary} applied with $U=U_{p^m}$, $h=g$.}

\Step{For $m\ge 1$ with $p^m>\lambda^2$: $\langle g,U_{p^m}g\rangle=0$.}
\Stepjust{By Remark~\ref{rem:truncate}: when $p^m>\lambda^2$, the supports of $g$ (in $[\lambda^{-1},\lambda]$)
and $U_{p^m}g$ (in $[p^m\lambda^{-1},p^m\lambda]$) are disjoint.}

\Step{In logarithmic coordinates:
$\|g-U_{p^m}g\|_2=\|\widetilde G-S_{m\log p}\widetilde G\|_{L^2(\R)}$.}
\Stepjust{The substitution $u=\log x$ converts $d^*x=du$, $g(x)\mapsto G(u)$, and
$(U_{p^m}g)(x)=g(x/p^m)\mapsto G(u-m\log p)=(S_{m\log p}\widetilde G)(u)$.
The $L^2(\R_+^*,d^*x)$ norm becomes the $L^2(\R,du)$ norm.
(Here $\widetilde G$ denotes the zero-extension of $G$ to $\R$; the identity
holds because $g$ is supported in $[\lambda^{-1},\lambda]$, so $G$ is supported
in $I=[-L,L]$, and the substitution $u=\log x$ applies simultaneously to both terms.)}

\Step{Assemble the formula for $-W_p(f)$.}

\Substep{From Steps~1--3, only terms with $p^m\le\lambda^2$ contribute, and for those terms:
\[
W_p(f)=(\log p)\sum_{\substack{m\ge 1\\ p^m\le\lambda^2}}p^{-m/2}\bigl(2\|g\|_2^2-\|g-U_{p^m}g\|_2^2\bigr).
\]}
\Substepjust{Combine Steps~1, 2, and 3: the terms with $p^m>\lambda^2$ vanish by Step~3; the remaining
terms are rewritten using Step~2.}

\Substep{Negate and use Step~4:
\[
-W_p(f)=\sum_{\substack{m\ge 1\\ p^m\le\lambda^2}}(\log p)\,p^{-m/2}\,
\|\widetilde G-S_{m\log p}\widetilde G\|_{L^2(\R)}^2
-2(\log p)\Bigl(\sum_{\substack{m\ge 1\\ p^m\le\lambda^2}}p^{-m/2}\Bigr)\|g\|_2^2.
\]}
\Substepjust{Negate Step~5.1 and replace $\|g-U_{p^m}g\|_2$ by $\|\widetilde G-S_{m\log p}\widetilde G\|_{L^2(\R)}$ using Step~4.}

\Substep{Set $c_p(\lambda):=-2(\log p)\sum_{\substack{m\ge 1,\, p^m\le\lambda^2}}p^{-m/2}$.
Then $c_p(\lambda)\in\R$ is finite (the sum has finitely many terms), and
$\|g\|_2^2=\|G\|_{L^2(I)}^2$.}
\Substepjust{The sum is finite because only finitely many integers $m$ satisfy $p^m\le\lambda^2$.
The norm identity $\|g\|_2=\|G\|_{L^2(I)}$ follows from the change of variables $u=\log x$.}

\QEDstep
\end{structuredproof}

\subsection{Archimedean contribution}
\begin{lemma}[Archimedean term as a continuum of difference energies plus a constant]
\label{lem:arch-energy}
Let $f=g*g^*$ with $g\in C_c^\infty([\lambda^{-1},\lambda])$, and let $G(u)=g(e^u)$.
Define the strictly positive weight on $(0,\infty)$,
\[
w(t):=\frac{e^{t/2}}{e^t-e^{-t}}=\frac{e^{t/2}}{2\sinh t}.
\]
Then
\[
-W_{\R}(f)
=\int_{0}^{2L} w(t)\,\|\widetilde G-S_t\widetilde G\|_{L^2(\R)}^2\,dt
\;+\;c_\infty(\lambda)\,\|G\|_{L^2(I)}^2,
\]
where $c_\infty(\lambda)\in \R$ is a finite constant depending only on $\lambda$.
\end{lemma}

\begin{structuredproof}
\Step{Rewrite $W_\R(f)$ by substituting $x=e^t$:
\[
W_{\R}(f)=(\log4\pi+\gamma)\,f(1)+\int_0^\infty\Bigl(f(e^t)+f(e^{-t})-2e^{-t/2}f(1)\Bigr)\,w(t)\,dt.
\]}
\Stepjust{Starting from~\eqref{eq:WR}, set $x=e^t$ so that $d^*x=dt$.
Then $x^{1/2}/(x-x^{-1})=e^{t/2}/(e^t-e^{-t})=w(t)$, and the integration range $x\in[1,\infty)$
becomes $t\in[0,\infty)$.
(Convergence: near $t=0$, $f(e^t)+f(e^{-t})-2e^{-t/2}f(1)=O(t)$ by Taylor expansion
of the smooth function~$f$, cancelling the $1/t$ singularity of $w(t)\sim 1/(2t)$;
for $t>2L$, $f(e^t)=f(e^{-t})=0$ by Remark~\ref{rem:truncate}
and the remaining term $2e^{-t/2}w(t)f(1)=f(1)/\sinh t$ is $O(e^{-t})$.)}

\Step{$f(1)=\|g\|_2^2$ and $f(e^t)+f(e^{-t})=2\Re\langle g,U_{e^t}g\rangle$.}
\Stepjust{The first identity is Lemma~\ref{lem:f-inner} with $a=1$.
The second is Lemma~\ref{lem:f-inner}: $f(a)+f(a^{-1})=2\Re\langle g,U_ag\rangle$ with $a=e^t$.}

\Step{$-2\Re\langle g,U_{e^t}g\rangle=\|g-U_{e^t}g\|_2^2-2\|g\|_2^2$.}
\Stepjust{Lemma~\ref{lem:unitary} with $U=U_{e^t}$, $h=g$:
$2\Re\langle g,U_{e^t}g\rangle=2\|g\|_2^2-\|g-U_{e^t}g\|_2^2$.  Negate both sides.}

\Step{Substituting Steps~2 and 3 into the integral from Step~1, the integrand of $-W_\R(f)$
(inside $\int_0^\infty$) equals
\[
w(t)\bigl(\|g-U_{e^t}g\|_2^2+2(e^{-t/2}-1)\|g\|_2^2\bigr).
\]}

\Substep{From Step~1 (negated), $-W_\R(f)=-(\log 4\pi+\gamma)\|g\|_2^2
+\int_0^\infty\bigl(-2\Re\langle g,U_{e^t}g\rangle+2e^{-t/2}\|g\|_2^2\bigr)w(t)\,dt$.}
\Substepjust{Negate Step~1 and use $f(1)=\|g\|_2^2$ from Step~2.
(Absolute convergence of this integral is verified in Steps~7--8 below;
we proceed with the algebraic manipulation.)}

\Substep{Replace $-2\Re\langle g,U_{e^t}g\rangle$ by $\|g-U_{e^t}g\|_2^2-2\|g\|_2^2$ (Step~3):
the integrand becomes $(\|g-U_{e^t}g\|_2^2-2\|g\|_2^2+2e^{-t/2}\|g\|_2^2)w(t)
=w(t)(\|g-U_{e^t}g\|_2^2+2(e^{-t/2}-1)\|g\|_2^2)$.}
\Substepjust{Algebra: $-2+2e^{-t/2}=2(e^{-t/2}-1)$.}

\Step{In logarithmic coordinates: $\|g-U_{e^t}g\|_2=\|\widetilde G-S_t\widetilde G\|_{L^2(\R)}$.}
\Stepjust{Same argument as in Lemma~\ref{lem:prime-energy}, Step~4.}

\Step{Split the integral at $t=2L$.  For $t>2L$, $\|\widetilde G-S_t\widetilde G\|_2^2=2\|G\|_2^2$.}
\Stepjust{By Remark~\ref{rem:truncate}, when $t>2L=2\log\lambda$, the supports of
$\widetilde G$ (contained in $[-L,L]$) and $S_t\widetilde G$ (contained in $[-L+t,L+t]$,
with $-L+t>L$) are disjoint.
Hence $\|\widetilde G-S_t\widetilde G\|_2^2=\|\widetilde G\|_2^2+\|S_t\widetilde G\|_2^2
=2\|G\|_2^2$.}

\Step{The tail integral over $(2L,\infty)$ is a finite constant times $\|G\|_2^2$.}

\Substep{For $t>2L$, the integrand from Step~4 becomes
$w(t)(2\|G\|_2^2+2(e^{-t/2}-1)\|G\|_2^2)=2e^{-t/2}w(t)\|G\|_2^2$.}
\Substepjust{Substitute $\|\widetilde G-S_t\widetilde G\|_2^2=2\|G\|_2^2$ from Step~6 into the
integrand from Step~4 (after applying Step~5):
$2\|G\|_2^2+2(e^{-t/2}-1)\|G\|_2^2=2e^{-t/2}\|G\|_2^2$.}

\Substep{$\int_{2L}^\infty 2e^{-t/2}w(t)\,dt<\infty$.}
\Substepjust{$2e^{-t/2}w(t)=2e^{-t/2}\cdot e^{t/2}/(2\sinh t)=1/\sinh t$.
For $t\ge 1$: $e^{-t}<1$ and $e^t>2$, so $e^t-e^{-t}>e^t/2$, giving
$\sinh t>e^t/4$, hence $1/\sinh t<4e^{-t}$.
On $(0,1]\cap(2L,\infty)$ (nonempty only if $2L<1$), $1/\sinh t$ is continuous and bounded.
Therefore $\int_{2L}^\infty 1/\sinh t\,dt\le C+4\int_1^\infty e^{-t}\,dt<\infty$.}

\Step{The integral over $[0,2L]$ contributes the main term plus a finite constant.}

\Substep{For $t\in[0,2L]$, the integrand (Step~4 with Step~5) splits as
$w(t)\|\widetilde G-S_t\widetilde G\|_2^2+2(e^{-t/2}-1)w(t)\|G\|_2^2$.}
\Substepjust{Rewrite Step~4 using Step~5.}

\Substep{$\int_0^{2L}2(e^{-t/2}-1)w(t)\,dt$ is finite.}
\Substepjust{By convexity of $e^{-x}$: $e^{-t/2}\ge 1-t/2$, so $|e^{-t/2}-1|\le t/2$.
Since $\sinh t\ge t$ for $t\ge 0$: $w(t)=e^{t/2}/(2\sinh t)\le e^{t/2}/(2t)$.
Hence $|2(e^{-t/2}-1)w(t)|\le 2\cdot(t/2)\cdot e^{t/2}/(2t)=e^{t/2}/2\le e^L/2$
on $[0,2L]$.  The integrand is bounded on a compact interval, so the integral
converges absolutely.}

\Step{Define $c_\infty(\lambda):=-(\log 4\pi+\gamma)+\int_0^{2L}2(e^{-t/2}-1)w(t)\,dt
+\int_{2L}^\infty 2e^{-t/2}w(t)\,dt$.}
\Stepjust{Collect all $\|G\|_2^2$-proportional terms: the $-(\log 4\pi+\gamma)$ from
Step~4.1, the integral $\int_0^{2L}2(e^{-t/2}-1)w(t)\,dt$ from Step~8.2,
and the tail integral from Step~7.2.  Each is finite, so $c_\infty(\lambda)\in\R$.}

\Step{Conclusion: $-W_\R(f)=\int_0^{2L}w(t)\|\widetilde G-S_t\widetilde G\|_2^2\,dt
+c_\infty(\lambda)\|G\|_{L^2(I)}^2$.}
\Stepjust{Combine Steps~8.1, 7, and 9, noting $\|g\|_2^2=\|G\|_{L^2(I)}^2$
(by the isometry $u=\log x$, as in Lemma~\ref{lem:prime-energy}, Step~5.3).}

\QEDstep
\end{structuredproof}

\subsection{Global quadratic form on the interval}
\label{sec:globalform}

\begin{definition}[Difference-energy form]
\label{def:E}
Fix $\lambda>1$ and $L=\log\lambda$. For $G\in L^2(I)$ define
\begin{align}
\mathcal E_\lambda(G)
&:=\int_{0}^{2L} w(t)\,\|\widetilde G-S_t\widetilde G\|_{L^2(\R)}^2\,dt
+\sum_{\substack{p\ \mathrm{prime}\\ p\le \lambda^2}}\ \sum_{\substack{m\ge 1\\ p^m\le \lambda^2}}
(\log p)p^{-m/2}\,\|\widetilde G-S_{m\log p}\widetilde G\|_{L^2(\R)}^2.
\end{align}
\end{definition}

\begin{remark}[What we have proved so far]
Lemmas~\ref{lem:prime-energy} and \ref{lem:arch-energy} show that for $f=g*g^*$ with
$g\in C_c^\infty([\lambda^{-1},\lambda])$, the quantity
\[
-\sum_{v\in\{\infty\}\cup\{p\}} W_v(f)
\]
equals $\mathcal E_\lambda(G)$ plus an additive constant multiple of $\|G\|_2^2$.
The form $\mathcal E_\lambda$ (Definition~\ref{def:E}) is then defined for all $G\in L^2(I)$
as an extended-real-valued quadratic form; from this point onward, all arguments use only the
abstract properties of $\mathcal E_\lambda$ and do not depend on the explicit-formula derivation.
Since adding a constant multiple of $\|G\|_2^2$ only shifts the spectrum of the associated
operator, it does not affect positivity/irreducibility properties of the semigroup and does not
affect eigenfunction parity considerations.

\medskip\noindent
\emph{Warning (positivity vs.\ spectral shape).}
Note that $c_p(\lambda)<0$ in Lemma~\ref{lem:prime-energy}, so the sign of the total
additive constant $C_\lambda$ is not controlled here.
This means the decomposition does \emph{not} by itself establish nonnegativity of the
full explicit-formula quadratic form (``Weil positivity'').
What it does establish is the Markov, irreducible, and compact-resolvent structure of the
nonnegative difference-energy part $\mathcal E_\lambda$, and hence the simplicity, strict
positivity, and evenness of the ground-state eigenfunction---properties that are unaffected
by adding a scalar multiple of $\|G\|_2^2$.
\end{remark}

%% =================================================================
\section{Markov property (normal contractions)}
%% =================================================================

\begin{definition}[Normal contraction]
A map $\Phi:\R\to\R$ is a normal contraction if $\Phi(0)=0$ and
$|\Phi(a)-\Phi(b)|\le |a-b|$ for all $a,b\in\R$.
\end{definition}

\begin{lemma}[Markov property]
\label{lem:markov}
For every normal contraction $\Phi$ and every $G\in L^2(I)$,
\[
\mathcal E_\lambda(\Phi\circ G)\le \mathcal E_\lambda(G).
\]
In particular, $\mathcal E_\lambda(|G|)\le \mathcal E_\lambda(G)$.
\end{lemma}

\begin{structuredproof}
\Step{$\Phi\circ G\in L^2(I)$.}
\Stepjust{Since $\Phi(0)=0$ and $|\Phi(a)-\Phi(b)|\le|a-b|$, setting $b=0$ gives
$|\Phi(G(u))|\le|G(u)|$ pointwise.  Hence
$\|\Phi\circ G\|_{L^2(I)}\le\|G\|_{L^2(I)}<\infty$.}

\Step{For each fixed $t\in\R$:
$\|\widetilde{\Phi\circ G}-S_t\widetilde{\Phi\circ G}\|_2^2
\le\|\widetilde G-S_t\widetilde G\|_2^2$.}

\Substep{$\widetilde{\Phi\circ G}=\Phi\circ\widetilde G$.}
\Substepjust{Since $\Phi(0)=0$, the extension by zero commutes with composition by~$\Phi$:
for $u\notin I$, $\widetilde G(u)=0$, so $\Phi(\widetilde G(u))=\Phi(0)=0=\widetilde{\Phi\circ G}(u)$;
for $u\in I$, both sides equal $\Phi(G(u))$.}

\Substep{$|\Phi(\widetilde G(u))-\Phi(\widetilde G(u-t))|^2\le|\widetilde G(u)-\widetilde G(u-t)|^2$
for every $u\in\R$.}
\Substepjust{$\Phi$ is $1$-Lipschitz by hypothesis (the ``normal contraction'' condition $|\Phi(a)-\Phi(b)|\le|a-b|$).
Apply this pointwise with $a=\widetilde G(u)$, $b=\widetilde G(u-t)$, and square.}

\Substep{Integrate Step~2.2 over $\R$ with respect to $du$:
\[
\int_\R|\Phi(\widetilde G(u))-\Phi(\widetilde G(u-t))|^2\,du
\le\int_\R|\widetilde G(u)-\widetilde G(u-t)|^2\,du.
\]}
\Substepjust{Integrate both sides of the pointwise inequality from Step~2.2.}

\Substep{Rewrite using Step~2.1 and the definition of $S_t$: this is precisely
$\|\widetilde{\Phi\circ G}-S_t\widetilde{\Phi\circ G}\|_2^2\le\|\widetilde G-S_t\widetilde G\|_2^2$.}
\Substepjust{$(S_t\widetilde{\Phi\circ G})(u)=(\Phi\circ\widetilde G)(u-t)=\Phi(\widetilde G(u-t))$.}

\Step{$\mathcal E_\lambda(\Phi\circ G)\le\mathcal E_\lambda(G)$.}
\Stepjust{When $\mathcal E_\lambda(G)=+\infty$ the inequality is trivial.
Otherwise, by Definition~\ref{def:E}, $\mathcal E_\lambda(G)$ is the integral of
$w(t)\|\widetilde G-S_t\widetilde G\|_2^2$ over $[0,2L]$ (with weight $w(t)\ge 0$)
plus a finite sum of terms $(\log p)p^{-m/2}\|\widetilde G-S_{m\log p}\widetilde G\|_2^2$
(all coefficients $\ge 0$).  Step~2 shows each summand decreases (or stays the same)
when $G$ is replaced by $\Phi\circ G$.  Since all weights are nonnegative,
the integral and sum are each $\le$ the corresponding quantity for~$G$.}

\Step{In particular, $\mathcal E_\lambda(|G|)\le\mathcal E_\lambda(G)$.}
\Stepjust{$\Phi(x)=|x|$ is a normal contraction: $\Phi(0)=0$ and
$\bigl||a|-|b|\bigr|\le|a-b|$ by the reverse triangle inequality.
Apply Step~3.}

\QEDstep
\end{structuredproof}

%% =================================================================
\section{A translation-invariance lemma on an interval}
%% =================================================================

\begin{lemma}[Local translation invariance forces null or conull]\label{lem:trans-inv}
Let $I\subset\R$ be a nontrivial open interval and let $B\subset I$ be measurable.
Assume that there exists $\varepsilon>0$ such that for every $t\in(0,\varepsilon)$,
\begin{equation}\label{eq:indicator-inv}
\1_{B}(u)=\1_{B}(u-t)\quad\text{for a.e.\ }u\in I\cap(I+t).
\end{equation}
Then either $m(B)=0$ or $m(I\setminus B)=0$.
\end{lemma}

\begin{structuredproof}
\Step{Set $f:=\1_B\in L^1_{\mathrm{loc}}(I)$.  Fix a compact subinterval $J\Subset I$.
Choose $0<\delta<\min\{\varepsilon,\mathrm{dist}(J,\partial I)\}$.}
\Stepjust{Since $J\Subset I$, we have $\mathrm{dist}(J,\partial I)>0$, so
$\delta$ as described exists.}

\Step{For every $t\in(0,\delta)$: $f(u+t)=f(u)$ for a.e.\ $u\in J$.}

\Substep{From~\eqref{eq:indicator-inv}: for $t\in(0,\varepsilon)$,
$f(u)=f(u-t)$ for a.e.\ $u\in I\cap(I+t)$.}
\Substepjust{Hypothesis of the lemma, since $0<t<\delta<\varepsilon$.}

\Substep{Substitute $u\mapsto u+t$: $f(u+t)=f(u)$ for a.e.\ $u\in (I-t)\cap I$.}
\Substepjust{The set $I\cap(I+t)$ becomes $(I-t)\cap I$ after the shift.}

\Substep{$J\subset (I-t)\cap I$.}
\Substepjust{For $u\in J$: since $\delta<\mathrm{dist}(J,\partial I)$, we have $u\in I$ and
$u+t\in I$ (as $|t|<\delta$), so $u\in I$ and $u\in I-t$, giving $u\in(I-t)\cap I$.}

\Substep{Combine: $f(u+t)=f(u)$ for a.e.\ $u\in J$.}
\Substepjust{Restrict the a.e.\ identity from Step~2.2 to the subset $J\subset(I-t)\cap I$ (Step~2.3).}

\Step{Summary: $f(u+t)=f(u)$ for a.e.\ $u\in J$, for all $t\in(0,\delta)$.
At $t=0$ the identity is trivial.}
\Stepjust{This is Step~2 restated for emphasis.  (The downstream mollification argument,
Steps~4--9, uses only $t\in(0,\delta/2)$, so no negative-$t$ extension is required.)}

\Step{Fix the compact interval $K:=[a+\delta,\,b-\delta]$ where $J=[a,b]$.
Shrink $J$ at the outset so that $K$ is nonempty.}
\Stepjust{Choosing $J$ with length $>2\delta$ ensures $K\ne\emptyset$.
Working on the fixed domain $K\Subset J$ will let us avoid any dependence on varying domains
in the subsequent mollification argument.}

\Step{Let $\rho$ be a standard nonneg.\ mollifier supported in $(-1,1)$ with $\int\rho=1$.
Set $\rho_\eta(s)=\eta^{-1}\rho(s/\eta)$ and $f_\eta:=f*\rho_\eta$ for $0<\eta<\delta/4$.}
\Stepjust{Standard construction; $f_\eta\in C^\infty(\R)$.
(Extend $f=\1_B$ to $\R$ by zero outside~$I$.  For $u\in K$ the convolution
samples $f$ only at points $u-s$ with $|s|<\eta<\delta/4$, so $u-s\in J\subset I$
and the extension choice is immaterial.)}

\Step{For every $0<\eta<\delta/4$ and every $t\in(0,\delta/2)$: $f_\eta(u+t)=f_\eta(u)$ for all $u\in K$.}
\Stepjust{Fix $u\in K$ and $|s|<\eta$.  Then
$u-s\in[a+\delta-\eta,\,b-\delta+\eta]\subset J$ (since $\eta<\delta/4$).
Let $h_t:=f(\cdot+t)-f(\cdot)$.  By Step~2 (with $t\in(0,\delta/2)\subset(0,\delta)$), $h_t=0$ a.e.\ on $J$, so $h_t=0$ in $L^1(J)$.
Hence
\[
f_\eta(u+t)-f_\eta(u)=(h_t*\rho_\eta)(u)=\int h_t(u-s)\rho_\eta(s)\,ds=0,
\]
because $u-s\in J$ on the support of $\rho_\eta$.}

\Step{For each $\eta\in(0,\delta/4)$, $f_\eta$ is constant on $K$; call this constant $c_\eta$.}
\Stepjust{$f_\eta\in C^\infty$ and Step~6 gives $f_\eta(u+t)=f_\eta(u)$ for all $u\in K$ and all $t\in(0,\delta/2)$.
For any $u\in K^\circ$:
\[
f_\eta'(u)=\lim_{t\downarrow 0}\frac{f_\eta(u+t)-f_\eta(u)}{t}=0.
\]
Hence $f_\eta$ is constant on the connected interval $K$.}

\Step{$(c_\eta)_{\eta\downarrow 0}$ is Cauchy, hence converges to some $c\in\R$.}
\Stepjust{Since $f\in L^1(K)$ and mollification converges in $L^1(K)$:
\[
m(K)|c_\eta-c_{\eta'}|=\|f_\eta-f_{\eta'}\|_{L^1(K)}
\le \|f_\eta-f\|_{L^1(K)}+\|f-f_{\eta'}\|_{L^1(K)}\to 0.
\]
Hence $(c_\eta)$ is Cauchy, so convergent.}

\Step{$f=\1_B$ equals $c$ a.e.\ on $K$, hence $c\in\{0,1\}$.}
\Stepjust{
\[
\|f-c\|_{L^1(K)}\le\|f-f_\eta\|_{L^1(K)}+\|f_\eta-c\|_{L^1(K)}
=\|f-f_\eta\|_{L^1(K)}+m(K)|c_\eta-c|\to 0.
\]
Thus $f=c$ a.e.\ on $K$.  Since $f=\1_B$ takes only values $0$ and $1$ a.e., we have $c\in\{0,1\}$.}

\Step{$f=\1_B$ is a.e.\ constant on~$I$, hence $m(B)=0$ or $m(I\setminus B)=0$.}
\Stepjust{Write $I=(\alpha,\beta)$.  For each integer $n\ge 1$ set
$J_n:=[\alpha+1/n,\,\beta-1/n]$ (nonempty for $n$ large) and
$\delta_n:=\min(\varepsilon,1/n)/2$.
Then $|J_n|=\beta-\alpha-2/n>2\delta_n$ for all sufficiently large~$n$
(since $\delta_n\le 1/(2n)$).
Apply Steps~1--9 with $J=J_n$, $\delta=\delta_n$: the function $\1_B$ equals a constant
$c_n\in\{0,1\}$ a.e.\ on $K_n:=[\alpha+1/n+\delta_n,\,\beta-1/n-\delta_n]$.
For $n$ large, $K_n\subset K_{n+1}^\circ$ (since $1/(n+1)+\delta_{n+1}<1/n+\delta_n$
eventually), so $K_n\cap K_{n+1}$ has positive measure and $c_n=c_{n+1}$.
Hence all $c_n$ agree for $n$ large, and $\bigcup_n K_n=I$ up to measure zero.
Thus $\1_B$ is a.e.\ constant on~$I$.}

\QEDstep
\end{structuredproof}

%% =================================================================
\section{Irreducibility from the archimedean continuum}
%% =================================================================

\subsection{A concrete criterion}

\begin{lemma}[Indicator-energy vanishes only for null/conull sets]
\label{lem:indicator-energy}
Let $B\subset I$ be measurable. If $\mathcal E_\lambda(\1_B)=0$, then $m(B)=0$ or $m(I\setminus B)=0$.
\end{lemma}

\begin{structuredproof}
\Step{The archimedean contribution to $\mathcal E_\lambda(\1_B)$ vanishes:
$\int_0^{2L}w(t)\|\widetilde{\1_B}-S_t\widetilde{\1_B}\|_2^2\,dt=0$.}
\Stepjust{By Definition~\ref{def:E}, $\mathcal E_\lambda(\1_B)$ is a sum of nonneg.\ terms (the archimedean
integral plus the prime sums).  If the total is $0$, each nonneg.\ summand must be $0$.
In particular, the archimedean integral (whose integrand is nonneg.) equals~$0$.}

\Step{$\|\widetilde{\1_B}-S_t\widetilde{\1_B}\|_2^2=0$ for a.e.\ $t\in(0,2L)$.}
\Stepjust{The integrand $w(t)\|\widetilde{\1_B}-S_t\widetilde{\1_B}\|_2^2$ is nonneg.\ and
$w(t)>0$ for all $t>0$ (since $w(t)=e^{t/2}/(2\sinh t)$ with numerator and denominator
both positive for $t>0$).
A nonneg.\ integral vanishing (Step~1) with a strictly positive weight implies the
other factor vanishes a.e.}

\Step{Upgrade to \emph{all} $t\in(0,2L)$:
$\|\widetilde{\1_B}-S_t\widetilde{\1_B}\|_2^2=0$ for every $t\in(0,2L)$.}

\Substep{The function $t\mapsto\|\phi-S_t\phi\|_2^2$ is continuous for any $\phi\in L^2(\R)$.}
\Substepjust{By strong continuity of the translation group $(S_t)_{t\in\R}$ on $L^2(\R)$
(which follows from dominated convergence: if $t_n\to t$ then $S_{t_n}\phi\to S_t\phi$ in $L^2$),
the map $t\mapsto S_t\phi$ is continuous $\R\to L^2(\R)$, and the squared norm is a continuous
function of its argument.}

\Substep{A continuous function that vanishes a.e.\ on an interval vanishes everywhere on that interval.}
\Substepjust{Let $h:(0,2L)\to[0,\infty)$ be continuous with $h=0$ a.e.
If $h(t_0)>0$ for some $t_0$, then by continuity $h>0$ on an open neighborhood of $t_0$,
which has positive Lebesgue measure---contradicting $h=0$ a.e.}

\Substep{Apply Steps~3.1 and 3.2 with $\phi=\widetilde{\1_B}$.}
\Substepjust{Step~2 says $\|\widetilde{\1_B}-S_t\widetilde{\1_B}\|_2^2=0$ for a.e.\ $t\in(0,2L)$;
Step~3.1 says this function of $t$ is continuous; Step~3.2 upgrades ``a.e.'' to ``all.''} %

\Step{For every $t\in(0,2L)$: $\1_B(u)=\1_B(u-t)$ for a.e.\ $u\in I\cap(I+t)$.}
\Stepjust{$\|\widetilde{\1_B}-S_t\widetilde{\1_B}\|_2^2=0$ (Step~3) means
$\widetilde{\1_B}(u)=\widetilde{\1_B}(u-t)$ for a.e.\ $u\in\R$.
Restricting to $u\in I\cap(I+t)$: both $u\in I$ and $u-t\in I$, so
$\widetilde{\1_B}(u)=\1_B(u)$ and $\widetilde{\1_B}(u-t)=\1_B(u-t)$.}

\Step{$m(B)=0$ or $m(I\setminus B)=0$.}
\Stepjust{Step~4 holds for all $t\in(0,2L)$, which contains an interval $(0,\varepsilon)$
for any $\varepsilon\le 2L$.  Lemma~\ref{lem:trans-inv} applies (with $\varepsilon=2L$)
and yields the conclusion.}

\QEDstep
\end{structuredproof}

\begin{remark}[In the nonconservative (killing) setting, only the null case can occur]
\label{rem:indicator-energy-sharp}
Since $\mathcal E_\lambda$ is defined by zero-extension (Definition~\ref{def:E}), one has
$\mathcal E_\lambda(1)>0$, so the conull alternative $m(I\setminus B)=0$ cannot occur under
the hypothesis $\mathcal E_\lambda(\1_B)=0$.  Hence that hypothesis actually forces $m(B)=0$.
\end{remark}

\begin{structuredproof}[Proof of Remark~\ref{rem:indicator-energy-sharp}]
\Step{$\mathcal E_\lambda(1)>0$.}
\Stepjust{For $0<t<2L$, $\widetilde 1=\1_{(-L,L)}$ and $S_t\widetilde 1=\1_{(-L+t,L+t)}$, so
$\|\widetilde 1-S_t\widetilde 1\|_2^2=m(I\,\Delta\,(I+t))=2t>0$
(the symmetric difference consists of $(-L,-L+t)$ and $(L,L+t)$, each of measure~$t$).
Since $w(t)>0$ on $(0,2L)$:
\[
\mathcal E_\lambda(1)\ge\int_0^{2L}w(t)\cdot 2t\,dt>0.
\]}

\Step{If $m(I\setminus B)=0$, then $\mathcal E_\lambda(\1_B)=\mathcal E_\lambda(1)>0$.}
\Stepjust{$m(I\setminus B)=0$ implies $\1_B=1$ in $L^2(I)$, hence
$\mathcal E_\lambda(\1_B)=\mathcal E_\lambda(1)>0$ by Step~1.}

\Step{Therefore $\mathcal E_\lambda(\1_B)=0$ forces $m(B)=0$.}
\Stepjust{Contrapositive of Step~2 combined with Lemma~\ref{lem:indicator-energy}.}

\QEDstep
\end{structuredproof}

%% -----------------------------------------------------------------
\subsection{Operator realization: closedness and compact resolvent}
%% =================================================================

\subsubsection{Ambient form on $L^2(\R)$ and Fourier representation}

Let $\mathcal F:L^2(\R)\to L^2(\R)$ denote the unitary Fourier transform
\[
\widehat{\phi}(\xi):=\int_{\R}\phi(u)e^{-iu\xi}\,du,
\qquad
\phi(u)=\frac1{2\pi}\int_{\R}\widehat{\phi}(\xi)e^{iu\xi}\,d\xi,
\]
so that Plancherel reads $\|\phi\|_{L^2(\R)}^2=\frac1{2\pi}\int_{\R}|\widehat{\phi}(\xi)|^2\,d\xi$.

Define the ``ambient'' quadratic form on $L^2(\R)$ by
\begin{align*}
\mathcal E_\lambda^{\R}(\phi)
&:=\int_{0}^{2L} w(t)\,\|\phi-S_t\phi\|_{L^2(\R)}^2\,dt\\
&\qquad\qquad
+\sum_{\substack{p\ \mathrm{prime}\\ p\le \lambda^2}}\ \sum_{\substack{m\ge 1\\ p^m\le \lambda^2}}
(\log p)p^{-m/2}\,\|\phi-S_{m\log p}\phi\|_{L^2(\R)}^2,
\end{align*}
with domain $\mathcal D(\mathcal E_\lambda^{\R}):=\{\phi\in L^2(\R):\mathcal E_\lambda^{\R}(\phi)<\infty\}$.
By definition, for $G\in L^2(I)$,
\[
\mathcal E_\lambda(G)=\mathcal E_\lambda^{\R}(\widetilde G).
\]

\begin{lemma}[Plancherel identity for translation differences]\label{lem:plancherel-diff}
For $\phi\in L^2(\R)$ and $t\in\R$,
\[
\|\phi-S_t\phi\|_{L^2(\R)}^2
=\frac1{2\pi}\int_{\R}|1-e^{-i\xi t}|^2\,|\widehat\phi(\xi)|^2\,d\xi
=\frac1{2\pi}\int_{\R}4\sin^2\!\Bigl(\frac{\xi t}{2}\Bigr)\,|\widehat\phi(\xi)|^2\,d\xi.
\]
\end{lemma}

\begin{structuredproof}
\Step{$\widehat{S_t\phi}(\xi)=e^{-i\xi t}\widehat\phi(\xi)$.}
\Stepjust{By definition, $\widehat{S_t\phi}(\xi)=\int_\R\phi(u-t)e^{-iu\xi}\,du$.
Substituting $v=u-t$, $du=dv$:
$=\int_\R\phi(v)e^{-i(v+t)\xi}\,dv=e^{-it\xi}\widehat\phi(\xi)$.}

\Step{$\|\phi-S_t\phi\|_2^2=\frac{1}{2\pi}\int_\R|1-e^{-i\xi t}|^2|\widehat\phi(\xi)|^2\,d\xi$.}
\Stepjust{By Plancherel, $\|\phi-S_t\phi\|_2^2=\frac{1}{2\pi}\int_\R|\widehat\phi(\xi)-\widehat{S_t\phi}(\xi)|^2\,d\xi$.
Step~1 gives $\widehat\phi(\xi)-\widehat{S_t\phi}(\xi)=(1-e^{-i\xi t})\widehat\phi(\xi)$.}

\Step{$|1-e^{-i\eta}|^2=4\sin^2(\eta/2)$.}
\Stepjust{$1-e^{-i\eta}=1-\cos\eta+i\sin\eta$, so
$|1-e^{-i\eta}|^2=(1-\cos\eta)^2+\sin^2\eta=2-2\cos\eta=4\sin^2(\eta/2)$
by the double-angle formula $\cos\eta=1-2\sin^2(\eta/2)$.}

\Step{Set $\eta=\xi t$ in Step~3 and substitute into Step~2.}
\Stepjust{$|1-e^{-i\xi t}|^2=4\sin^2(\xi t/2)$.  This gives the second equality.}

\QEDstep
\end{structuredproof}

\begin{lemma}[Fourier representation]\label{lem:FourierRep}
For $\phi\in L^2(\R)$,
\[
\mathcal E_\lambda^{\R}(\phi)=\frac1{2\pi}\int_{\R}\psi_\lambda(\xi)\,|\widehat{\phi}(\xi)|^2\,d\xi
\quad\text{in }[0,\infty],
\]
where
\begin{align}
\psi_\lambda(\xi)
&:=4\int_0^{2L} w(t)\,\sin^2\!\Bigl(\frac{\xi t}{2}\Bigr)\,dt\notag\\
&\qquad\qquad
+4\sum_{\substack{p\ \mathrm{prime}\\ p\le \lambda^2}}\ \sum_{\substack{m\ge 1\\ p^m\le \lambda^2}}
(\log p)p^{-m/2}\,\sin^2\!\Bigl(\frac{\xi m\log p}{2}\Bigr).\label{eq:defpsi}
\end{align}
In particular $\psi_\lambda$ is measurable, even, finite for each $\xi$, and $\psi_\lambda(\xi)\ge 0$.
\end{lemma}

\begin{structuredproof}
\Step{Apply Lemma~\ref{lem:plancherel-diff} to each translation-difference norm in $\mathcal E_\lambda^\R(\phi)$.}
\Stepjust{Each term $\|\phi-S_t\phi\|_2^2$ or $\|\phi-S_{m\log p}\phi\|_2^2$ in Definition~\ref{def:E}
equals $\frac{1}{2\pi}\int_\R 4\sin^2(\xi t/2)|\widehat\phi(\xi)|^2\,d\xi$ (resp.\ with $t$ replaced
by $m\log p$).}

\Step{Interchange integration/summation order by Tonelli's theorem:
\[
\mathcal E_\lambda^\R(\phi)=\frac{1}{2\pi}\int_\R\psi_\lambda(\xi)|\widehat\phi(\xi)|^2\,d\xi.
\]}
\Stepjust{The integrand $w(t)\cdot 4\sin^2(\xi t/2)\cdot|\widehat\phi(\xi)|^2$ is
jointly measurable in $(t,\xi)$: $w(t)$ is Borel on $(0,2L]$,
$\sin^2(\xi t/2)$ is jointly continuous, and $|\widehat\phi|^2$ is measurable.
All factors are nonneg.\ ($w(t)\ge 0$, $\sin^2\ge 0$, $|\widehat\phi|^2\ge 0$, $(\log p)p^{-m/2}\ge 0$).
Tonelli's theorem permits interchange of the $\xi$-integral with the $t$-integral and the finite sums.
The resulting multiplier of $|\widehat\phi(\xi)|^2$ is precisely $\psi_\lambda(\xi)$.}

\Step{$\psi_\lambda$ is measurable, even, finite for each $\xi$, and $\ge 0$.}

\Substep{Measurability: $\psi_\lambda$ is a sum of continuous functions of~$\xi$.}
\Substepjust{Each $\sin^2(\xi t/2)$ is continuous in $\xi$.  The integral $\int_0^{2L}w(t)\sin^2(\xi t/2)\,dt$
is continuous in $\xi$ by dominated convergence: for $|\xi|\le M$, the integrand is bounded by
$w(t)\min\bigl(1,(Mt/2)^2\bigr)$, which is integrable on $(0,2L)$
(near $t=0$: $w(t)(Mt/2)^2\le M^2t^2 e^{t/2}/(8t)=M^2te^{t/2}/8$, integrable;
away from $0$: $w$ is bounded on $[1,2L]$).
The prime sum is a finite sum of continuous functions.}

\Substep{Evenness: $\psi_\lambda(-\xi)=\psi_\lambda(\xi)$.}
\Substepjust{$\sin^2((-\xi)t/2)=\sin^2(\xi t/2)$.}

\Substep{Finiteness: for each fixed $\xi$, $\psi_\lambda(\xi)<\infty$.}
\Substepjust{Split $(0,2L)=(0,1)\cup[1,2L)$.
On $[1,2L)$, $w$ is bounded and $\sin^2\le 1$, so the integral over $[1,2L)$ is finite.
On $(0,1)$, use $\sin^2(\xi t/2)\le(\xi t/2)^2$, giving
$w(t)\sin^2(\xi t/2)\le\xi^2 t^2 w(t)/4\le\xi^2 te^{t/2}/8$
(using $w(t)\le e^{t/2}/(2t)$ from $\sinh t\ge t$), which is integrable near $0$.
The prime sum is finite (finitely many terms, each finite).}

\Substep{Nonnegativity: $\psi_\lambda(\xi)\ge 0$.}
\Substepjust{Every summand is a product of nonneg.\ factors.}

\QEDstep
\end{structuredproof}

\begin{proposition}[Closedness on $L^2(\R)$]\label{prop:closedR}
The form $\mathcal E_\lambda^{\R}$ is densely defined, symmetric, nonneg., and closed on $L^2(\R)$.
Moreover,
\[
\mathcal D(\mathcal E_\lambda^{\R})
=\Bigl\{\phi\in L^2(\R):\int_{\R}\psi_\lambda(\xi)\,|\widehat\phi(\xi)|^2\,d\xi<\infty\Bigr\},
\]
and $\mathcal D(\mathcal E_\lambda^{\R})$ is a Hilbert space for the norm
$\|\phi\|_{\mathcal D}^2:=\|\phi\|_{L^2(\R)}^2+\mathcal E_\lambda^{\R}(\phi)$.
\end{proposition}

\begin{structuredproof}
\Step{$\mathcal E_\lambda^\R$ is the quadratic form of multiplication by $\psi_\lambda$ in Fourier space.}
\Stepjust{Lemma~\ref{lem:FourierRep}: $\mathcal E_\lambda^\R(\phi)=\frac{1}{2\pi}\int_\R\psi_\lambda(\xi)|\widehat\phi(\xi)|^2\,d\xi$.}

\Step{$\mathcal D(\mathcal E_\lambda^\R)=\{\phi\in L^2(\R):\int\psi_\lambda(\xi)|\widehat\phi(\xi)|^2\,d\xi<\infty\}$.}
\Stepjust{Immediate from Step~1: $\phi\in\mathcal D(\mathcal E_\lambda^\R)$ iff $\mathcal E_\lambda^\R(\phi)<\infty$
iff $\int\psi_\lambda|\widehat\phi|^2<\infty$.}

\Step{$\mathcal D(\mathcal E_\lambda^\R)$ with the norm $\|\phi\|_{\mathcal D}^2=\frac{1}{2\pi}\int(1+\psi_\lambda(\xi))|\widehat\phi(\xi)|^2\,d\xi$ is a Hilbert space.}
\Stepjust{Via $\phi\mapsto\widehat\phi$, this norm space is isometrically isomorphic to
$L^2(\R,(1+\psi_\lambda(\xi))\frac{d\xi}{2\pi})$, which is a weighted $L^2$ space with a
nonneg.\ measurable weight, hence complete.}

\Step{The form is closed.}
\Stepjust{By Kato~\cite[Thm.\,VI.1.17]{KatoPerturbation}, a nonneg.\ symmetric form is closed iff its domain
equipped with the graph norm $\|\cdot\|_{\mathcal D}$ is complete.
Step~3 verifies this.}

\Step{Nonnegativity and symmetry are immediate.}
\Stepjust{Nonnegativity: $\psi_\lambda\ge 0$ implies $\mathcal E_\lambda^\R(\phi)\ge 0$.
Symmetry: $\mathcal E_\lambda^\R$ is defined on real-valued or complex-valued functions via a real
nonneg.\ multiplier; the associated bilinear form inherits symmetry from the pointwise identity.}

\Step{$\mathcal E_\lambda^\R$ is densely defined: $C_c^\infty(\R)\subset\mathcal D(\mathcal E_\lambda^\R)$.}

\Substep{For $\phi\in C_c^\infty(\R)$: $\|\phi-S_t\phi\|_2\le|t|\|\phi'\|_2$.}
\Substepjust{By Plancherel:
$\|\phi-S_t\phi\|_2^2=\frac{1}{2\pi}\int 4\sin^2(\xi t/2)|\widehat\phi(\xi)|^2\,d\xi
\le t^2\frac{1}{2\pi}\int\xi^2|\widehat\phi(\xi)|^2\,d\xi=t^2\|\phi'\|_2^2$,
using $\sin^2 x\le x^2$ and $\frac{1}{2\pi}\int\xi^2|\widehat\phi|^2\,d\xi=\|\phi'\|_2^2$.}

\Substep{$\int_0^{2L}w(t)t^2\,dt<\infty$.}
\Substepjust{Since $\sinh t\ge t$: $w(t)=e^{t/2}/(2\sinh t)\le e^{t/2}/(2t)$, so
$w(t)t^2\le te^{t/2}/2$, which is bounded and integrable on $[0,2L]$.}

\Substep{$\mathcal E_\lambda^\R(\phi)\le\|\phi'\|_2^2\bigl(\int_0^{2L}w(t)t^2\,dt+\sum_{p,m}(\log p)p^{-m/2}(m\log p)^2\bigr)<\infty$.}
\Substepjust{From Step~6.1: each $\|\phi-S_t\phi\|_2^2\le t^2\|\phi'\|_2^2$.
Apply this to the archimedean integral and each prime term.  The prime sum is finite
(finitely many terms, each $\le(\log p)p^{-m/2}(m\log p)^2\|\phi'\|_2^2$).}

\Substep{$C_c^\infty(\R)$ is dense in $L^2(\R)$.}
\Substepjust{Standard fact in functional analysis.}

\QEDstep
\end{structuredproof}

\begin{proposition}[Closedness on $L^2(I)$]\label{prop:closedI}
The form $\mathcal E_\lambda$ on $H=L^2(I)$ is densely defined, symmetric, nonneg., and closed.
\end{proposition}

\begin{structuredproof}
\Step{$G\mapsto\widetilde G$ is an isometry from $L^2(I)$ onto
$H_I:=\{\phi\in L^2(\R):\phi=0\text{ a.e.\ on }\R\setminus I\}$.}
\Stepjust{$\|\widetilde G\|_{L^2(\R)}^2=\int_\R|\widetilde G(u)|^2\,du=\int_I|G(u)|^2\,du=\|G\|_{L^2(I)}^2$.
Surjectivity: any $\phi\in H_I$ is the zero-extension of $G:=\phi|_I$.}

\Step{$\mathcal E_\lambda(G)=\mathcal E_\lambda^\R(\widetilde G)$.}
\Stepjust{By definition (Definition~\ref{def:E} and the definition of $\mathcal E_\lambda^\R$).}

\Step{$\mathcal E_\lambda$ is the restriction of $\mathcal E_\lambda^\R$ to $H_I$.}
\Stepjust{Steps~1 and 2: $H_I$ is a closed subspace of $L^2(\R)$, and $\mathcal E_\lambda$ on $L^2(I)$
corresponds (via the isometry) to $\mathcal E_\lambda^\R$ restricted to~$H_I$.}

\Step{$\mathcal E_\lambda$ is closed.}
\Stepjust{Suppose $G_n\to G$ in $L^2(I)$ and $\mathcal E_\lambda(G_n-G_m)\to 0$.  Then $\widetilde G_n\to\widetilde G$
in $L^2(\R)$ (Step~1) and $\mathcal E_\lambda^\R(\widetilde G_n-\widetilde G_m)\to 0$ (Step~2),
so closedness of $\mathcal E_\lambda^\R$ (Prop.~\ref{prop:closedR}) gives
$\widetilde G\in\mathcal D(\mathcal E_\lambda^\R)$ and $\mathcal E_\lambda^\R(\widetilde G_n-\widetilde G)\to 0$.
Since $H_I$ is closed in $L^2(\R)$ and each $\widetilde G_n\in H_I$, the limit $\widetilde G\in H_I$,
whence $G\in\mathcal D(\mathcal E_\lambda)$ and $\mathcal E_\lambda(G_n-G)\to 0$.}

\Step{$\mathcal E_\lambda$ is densely defined.}
\Stepjust{$C_c^\infty(I)\subset\mathcal D(\mathcal E_\lambda)$: for $G\in C_c^\infty(I)$,
$\widetilde G\in C_c^\infty(\R)\subset\mathcal D(\mathcal E_\lambda^\R)$
(Proposition~\ref{prop:closedR}, Step~6), so $G\in\mathcal D(\mathcal E_\lambda)$.
Since $C_c^\infty(I)$ is dense in $L^2(I)$, the form is densely defined.}

\Step{Nonnegativity and symmetry follow from those of $\mathcal E_\lambda^\R$.}
\Stepjust{$\mathcal E_\lambda(G)=\mathcal E_\lambda^\R(\widetilde G)\ge 0$ (Prop.~\ref{prop:closedR}).}

\QEDstep
\end{structuredproof}

\subsubsection{A coercive lower bound for the symbol $\psi_\lambda$}

\begin{lemma}[A lower bound for $w(t)$]\label{lem:wlower}
Let $t_0:=\min(1,2L)$. There exists $c_0=c_0(L)>0$ such that for all $t\in(0,t_0]$,
\[
w(t)=\frac{e^{t/2}}{2\sinh t}\ge \frac{c_0}{t}.
\]
\end{lemma}

\begin{structuredproof}
\Step{$\sinh t\le te^t$ for all $t>0$.}
\Stepjust{$\sinh t=\sum_{n=0}^\infty\frac{t^{2n+1}}{(2n+1)!}\le t\sum_{n=0}^\infty\frac{t^{2n}}{(2n)!}
=t\cosh t\le te^t$.}

\Step{$w(t)\ge\frac{e^{-t/2}}{2t}$ for $t>0$.}
\Stepjust{$w(t)=\frac{e^{t/2}}{2\sinh t}\ge\frac{e^{t/2}}{2te^t}=\frac{e^{-t/2}}{2t}$, using Step~1.}

\Step{For $t\in(0,1]$: $e^{-t/2}\ge e^{-1/2}$.}
\Stepjust{$e^{-t/2}$ is decreasing; its minimum on $(0,1]$ is at $t=1$.}

\Step{Set $c_0:=e^{-1/2}/2$.  Then for $t\in(0,t_0]\subset(0,1]$: $w(t)\ge\frac{c_0}{t}$.}
\Stepjust{$w(t)\ge\frac{e^{-t/2}}{2t}\ge\frac{e^{-1/2}}{2t}=\frac{c_0}{t}$, combining Steps~2 and~3.}

\QEDstep
\end{structuredproof}

\begin{lemma}[Logarithmic growth of $\psi_\lambda$]\label{lem:psilog}
There exist constants $c_1,c_2>0$ and $\xi_0\ge 2$ (depending only on $L$) such that for all $|\xi|\ge \xi_0$,
\[
\psi_\lambda(\xi)\ge c_1\log|\xi|-c_2.
\]
In particular $\psi_\lambda(\xi)\to\infty$ as $|\xi|\to\infty$.
\end{lemma}

\begin{structuredproof}
\Step{Drop the nonneg.\ prime sum:
$\psi_\lambda(\xi)\ge 4\int_0^{t_0}w(t)\sin^2(\xi t/2)\,dt$.}
\Stepjust{The prime sum in~\eqref{eq:defpsi} is $\ge 0$, and we restrict the archimedean
integral from $[0,2L]$ to $[0,t_0]\subset[0,2L]$ (the integrand is nonneg.).}

\Step{Apply Lemma~\ref{lem:wlower}: $\psi_\lambda(\xi)\ge 4c_0\int_0^{t_0}\frac{1}{t}\sin^2(\xi t/2)\,dt$.}
\Stepjust{For $t\in(0,t_0]$, $w(t)\ge c_0/t$ by Lemma~\ref{lem:wlower}.}

\Step{For $|\xi|\ge\xi_0:=4\pi/t_0$, define intervals
$J_n:=\bigl[\frac{2\pi n+\pi/2}{|\xi|},\frac{2\pi n+3\pi/2}{|\xi|}\bigr]$ for $n\ge 0$.
Then $\sin^2(\xi t/2)\ge 1/2$ for $t\in J_n$.}
\Stepjust{Set $\theta=|\xi|t/2$.  For $t\in J_n$, $\theta\in[\pi n+\pi/4,\pi n+3\pi/4]$.
On this interval, $|\sin\theta|\ge\sin(\pi/4)=1/\sqrt{2}$, so $\sin^2\theta\ge 1/2$.
Since $\sin^2(\xi t/2)=\sin^2(|\xi|t/2)=\sin^2\theta$, the claim follows.
The $J_n$ are pairwise disjoint: the left endpoint of $J_{n+1}$ exceeds the right endpoint of $J_n$
by $(2\pi(n+1)+\pi/2-2\pi n-3\pi/2)/|\xi|=\pi/|\xi|>0$.}

\Step{Let $N$ be the largest integer with $J_{N-1}\subset(0,t_0]$.  Then $N\asymp|\xi|$.}
\Stepjust{The right endpoint of $J_{N-1}$ is $\frac{2\pi(N-1)+3\pi/2}{|\xi|}\le t_0$, giving
$N\le\frac{t_0|\xi|}{2\pi}+\frac{1}{4}$.  The condition $|\xi|\ge 4\pi/t_0$ ensures $N\ge 1$.
Thus $N$ is of order $|\xi|$ with constants depending on $t_0$.}

\Step{$\int_0^{t_0}\frac{1}{t}\sin^2(\xi t/2)\,dt\ge\frac{1}{2}\sum_{n=0}^{N-1}\log\frac{2\pi n+3\pi/2}{2\pi n+\pi/2}$.}
\Stepjust{Restrict the integral to $\bigcup_{n=0}^{N-1}J_n\subset(0,t_0]$.
On $J_n$, $\sin^2(\xi t/2)\ge 1/2$ (Step~3), so
$\int_{J_n}\frac{1}{t}\cdot\frac{1}{2}\,dt=\frac{1}{2}\log\frac{2\pi n+3\pi/2}{2\pi n+\pi/2}$.}

\Step{$\sum_{n=0}^{N-1}\log\frac{2\pi n+3\pi/2}{2\pi n+\pi/2}\ge c'\log N$ for an absolute constant $c'>0$.}

\Substep{$\log\frac{2\pi n+3\pi/2}{2\pi n+\pi/2}=\log\bigl(1+\frac{\pi}{2\pi n+\pi/2}\bigr)\ge\frac{c}{n+1}$
for some absolute $c>0$.}
\Substepjust{Using $\log(1+x)\ge x/(1+x)$: with $x=\pi/(2\pi n+\pi/2)$,
$\log(1+x)\ge\frac{\pi/(2\pi n+\pi/2)}{1+\pi/(2\pi n+\pi/2)}=\frac{\pi}{2\pi n+3\pi/2}\ge\frac{\pi}{2\pi(n+1)+3\pi/2}\ge\frac{c}{n+1}$.}

\Substep{$\sum_{n=0}^{N-1}\frac{1}{n+1}=H_N\ge\log N$ (where $H_N$ is the $N$-th harmonic number).}
\Substepjust{Standard lower bound for harmonic numbers: $H_N\ge\log N$ for $N\ge 1$.}

\Substep{Combine: the sum $\ge c\cdot\log N$.}
\Substepjust{Multiply Step~6.1 by $1$ and sum, then apply Step~6.2.}

\Step{Since $N\asymp|\xi|$ (Step~4): $\log N=\log|\xi|+O(1)$.}
\Stepjust{$N=\Theta(|\xi|)$ implies $\log N=\log|\xi|+\log(N/|\xi|)=\log|\xi|+O(1)$.}

\Step{Conclusion: $\psi_\lambda(\xi)\ge 4c_0\cdot\frac{1}{2}\cdot c'\cdot(\log|\xi|-C')=c_1\log|\xi|-c_2$.}
\Stepjust{Chain Steps~2, 5, 6, and 7: $\psi_\lambda(\xi)\ge 4c_0\cdot\frac{1}{2}\cdot c'\log N
= 2c_0c'\log N = 2c_0c'(\log|\xi|+O(1))$.  Set $c_1:=2c_0c'$ and absorb the $O(1)$ into~$c_2$.}

\QEDstep
\end{structuredproof}

\begin{corollary}[Energy controls a logarithmic frequency moment]\label{cor:tail}
There exist constants $a,b>0$ (depending only on $L$) such that for every $\phi\in\mathcal D(\mathcal E_\lambda^{\R})$,
\[
\int_\R \log(2+|\xi|)\,|\widehat\phi(\xi)|^2\,d\xi
\le a\,\|\phi\|_{L^2(\R)}^2 + b\,\int_\R \psi_\lambda(\xi)\,|\widehat\phi(\xi)|^2\,d\xi.
\]
\end{corollary}

\begin{structuredproof}
\Step{For all $\xi\in\R$: $\log(2+|\xi|)\le a'+b'\psi_\lambda(\xi)$ for suitable $a',b'>0$.}

\Substep{For $|\xi|\ge\xi_0$: $\psi_\lambda(\xi)\ge c_1\log|\xi|-c_2$ (Lemma~\ref{lem:psilog}).}
\Substepjust{Direct application of Lemma~\ref{lem:psilog}.}

\Substep{Hence $\log(2+|\xi|)\le\log|\xi|+\log 3\le\frac{1}{c_1}(\psi_\lambda(\xi)+c_2)+\log 3$ for $|\xi|\ge\xi_0$.}
\Substepjust{$\log(2+|\xi|)\le\log(3|\xi|)=\log 3+\log|\xi|$, and from Step~1.1:
$\log|\xi|\le(\psi_\lambda(\xi)+c_2)/c_1$.}

\Substep{For $|\xi|<\xi_0$: $\log(2+|\xi|)\le\log(2+\xi_0)$, a finite constant.}
\Substepjust{$\log(2+|\xi|)$ is bounded on bounded sets.}

\Substep{Combine: set $b':=1/c_1$ and $a':=c_2/c_1+\log 3+\log(2+\xi_0)$.
Then $\log(2+|\xi|)\le a'+b'\psi_\lambda(\xi)$ for all~$\xi$.}
\Substepjust{For $|\xi|\ge\xi_0$, use Step~1.2.  For $|\xi|<\xi_0$, $a'\ge\log(2+\xi_0)\ge\log(2+|\xi|)$
and $b'\psi_\lambda(\xi)\ge 0$.}

\Step{Multiply by $|\widehat\phi(\xi)|^2$ and integrate:
$\int\log(2+|\xi|)|\widehat\phi(\xi)|^2\,d\xi\le a'\int|\widehat\phi(\xi)|^2\,d\xi+b'\int\psi_\lambda(\xi)|\widehat\phi(\xi)|^2\,d\xi$.}
\Stepjust{Pointwise inequality from Step~1 times $|\widehat\phi(\xi)|^2\ge 0$, integrated.}

\Step{By Plancherel, $\frac{1}{2\pi}\int|\widehat\phi|^2\,d\xi=\|\phi\|_2^2$.  Set $a:=2\pi a'$, $b:=b'$.}
\Stepjust{Rewrite Step~2 in terms of $\|\phi\|_2^2$ and $\mathcal E_\lambda^\R(\phi)$ (via Lemma~\ref{lem:FourierRep}).}

\QEDstep
\end{structuredproof}

\subsubsection{Compact embedding and compact resolvent}

\begin{theorem}[Kolmogorov--Riesz compactness criterion in $L^2(\R)$]\label{thm:KR}
A set $\mathcal K\subset L^2(\R)$ is relatively compact if and only if:
\begin{enumerate}
\item[(i)] (\emph{tightness}) for every $\varepsilon>0$ there exists $R>0$ such that
$\int_{|u|>R}|\phi(u)|^2\,du<\varepsilon^2$ for all $\phi\in\mathcal K$;
\item[(ii)] (\emph{translation equicontinuity}) for every $\varepsilon>0$ there exists $\delta>0$ such that
$\|\phi-S_h\phi\|_{2}<\varepsilon$ for all $\phi\in\mathcal K$ and all $|h|<\delta$.
\end{enumerate}
\end{theorem}

\begin{remark}
See Brezis~\cite[Cor.~4.27]{BrezisFA} for a proof of Theorem~\ref{thm:KR}.
(Brezis Thm.\,4.26 covers the bounded-domain case; Cor.\,4.27 extends to full $L^2(\mathbb{R})$
and requires the tightness condition~(i), which in Proposition~\ref{prop:compactEmbed}
is supplied by compact support of all $\phi_n$ in $\overline{I}$.)
\end{remark}

\begin{lemma}[Uniform translation control from the form norm]\label{lem:trans}
Fix $M>0$ and define
$\mathcal K_M:=\{\phi\in H_I:\ \|\phi\|_2^2+\mathcal E_\lambda^{\R}(\phi)\le M\}$.
Then $\mathcal K_M$ satisfies condition (ii) in Theorem~\ref{thm:KR}.
\end{lemma}

\begin{structuredproof}
\Step{For $\phi\in\mathcal K_M$ and $h\in\R$:
$\|\phi-S_h\phi\|_2^2=\frac{1}{2\pi}\int_\R 4\sin^2(\xi h/2)|\widehat\phi(\xi)|^2\,d\xi$.}
\Stepjust{Lemma~\ref{lem:plancherel-diff}.}

\Step{Split the integral at a parameter $R\ge 1$ into low and high frequencies.}

\Substep{Low-frequency bound ($|\xi|\le R$):
$\int_{|\xi|\le R}4\sin^2(\xi h/2)|\widehat\phi(\xi)|^2\,d\xi\le(Rh)^2\cdot 2\pi\|\phi\|_2^2$.}
\Substepjust{$\sin^2(x)\le x^2$ gives $4\sin^2(\xi h/2)\le(\xi h)^2\le(Rh)^2$ for $|\xi|\le R$.
Then $\int_{|\xi|\le R}(Rh)^2|\widehat\phi|^2\le(Rh)^2\int_\R|\widehat\phi|^2=(Rh)^2\cdot 2\pi\|\phi\|_2^2$.}

\Substep{High-frequency bound ($|\xi|>R$):
$\int_{|\xi|>R}4\sin^2(\xi h/2)|\widehat\phi(\xi)|^2\,d\xi\le\frac{4}{\log(2+R)}\int_\R\log(2+|\xi|)|\widehat\phi(\xi)|^2\,d\xi$.}
\Substepjust{$\sin^2\le 1$ gives the left side $\le 4\int_{|\xi|>R}|\widehat\phi|^2\,d\xi$.
For $|\xi|>R$, $1\le\frac{\log(2+|\xi|)}{\log(2+R)}$, so
$\int_{|\xi|>R}|\widehat\phi|^2\le\frac{1}{\log(2+R)}\int_{|\xi|>R}\log(2+|\xi|)|\widehat\phi|^2
\le\frac{1}{\log(2+R)}\int_\R\log(2+|\xi|)|\widehat\phi|^2$.}

\Step{By Corollary~\ref{cor:tail}, $\int_\R\log(2+|\xi|)|\widehat\phi|^2\le C(M,L)$ uniformly
over $\phi\in\mathcal K_M$.}
\Stepjust{$\|\phi\|_2^2+\mathcal E_\lambda^\R(\phi)\le M$ by definition of $\mathcal K_M$.
Corollary~\ref{cor:tail} bounds $\int\log(2+|\xi|)|\widehat\phi|^2$ by
$a\|\phi\|_2^2+b\int\psi_\lambda|\widehat\phi|^2\,d\xi
=a\|\phi\|_2^2+2\pi b\,\mathcal E_\lambda^\R(\phi)\le(a+2\pi b)M=:C(M,L)$.}

\Step{Combine: $\|\phi-S_h\phi\|_2^2\le(Rh)^2M+\frac{C'(M,L)}{\log(2+R)}$.}
\Stepjust{Add the bounds from Steps~2.1 and 2.2, using Step~3 to bound the high-frequency term.
Here $C'(M,L)=4C(M,L)/(2\pi)$ (absorbing constants).}

\Step{Given $\varepsilon>0$, choose $R$ then $\delta$ to make each term $\le\varepsilon^2/2$.}

\Substep{Choose $R$ so that $\frac{C'(M,L)}{\log(2+R)}\le\varepsilon^2/2$.}
\Substepjust{$\log(2+R)\to\infty$ as $R\to\infty$, so such $R$ exists.}

\Substep{Then choose $\delta>0$ so that $(R\delta)^2M\le\varepsilon^2/2$.}
\Substepjust{Take $\delta:=\varepsilon/(R\sqrt{2M})$ (assuming $M>0$; if $M=0$ then $\mathcal K_M=\{0\}$
and the result is trivial).}

\Substep{For $|h|<\delta$: $\|\phi-S_h\phi\|_2^2\le\varepsilon^2/2+\varepsilon^2/2=\varepsilon^2$.}
\Substepjust{Substitute into Step~4.}

\QEDstep
\end{structuredproof}

\begin{proposition}[Compact embedding of the form domain]\label{prop:compactEmbed}
The embedding $(\mathcal D(\mathcal E_\lambda),\|\cdot\|_{\mathcal D})\hookrightarrow L^2(I)$ is compact.
\end{proposition}

\begin{structuredproof}
\Step{Let $\{G_n\}\subset\mathcal D(\mathcal E_\lambda)$ with $\|G_n\|_2^2+\mathcal E_\lambda(G_n)\le M$.
Set $\phi_n:=\widetilde G_n\in H_I$.}
\Stepjust{Setup.  Then $\|\phi_n\|_2^2+\mathcal E_\lambda^\R(\phi_n)=\|G_n\|_2^2+\mathcal E_\lambda(G_n)\le M$,
so $\phi_n\in\mathcal K_M$.}

\Step{$\{\phi_n\}$ satisfies tightness condition~(i) in Theorem~\ref{thm:KR}.}
\Stepjust{Each $\phi_n$ is supported in $\overline I=[-L,L]$.  For any $R>L$,
$\int_{|u|>R}|\phi_n(u)|^2\,du=0<\varepsilon^2$.}

\Step{$\{\phi_n\}$ satisfies translation equicontinuity (ii) in Theorem~\ref{thm:KR}.}
\Stepjust{Lemma~\ref{lem:trans}: $\mathcal K_M$ satisfies~(ii).}

\Step{$\{\phi_n\}$ is relatively compact in $L^2(\R)$.}
\Stepjust{Theorem~\ref{thm:KR} applied to $\mathcal K:=\{\phi_n:n\ge 1\}$, using Steps~2 and 3.
(The required $L^2$-boundedness holds: $\|\phi_n\|_2^2\le M$ by Step~1.)}

\Step{$\{G_n\}$ is relatively compact in $L^2(I)$.}
\Stepjust{Since $H_I$ is closed in $L^2(\R)$, every subsequential $L^2(\R)$-limit of $\{\phi_n\}$
lies in $H_I$.  The map $\phi\mapsto\phi|_I$ is a continuous surjection (indeed, isometry)
$H_I\to L^2(I)$.  Continuous images of relatively compact sets are relatively compact.
So from Step~4, $\{G_n=\phi_n|_I\}$ is relatively compact in $L^2(I)$.}

\QEDstep
\end{structuredproof}

\begin{theorem}[Closed form, associated operator, and compact resolvent]\label{thm:operator}
There exists a unique selfadjoint operator $A_\lambda\ge 0$ on $L^2(I)$ associated to the closed form
$\mathcal E_\lambda$ (Proposition~\ref{prop:closedI}) in the sense of the representation theorem for closed forms.
Moreover, $A_\lambda$ has compact resolvent.
\end{theorem}

\begin{structuredproof}
\Step{There exists a unique selfadjoint operator $A_\lambda\ge 0$ with
$\mathcal D(\mathcal E_\lambda)=\mathcal D(A_\lambda^{1/2})$ and
$\mathcal E_\lambda(G)=\|A_\lambda^{1/2}G\|_2^2$.}
\Stepjust{By the First and Second Representation Theorems for densely defined, closed,
lower-bounded symmetric forms
(Kato~\cite[Thm.~VI.2.1, Thm.~VI.2.23]{KatoPerturbation}; in the Dirichlet-form setting,
Fukushima--Oshima--Takeda~\cite[Thm.~1.3.1]{FOT}).
Kato Thm.\,VI.2.1 gives existence of the unique associated selfadjoint operator;
Kato Thm.\,VI.2.23 (Second Representation Theorem) establishes
$\mathcal{D}(\mathcal{E}_\lambda)=\mathcal{D}(A_\lambda^{1/2})$ and the identity
$\mathcal{E}_\lambda(G)=\|A_\lambda^{1/2}G\|_2^2$.
Proposition~\ref{prop:closedI} verified that $\mathcal E_\lambda$ satisfies all hypotheses.}

\Step{$(A_\lambda+1)^{-1}$ is compact on $L^2(I)$.}

\Substep{Let $\{f_n\}$ be bounded in $L^2(I)$: $\|f_n\|_2\le C$.  Set $u_n:=(A_\lambda+1)^{-1}f_n$.}
\Substepjust{$A_\lambda+1$ is invertible because $A_\lambda\ge 0$, so $A_\lambda+1\ge 1>0$.}

\Substep{$u_n\in\mathcal D(A_\lambda)\subset\mathcal D(\mathcal E_\lambda)$ and $(A_\lambda+1)u_n=f_n$.}
\Substepjust{Definition of the resolvent.}

\Substep{$\mathcal E_\lambda(u_n)+\|u_n\|_2^2=\langle f_n,u_n\rangle$.}
\Substepjust{Take the $L^2$ inner product of $(A_\lambda+1)u_n=f_n$ with $u_n$:
$\langle A_\lambda u_n,u_n\rangle+\|u_n\|_2^2=\langle f_n,u_n\rangle$.
By the form identity, $\langle A_\lambda u_n,u_n\rangle=\mathcal E_\lambda(u_n)$.}

\Substep{$\|u_n\|_2^2+\mathcal E_\lambda(u_n)\le\|f_n\|_2^2$.}
\Substepjust{From Step~2.3: $\mathcal E_\lambda(u_n)+\|u_n\|_2^2=\langle f_n,u_n\rangle\le\|f_n\|_2\|u_n\|_2$
by Cauchy--Schwarz.  Since $\|u_n\|_2^2\le\mathcal E_\lambda(u_n)+\|u_n\|_2^2\le\|f_n\|_2\|u_n\|_2$,
we get $\|u_n\|_2\le\|f_n\|_2$ (if $u_n=0$ the bound is trivial), hence
$\mathcal E_\lambda(u_n)+\|u_n\|_2^2\le\|f_n\|_2\|u_n\|_2\le\|f_n\|_2^2$.}

\Substep{$\{u_n\}$ has a convergent subsequence in $L^2(I)$.}
\Substepjust{Step~2.4 shows $\{u_n\}$ is bounded in the form norm
(with $M:=C^2$), so Proposition~\ref{prop:compactEmbed} gives
relative compactness in $L^2(I)$.}

\Substep{$(A_\lambda+1)^{-1}$ maps bounded sequences to sequences with convergent subsequences,
hence is compact.}
\Substepjust{This is the definition of a compact operator: it maps bounded sets to relatively compact sets.
Steps~2.1--2.5 verify this.}

\QEDstep
\end{structuredproof}

%% =================================================================
\subsection{Semigroup and irreducibility}
%% =================================================================

\begin{definition}[Irreducibility for semigroups on $L^2(I)$]
A closed ideal in $L^2(I)$ has the form $L^2(B)$ for some measurable $B\subset I$.
We call $T$ \emph{irreducible} if the only invariant closed ideals are $\{0\}$ and $L^2(I)$.
\end{definition}

\begin{lemma}[Invariant ideals and splitting of the form]\label{lem:invariance_split}
Assume Theorem~\ref{thm:operator}. Let $B\subset I$ be measurable and suppose
$L^2(B)$ is invariant under $T(t)=e^{-tA_\lambda}$.
Then for every $G\in\mathcal D(\mathcal E_\lambda)$:
$\1_B G,\1_{I\setminus B}G\in\mathcal D(\mathcal E_\lambda)$ and
$\mathcal E_\lambda(G)=\mathcal E_\lambda(\1_B G)+\mathcal E_\lambda(\1_{I\setminus B}G)$.
\end{lemma}

\begin{structuredproof}
\Step{Let $P=M_{\1_B}$ (multiplication by $\1_B$) and $Q=I-P$.  Then $P$ is an orthogonal projection
with $\mathrm{Ran}(P)=L^2(B)$.}
\Stepjust{$P^2=P$ ($\1_B^2=\1_B$) and $P^*=P$ ($\1_B$ is real), so $P$ is an orthogonal projection.
$Pf=\1_Bf$ vanishes outside $B$, so $\mathrm{Ran}(P)=L^2(B)$.}

\Step{$P$ commutes with $T(t)$ for all $t\ge 0$.}

\Substep{Invariance of $L^2(B)=\mathrm{Ran}(P)$ means $T(t)(\mathrm{Ran}(P))\subset\mathrm{Ran}(P)$,
i.e.\ $PT(t)P=T(t)P$.}
\Substepjust{$T(t)Pf\in L^2(B)$ for all $f$, so $PT(t)Pf=T(t)Pf$.}

\Substep{Take adjoints: $(PT(t)P)^*=(T(t)P)^*$, giving $PT(t)P=PT(t)$.}
\Substepjust{$P^*=P$ and $T(t)^*=T(t)$ (selfadjointness of $A_\lambda$ implies selfadjointness of $e^{-tA_\lambda}$).
So $(PT(t)P)^*=P^*T(t)^*P^*=PT(t)P$ and $(T(t)P)^*=P^*T(t)^*=PT(t)$.
From Step~2.1: $PT(t)P=T(t)P$; equating with the adjoint computation: $PT(t)P=PT(t)$.
Together: $T(t)P=PT(t)$.}

\Step{$P$ commutes with $(A_\lambda+\alpha)^{-1}$ for every $\alpha>0$.}
\Stepjust{By the Laplace-transform formula for $C_0$-semigroups
(Engel--Nagel~\cite[Cor.\,II.1.11]{EngelNagel}):
\[
(A_\lambda+\alpha)^{-1}=\int_0^\infty e^{-\alpha t}\,e^{-tA_\lambda}\,dt
\]
holds as a Bochner integral in $\mathcal B(L^2(I))$
(convergence: $\|e^{-\alpha t}T(t)\|\le e^{-\alpha t}$, integrable for $\alpha>0$;
$\alpha$ lies in the resolvent set of $-A_\lambda$ since $\sigma(A_\lambda)\subset[0,\infty)$).
By Step~2, $P$ commutes with $e^{-tA_\lambda}$ for every $t\ge 0$, hence commutes with the Bochner integral.}

\Step{$P$ commutes with $A_\lambda^{1/2}$.}
\Stepjust{Let $R:=(A_\lambda+1)^{-1}$, which is bounded and selfadjoint.
Step~3 (with $\alpha=1$) gives $PR=RP$.
Since $PR=RP$, $P$ commutes with every bounded Borel function of $R$
(Reed--Simon~\cite[Cor.\ to Thm.\,VIII.5]{ReedSimonI}:
$PR^n=R^nP$ by induction; extend to polynomials by linearity,
to $C(\sigma(R))$ by Weierstrass, and to bounded Borel functions via
strong-operator-topology limits).
In particular, $P$ commutes with the spectral projections of $R$, hence with $E_R$.
Since $R=\varphi(A_\lambda)$ with $\varphi(\mu)=(\mu+1)^{-1}$, a continuous strictly
decreasing bijection $[0,\infty)\to(0,1]$, the spectral measures satisfy
$E_{A_\lambda}(\Delta)=E_R(\varphi(\Delta))$ for Borel $\Delta\subset[0,\infty)$.
Thus $PE_{A_\lambda}(\Delta)=E_{A_\lambda}(\Delta)P$ for all Borel $\Delta$.
\emph{Domain preservation:}
For $u\in\mathcal D(A_\lambda^{1/2})$,
$\int_0^\infty\mu\,d\|E_{A_\lambda}(\mu)Pu\|^2
=\int_0^\infty\mu\,d\|PE_{A_\lambda}(\mu)u\|^2
\le\int_0^\infty\mu\,d\|E_{A_\lambda}(\mu)u\|^2
=\|A_\lambda^{1/2}u\|^2<\infty$,
so $Pu\in\mathcal D(A_\lambda^{1/2})$.
\emph{Commutativity on the domain:}
$A_\lambda^{1/2}Pu=\int_0^\infty\mu^{1/2}\,dE_{A_\lambda}(\mu)Pu
=\int_0^\infty\mu^{1/2}\,PE_{A_\lambda}(\mu)u\,d\mu
=P\int_0^\infty\mu^{1/2}\,dE_{A_\lambda}(\mu)u=PA_\lambda^{1/2}u$
(interchange of $P$ with the spectral integral is justified since $P$ is bounded
and commutes with each $E_{A_\lambda}(\Delta)$).}

\Step{$P(\mathcal D(\mathcal E_\lambda))\subset\mathcal D(\mathcal E_\lambda)$ and
$Q(\mathcal D(\mathcal E_\lambda))\subset\mathcal D(\mathcal E_\lambda)$.}
\Stepjust{$\mathcal D(\mathcal E_\lambda)=\mathcal D(A_\lambda^{1/2})$.
If $u\in\mathcal D(A_\lambda^{1/2})$, then $A_\lambda^{1/2}Pu=PA_\lambda^{1/2}u\in L^2(I)$ (Step~4),
so $Pu\in\mathcal D(A_\lambda^{1/2})$.  Similarly for $Qu=(I-P)u$.}

\Step{$\mathcal E_\lambda(G)=\mathcal E_\lambda(PG)+\mathcal E_\lambda(QG)$ for $G\in\mathcal D(\mathcal E_\lambda)$.}

\Substep{$\mathcal E_\lambda(G)=\|A_\lambda^{1/2}G\|_2^2$.}
\Substepjust{Form identity from the representation theorem (Step~1 of Theorem~\ref{thm:operator}).}

\Substep{$A_\lambda^{1/2}G=A_\lambda^{1/2}PG+A_\lambda^{1/2}QG=PA_\lambda^{1/2}G+QA_\lambda^{1/2}G$.}
\Substepjust{$G=PG+QG$ and $A_\lambda^{1/2}$ commutes with $P$ and $Q$ (Step~4).}

\Substep{$PA_\lambda^{1/2}G$ and $QA_\lambda^{1/2}G$ are orthogonal in $L^2(I)$.}
\Substepjust{$\langle Pv,Qv\rangle=\langle Pv,(I-P)v\rangle=\langle Pv,v\rangle-\langle Pv,Pv\rangle
=\langle P^2v,v\rangle-\|Pv\|^2=\|Pv\|^2-\|Pv\|^2=0$.  Apply with $v=A_\lambda^{1/2}G$.}

\Substep{$\|A_\lambda^{1/2}G\|^2=\|PA_\lambda^{1/2}G\|^2+\|QA_\lambda^{1/2}G\|^2
=\|A_\lambda^{1/2}PG\|^2+\|A_\lambda^{1/2}QG\|^2
=\mathcal E_\lambda(PG)+\mathcal E_\lambda(QG)$.}
\Substepjust{Pythagorean theorem (Step~6.3), then $PA_\lambda^{1/2}G=A_\lambda^{1/2}PG$ (Step~4),
then the form identity (Step~6.1) applied to $PG$ and $QG$ (which lie in $\mathcal D(\mathcal E_\lambda)$
by Step~5).}

\QEDstep
\end{structuredproof}

\begin{proposition}[Triviality of invariant ideals for $\mathcal E_\lambda$]\label{prop:irreducible}
Assume Theorem~\ref{thm:operator}. Let $B\subset I$ be measurable and assume
$L^2(B)$ is invariant under $T(t)=e^{-tA_\lambda}$.
Then $m(B)=0$ or $m(I\setminus B)=0$.
\end{proposition}

\begin{structuredproof}
\Step{$1\in\mathcal D(\mathcal E_\lambda)$ (the constant function $G\equiv 1$ on $I$).}

\Substep{For each shift $s>0$: $\|\widetilde 1-S_s\widetilde 1\|_2^2=m(I\,\Delta\,(I+s))$.}
\Substepjust{$\widetilde 1=\1_{(-L,L)}$ and $S_s\widetilde 1=\1_{(-L+s,L+s)}$.
$\|\widetilde 1-S_s\widetilde 1\|_2^2=m(I\,\Delta\,(I+s))$, the measure of the symmetric difference.}

\Substep{For $0<s<2L$: $m(I\,\Delta\,(I+s))=2s$.}
\Substepjust{$I\setminus(I+s)=(-L,-L+s)$ has measure $s$; $(I+s)\setminus I=(L,L+s)$ has measure $s$.}

\Substep{The archimedean integral $\int_0^{2L}w(t)\cdot 2t\,dt<\infty$.}
\Substepjust{Since $\sinh t\ge t$: $w(t)\le e^{t/2}/(2t)$, so $w(t)\cdot 2t\le e^{t/2}\le e^L$
on $[0,2L]$.  The integrand is bounded on a compact interval, hence integrable.}

\Substep{The prime sum in $\mathcal E_\lambda(1)$ is finite.}
\Substepjust{Finitely many shift sizes $m\log p$ with $p^m\le\lambda^2$, each giving
$\|\widetilde 1-S_{m\log p}\widetilde 1\|_2^2=2m\log p<\infty$.}

\Substep{$\mathcal E_\lambda(1)<\infty$, hence $1\in\mathcal D(\mathcal E_\lambda)$.}
\Substepjust{Combine Steps~1.3 and~1.4.}

\Step{By Lemma~\ref{lem:invariance_split}, for every $G\in\mathcal D(\mathcal E_\lambda)$:
\begin{equation}\label{eq:Einvariance}
\mathcal E_\lambda(G)=\mathcal E_\lambda(\1_BG)+\mathcal E_\lambda(\1_{I\setminus B}G).
\end{equation}}
\Stepjust{Direct application of Lemma~\ref{lem:invariance_split}.}

\Step{Apply~\eqref{eq:Einvariance} with $G\equiv 1$ (justified by Step~1).
Set $f:=\widetilde{\1_B}$, $g:=\widetilde{\1_{B^c}}$ where $B^c:=I\setminus B$.
Then $\widetilde 1=f+g$ and $fg=0$ a.e.}
\Stepjust{$\1_B+\1_{B^c}=1$ on $I$ and $\1_B\cdot\1_{B^c}=0$.
Zero-extending: $f+g=\widetilde 1$ and $fg=0$ a.e.\ on $\R$.}

\Step{For each shift $s>0$:
$\|(f+g)-S_s(f+g)\|_2^2=\|f-S_sf\|_2^2+\|g-S_sg\|_2^2+2\langle f-S_sf,g-S_sg\rangle$.}
\Stepjust{Expand $\|a+b\|^2=\|a\|^2+\|b\|^2+2\Re\langle a,b\rangle$
with $a=f-S_sf$ and $b=g-S_sg$.  Since $f,g$ are real-valued, $\langle a,b\rangle\in\R$.}

\Step{Substituting Step~4 into~\eqref{eq:Einvariance} (which, via Definition~\ref{def:E}, is
an identity between weighted sums/integrals of $\|\cdot-S_s\cdot\|_2^2$ terms):
the cross-terms sum to zero.}

\Substep{$\mathcal E_\lambda(1)=\mathcal E_\lambda(\1_B)+\mathcal E_\lambda(\1_{B^c})$ (from Step~2 with $G=1$).}
\Substepjust{Equation~\eqref{eq:Einvariance} with $G\equiv 1$.}

\Substep{Expand each $\mathcal E_\lambda$ using Definition~\ref{def:E} and Step~4:
each shift-size gives the identity from Step~4.  After subtracting $\mathcal E_\lambda(\1_B)+\mathcal E_\lambda(\1_{B^c})$
from $\mathcal E_\lambda(1)$, the remainder is
\[
\textstyle
\int_0^{2L}w(t)\cdot 2\langle f-S_tf,g-S_tg\rangle\,dt
+\sum_{p,m}(\log p)p^{-m/2}\cdot 2\langle f-S_{m\log p}f,g-S_{m\log p}g\rangle=0.
\]}
\Substepjust{Combine Step~4 for each shift size with Step~5.1.  The ``diagonal'' terms cancel, leaving the cross terms.}

\Step{For any $s>0$: $\langle f-S_sf,g-S_sg\rangle\le 0$.}

\Substep{$\langle f-S_sf,g-S_sg\rangle=\langle f,g\rangle-\langle f,S_sg\rangle-\langle S_sf,g\rangle+\langle S_sf,S_sg\rangle$.}
\Substepjust{Bilinearity of the inner product.}

\Substep{$\langle f,g\rangle=0$.}
\Substepjust{$fg=0$ a.e.\ (Step~3), so $\int_\R f(u)g(u)\,du=0$.}

\Substep{$\langle S_sf,S_sg\rangle=\langle f,g\rangle=0$.}
\Substepjust{$S_s$ is unitary: $\langle S_sf,S_sg\rangle=\langle f,g\rangle$.  Then Step~6.2.}

\Substep{$\langle f,S_sg\rangle\ge 0$ and $\langle S_sf,g\rangle=\langle g,S_sf\rangle\ge 0$.}
\Substepjust{$f=\widetilde{\1_B}\ge 0$ and $S_sg=\widetilde{\1_{B^c}}(\cdot-s)\ge 0$, so
$\langle f,S_sg\rangle=\int f\cdot S_sg\ge 0$.  Similarly for $\langle g,S_sf\rangle$.}

\Substep{Combine: $\langle f-S_sf,g-S_sg\rangle=0-\langle f,S_sg\rangle-\langle g,S_sf\rangle+0\le 0$.}
\Substepjust{Steps~6.1--6.4.}

\Step{$\langle f,S_tg\rangle=\langle g,S_tf\rangle=0$ for a.e.\ $t\in(0,2L)$.}
\Stepjust{From Step~5.2, the weighted sum of the cross terms is $0$.
All weights are $\ge 0$, and $w(t)>0$ for $t>0$.
Each cross term $\langle f-S_tf,g-S_tg\rangle\le 0$ (Step~6).
A sum of nonpositive terms with positive weights equaling zero forces each term to be zero a.e.
Hence $\langle f-S_tf,g-S_tg\rangle=0$ for a.e.\ $t\in(0,2L)$.
From Step~6.5: this means $\langle f,S_tg\rangle+\langle g,S_tf\rangle=0$ a.e.
Since both are $\ge 0$ (Step~6.4), each is $0$.}

\Step{Upgrade to \emph{all} $t\in(0,2L)$: $\langle f,S_tg\rangle=0$ for every $t\in(0,2L)$.}
\Stepjust{$t\mapsto\langle f,S_tg\rangle$ is continuous (strong continuity of $S_t$ on $L^2(\R)$).
A continuous nonneg.\ function vanishing a.e.\ on $(0,2L)$ vanishes everywhere on $(0,2L)$.
(Same argument as in Lemma~\ref{lem:indicator-energy}, Step~3.)}

\Step{For every $t\in(0,2L)$: $\1_B(u)=\1_B(u-t)$ for a.e.\ $u\in I\cap(I+t)$.}

\Substep{$0=\langle f,S_tg\rangle=\int_{I\cap(I+t)}\1_B(u)\cdot\1_{B^c}(u-t)\,du$.}
\Substepjust{Unwinding definitions: $f=\widetilde{\1_B}$, $S_tg(u)=\widetilde{\1_{B^c}}(u-t)$.
The integrand is nonzero only where both $u\in I$ (so $f(u)=\1_B(u)$) and
$u-t\in I$ (so $g(u-t)=\1_{B^c}(u-t)$), i.e.\ $u\in I\cap(I+t)$.}

\Substep{Since $\1_B(u)\cdot\1_{B^c}(u-t)\ge 0$ and the integral is $0$:
$\1_B(u)\cdot\1_{B^c}(u-t)=0$ for a.e.\ $u\in I\cap(I+t)$.}
\Substepjust{A nonneg.\ integrable function with zero integral vanishes a.e.}

\Substep{Hence $\1_B(u)\le\1_B(u-t)$ for a.e.\ $u\in I\cap(I+t)$.}
\Substepjust{Step~9.2 says: wherever $\1_B(u)=1$, we must have $\1_{B^c}(u-t)=0$, i.e.\ $\1_B(u-t)=1$.}

\Substep{Similarly, $\langle g,S_tf\rangle=0$ gives $\1_B(u-t)\le\1_B(u)$ a.e.\ on $I\cap(I+t)$.}
\Substepjust{$\langle g,S_tf\rangle=\int_{I\cap(I+t)}\1_{B^c}(u)\1_B(u-t)\,du=0$ (Step~8).
Same argument as Steps~9.2--9.3 with $B$ and $B^c$ swapped.}

\Substep{Combine: $\1_B(u)=\1_B(u-t)$ a.e.\ on $I\cap(I+t)$.}
\Substepjust{Steps~9.3 and 9.4.}

\Step{$m(B)=0$ or $m(I\setminus B)=0$.}
\Stepjust{Step~9.5 holds for every $t\in(0,2L)$.  This provides the hypothesis of
Lemma~\ref{lem:trans-inv} with $\varepsilon=2L$.  The conclusion follows.}

\QEDstep
\end{structuredproof}

\begin{remark}[Why we do not use $\mathcal E_\lambda(\1_B)=0$]\label{rem:killing}
Because $\mathcal E_\lambda$ is defined using zero-extension to $\R$ (Definition~\ref{def:E}), the form is
typically non-conservative: in general $\mathcal E_\lambda(1)>0$.
In the conservative case ($\mathcal E(1)=0$) one often has an equivalence between invariance and the condition
$\mathcal E(\1_B)=0$.  Here, the presence of killing means $\mathcal E_\lambda(\1_B)=0$ is a \emph{stronger}
condition than invariance, so we instead argue directly from the correct invariance identity
\eqref{eq:Einvariance}, which depends only on the interaction/jump part.
\end{remark}

\begin{corollary}[Irreducibility for $\mathcal E_\lambda$]\label{cor:irreducible}
Assume Theorem~\ref{thm:operator}. Then $T(t)=e^{-tA_\lambda}$ is irreducible.
\end{corollary}

\begin{structuredproof}
\Step{Let $J\subset L^2(I)$ be a closed $T(t)$-invariant ideal.  Then $J=L^2(B)$ for some
measurable $B\subset I$.}
\Stepjust{Standard lattice-theory fact: every closed ideal in $L^2(I)$ has the form $L^2(B)$
for some measurable $B\subset I$
(Schaefer~\cite[Sect.~III.1, Ex.~1]{SchaeferBanachLattices}).}

\Step{$m(B)=0$ or $m(I\setminus B)=0$.}
\Stepjust{Proposition~\ref{prop:irreducible}.}

\Step{$J=\{0\}$ or $J=L^2(I)$.}
\Stepjust{If $m(B)=0$, then $L^2(B)=\{0\}$.  If $m(I\setminus B)=0$, then $L^2(B)=L^2(I)$.}

\QEDstep
\end{structuredproof}

%% =================================================================
\section{Positivity improving and the ground state}
\label{sec:PF}
%% =================================================================

\subsection{External theorems used}

\begin{theorem}[Positivity improving from positivity + irreducibility + holomorphy]
\label{thm:ABHN}
Let $E$ be a Banach lattice and $S$ a positive, irreducible, holomorphic $C_0$-semigroup on $E$.
Then $S$ is positivity improving: for each $t>0$ and each $0\le f\in E$ with $f\ne 0$,
one has $S(t)f>0$ (in the lattice sense; on $L^2$ this means $>0$ a.e.).
\end{theorem}

\begin{remark}[Source]
This is stated (for general Banach lattices) as Theorem~2.3 in Arendt--ter~Elst--Gl{\"u}ck~\cite{ATGStrictPos2020}.
\end{remark}

\begin{theorem}[Simplicity of the principal eigenvalue under compact resolvent]
\label{thm:principal-simple}
Let $A$ be selfadjoint and lower bounded on $L^2(I)$ with compact resolvent, and let $S(t)=e^{-tA}$.
If $S$ is positivity improving, then all four conclusions of
Arendt--ter~Elst--Gl\"{u}ck~\cite[Prop.~2.4]{ATGStrictPos2020} hold:
\begin{enumerate}
\item[(a)] $\sigma(A)\ne\emptyset$ (automatic for a selfadjoint operator on a nonzero Hilbert space);
\item[(b)] $\lambda_1:=\inf\{\operatorname{Re}\lambda:\lambda\in\sigma(A)\}$ is an eigenvalue
  (so the infimum is a minimum);
\item[(c)] the associated eigenfunction satisfies $\psi>0$ a.e.\ (a strictly positive quasi-interior
  point in $L^2(I)$);
\item[(d)] $\lambda_1$ has algebraic multiplicity one (simple eigenvalue).
\end{enumerate}
In summary: $\min\sigma(A)$ is a simple eigenvalue admitting an eigenfunction strictly positive a.e.
\end{theorem}

\begin{remark}[Source]
Proposition~2.4 in the same paper of Arendt et al.,
a Perron--Frobenius/Krein--Rutman/Jentzsch consequence for compact positive operators.
\end{remark}

\subsection{Application to $A_\lambda$}

\begin{proposition}[Positivity improving and simple ground state for $A_\lambda$]
\label{prop:groundstate}
Assume Theorem~\ref{thm:operator}. Then:
\begin{enumerate}
\item The semigroup $T(t)=e^{-tA_\lambda}$ is positivity preserving (Markovian).
\item $T(t)$ is irreducible.
\item $T(t)$ is holomorphic.
\end{enumerate}
Consequently $T(t)$ is positivity improving, and the lowest eigenvalue of $A_\lambda$ is simple with a
strictly positive a.e.\ eigenfunction.
\end{proposition}

\begin{structuredproof}
\Step{$T(t)$ is positivity preserving (Markovian).}
\Stepjust{Lemma~\ref{lem:markov} shows that $\mathcal E_\lambda$ satisfies the Markov
(normal contraction) property.  By the general correspondence between Dirichlet forms
and positivity-preserving semigroups (Fukushima--Oshima--Takeda~\cite[Thm.~1.4.1]{FOT};
Ouhabaz~\cite[Thm.~1.4.1]{OuhabazHeatEq}), the associated semigroup $T(t)$
is positivity preserving.}

\Step{$T(t)$ is irreducible.}
\Stepjust{Corollary~\ref{cor:irreducible}.}

\Step{$T(t)$ is holomorphic.}
\Stepjust{$T(t)$ is strongly continuous ($C_0$): by the spectral theorem,
$\|T(t)f-f\|_2^2=\int_0^\infty|e^{-t\mu}-1|^2\,d\|E_{A_\lambda}(\mu)f\|^2\to 0$
as $t\to 0^+$ by dominated convergence.
Moreover, $A_\lambda$ is selfadjoint and lower bounded ($A_\lambda\ge 0$), hence m-sectorial
with numerical range in $[0,\infty)$.
By Kato~\cite[Ex.~IX.1.25]{KatoPerturbation} (backed by
Thm.\,IX.1.24), $e^{-zA_\lambda}$ is bounded and holomorphic on
$\{z\in\C:\Re z>0\}$ with $\|e^{-zA_\lambda}\|\le 1$.
In particular, $T(t)=e^{-tA_\lambda}$ extends to a
holomorphic semigroup.}

\Step{$T(t)$ is positivity improving.}
\Stepjust{Apply Theorem~\ref{thm:ABHN}: $T(t)$ is positive (Step~1), irreducible (Step~2),
and holomorphic (Step~3).  Therefore it is positivity improving.}

\Step{The lowest eigenvalue of $A_\lambda$ is simple, with a strictly positive a.e.\ eigenfunction.}
\Stepjust{Apply Theorem~\ref{thm:principal-simple}: $A_\lambda$ is selfadjoint, lower bounded,
has compact resolvent (Theorem~\ref{thm:operator}), and $T(t)=e^{-tA_\lambda}$ is positivity
improving (Step~4).
All four conclusions of Theorem~\ref{thm:principal-simple}
(ATG~\cite[Prop.~2.4]{ATGStrictPos2020}) hold:
(a)~$\sigma(A_\lambda)\ne\emptyset$;
(b)~$\mu_0:=\min\sigma(A_\lambda)$ exists and is an eigenvalue
(we write $\mu_0$ instead of the $\lambda_1$ of Theorem~\ref{thm:principal-simple}
to avoid confusion with the form parameter~$\lambda$);
(c)~the corresponding eigenfunction is strictly positive a.e.;
(d)~$\mu_0$ is a simple eigenvalue (multiplicity~one).}

\QEDstep
\end{structuredproof}

%% =================================================================
\section{Evenness of the ground state from inversion symmetry}
%% =================================================================

\begin{proposition}[Inversion (reflection) symmetry]\label{prop:reflection}
Let $R:L^2(I)\to L^2(I)$ be the unitary involution $(RG)(u):=G(-u)$.
Then $R(\mathcal D(\mathcal E_\lambda))=\mathcal D(\mathcal E_\lambda)$ and
$\mathcal E_\lambda(RG)=\mathcal E_\lambda(G)$ for all $G\in\mathcal D(\mathcal E_\lambda)$.
Consequently, $A_\lambda$ commutes with $R$.
\end{proposition}

\begin{structuredproof}
\Step{$R$ is a well-defined unitary involution on $L^2(I)$, and $R$ preserves $H_I\subset L^2(\R)$.}
\Stepjust{$I=(-L,L)$ is symmetric about $0$: if $u\in I$ then $-u\in I$.
So $(RG)(u)=G(-u)$ maps $L^2(I)$ to itself.  $R$ is unitary ($\|RG\|_2=\|G\|_2$ by substitution
$u\mapsto -u$) and $R^2=\mathrm{Id}$.
If $\phi\in H_I$ (i.e.\ $\phi=0$ outside $I$), then $R\phi(u)=\phi(-u)$ vanishes for $u\notin I$
(since $-u\notin I$), so $R\phi\in H_I$.}

\Step{$RS_t=S_{-t}R$ on $L^2(\R)$.}
\Stepjust{$(RS_t\phi)(u)=(S_t\phi)(-u)=\phi(-u-t)$ and
$(S_{-t}R\phi)(u)=(R\phi)(u-(-t))=(R\phi)(u+t)=\phi(-u-t)$.  They agree.}

\Step{$\|\widetilde{RG}-S_t\widetilde{RG}\|_2=\|\widetilde G-S_t\widetilde G\|_2$ for every $t\in\R$.}

\Substep{$\widetilde{RG}=R\widetilde G$ (extension by zero commutes with reflection, since $I$ is symmetric).}
\Substepjust{For $u\in I$: $\widetilde{RG}(u)=(RG)(u)=G(-u)=(R\widetilde G)(u)$.
For $u\notin I$: $\widetilde{RG}(u)=0$ and $(R\widetilde G)(u)=\widetilde G(-u)=0$ (since $-u\notin I$).}

\Substep{$\|R\widetilde G-S_tR\widetilde G\|_2
=\|R(\widetilde G-S_{-t}\widetilde G)\|_2$ (using Step~2: $S_tR=RS_{-t}$, so $S_tR\widetilde G=RS_{-t}\widetilde G$).}
\Substepjust{$R\widetilde G-S_tR\widetilde G=R\widetilde G-RS_{-t}\widetilde G=R(\widetilde G-S_{-t}\widetilde G)$.}

\Substep{$\|R(\widetilde G-S_{-t}\widetilde G)\|_2=\|\widetilde G-S_{-t}\widetilde G\|_2$ ($R$ is unitary).}
\Substepjust{$R$ is unitary on $L^2(\R)$ by the same substitution $v=-u$ as in Step~1
(the isometry and surjectivity arguments carry over verbatim from $L^2(I)$ to $L^2(\R)$).}

\Substep{$\|\widetilde G-S_{-t}\widetilde G\|_2=\|\widetilde G-S_t\widetilde G\|_2$.}
\Substepjust{Substituting $v=u+t$:
$\|\phi-S_{-t}\phi\|_2^2=\int|\phi(u)-\phi(u+t)|^2\,du
=\int|\phi(v-t)-\phi(v)|^2\,dv=\|\phi-S_t\phi\|_2^2$.}

\Substep{Chain Steps~3.1--3.4:
$\|\widetilde{RG}-S_t\widetilde{RG}\|_2=\|R\widetilde G-S_tR\widetilde G\|_2=\|\widetilde G-S_{-t}\widetilde G\|_2=\|\widetilde G-S_t\widetilde G\|_2$.}

\Step{$\mathcal E_\lambda(RG)=\mathcal E_\lambda(G)$.}
\Stepjust{By Definition~\ref{def:E}, $\mathcal E_\lambda$ is built from terms of the form
$\|\widetilde G-S_t\widetilde G\|_2^2$ with nonneg.\ weights.
Step~3 shows each such term is the same for $RG$ as for $G$.
Therefore $\mathcal E_\lambda(RG)=\mathcal E_\lambda(G)$.
In particular, $RG\in\mathcal D(\mathcal E_\lambda)$ iff $G\in\mathcal D(\mathcal E_\lambda)$.}

\Step{$A_\lambda$ commutes with $R$.}

\Substep{For $u\in\mathcal D(A_\lambda)$ and $v\in\mathcal D(\mathcal E_\lambda)$:
$\mathcal E_\lambda(Ru,v)=\mathcal E_\lambda(u,Rv)$.}
\Substepjust{$Ru\in\mathcal D(\mathcal E_\lambda)$ (since $R$ preserves $\mathcal D(\mathcal E_\lambda)$ by Step~4).
Invariance of $\mathcal E_\lambda$ under $R$ (Step~4) implies, by polarization,
$\mathcal E_\lambda(Ru,Rv)=\mathcal E_\lambda(u,v)$ for all $u,v\in\mathcal D(\mathcal E_\lambda)$.
Set $v\mapsto Rv$: $\mathcal E_\lambda(Ru,R(Rv))=\mathcal E_\lambda(Ru,v)=\mathcal E_\lambda(u,Rv)$
(using $R^2=\mathrm{Id}$).}

\Substep{$\mathcal E_\lambda(u,Rv)=\langle A_\lambda u,Rv\rangle=\langle RA_\lambda u,v\rangle$.}
\Substepjust{Form identity: $\mathcal E_\lambda(u,Rv)=\langle A_\lambda u,Rv\rangle$ (since $u\in\mathcal D(A_\lambda)$
and $Rv\in\mathcal D(\mathcal E_\lambda)$).
Then $\langle A_\lambda u,Rv\rangle=\langle R^*A_\lambda u,v\rangle=\langle RA_\lambda u,v\rangle$
since $R^*=R$ ($R$ is selfadjoint: $R=R^{-1}=R^*$).}

\Substep{Combining: $\mathcal E_\lambda(Ru,v)=\langle RA_\lambda u,v\rangle$ for all $v\in\mathcal D(\mathcal E_\lambda)$.}
\Substepjust{Steps~5.1 and 5.2.}

\Substep{$Ru\in\mathcal D(A_\lambda)$ and $A_\lambda Ru=RA_\lambda u$.}
\Substepjust{By the representation theorem (Kato~\cite[Thm.~VI.2.1]{KatoPerturbation}),
$w\in\mathcal D(A_\lambda)$ if and only if there exists $h\in L^2(I)$ such that
$\mathcal E_\lambda(w,v)=\langle h,v\rangle$ for all $v\in\mathcal D(\mathcal E_\lambda)$,
in which case $A_\lambda w=h$.
Step~5.3 provides exactly this with $w=Ru$ and $h=RA_\lambda u\in L^2(I)$.}

\QEDstep
\end{structuredproof}

\begin{corollary}[Even ground state]
\label{cor:even}
Assume Theorem~\ref{thm:operator} and Proposition~\ref{prop:reflection}. Let $\psi$ be the strictly positive ground-state
eigenfunction from Proposition~\ref{prop:groundstate}. Then $\psi$ is even: $\psi(-u)=\psi(u)$ a.e.
\end{corollary}

\begin{structuredproof}
\Step{Define $\psi^\sharp:=R\psi$.  Then $\psi^\sharp$ is an eigenfunction of $A_\lambda$ for the same
eigenvalue $\mu_0:=\min\sigma(A_\lambda)$.}
\Stepjust{$A_\lambda\psi=\mu_0\psi$.  Since $A_\lambda R=RA_\lambda$ (Proposition~\ref{prop:reflection}):
$A_\lambda\psi^\sharp=A_\lambda R\psi=RA_\lambda\psi=R(\mu_0\psi)=\mu_0 R\psi=\mu_0\psi^\sharp$.}

\Step{$\psi^\sharp>0$ a.e.}
\Stepjust{$\psi>0$ a.e.\ (Proposition~\ref{prop:groundstate}).
$\psi^\sharp(u)=\psi(-u)$.  Since $I$ is symmetric and $\psi>0$ a.e.\ on $I$,
the reflection $\psi^\sharp>0$ a.e.\ on~$I$.}

\Step{$\psi^\sharp=c\psi$ for some $c\in\R$.}
\Stepjust{Proposition~\ref{prop:groundstate} says the eigenspace for $\mu_0$ is one-dimensional (simple eigenvalue).
Both $\psi$ and $\psi^\sharp$ lie in this eigenspace (Steps~1 and the original eigenvalue equation).
Hence $\psi^\sharp=c\psi$ for some scalar~$c$.  Since both are real-valued (reflection preserves real-valuedness), $c\in\R$.}

\Step{$c>0$.}
\Stepjust{$\psi^\sharp>0$ a.e.\ (Step~2) and $\psi>0$ a.e.\ (Proposition~\ref{prop:groundstate}).
If $c\le 0$, then $\psi^\sharp=c\psi\le 0$ a.e., contradicting $\psi^\sharp>0$ a.e.}

\Step{$c=1$.}
\Stepjust{$\|\psi^\sharp\|_2=\|R\psi\|_2=\|\psi\|_2$ ($R$ is unitary).
From $\psi^\sharp=c\psi$: $\|c\psi\|_2=|c|\|\psi\|_2=\|\psi\|_2$.
Since $\|\psi\|_2>0$ ($\psi\ne 0$), $|c|=1$.  Combined with $c>0$ (Step~4): $c=1$.}

\Step{$\psi(-u)=\psi(u)$ a.e.}
\Stepjust{$\psi^\sharp=\psi$ (Step~5), i.e.\ $R\psi=\psi$, i.e.\ $\psi(-u)=\psi(u)$ a.e.}

\QEDstep
\end{structuredproof}

%% =================================================================
\section{Conclusion}
%% =================================================================

\begin{itemize}
\item Starting solely from the explicit local formulas \eqref{eq:Wp}--\eqref{eq:WR}, we derived
a representation of $-\sum_v W_v(g*g^*)$ (up to an additive constant multiple of $\|g\|_2^2$) as a
positive combination of translation-difference energies in log-coordinates
(Definition~\ref{def:E}, Lemmas~\ref{lem:prime-energy}--\ref{lem:arch-energy}).
\item We proved the Markov/normal contraction inequality for this form (Lemma~\ref{lem:markov}).
\item Using only measure theory (Lebesgue density), we proved that invariance under all sufficiently small
translations forces a measurable subset of an interval to be null or conull (Lemma~\ref{lem:trans-inv}),
and we used it to show that $\mathcal E_\lambda(\1_B)=0$ implies $B$ is null or conull
(Lemma~\ref{lem:indicator-energy}).
\item We proved that the quadratic form is closed (Propositions~\ref{prop:closedR}--\ref{prop:closedI}),
established a logarithmic lower bound on its Fourier symbol (Lemma~\ref{lem:psilog}), and used
the Kolmogorov--Riesz compactness criterion to show that the associated selfadjoint operator
has compact resolvent (Theorem~\ref{thm:operator}).
\item From this operator setup we obtained irreducibility and then (by a standard external theorem)
positivity improving of the semigroup,
hence simplicity and strict positivity of the ground state (Proposition~\ref{prop:groundstate}).
\item Finally, inversion symmetry forces that strictly positive simple ground state to be even
(Corollary~\ref{cor:even}).
\end{itemize}

\begin{remark}[Generalization to Dedekind zeta functions and Rankin--Selberg $L$-functions]
\label{rem:generalization}
The proof relies on two structural features of the Weil explicit formula:
(i)~the local terms at non-archimedean places produce translation-difference
energies with \emph{non-negative} weights, and (ii)~the archimedean term
produces a continuum of such energies with a strictly positive weight
$w(t)>0$.  Everything else---the Markov property, irreducibility from
the archimedean continuum, the logarithmic lower bound on the Fourier
symbol, and the Perron--Frobenius conclusion---follows formally from
(i) and (ii).

\smallskip\noindent
\emph{Dedekind zeta functions.}
For a number field~$K$, the Weil explicit formula for $\zeta_K(s)$
has the same structure: a sum over prime ideals~$\mathfrak p$
(with shift sizes $m\log N(\mathfrak p)$ in place of $m\log p$)
plus archimedean contributions from each infinite place.
Each archimedean place contributes its own integral term with
a strictly positive weight, so the logarithmic lower bound on
the Fourier symbol and the irreducibility argument both carry over.
The generalization requires only notational changes.

\smallskip\noindent
\emph{Rankin--Selberg $L$-functions.}
The method may be most naturally suited to $L$-functions of the form
$L(s,\pi\times\widetilde\pi)$, where $\pi$ is a cuspidal automorphic
representation of $\mathrm{GL}(n)$.  The self-convolution structure
makes property~(i) automatic: the Dirichlet coefficients at
unramified primes are $|a_\pi(\mathfrak p)|^2$ (and sums thereof at
higher prime powers), so the prime-energy weights are manifestly
non-negative without any algebraic manipulation.
For property~(ii), the archimedean $L$-factor of
$L(s,\pi\times\widetilde\pi)$ is a product of~$n^2$ Gamma-type
terms, each contributing a digamma-type weight to the integral.
This only strengthens the logarithmic lower bound.
Since GRH for $L(s,\pi\times\widetilde\pi)$ implies GRH for
$L(s,\pi)$ (the zeros of the former include those of the latter),
establishing the spectral setup---compact resolvent, simple ground
state, Perron--Frobenius---for Rankin--Selberg $L$-functions would
bring a wide class of automorphic $L$-functions under a single
framework.
\end{remark}

%% =================================================================
\section{Bibliographic pointers}
\label{sec:bib}
%% =================================================================

\begin{thebibliography}{99}

\bibitem{ATGStrictPos2020}
W.~Arendt, A.~F.~M.~ter~Elst, and J.~Gl{\"u}ck.
\newblock Strict positivity for the principal eigenfunction of elliptic operators with various boundary conditions.
\newblock \emph{Adv.\ Nonlinear Stud.} \textbf{20} (2020), no.~3, 633--650.
\newblock DOI:~10.1515/ans-2020-2091.
\newblock Also available as arXiv:1909.12194.

\bibitem{FOT}
M.~Fukushima, Y.~Oshima, and M.~Takeda.
\newblock \emph{Dirichlet Forms and Symmetric Markov Processes}.
\newblock 2nd revised and extended ed., De~Gruyter Studies in Mathematics, vol.~19,
Walter de~Gruyter, Berlin, 2011.
\newblock DOI:~10.1515/9783110218091.

\bibitem{LamportHowToWrite}
L.~Lamport.
\newblock How to write a proof.
\newblock \emph{Amer.\ Math.\ Monthly} \textbf{102} (1995), no.~7, 600--608.
\newblock DOI:~10.2307/2974556.

\bibitem{OuhabazHeatEq}
E.-M.~Ouhabaz.
\newblock \emph{Analysis of Heat Equations on Domains}.
\newblock London Mathematical Society Monographs Series, vol.~31, Princeton University Press,
Princeton, NJ, 2005.
\newblock DOI:~10.1515/9781400826483.

\bibitem{KatoPerturbation}
T.~Kato.
\newblock \emph{Perturbation Theory for Linear Operators}.
\newblock Classics in Mathematics, Springer, Berlin, 1995.
\newblock DOI:~10.1007/978-3-642-66282-9.

\bibitem{BrezisFA}
H.~Brezis.
\newblock \emph{Functional Analysis, Sobolev Spaces and Partial Differential Equations}.
\newblock Universitext, Springer, New York, 2011.
\newblock ISBN:~978-0-387-70913-0.
\newblock DOI:~10.1007/978-0-387-70914-7.

\bibitem{SchaeferBanachLattices}
H.~H.~Schaefer.
\newblock \emph{Banach Lattices and Positive Operators}.
\newblock Grundlehren der mathematischen Wissenschaften, vol.~215,
Springer-Verlag, Berlin, 1974.
\newblock ISBN:~978-3-540-06936-2.
\newblock DOI:~10.1007/978-3-642-65970-6.

\bibitem{EngelNagel}
K.-J.~Engel and R.~Nagel.
\newblock \emph{One-Parameter Semigroups for Linear Evolution Equations}.
\newblock Graduate Texts in Mathematics, vol.~194,
Springer-Verlag, New York, 2000.
\newblock ISBN:~0-387-98463-1.
\newblock DOI:~10.1007/b97696.

\bibitem{ReedSimonI}
M.~Reed and B.~Simon.
\newblock \emph{Methods of Modern Mathematical Physics.\ I:\ Functional Analysis}.
\newblock Academic Press, New York, revised and enlarged edition, 1980.
\newblock ISBN:~0-12-585050-6.

\end{thebibliography}

\end{document}
