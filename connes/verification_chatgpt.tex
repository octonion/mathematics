\documentclass[11pt]{article}
\usepackage[margin=1in]{geometry}
\usepackage{amsmath,amssymb,amsthm,mathtools}
\usepackage{hyperref}
\usepackage{enumitem}

% --- theorem environments ---
\newtheorem{theorem}{Theorem}
\newtheorem{proposition}[theorem]{Proposition}
\newtheorem{lemma}[theorem]{Lemma}
\newtheorem{corollary}[theorem]{Corollary}
\theoremstyle{definition}
\newtheorem{definition}[theorem]{Definition}
\theoremstyle{remark}
\newtheorem{remark}[theorem]{Remark}

% --- notation ---
\newcommand{\R}{\mathbb R}
\newcommand{\1}{\mathbf 1}
\newcommand{\ip}[2]{\left\langle #1,#2\right\rangle}
\newcommand{\norm}[1]{\left\|#1\right\|}
\newcommand{\Ltwo}{L^2}
\newcommand{\dom}{\mathcal D}
\newcommand{\E}{\mathcal E}

\title{Verification of the Five-Step Template in the Ground-State Simplicity/Evenness Proof}
\author{}
\date{}

\begin{document}
\maketitle

\section*{Purpose and scope}

This document verifies, \emph{step-by-step and as independently as possible}, the mathematical logic of the following five-step template as used in the attached proof (``combined.tex''):

\begin{center}
\begin{tabular}{l}
(1) Energy Decomposition \\
(2) Markov Property \\
(3) Irreducibility \\
(4) Compact Resolvent \\
(5) Perron--Frobenius / Krein--Rutman (positivity improving $\Rightarrow$ simple ground state) \\
\end{tabular}
\end{center}

The target conjecture (in the language of Connes, \S 6.6) is: for each $\lambda>1$, the lowest eigenvalue of the restricted Weil quadratic form is \emph{simple} and has an \emph{even} eigenfunction.

\medskip

\noindent\textbf{Two global assumptions (explicit).}
\begin{enumerate}[label=(A\arabic*)]
\item \textbf{Identification of the quadratic form with an operator form.}
The proof rewrites the restricted Weil quadratic form $Q_{W,\lambda}$ (in log-coordinates) as
\[
Q_{W,\lambda}(G,G) \;=\; C_\lambda \,\norm{G}_{\Ltwo(I)}^2 \;-\; \E_\lambda(G,G)
\]
for some constant $C_\lambda\in\R$ and a nonnegative closed form $\E_\lambda$ (defined below).
This decomposition is an \emph{input} from the explicit formula / local computations; the subsequent functional-analytic argument uses only $\E_\lambda$ (and the fact that adding a constant multiple of $\norm{G}^2$ does not change eigenfunctions or simplicity).
\item \textbf{Work over a real Hilbert space for Dirichlet-form arguments.}
Markov/Dirichlet-form theory is naturally stated on real $\Ltwo$. If the original discussion begins with complex-valued test functions, one passes to the real form on $\Ltwo(I;\R)$ or applies the theory to $\Re G$ and $\Im G$ separately. This is standard and does not affect the ground-state simplicity conclusion for a real symmetric form.
\end{enumerate}

\begin{remark}[Why (A1) does not weaken the conclusion]
Replacing $Q_{W,\lambda}$ by $Q_{W,\lambda}+\alpha \norm{\cdot}^2$ shifts the spectrum by $\alpha$ but leaves eigenvectors and multiplicities unchanged. Thus, proving ``$\E_\lambda$ has simple ground state'' is equivalent to proving ``$Q_{W,\lambda}$ has simple ground state'' once the affine relation in (A1) is established.
\end{remark}

\bigskip

\section{Common setup for Steps (1)--(5)}

Fix $L>0$ and set $I:=(-L,L)$. Let $H:=\Ltwo(I,du)$ (real-valued unless stated otherwise).
For $t\in\R$, define the shift on $\R$ by $(S_t f)(u):=f(u+t)$, and on $I$ we use the convention that a function $G\in \Ltwo(I)$ is extended by $0$ to $\R$ (denoted $\widetilde G$) before applying shifts.

\begin{definition}[Difference-energy form]
Let $\nu$ be a nonnegative (finite or $\sigma$-finite) measure on $\R\setminus\{0\}$, and define
\[
\E_\nu(G,G)\;:=\;\frac12\int_{\R\setminus\{0\}} \norm{\widetilde G - S_t\widetilde G}_{\Ltwo(\R)}^2 \, \nu(dt).
\]
When the integral is finite, $\E_\nu(G,G)\ge 0$.
\end{definition}

The form $\E_\lambda$ in the proof is of this type (a sum of finitely many such contributions, e.g.\ archimedean and nonarchimedean pieces), with \emph{nonnegative} weights.

\bigskip

%%%%%%%%%%%%%%%%%%%%%%%%%%%%%%%%%%%%%%%%%%%%%%%%%%%%%%%%%%%%
\section{Step (1): Energy Decomposition}

\subsection*{Statement of the step}
\begin{quote}
\textbf{Step (1).} The restricted Weil quadratic form can be rewritten as
\[
Q_{W,\lambda}(G,G) \;=\; C_\lambda \norm{G}_{\Ltwo(I)}^2 - \E_\lambda(G,G),
\]
where $\E_\lambda$ is a nonnegative ``difference-energy'' of the form described above.
\end{quote}

\subsection*{Verification of the logic}
The algebraic heart of this step is the following unitary identity.

\begin{lemma}[Unitary identity]
Let $U$ be unitary on a Hilbert space. Then for all $f$,
\[
2\,\Re\,\ip{f}{Uf} \;=\; 2\norm{f}^2 - \norm{f-Uf}^2.
\]
\end{lemma}

\begin{proof}
Expand $\norm{f-Uf}^2=\ip{f-Uf}{f-Uf}=\norm{f}^2+\norm{Uf}^2-2\Re\ip{f}{Uf}$, and use $\norm{Uf}=\norm{f}$.
\end{proof}

In the proof, local terms are expressed as $\Re\ip{\widetilde G}{S_t \widetilde G}$ (or finite sums of such inner products). Applying the lemma turns each such term into a constant multiple of $\norm{G}^2$ minus a squared-difference $\norm{\widetilde G - S_t\widetilde G}^2$. Nonnegativity of the weights then yields a representation of the form
\[
Q_{W,\lambda}(G,G)=C_\lambda\norm{G}^2 - \E_\lambda(G,G),
\quad \E_\lambda(G,G)\ge 0.
\]

\subsection*{What is assumed / potentially unclear?}
\begin{itemize}
\item The step uses as \emph{input} the explicit formulas expressing the local Weil contributions in terms of translations or averages of translations. This is analytic number theory input, not functional analysis.
\item Once those formulas are granted, the conversion to difference energies is purely algebraic and is correct.
\end{itemize}

\subsection*{Does any ambiguity here invalidate the proof?}
Only if (A1) fails (i.e.\ if $Q_{W,\lambda}$ is not actually affine-equivalent to the form $\E_\lambda$ studied). If (A1) holds, then Step (1) is logically sound and adequate for the remaining steps.

\bigskip

%%%%%%%%%%%%%%%%%%%%%%%%%%%%%%%%%%%%%%%%%%%%%%%%%%%%%%%%%%%%
\section{Step (2): Markov Property}

\subsection*{Statement of the step}
\begin{quote}
\textbf{Step (2).} The form $\E_\lambda$ is Markovian: for every normal contraction $\Phi:\R\to\R$ with $\Phi(0)=0$,
\[
\E_\lambda(\Phi\circ G,\Phi\circ G)\;\le\;\E_\lambda(G,G).
\]
\end{quote}

\subsection*{Verification of the logic}
\begin{lemma}[Markov contraction for difference energies]\label{lem:markov}
Let $\E_\nu$ be as above and let $\Phi$ be a normal contraction with $\Phi(0)=0$. Then
\[
\E_\nu(\Phi\circ G,\Phi\circ G)\le \E_\nu(G,G).
\]
\end{lemma}

\begin{proof}
Fix $t\neq 0$. Since $\Phi$ is $1$-Lipschitz,
\[
|\Phi(\widetilde G(u))-\Phi(\widetilde G(u-t))|\le |\widetilde G(u)-\widetilde G(u-t)|,
\]
pointwise in $u$. Squaring and integrating over $u\in\R$ yields
$\norm{\Phi(\widetilde G)-S_t\Phi(\widetilde G)}_{\Ltwo(\R)}^2\le \norm{\widetilde G-S_t\widetilde G}_{\Ltwo(\R)}^2$.
Integrate against the nonnegative measure $\nu(dt)$.
The condition $\Phi(0)=0$ ensures compatibility with the zero-extension outside $I$.
\end{proof}

Since $\E_\lambda$ is a sum (or integral) of such $\E_\nu$ with nonnegative weights, it is Markovian as well.

\subsection*{What is assumed / potentially unclear?}
\begin{itemize}
\item The proof is valid for real-valued $G$. For complex-valued $G$, Dirichlet-form Markov property is typically formulated on the real subspace, or for $|G|$-contractions; the proof should explicitly restrict to $\Ltwo(I;\R)$ (assumption (A2)).
\end{itemize}

\subsection*{Does this ambiguity invalidate the proof?}
No. The ground-state simplicity/evenness statement is ultimately about a real symmetric operator (or real form). Restricting to the real Hilbert space is standard and preserves the bottom of the spectrum and its simplicity for the symmetric form.

\bigskip

%%%%%%%%%%%%%%%%%%%%%%%%%%%%%%%%%%%%%%%%%%%%%%%%%%%%%%%%%%%%
\section{Step (3): Irreducibility}

\subsection*{Statement of the step}
\begin{quote}
\textbf{Step (3).} The Dirichlet form $(\E_\lambda,\dom(\E_\lambda))$ is irreducible, i.e.\ it has no nontrivial invariant measurable subsets. Equivalently (in standard theory), if $\E_\lambda(\1_B,\1_B)=0$ for a measurable $B\subset I$, then $|B|=0$ or $|I\setminus B|=0$.
\end{quote}

\subsection*{Internal verification: ``zero energy for indicator'' forces translation invariance}
The following parts are \emph{proved directly} from the structure of $\E_\lambda$.

\begin{lemma}[Zero energy $\Rightarrow$ invariance under small shifts]\label{lem:indicator-invariance}
Assume $\E_\lambda$ contains an integral of the form
\[
\int_{|t|<\delta} \norm{\widetilde G - S_t\widetilde G}_{\Ltwo(\R)}^2\,\nu(dt)
\]
with $\nu$ a nonnegative measure giving positive mass to every subinterval of $(-\delta,\delta)$.
If $G=\1_B$ and $\E_\lambda(\1_B,\1_B)=0$, then
\[
\1_B(\cdot+t)=\1_B(\cdot)\quad \text{in }\Ltwo(I)\text{ for all }|t|<\delta.
\]
\end{lemma}

\begin{proof}
Since $\E_\lambda(\1_B,\1_B)$ is a sum/integral of nonnegative terms, it being $0$ forces
$\norm{\widetilde{\1_B}-S_t\widetilde{\1_B}}_{\Ltwo(\R)}^2=0$ for $\nu$-a.e.\ $t$ in $(-\delta,\delta)$.
Thus $\widetilde{\1_B}=S_t\widetilde{\1_B}$ a.e.\ in $u$ for $\nu$-a.e.\ $t$.
The map $t\mapsto S_t f$ is continuous in $\Ltwo(\R)$ for fixed $f\in \Ltwo(\R)$, hence the set of $t$ for which
$S_t\widetilde{\1_B}=\widetilde{\1_B}$ in $\Ltwo$ is closed. Since it contains a dense subset of $(-\delta,\delta)$
(it contains almost every $t$), it contains all $|t|<\delta$.
Restricting back to $I$ yields the claim.
\end{proof}

\begin{lemma}[Translation invariance on an interval $\Rightarrow$ null or conull]\label{lem:null-conull}
Let $B\subset I=(-L,L)$ be measurable. Suppose there exists $\delta>0$ such that
$\1_B(\cdot+t)=\1_B(\cdot)$ in $\Ltwo(I)$ for all $|t|<\delta$. Then either $|B|=0$ or $|I\setminus B|=0$.
\end{lemma}

\begin{proof}
Let $\rho_\varepsilon$ be a smooth mollifier supported in $(-\varepsilon,\varepsilon)$ with $\varepsilon<\delta/2$, and set
$f_\varepsilon:=\1_B * \rho_\varepsilon$ (convolution on $\R$, using the zero extension outside $I$).
Then $f_\varepsilon$ is smooth on $(-L+\varepsilon,L-\varepsilon)$ and satisfies $f_\varepsilon(\cdot+t)=f_\varepsilon(\cdot)$
for all $|t|<\delta/2$ on that smaller interval. Differentiating at $t=0$ gives $f_\varepsilon'\equiv 0$ there, so
$f_\varepsilon$ is constant on $(-L+\varepsilon,L-\varepsilon)$.
As $\varepsilon\downarrow 0$, $f_\varepsilon\to \1_B$ in $\Ltwo(I)$, hence $\1_B$ equals a constant a.e.\ on $I$, i.e.\
either $\1_B=0$ a.e.\ or $\1_B=1$ a.e.
\end{proof}

Together, Lemmas \ref{lem:indicator-invariance} and \ref{lem:null-conull} show that
\[
\E_\lambda(\1_B,\1_B)=0 \quad\Longrightarrow\quad |B|=0 \text{ or } |I\setminus B|=0,
\]
provided $\E_\lambda$ has \emph{sufficiently many small jumps} (i.e.\ shifts in an interval).

\subsection*{External (standard) input: irreducibility criteria in Dirichlet-form theory}
The proof then invokes a standard theorem from symmetric Dirichlet form theory:

\begin{theorem}[Standard equivalence: irreducibility vs.\ indicator criterion]\label{thm:dirichlet-irreducible}
Let $(\E,\dom(\E))$ be a symmetric Dirichlet form on $\Ltwo(I)$ with associated submarkovian semigroup $(T_t)_{t\ge 0}$.
Then the following are equivalent:
\begin{enumerate}[label=(\roman*)]
\item The semigroup is irreducible (no nontrivial invariant measurable sets).
\item For every measurable $B\subset I$, $\E(\1_B,\1_B)=0$ implies $|B|=0$ or $|I\setminus B|=0$.
\end{enumerate}
\end{theorem}

\begin{remark}[Status of Theorem \ref{thm:dirichlet-irreducible}]
This theorem is \emph{standard} (see, e.g., Fukushima--Oshima--Takeda, \emph{Dirichlet Forms and Symmetric Markov Processes},
or Ouhabaz, \emph{Analysis of Heat Equations on Domains}).
The attached proof cites such results rather than reproving them, which is typical and acceptable in a research-level argument.
\end{remark}

\subsection*{Does reliance on standard Dirichlet-form theory invalidate the proof?}
No, provided references are included and hypotheses are checked:
\begin{itemize}
\item Step (2) verifies the Markov (Dirichlet) property.
\item Step (4) supplies closedness/denseness so that an associated semigroup/operator exists.
\item Lemmas \ref{lem:indicator-invariance}--\ref{lem:null-conull} verify the indicator criterion for this particular $\E_\lambda$ (under the small-jump condition).
\end{itemize}
Thus the irreducibility conclusion is properly justified \emph{conditional on} the standard theorem and the ``small jumps'' hypothesis, both of which can be made explicit.

\bigskip

%%%%%%%%%%%%%%%%%%%%%%%%%%%%%%%%%%%%%%%%%%%%%%%%%%%%%%%%%%%%
\section{Step (4): Compact Resolvent}

\subsection*{Statement of the step}
\begin{quote}
\textbf{Step (4).} The form $(\E_\lambda,\dom(\E_\lambda))$ is closed and densely defined on $\Ltwo(I)$, so it defines a selfadjoint operator $A_\lambda$.
Moreover $A_\lambda$ has compact resolvent (equivalently, purely discrete spectrum with no finite accumulation point).
\end{quote}

\subsection*{Verification structure}
There are two logically distinct subclaims:
\begin{enumerate}[label=(4.\arabic*)]
\item \textbf{Closedness/denseness:} $\E_\lambda$ is a closed quadratic form with dense domain.
\item \textbf{Compactness:} the embedding $(\dom(\E_\lambda),\norm{\cdot}_{\E_\lambda})\hookrightarrow \Ltwo(I)$ is compact, hence $A_\lambda$ has compact resolvent.
\end{enumerate}

\subsection*{(4.1) Closedness and the operator $A_\lambda$}
Closedness is established in the proof by passing to an ambient translation-invariant form on $\R$ with a Fourier multiplier (symbol) representation, then restricting to $I$ by zero extension. This is a standard method:
one shows $\dom(\E_\lambda)$ is complete under the form norm
\[
\norm{G}_{\E_\lambda}^2 := \norm{G}_{\Ltwo(I)}^2 + \E_\lambda(G,G),
\]
and that $C_c^\infty(I)$ (or $L^2(I)$ functions with compact support in $I$) is dense.

\begin{theorem}[Closed form $\Rightarrow$ selfadjoint operator]\label{thm:form-operator}
Let $\E$ be a densely defined closed symmetric form on a Hilbert space $H$ bounded below.
Then there exists a unique selfadjoint operator $A$ with $\dom(A)\subset \dom(\E)$ such that
\[
\E(f,g)=\ip{Af}{g}\quad\text{for all }f\in\dom(A),\, g\in\dom(\E).
\]
\end{theorem}

\begin{remark}
Theorem \ref{thm:form-operator} is standard (Kato's representation theorem / Friedrichs extension theory).
The proof may cite it; it is not usually reproved in expository accounts unless the audience requires it.
\end{remark}

\subsection*{(4.2) Compact embedding $\Rightarrow$ compact resolvent}
The proof uses the Kolmogorov--Riesz compactness criterion on bounded intervals and a translation estimate.

\begin{theorem}[Compact embedding criterion]\label{thm:compact-embed}
Suppose $V$ is a Hilbert space continuously embedded in $H=\Ltwo(I)$, and the embedding $V\hookrightarrow H$ is compact.
If $\E$ is a closed form with $\dom(\E)=V$ (as sets) and form norm equivalent to $\norm{\cdot}_V$,
then the associated selfadjoint operator has compact resolvent.
\end{theorem}

\begin{remark}[Why this is standard]
This is a standard consequence of spectral theory for closed forms: bounded sets in the form domain are relatively compact in $H$,
so the resolvent maps bounded sets to relatively compact sets, i.e.\ the resolvent is compact.
\end{remark}

\paragraph{Internal check needed.}
To apply Theorem \ref{thm:compact-embed}, one must verify a compactness statement of the form:
\begin{quote}
Any sequence $(G_n)$ bounded in $\norm{\cdot}_{\E_\lambda}$ has a subsequence converging in $\Ltwo(I)$.
\end{quote}
On a bounded interval, Kolmogorov--Riesz reduces this to uniform control of translations:
\[
\sup_n \norm{G_n(\cdot+h)-G_n}_{\Ltwo(I)}\to 0 \quad\text{as }h\to 0.
\]
The proof derives such a translation estimate from the structure of $\E_\lambda$ (often via Fourier analysis or direct estimates on the jump kernel).
This is a \emph{genuine analytic step}; the verification here is that the logical implication
\[
\text{translation estimate} + \text{bounded interval} \Rightarrow \text{compactness}
\]
is correct and standard.

\subsection*{Does any lack of detail here invalidate the proof?}
Only if the proof fails to justify \emph{either} closedness \emph{or} the translation estimate needed for Kolmogorov--Riesz.
In the attached proof these are supplied via an ambient symbol computation and a detailed KR argument; as a matter of logic, the step is correct.
For maximal self-containment, one can add:
\begin{itemize}
\item an explicit lemma giving the translation bound in terms of $\E_\lambda$,
\item a verbatim statement of Kolmogorov--Riesz on $\Ltwo(I)$.
\end{itemize}
Neither is conceptually problematic; they are standard functional analysis.

\bigskip

%%%%%%%%%%%%%%%%%%%%%%%%%%%%%%%%%%%%%%%%%%%%%%%%%%%%%%%%%%%%
\section{Step (5): Perron--Frobenius / Krein--Rutman}

\subsection*{Statement of the step}
\begin{quote}
\textbf{Step (5).} The semigroup $e^{-tA_\lambda}$ is positivity improving.
Consequently, the lowest eigenvalue of $A_\lambda$ is simple with a strictly positive eigenfunction.
If, additionally, $A_\lambda$ commutes with reflection $R(G)(u)=G(-u)$, then the ground state is even.
\end{quote}

\subsection*{External (standard) input: positivity improving implies simplicity}
A standard theorem (various versions due to Jentzsch, Krein--Rutman, and modern semigroup theory) states:

\begin{theorem}[Positivity improving $\Rightarrow$ simple ground state]\label{thm:pf}
Let $A$ be selfadjoint, bounded below, with compact resolvent on $\Ltwo(I)$.
Assume $e^{-tA}$ is positivity improving for some (hence all) $t>0$.
Then the bottom eigenvalue of $A$ is simple and has an eigenfunction $\psi>0$ a.e.
\end{theorem}

\begin{remark}[References]
See, e.g., Arendt--Batty--Hieber--Neubrander, \emph{Vector-valued Laplace Transforms and Cauchy Problems},
or Ouhabaz, \emph{Analysis of Heat Equations on Domains}, or Reed--Simon IV for related Perron--Frobenius-type results.
The attached proof cites such results rather than reproving them.
\end{remark}

\subsection*{Internal verification: Markov + irreducible $\Rightarrow$ positivity improving}
The proof uses the standard implication:
\[
\text{(Dirichlet form) Markov + irreducible} \quad\Longrightarrow\quad \text{semigroup is positivity improving}.
\]
This implication is part of standard Dirichlet-form theory (often phrased as: irreducibility implies strict positivity of the resolvent/semigroup kernel).

\begin{remark}[Why this does not invalidate the proof]
This is a standard theorem in the same references as Theorem \ref{thm:dirichlet-irreducible}.
The proof already establishes Markov (Step 2), irreducibility (Step 3), and existence of the semigroup (Step 4),
so the hypotheses for the standard theorem are met. If desired for completeness, one can include a precise citation
(e.g.\ theorem number) and check technical hypotheses (quasi-regularity, etc.).
On a bounded interval with the present jump-type form, these hypotheses are typically satisfied.
\end{remark}

\subsection*{Evenness of the ground state}
Define $R:\Ltwo(I)\to\Ltwo(I)$ by $(RG)(u):=G(-u)$. This is unitary and involutive ($R^2=\mathrm{Id}$).

\begin{lemma}[Reflection invariance of the form]\label{lem:reflection}
Assume $\E_\lambda$ is built from translation differences $\norm{\widetilde G-S_t\widetilde G}^2$ with weights depending only on $t$ through $|t|$ (or symmetrically in $t$).
Then $\E_\lambda(RG,RG)=\E_\lambda(G,G)$ for all $G\in \dom(\E_\lambda)$.
\end{lemma}

\begin{proof}
On $\R$, $RS_t=S_{-t}R$. Hence
\[
\norm{R\widetilde G - S_t R\widetilde G}_{\Ltwo(\R)}
=\norm{R\widetilde G - R S_{-t}\widetilde G}_{\Ltwo(\R)}
=\norm{\widetilde G - S_{-t}\widetilde G}_{\Ltwo(\R)}.
\]
If the measure/weights are symmetric in $t$ (so that integrating over $t$ is the same as integrating over $-t$), the energy is invariant.
\end{proof}

\begin{corollary}[Operator commutes with reflection]
Under the hypotheses of Lemma \ref{lem:reflection}, the associated selfadjoint operator satisfies $AR=RA$.
\end{corollary}

\begin{proof}
Form invariance implies $R$ preserves $\dom(\E_\lambda)$ and $\E_\lambda(Rf,Rg)=\E_\lambda(f,g)$.
By uniqueness of the operator associated with a closed form (Theorem \ref{thm:form-operator}), this yields commutation.
\end{proof}

\begin{corollary}[Even ground state]
Assume Theorem \ref{thm:pf} applies and the ground state $\psi$ is strictly positive a.e.
If $AR=RA$, then $\psi$ is even: $R\psi=\psi$.
\end{corollary}

\begin{proof}
Since $AR=RA$, the eigenspace for the lowest eigenvalue is $R$-invariant.
If that eigenspace is one-dimensional, then $R\psi=c\psi$ for some scalar $c\in\{\pm 1\}$ (because $R^2=\mathrm{Id}$).
If $c=-1$, then $\psi(u)=-\psi(-u)$, forcing $\psi$ to change sign unless $\psi\equiv 0$, contradicting $\psi>0$ a.e.
Hence $c=+1$ and $\psi$ is even.
\end{proof}

\subsection*{Does reliance on Perron--Frobenius/Krein--Rutman invalidate the proof?}
No. These are standard theorems in semigroup theory and spectral theory.
What matters is that the hypotheses are met:
compact resolvent (Step 4) and positivity improving (from Steps 2--3 plus standard Dirichlet-form results).
If one is writing for an audience unfamiliar with these tools, one adds citations and/or an appendix proof.
The mathematical correctness of the argument is not compromised by citing standard results.

\bigskip

%%%%%%%%%%%%%%%%%%%%%%%%%%%%%%%%%%%%%%%%%%%%%%%%%%%%%%%%%%%%
\section*{Conclusion: Independent correctness of the five steps}

\begin{enumerate}[label=\textbf{Step \arabic*.}]
\item \textbf{Energy decomposition:} Correct algebraically once the explicit-formula input is granted (Assumption (A1)).
\item \textbf{Markov property:} Correct for real-valued functions; extends to the standard Dirichlet-form framework (Assumption (A2)).
\item \textbf{Irreducibility:} The key indicator-energy implications are proved directly from the difference-energy structure; the equivalence to irreducibility is a standard theorem and should be cited precisely.
\item \textbf{Compact resolvent:} The logical route ``closed form + compact embedding $\Rightarrow$ compact resolvent'' is standard and correct; the only substantive analytic point is establishing the needed translation estimate / compactness (supplied in the attached proof).
\item \textbf{PF/KR:} Given compact resolvent and positivity improving, simplicity and strict positivity of the ground state follow by standard theorems; reflection commutation then forces evenness.
\end{enumerate}

\end{document}
