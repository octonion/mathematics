\documentclass[11pt]{article}
\usepackage[margin=1in]{geometry}
\usepackage{amsmath,amssymb,amsthm}
\usepackage{mathrsfs}
\usepackage{hyperref}
\usepackage{enumitem}
\usepackage{booktabs}
\usepackage{xcolor}

\newtheorem{theorem}{Theorem}[section]
\newtheorem{lemma}[theorem]{Lemma}
\newtheorem{proposition}[theorem]{Proposition}
\newtheorem{corollary}[theorem]{Corollary}
\newtheorem{conjecture}[theorem]{Conjecture}
\theoremstyle{definition}
\newtheorem{definition}[theorem]{Definition}
\newtheorem{example}[theorem]{Example}
\newtheorem{question}[theorem]{Question}
\theoremstyle{remark}
\newtheorem{remark}[theorem]{Remark}

\newcommand{\R}{\mathbb{R}}
\newcommand{\C}{\mathbb{C}}
\newcommand{\Z}{\mathbb{Z}}
\newcommand{\Q}{\mathbb{Q}}
\newcommand{\N}{\mathbb{N}}
\newcommand{\A}{\mathbb{A}}
\newcommand{\cE}{\mathcal{E}}
\newcommand{\cD}{\mathcal{D}}
\newcommand{\cF}{\mathcal{F}}
\newcommand{\cH}{\mathcal{H}}
\newcommand{\cO}{\mathcal{O}}
\newcommand{\GL}{\mathrm{GL}}
\newcommand{\Spec}{\mathrm{Spec}}
\newcommand{\re}{\mathrm{Re}}

\title{Potential Consequences of the Energy-Decomposition\\
Method Beyond the Riemann Zeta Function:\\[4pt]
\large An Assessment of Three Directions}
\author{}
\date{February 2026}

\begin{document}
\maketitle

\begin{abstract}
The energy-decomposition method---which proves ground state simplicity
for the restricted Weil quadratic form operator by establishing a
Dirichlet form structure, irreducibility, and compact
resolvent---raises a natural question: where else does this strategy
apply, and what would the consequences be? We analyze three concentric
extensions: (I)~Dirichlet $L$-functions with real characters,
(II)~non-local operators with arithmetic kernels, and (III)~operators
on adelic and locally compact groups. For each, we describe the
expected form of the extension, identify the principal mathematical
consequences, and assess the technical obstacles. We argue that the
three directions are not independent but form a coherent program whose
full realization would establish a Perron--Frobenius theory for
operators on adelic groups, connecting ground state simplicity to the
Euler product structure of automorphic $L$-functions and to
multiplicity-one theorems in the Langlands program.
\end{abstract}

\tableofcontents
\newpage

%% ===================================================================
\section{Introduction}
\label{sec:intro}
%% ===================================================================

\subsection{Context}

The energy-decomposition method, as developed for the restricted Weil
quadratic form operator $A_\lambda$, establishes three properties:
compact resolvent, simplicity of the lowest eigenvalue, and evenness
of the ground state eigenfunction. The proof follows a five-step
pipeline:
\begin{center}
\fbox{\parbox{0.88\textwidth}{\centering
Energy Decomposition $\longrightarrow$ Markov Property
$\longrightarrow$ Irreducibility \\[3pt]
$\longrightarrow$ Compact Resolvent $\longrightarrow$
Perron--Frobenius / Krein--Rutman}}
\end{center}
These results resolve what Connes and collaborators identify as ``the
key difficulty'' \cite{ConnesvanSuijlekom2025} and one of ``the
missing steps'' \cite{ConnesConsaniMoscovici2025} in their program
toward the Riemann Hypothesis. The conditional theorems of
Connes--van Suijlekom \cite{ConnesvanSuijlekom2025} and
Connes--Consani--Moscovici \cite{ConnesConsaniMoscovici2025} then
imply that the approximate zeros of the associated entire functions
lie on the critical line.

The method's key structural features---the decomposition of an
arithmetic quadratic form into non-negative translation-difference
energies, the role of the archimedean continuum in forcing
irreducibility, and the use of Fourier-analytic coercivity for
compactness---are not specific to the Riemann zeta function. This
paper explores three natural directions in which the method might
extend, and assesses the mathematical consequences of each.

\subsection{The three directions}

We consider three extensions of increasing generality and ambition:

\begin{description}[leftmargin=1.5cm, labelwidth=1.2cm]
\item[Direction I.] Extension to Dirichlet $L$-functions $L(s,\chi)$
  with real characters $\chi$.
\item[Direction II.] Application to non-local operators whose kernels
  are defined by arithmetic data (partial Euler products, Epstein zeta
  functions, Rankin--Selberg convolutions).
\item[Direction III.] Formulation on locally compact abelian groups
  and adelic groups, connecting to the Langlands program and
  non-commutative geometry.
\end{description}

These form concentric circles: Direction~I validates the method across
a family; Direction~II transforms it into a criterion for a class of
operators; Direction~III embeds the framework in its natural algebraic
home.

%% ===================================================================
\section{Direction I: Dirichlet \texorpdfstring{$L$}{L}-functions
  with real characters}
\label{sec:dirichlet}
%% ===================================================================

\subsection{Setup}

Let $\chi$ be a real primitive Dirichlet character modulo $q$ (i.e.,
a Kronecker symbol $\chi = \left(\frac{d}{\cdot}\right)$ for a
fundamental discriminant~$d$). The Weil explicit formula for
$L(s,\chi)$ takes the form
\[
-\sum_{v} W_v^\chi(g * \tilde g)
= \sum_\gamma h(\gamma) + \text{(polar terms)},
\]
where the sum on the left runs over all places $v$ (primes and the
archimedean place), $W_v^\chi$ are the local distributions twisted by
$\chi$, and the sum on the right runs over zeros
$\rho = \tfrac{1}{2} + i\gamma$ of $L(s,\chi)$.

Restricting to test functions supported on $[\lambda^{-1},\lambda]$
(or equivalently on $[-L,L]$ in logarithmic coordinates, with
$L = \log\lambda$) defines a quadratic form $\cE_\lambda^\chi$ and an
associated self-adjoint operator $A_\lambda^\chi$ on $L^2([-L,L])$.

\subsection{How the energy decomposition transfers}

The local distributions decompose as:
\begin{itemize}[nosep]
\item \textbf{Unramified primes with $\chi(p) = +1$}: The
  contribution from $p$ involves the same difference-energy structure
  as the $\zeta(s)$ case:
  \[
  \cE_p^\chi(G,G) = \sum_{k=1}^\infty \frac{1}{kp^k}
  \|\tilde G - S_{k\ell_p}\tilde G\|^2_{L^2},
  \quad \ell_p = \log p.
  \]
  These terms are manifestly non-negative.

\item \textbf{Unramified primes with $\chi(p) = -1$}: The character
  twist replaces the shift $S_{k\ell_p}$ with a \emph{sign-alternating}
  shift. For odd powers of $p$, the term becomes:
  \[
  \frac{1}{kp^k}\|\tilde G + S_{k\ell_p}\tilde G\|^2_{L^2},
  \]
  which is the difference energy for the anti-symmetric combination.
  This is still a non-negative quadratic form, since
  $\|f + g\|^2 = \|f - (-g)\|^2$ is a difference energy with
  reflected shift. The key point is that $\chi(p) = -1$ changes which
  combination appears but does not introduce negative signs in the
  energy.

\item \textbf{Ramified primes} ($p \mid q$, $\chi(p) = 0$): The
  local contribution $\cE_p^\chi$ vanishes. These primes provide no
  energy and no shifts, but they do not obstruct the argument either.

\item \textbf{Archimedean place}: For even characters $\chi(-1) = +1$,
  the archimedean distribution $W_\R^\chi$ is identical to $W_\R$
  for $\zeta(s)$. For odd characters $\chi(-1) = -1$, there is a
  modification involving the digamma function $\psi$ evaluated at a
  shifted argument, but the resulting distribution still gives rise
  to a non-negative continuous superposition of translation-difference
  energies over $t \in (0,2L)$ with a strictly positive weight.
\end{itemize}

\begin{proposition}[Expected]
\label{prop:dirichlet_markov}
For every real primitive character $\chi$ and every $\lambda > 1$, the
quadratic form $\cE_\lambda^\chi$ satisfies the Markov property,
is irreducible (via the archimedean continuum), and has a coercive
Fourier symbol with logarithmic growth. Consequently, the operator
$A_\lambda^\chi$ has compact resolvent, simple lowest eigenvalue, and
a strictly positive eigenfunction with the appropriate symmetry.
\end{proposition}

\subsection{Mathematical consequences}

\subsubsection{Uniform verification of the Connes--van Suijlekom
  hypotheses}

The conditional theorem of Connes--van Suijlekom
\cite{ConnesvanSuijlekom2025} is stated in sufficient generality to
cover operators beyond $\zeta(s)$. Proposition~\ref{prop:dirichlet_markov}
would verify the hypotheses---simplicity and appropriate
symmetry---for the restricted operator associated with every real
Dirichlet $L$-function simultaneously. This gives:

\begin{corollary}[Expected]
\label{cor:real_zeros}
For every real primitive character $\chi$, every $\lambda > 1$, and
the ground state eigenfunction $\xi_\lambda^\chi$ of $A_\lambda^\chi$:
all zeros of the entire function $\hat\xi_\lambda^\chi(z)$ lie on
the real line (equivalently, on the critical line $\re(s) = \tfrac{1}{2}$
after the usual change of variables).
\end{corollary}

This would be a uniform statement across an infinite family of
$L$-functions, established by a single structural argument rather than
case-by-case analysis.

\subsubsection{Implications for Siegel zeros}

Real Dirichlet $L$-functions are precisely the family where
\emph{Siegel zeros}---hypothetical zeros very close to $s = 1$---are
most dangerous and hardest to exclude. The existence of a Siegel zero
for $L(s,\chi)$ would have dramatic consequences for the distribution
of primes in arithmetic progressions.

The energy-decomposition result at finite $\lambda$ does not directly
exclude Siegel zeros, since that would require the $\lambda \to \infty$
convergence. However, it establishes that the finite-approximation
framework is well-posed for every real character: the approximate zeros
lie on the critical line, and the spectral gap $\lambda_2 - \lambda_1$
is strictly positive. If quantitative lower bounds on the spectral gap
could be established (uniformly in $\chi$), these would translate into
zero-free regions for $L(s,\chi)$ near $s = 1$, providing a new
analytic approach to the Siegel zero problem.

\subsubsection{Robustness of the Connes program}

Perhaps most importantly, extending the result to real characters
would demonstrate that the energy-decomposition method is not an
artifact of the special structure of $\zeta(s)$ but is robust across
the family of $L$-functions with Euler products. This matters for the
broader Connes program, which is naturally formulated for all
automorphic $L$-functions via the ad\`ele class space and the
semi-local trace formula. Showing that the analytic step (ground state
simplicity) works uniformly across the real family would be a
meaningful indication that the method has the right level of
generality.

\subsection{Assessment of difficulty}

\textbf{Low to moderate.} The character twist does not fundamentally
alter the energy decomposition. The main work is bookkeeping: tracking
the sign of $\chi(p)$ through the local computations and verifying
that the archimedean term for odd characters retains a non-negative
decomposition. No new ideas appear to be required beyond those in the
original argument.

%% ===================================================================
\section{Direction II: Non-local operators with arithmetic kernels}
\label{sec:arithmetic}
%% ===================================================================

\subsection{Setup}

We consider integral operators on $L^2([-L,L])$ (or on bounded
domains in $\R^d$) whose kernels or Fourier symbols are defined by
arithmetic data. Three natural classes present themselves:

\begin{definition}[Partial Euler product operators]
\label{def:euler_op}
For a finite set of primes $S$ and $s_0 \in \C$, define the Fourier
symbol
\[
\psi_S(\xi) = -\sum_{p \in S}
\log|1 - p^{-s_0 - i\xi}|^2
= -\sum_{p \in S} \sum_{k=1}^\infty \frac{2\cos(k\xi \log p)}{kp^{k\re(s_0)}}.
\]
The associated operator $A_S$ on $L^2([-L,L])$ has this function as
its Fourier symbol (up to boundary corrections from the restriction
to a finite interval).
\end{definition}

\begin{definition}[Epstein zeta operators]
\label{def:epstein_op}
For a positive definite quadratic form $Q$ in $n$ variables, the
Epstein zeta function $Z_Q(s) = \sum'_{\mathbf{m} \in \Z^n}
Q(\mathbf{m})^{-s}$ satisfies a functional equation and defines a
quadratic form via the explicit formula, analogous to the Weil form.
The restricted operator $A_Q^\lambda$ acts on $L^2$ of a bounded
domain.
\end{definition}

\begin{definition}[Rankin--Selberg operators]
\label{def:RS_op}
For two cuspidal automorphic representations $\pi, \pi'$ on
$\GL_n(\A_\Q)$, the Rankin--Selberg $L$-function $L(s, \pi \times
\tilde\pi')$ has an Euler product and an explicit formula. The
associated restricted operator $A_{\pi,\pi'}^\lambda$ is defined by
the corresponding quadratic form.
\end{definition}

\subsection{The energy-decomposition criterion}

The abstract framework (Section~5 of the companion paper
\cite{FracLaplacian2026}) gives a clean criterion: an operator with
Fourier symbol $\psi$ on a bounded domain has a simple ground state
if:
\begin{enumerate}[nosep,label=(H\arabic*)]
\item The symbol admits a representation
  $\psi(\xi) = \int |e^{2\pi i h\xi} - 1|^2 \, d\mu(h)$ for a
  non-negative measure $\mu$.
\item The support of $\mu$ is dense (irreducibility).
\item $\psi(\xi) \to \infty$ as $|\xi| \to \infty$ (coercivity).
\end{enumerate}
The question for each class of arithmetic operators is: which of these
hypotheses hold?

\subsection{Analysis by class}

\subsubsection{Partial Euler product operators}

For partial Euler products at $s_0 = \tfrac{1}{2}$ (the case relevant
to the Riemann Hypothesis), the symbol $\psi_S(\xi)$ is a
\emph{finite} sum of cosines. Writing out:
\[
\psi_S(\xi) = \sum_{p \in S} \sum_{k=1}^\infty
\frac{2(1 - \cos(k\xi\log p))}{kp^{k/2}}
= \int |e^{2\pi i h\xi} - 1|^2 \, d\mu_S(h),
\]
where $\mu_S$ is a discrete measure supported on $\{k\log p : p \in S,
\, k \ge 1\}$. The measure is non-negative (H1~holds), but its support
is discrete and does not generate a dense subgroup of~$\R$ unless
$S$ contains at least two primes with $\log p_1/\log p_2$ irrational
(which is the case for any two distinct primes, since $\log 2 / \log 3$
is irrational). However, with only finitely many primes, the support
is contained in a lattice-like set, and irreducibility requires
careful analysis.

Moreover, $\psi_S(\xi)$ is \emph{bounded} (it is a finite sum of
bounded terms), so $\psi_S(\xi) \not\to \infty$. \textbf{Coercivity
fails.} This means that for finite Euler products without the
archimedean contribution, the operator does not have compact resolvent,
and the energy-decomposition pipeline breaks at Step~4.

This is consistent with the expectation that the archimedean place
plays an essential role. The Weil form includes both the prime
contributions \emph{and} the archimedean term $W_\R$, and it is the
latter that provides logarithmic coercivity. Partial Euler products
alone are insufficient.

\subsubsection{Epstein zeta operators}

Epstein zeta functions $Z_Q(s)$ do \emph{not} in general satisfy the
Riemann Hypothesis---this has been known since Davenport and Heilbronn
\cite{DavenportHeilbronn1936}. The energy-decomposition method must
therefore fail somewhere for Epstein operators. Identifying
\emph{where} it fails is diagnostically valuable.

The explicit formula for $Z_Q$ gives local contributions that, unlike
the Weil form, do not decompose into an Euler product. The quadratic
form associated to $Z_Q$ has a ``global'' structure without a natural
decomposition into independent local pieces. This means:
\begin{itemize}[nosep]
\item Hypothesis (H1) (non-negative energy decomposition) is
  \emph{not guaranteed}. Without an Euler product, there is no reason
  for the quadratic form to decompose as a sum of non-negative
  translation-difference energies. The Markov property may fail.
\item Even if (H1) holds, the ``shifts'' arising from the lattice
  $\Z^n$ of summation in $Z_Q$ are discrete, and irreducibility (H2)
  depends on the geometry of the lattice relative to the domain of
  restriction.
\end{itemize}

\begin{question}
\label{q:epstein}
For which positive definite forms $Q$ does the associated quadratic
form admit a non-negative energy decomposition? Is the failure of
such a decomposition equivalent to (or correlated with) the existence
of zeros off the critical line?
\end{question}

An affirmative answer to the second part of
Question~\ref{q:epstein} would give a new structural explanation for
\emph{why} certain zeta functions satisfy the Riemann Hypothesis and
others do not: the dividing line would be the Markov property of the
associated Dirichlet form, which in turn reflects the presence or
absence of an Euler product.

\subsubsection{Rankin--Selberg operators}

Rankin--Selberg $L$-functions $L(s, \pi \times \tilde\pi')$ have
Euler products:
\[
L(s, \pi \times \tilde\pi') = \prod_p L_p(s, \pi_p \times
\tilde\pi'_p),
\]
where each local factor $L_p$ is an inverse polynomial in $p^{-s}$.
The explicit formula therefore decomposes into local contributions
$W_p^{\pi,\pi'}$, and the energy decomposition should proceed
place-by-place, just as for $\zeta(s)$ and Dirichlet $L$-functions.

The key subtlety is at \emph{ramified primes}: when $\pi$ or $\pi'$
is ramified at $p$, the local factor $L_p$ has a different structure
(fewer terms, modified coefficients), and the corresponding local
energy $\cE_p^{\pi,\pi'}$ may have a more complex form. The question
is whether these local energies remain non-negative quadratic forms.

For $\GL_2$ (classical modular forms), the local factors at ramified
primes are either $1$ (supercuspidal) or $(1 - \alpha_p p^{-s})^{-1}$
(Steinberg), both of which give non-negative local energies (the
former trivially, the latter by the same argument as for unramified
primes with one fewer term). For $\GL_n$ with $n \ge 3$, the
ramified local factors can be more complex, and a case-by-case
analysis is needed.

\subsection{Mathematical consequences}

\subsubsection{Functoriality and the Langlands program}

The Langlands functoriality conjectures predict that for a morphism
of $L$-groups ${}^LG \to {}^LG'$, there is a ``transfer'' of
automorphic representations from $G$ to $G'$, compatible with
$L$-functions. These transfers are often established by analyzing
the analytic properties of Rankin--Selberg $L$-functions.

If the energy-decomposition method verifies ground state simplicity
for the restricted operators associated with Rankin--Selberg
$L$-functions $L(s, \pi \times \tilde\pi')$, this would establish
that the approximate zeros of these $L$-functions lie on the critical
line. While this is a finite-$\lambda$ statement (and hence does not
prove GRH for Rankin--Selberg $L$-functions), it would provide
structural constraints on the analytic behavior of these $L$-functions
that could serve as inputs to functoriality arguments.

More concretely, the spectral gap $\lambda_2 - \lambda_1$ of the
restricted operator encodes information about the spacing of zeros.
Quantitative lower bounds on this spectral gap would give
\emph{zero-density estimates} for Rankin--Selberg $L$-functions,
which are among the most useful analytic inputs to the Langlands
program.

\subsubsection{Subconvexity bounds}

Subconvexity bounds---estimates of the form $|L(\tfrac{1}{2}, \pi)|
\ll C(\pi)^{1/4 - \delta}$ for some $\delta > 0$, where $C(\pi)$ is
the analytic conductor---are central tools in modern analytic number
theory. They have applications to equidistribution of lattice points,
quantum unique ergodicity, and mass equidistribution on arithmetic
surfaces.

The spectral gap of the restricted Weil-type operator is related to
the rate at which approximate zeros separate from the critical line.
If the energy-decomposition method could provide \emph{quantitative}
lower bounds on $\lambda_2 - \lambda_1$ (uniformly in the
representation $\pi$), these would translate into
subconvexity-type estimates for the associated $L$-functions.

This connection is speculative but structurally motivated: the
spectral gap controls the decay rate of correlations in the associated
Markov semigroup, and this decay rate is directly related to the
analytic behavior of $L(s,\pi)$ near the critical line.

\subsubsection{Diagnostic value for Epstein zeta functions}

As noted in Section~\ref{sec:arithmetic}.3.2, the energy-decomposition
method is expected to \emph{fail} for Epstein zeta functions (which
violate the Riemann Hypothesis). Identifying the precise failure
mode---does the Markov property break down? does irreducibility fail?
does coercivity fail?---would give a new structural characterization
of the Euler product as the arithmetic feature that ``protects'' the
Riemann Hypothesis.

\begin{conjecture}[Informal]
\label{conj:diagnostic}
An $L$-function $L(s)$ with functional equation and analytic
continuation satisfies a Riemann Hypothesis if and only if the
associated restricted quadratic form admits a non-negative energy
decomposition (Markov property) with an irreducible shift structure.
The Euler product is the arithmetic source of both properties.
\end{conjecture}

This conjecture is deliberately informal; making it precise would
require specifying the class of $L$-functions under consideration and
the sense in which the energy decomposition is ``associated'' to the
$L$-function. But even as a heuristic, it suggests a fundamentally
new perspective: the Riemann Hypothesis as a \emph{positivity and
irreducibility} statement about Dirichlet forms, rather than a
statement about the location of zeros.

\subsection{Assessment of difficulty}

\textbf{Moderate to high.} The main challenge is that for general
arithmetic kernels, the energy decomposition into non-negative pieces
may not exist. The Euler product structure of $L$-functions is what
makes the Weil form decompose nicely; without it, the local terms may
lack definite sign. Each class of operators requires individual
analysis. The Rankin--Selberg case for $\GL_2$ is likely tractable;
the general $\GL_n$ case and the Epstein diagnostic problem are
substantially harder.

%% ===================================================================
\section{Direction III: Operators on groups}
\label{sec:groups}
%% ===================================================================

\subsection{Setup}

The translation-difference structure at the heart of the
energy-decomposition method generalizes naturally to locally compact
abelian groups. Let $G$ be such a group with Haar measure $\mu$,
Pontryagin dual $\hat G$, and let $K \subset G$ be a compact subset
with $\mu(K) > 0$.

\begin{definition}[Energy form on a group]
\label{def:group_energy}
Given a non-negative L\'evy-type measure $\nu$ on $G$ (i.e., a
Radon measure with $\nu(\{0\}) = 0$ and
$\int_G \min(1, d(g,0)^2) \, d\nu(g) < \infty$ for some metric $d$),
define
\[
\cE_\nu(u,u) = \int_G \|\tau_g \tilde u - \tilde u\|_{L^2(G)}^2
\, d\nu(g),
\]
where $\tilde u$ is the extension of $u \in L^2(K)$ by zero and
$\tau_g$ is translation by $g$.
\end{definition}

The associated Fourier symbol on $\hat G$ is the continuous negative
definite function
\[
\psi(\hat g) = \int_G |1 - \hat g(g)|^2 \, d\nu(g),
\quad \hat g \in \hat G.
\]
The abstract criterion (hypotheses H1--H3 of
Section~\ref{sec:arithmetic}.2) becomes:
\begin{enumerate}[nosep, label=(H\arabic*${}'$)]
\item $\nu$ is a non-negative measure (automatic from the definition).
\item The support of $\nu$ generates a dense subgroup of $G$
  (irreducibility).
\item $\psi(\hat g) \to \infty$ as $\hat g \to \infty$ in $\hat G$
  (coercivity).
\end{enumerate}

\subsection{Natural settings}

\subsubsection{The id\`ele class group}

The most natural home for the Weil distribution is the id\`ele class
group $C_\Q = \A_\Q^*/\Q^*$, where $\A_\Q$ is the ad\`ele ring of
$\Q$. Via the logarithmic map on the archimedean component,
$C_\Q \cong \R_{>0} \times \prod_p \Z_p^*$ (as a topological group
modulo the discrete action of $\Q^*$). The Weil distribution is
naturally a distribution on $C_\Q$, and the local decomposition
\[
W = \sum_p W_p + W_\R
\]
reflects the factorization of $C_\Q$ into local components.

Working directly on $C_\Q$ rather than projecting to $\R$ would give
a more canonical formulation of the energy decomposition. The
``shifts'' by $\log p$ and by the archimedean continuum become
translations within $C_\Q$, and the irreducibility question becomes:
does the support of the Weil distribution generate a dense subgroup of
$C_\Q$?

The answer is yes: the elements $\{p\}_{p \text{ prime}}$, viewed in
$C_\Q$, generate a dense subgroup (this is essentially the content of
the Artin reciprocity law for $\Q$). Combined with the archimedean
component, the full support of the Weil distribution generates all of
$C_\Q$.

\subsubsection{Adelic groups}

For automorphic $L$-functions on $\GL_n$, the natural setting is the
quotient $\GL_n(\A_\Q) / \GL_n(\Q)$. Automorphic forms are functions
on this quotient, and the trace formula---the principal tool for
studying their spectral properties---is fundamentally a statement
about operators on this space.

The energy-decomposition framework would need to be reformulated for
\emph{non-abelian} groups, since $\GL_n$ for $n \ge 2$ is not
abelian. The translations $\tau_g$ would be replaced by the regular
representation, and the ``Fourier symbol'' would be replaced by the
spectral decomposition into automorphic representations. The Markov
property, irreducibility, and coercivity would need non-commutative
analogues.

\subsubsection{$p$-adic groups}

The local distributions $W_p$ are naturally defined on $\Q_p^*$
(the multiplicative group of $p$-adic numbers), and the local energy
$\cE_p$ is a Dirichlet form on $L^2(\Q_p^*)$. Studying these local
forms individually, before taking the product over places, could
provide insight into the local structure of the energy decomposition.

The $p$-adic groups $\Q_p^*$ are totally disconnected, which makes
the analysis of Dirichlet forms rather different from the archimedean
case. The ``shifts'' are multiplication by powers of $p$, which act
on the ultrametric space $\Q_p^*$ in a way that has no Euclidean
analogue. Nevertheless, the theory of Dirichlet forms on ultrametric
spaces has been developed (see \cite{AlbrechtoBogachevRockner2009}
and references therein), and could provide the necessary tools.

\subsection{Mathematical consequences}

\subsubsection{Adelic Perron--Frobenius theory}

If the semigroup generated by the Weil form on the id\`ele class group
(or more generally on $\GL_n(\A)/\GL_n(\Q)$) is positivity-improving
in an appropriate sense, this would give a \emph{representation-theoretic}
proof of ground state simplicity. The ground state would be the
spherical (unramified) vector in the trivial automorphic
representation, and simplicity would reflect the
\emph{multiplicity-one theorem} for automorphic representations of
$\GL_n$ \cite{Shalika1974,PiatetskiShapiro1979}.

This would be a deep connection: the multiplicity-one theorem is
proved by purely algebraic/representation-theoretic methods (local
uniqueness of Whittaker models), while ground state simplicity is
proved by analytic methods (Dirichlet forms and Perron--Frobenius).
If these two results could be identified as different manifestations
of the same underlying structure, it would unify two of the most
important tools in the theory of automorphic forms.

\begin{conjecture}[Adelic Perron--Frobenius]
\label{conj:adelic_PF}
Let $\pi$ be a cuspidal automorphic representation of $\GL_n(\A_\Q)$,
and let $\cE_\lambda^\pi$ be the quadratic form obtained from the
explicit formula for $L(s,\pi)$ restricted to a cutoff $\lambda$.
Then $\cE_\lambda^\pi$ is an irreducible Dirichlet form, and the
corresponding operator has a simple ground state if and only if $\pi$
satisfies multiplicity one.
\end{conjecture}

\subsubsection{Trace formula interpretation}

The Arthur--Selberg trace formula equates spectral data (automorphic
representations) with geometric data (orbital integrals):
\[
\sum_\pi m(\pi) \, \mathrm{tr} \, \pi(f) = \sum_{\{\gamma\}}
\mathrm{vol}(G_\gamma(\Q) \backslash G_\gamma(\A)) \cdot
\cO_\gamma(f),
\]
where the left side sums over automorphic representations with
multiplicity $m(\pi)$ and the right side sums over conjugacy classes
$\{\gamma\}$ with orbital integrals $\cO_\gamma(f)$.

The energy decomposition of the Weil form, lifted to the adelic
setting, would give a new interpretation of the \emph{spectral side}
of the trace formula:
\begin{itemize}[nosep]
\item The ground state simplicity at finite cutoff corresponds to the
  \textbf{isolation of the trivial representation} in the spectral
  decomposition.
\item The spectral gap $\lambda_2 - \lambda_1$ encodes information
  about the \textbf{first non-trivial automorphic representation}
  (the one closest to the trivial representation in the spectral
  ordering).
\item The rate of convergence of the spectral gap as
  $\lambda \to \infty$ would be controlled by the
  \textbf{Ramanujan conjecture} (which bounds the local components
  of cuspidal representations).
\end{itemize}

This perspective suggests that the energy-decomposition method could
provide a new approach to the Ramanujan conjecture, at least in the
form of bounds on the spectral gap. The Ramanujan conjecture for
$\GL_2$ over $\Q$ is known (by the work of Deligne on the Weil
conjectures for varieties over finite fields
\cite{Deligne1974}), but for $\GL_n$ with $n \ge 3$ it remains open
in general.

\subsubsection{Non-commutative geometry}

Connes' broader program places the Riemann Hypothesis within the
framework of non-commutative geometry, where the ad\`ele class space
$\A_\Q / \Q^*$ is treated as a ``non-commutative space'' and the
zeros of $\zeta(s)$ are realized as the spectrum of a suitable
operator.

The Dirichlet form framework has non-commutative analogues: the
Cipriani--Sauvageot theory of \emph{non-commutative Dirichlet forms}
\cite{CiprianiSauvageot2003} extends the classical theory to the
setting of von Neumann algebras and $C^*$-algebras. Extending the
energy-decomposition method to this setting would connect to Connes'
spectral realization of zeros on the non-commutative ad\`ele class
space.

Specifically, the non-commutative Dirichlet form associated to the
``Weil operator'' on the ad\`ele class space would encode both the
spectral properties of $\zeta(s)$ and the positivity/irreducibility
structure that implies ground state simplicity. The Markov property in
the non-commutative setting is related to \emph{complete positivity}
of the associated semigroup, and irreducibility is related to
\emph{ergodicity} of the corresponding non-commutative dynamical
system.

\subsection{Assessment of difficulty}

\textbf{High to very high.} The principal obstacles are:
\begin{enumerate}[nosep]
\item \textbf{Compactness on groups.} The Kolmogorov--Riesz
  compactness criterion is specific to $\R^d$. Extending it to
  locally compact abelian groups (especially those with totally
  disconnected components, such as $\prod_p \Z_p^*$) requires
  developing new compactness criteria adapted to the group topology.
\item \textbf{Coercivity on non-Euclidean duals.} The ``growth at
  infinity'' of the Fourier symbol $\psi(\hat g)$ must be formulated
  in terms of the topology of $\hat G$, which for adelic groups is
  a restricted direct product with a complicated structure at infinity.
\item \textbf{Non-commutativity.} For $\GL_n$ with $n \ge 2$, the
  group is non-abelian, and the Pontryagin dual is replaced by the
  unitary dual (the set of equivalence classes of irreducible unitary
  representations). The ``Fourier symbol'' becomes an operator-valued
  function on this dual, and the Dirichlet form framework must be
  replaced by its non-commutative analogue.
\item \textbf{Convergence as $\lambda \to \infty$.} Even in the
  abelian case, the convergence of ground states and eigenvalues as
  the cutoff $\lambda \to \infty$ is a hard problem that requires
  controlling the growth of the operator in the limit. This is the
  ``second missing step'' in the Connes program, independent of
  (and almost certainly harder than) the ground state simplicity
  question.
\end{enumerate}

%% ===================================================================
\section{The three directions together}
\label{sec:synthesis}
%% ===================================================================

\subsection{Concentric structure}

The three directions form concentric circles of a single program:

\begin{center}
\renewcommand{\arraystretch}{1.4}
\begin{tabular}{@{}p{2.5cm}p{4.8cm}p{5.5cm}@{}}
\toprule
\textbf{Direction} & \textbf{Role} & \textbf{Key question} \\
\midrule
I. Real $\chi$ &
Proof of concept: validates the method across a family &
Does the character twist preserve non-negativity? \\
II. Arithmetic kernels &
Criterion: characterizes which operators have simple ground states &
Which arithmetic structures yield Markov property + irreducibility? \\
III. Groups &
Foundation: embeds the framework in its natural algebraic home &
Does Perron--Frobenius on adelic groups connect to multiplicity one? \\
\bottomrule
\end{tabular}
\end{center}

\subsection{The unified vision}

If all three extensions were carried out, the combined result would
amount to the following:

\begin{quote}
\emph{There exists a Perron--Frobenius theory for operators on adelic
groups, in which ground state simplicity is equivalent to
irreducibility of the associated Dirichlet form, and this
irreducibility is guaranteed by the Euler product structure of
automorphic $L$-functions.}
\end{quote}

This would have four major consequences:

\subsubsection{RH-type results as consequences of representation theory}

The chain
\[
\text{Euler product} \implies \text{non-negative energy decomposition}
\implies \text{Markov property}
\]
\[
\implies \text{irreducibility} \implies \text{positivity-improving}
\implies \text{simple ground state}
\]
\[
\implies \text{approximate zeros on critical line}
\]
would establish that the location of approximate zeros is a
\emph{consequence of the multiplicative structure of
$L$-functions}---that is, of the Euler product, which is the
arithmetic incarnation of the local-global principle. Combined with
the multiplicity-one connection
(Conjecture~\ref{conj:adelic_PF}), this would give a
representation-theoretic explanation for why RH-type statements are
true: they are shadow of the uniqueness of automorphic forms.

\subsubsection{A diagnostic for the Riemann Hypothesis}

For objects without Euler products (Epstein zeta functions, linear
combinations of $L$-functions, Beurling zeta functions), the energy
decomposition would fail, and the \emph{specific failure
mode}---which hypothesis (H1), (H2), or (H3) breaks down---would
explain which zeros leave the critical line and why:

\begin{center}
\renewcommand{\arraystretch}{1.3}
\begin{tabular}{@{}p{4cm}p{4cm}p{4.5cm}@{}}
\toprule
\textbf{Failure mode} & \textbf{Consequence} & \textbf{Example} \\
\midrule
(H1) Markov fails: energy decomposition has indefinite terms &
Ground state may not be positive; semigroup not positivity-preserving &
Epstein zeta with no Euler product \\
(H2) Irreducibility fails: shifts too sparse &
Ground state may not be unique; degenerate eigenspaces possible &
Partial Euler product with too few primes \\
(H3) Coercivity fails: symbol bounded &
Resolvent not compact; continuous spectrum possible &
Finite Euler product without archimedean term \\
\bottomrule
\end{tabular}
\end{center}

\subsubsection{A new analytic foundation for the Connes program}

Currently, the Connes program rests on two principal tools: the
semi-local trace formula and the spectral realization of zeros. The
energy-decomposition/Dirichlet-form approach would add a third
component---\emph{positivity and irreducibility in the sense of Markov
processes}---that provides the ``missing step'' not just for $\zeta(s)$
but systematically across all $L$-functions with Euler products.

This three-legged structure (trace formula + spectral realization +
Dirichlet form theory) would be more robust than any two components
alone, because the Dirichlet form approach provides the
\emph{analytic} mechanism (positivity-improving semigroup) that
converts the \emph{algebraic} input (Euler product) into the
\emph{spectral} conclusion (ground state simplicity).

\subsubsection{A bridge between analytic number theory and PDE}

The fractional Laplacian paper \cite{FracLaplacian2026} demonstrates
that the same abstract machinery governs both the Weil quadratic form
and the fractional Laplacian. A fully developed theory would mean that
techniques from non-local PDE---regularity theory, heat kernel
estimates, spectral gap bounds, Harnack inequalities---could be
imported into number theory, and vice versa.

This kind of cross-pollination has historical precedent: the Selberg
trace formula was inspired by the spectral theory of the Laplacian on
hyperbolic surfaces; the Langlands program draws on the
representation theory of Lie groups; and the proof of the Weil
conjectures by Deligne used \'etale cohomology, a tool imported from
algebraic topology. The energy-decomposition method suggests a new
channel of communication: between the theory of non-local operators
(fractional Laplacians, L\'evy processes, Dirichlet forms on metric
measure spaces) and the analytic theory of $L$-functions.

Specific techniques that could transfer include:
\begin{itemize}[nosep]
\item \textbf{Spectral gap estimates} from the theory of Dirichlet
  forms on fractals and metric measure spaces, adapted to give
  quantitative bounds on $\lambda_2 - \lambda_1$ for
  arithmetic operators.
\item \textbf{Heat kernel bounds} (Li--Yau type, or sub-Gaussian
  estimates) for the semigroup generated by the Weil operator,
  translating into zero-density estimates for $L$-functions.
\item \textbf{Functional inequalities} (Poincar\'e, log-Sobolev,
  Nash) for arithmetic Dirichlet forms, giving new analytic inputs to
  the study of $L$-functions near the critical line.
\end{itemize}

\subsection{Principal caveats}

The vision described above is contingent on assumptions that are not
yet verified. The most important caveats are:

\begin{enumerate}[nosep]
\item \textbf{Non-negativity of the energy decomposition.} The entire
  framework depends on the quadratic form decomposing into
  non-negative pieces. For $\zeta(s)$ and real Dirichlet
  $L$-functions, this appears to hold. For general automorphic
  $L$-functions, especially at ramified primes, it is unproven and
  could fail. The boundary of where non-negativity holds would itself
  be an important discovery.

\item \textbf{The convergence problem.} Even if ground state
  simplicity is established for every finite cutoff $\lambda$, the
  full Riemann Hypothesis requires $\lambda \to \infty$. The
  convergence of approximate zeros to actual zeros is a separate and
  almost certainly harder problem, identified by Connes as the other
  ``missing step.'' The energy-decomposition method, as currently
  formulated, does not address this.

\item \textbf{Non-commutative obstacles.} The extension to non-abelian
  groups ($\GL_n$ for $n \ge 2$) is genuinely difficult and may
  require new ideas beyond the scope of classical Dirichlet form
  theory. The Cipriani--Sauvageot framework provides a starting point,
  but substantial development would be needed.

\item \textbf{Quantitative vs.\ qualitative.} The current method gives
  \emph{qualitative} results (the ground state is simple; the
  eigenfunction is positive) but not \emph{quantitative} ones (how
  large is the spectral gap? how positive is the eigenfunction?). For
  the deeper applications (subconvexity bounds, Ramanujan conjecture,
  zero-density estimates), quantitative control is essential and
  would require new techniques beyond the abstract
  Perron--Frobenius framework.
\end{enumerate}

%% ===================================================================
\section{Summary}
\label{sec:summary}
%% ===================================================================

The energy-decomposition method, in its current form, resolves a
specific open problem (ground state simplicity for the restricted Weil
operator) by a specific route (Dirichlet form theory and
Perron--Frobenius). But the structural features of the
argument---non-negative decomposition, irreducibility via rich shift
structures, coercivity via Fourier growth---are not tied to the
Riemann zeta function. They suggest a broader framework in which:

\begin{itemize}[nosep]
\item Direction~I (real characters) validates the method's robustness
  and provides the first uniform result across a family of
  $L$-functions, with potential implications for the Siegel zero
  problem.

\item Direction~II (arithmetic kernels) transforms the method into a
  diagnostic tool that characterizes which $L$-functions ``should''
  satisfy the Riemann Hypothesis based on the Markov/irreducibility
  properties of their associated Dirichlet forms, with connections to
  functoriality and subconvexity.

\item Direction~III (operators on groups) embeds the framework in the
  Langlands program and non-commutative geometry, potentially
  connecting ground state simplicity to multiplicity-one theorems and
  providing a new analytic foundation for the Connes program.
\end{itemize}

The full vision---an adelic Perron--Frobenius theory connecting Euler
products, Dirichlet form irreducibility, and automorphic multiplicity
one---is ambitious and largely conjectural. But each step along the
way (starting with Direction~I, which appears to require minimal new
ideas) would be a concrete mathematical contribution, and the overall
direction suggests a genuine structural insight: the Riemann
Hypothesis, at its core, may be a statement about positivity and
irreducibility.

\begin{thebibliography}{99}

\bibitem{AlbrechtoBogachevRockner2009}
S.~Albeverio, V.I.~Bogachev, M.~R\"ockner, Dirichlet form methods
in the theory of Kolmogorov operators and generalized stochastic
Burgers equations, in: \emph{Stochastic PDE and Applications}, CIME
Lecture Notes, Springer, 2009.

\bibitem{CiprianiSauvageot2003}
F.~Cipriani, J.-L.~Sauvageot, Derivations as square roots of
Dirichlet forms, \emph{J.\ Funct.\ Anal.}\ \textbf{201} (2003),
78--120.

\bibitem{Connes2026}
A.~Connes, The Riemann Hypothesis: Past, Present and a Letter Through
Time, arXiv:2602.04022 (2026).

\bibitem{ConnesConsaniMoscovici2025}
A.~Connes, C.~Consani, H.~Moscovici, Zeta Spectral Triples,
arXiv:2511.22755 (2025).

\bibitem{ConnesvanSuijlekom2025}
A.~Connes, W.\,D.~van Suijlekom, Quadratic Forms, Real Zeros and
Echoes of the Spectral Action, \emph{Commun.\ Math.\ Phys.}\
\textbf{406}, 12 (2025).

\bibitem{DavenportHeilbronn1936}
H.~Davenport, H.~Heilbronn, On the zeros of certain Dirichlet
series, \emph{J.\ London Math.\ Soc.}\ \textbf{11} (1936), 181--185.

\bibitem{Deligne1974}
P.~Deligne, La conjecture de Weil.~I, \emph{Inst.\ Hautes \'Etudes
Sci.\ Publ.\ Math.}\ \textbf{43} (1974), 273--307.

\bibitem{FracLaplacian2026}
Ground State Simplicity via Energy Decomposition: A Dirichlet Form
Proof for the Fractional Laplacian, with a View Toward the Weil
Quadratic Form (2026). Companion paper.

\bibitem{FukushimaOshimaTakeda2011}
M.~Fukushima, Y.~\=Oshima, M.~Takeda, \emph{Dirichlet Forms and
Symmetric Markov Processes}, 2nd ed., de~Gruyter, 2011.

\bibitem{PiatetskiShapiro1979}
I.\,I.~Piatetski-Shapiro, Multiplicity one theorems, in:
\emph{Automorphic Forms, Representations and $L$-functions}, Proc.\
Sympos.\ Pure Math.\ \textbf{33}, Part~1, AMS, 1979, 209--212.

\bibitem{Shalika1974}
J.\,A.~Shalika, The multiplicity one theorem for $\GL_n$, \emph{Ann.\
of Math.}\ \textbf{100} (1974), 171--193.

\end{thebibliography}

\end{document}
