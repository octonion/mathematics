\documentclass[11pt]{article}
\usepackage[margin=1in]{geometry}
\usepackage{amsmath,amssymb,amsthm,mathtools}
\usepackage{hyperref}
\usepackage{enumitem}
\usepackage{tikz}
\usetikzlibrary{arrows.meta,positioning,shapes.geometric,calc,fit,backgrounds}

\newtheorem{theorem}{Theorem}
\newtheorem{proposition}[theorem]{Proposition}
\newtheorem{lemma}[theorem]{Lemma}
\newtheorem{corollary}[theorem]{Corollary}
\theoremstyle{remark}
\newtheorem{remark}[theorem]{Remark}
\newtheorem{definition}[theorem]{Definition}
\newcommand{\R}{\mathbb{R}}
\newcommand{\C}{\mathbb{C}}
\newcommand{\1}{\mathbf{1}}
\DeclareMathOperator{\supp}{supp}
\DeclareMathOperator{\Re}{Re}

\title{Verification and Structural Analysis of\\
``Energy-Decomposition and Perron--Frobenius Consequences\\
for the Restricted Weil Quadratic Form''}
\author{}
\date{}

\begin{document}
\maketitle

\begin{abstract}
We present a detailed verification of the mathematical logic in the paper
\emph{Energy-Decomposition and Perron--Frobenius Consequences for the Restricted
Weil Quadratic Form}, summarise its logical flow, provide a dependency flowchart
connecting all sections and results, and catalogue the external (unproven)
assumptions on which the argument rests.
\end{abstract}

\tableofcontents
\newpage

%% ===================================================================
\section{Overview of the paper's goal}
%% ===================================================================

The paper studies the quadratic form that arises in the spectral approach to
Weil's explicit-formula criterion, restricted to test functions supported on a
compact multiplicative interval $[\lambda^{-1},\lambda]\subset\R_+^*$.  Its
main conclusion is:

\medskip
\noindent\textbf{Main Result.}
\emph{The self-adjoint operator $A_\lambda$ associated with the restricted Weil
quadratic form has compact resolvent, and its lowest eigenvalue is simple with an
eigenfunction that is strictly positive a.e.\ and even (symmetric under
$u\mapsto -u$ in logarithmic coordinates).}

\medskip
The argument proceeds in five stages: (1) energy decomposition, (2) the Markov
property, (3) irreducibility, (4) operator-theoretic realisation, and (5) the
Perron--Frobenius conclusion with evenness.

%% ===================================================================
\section{Verification of mathematical correctness}
%% ===================================================================

We verify each proof in the paper.  In all cases the logic is
\textbf{correct}.

%% -------------------------------------------------------------------
\subsection{Lemma 1: Convolution inner-product identity}
%% -------------------------------------------------------------------

The computation
\[
  (g*g^*)(a)
  = \int g(y)\,\overline{g(y/a)}\,d^*y
  = \langle g, U_a g\rangle_{L^2(d^*x)}
\]
is a direct unfolding of the definitions of multiplicative convolution,
involution $g^*(x)=\overline{g(x^{-1})}$, and the dilation operator
$U_a g(x)=g(x/a)$.  The relation $f(a^{-1})=\overline{f(a)}$ follows by
substituting $a\mapsto a^{-1}$ and complex-conjugating.  \textbf{Correct.}

%% -------------------------------------------------------------------
\subsection{Lemma 2: Basic unitary identity}
%% -------------------------------------------------------------------

The identity
\[
  2\Re\langle h,Uh\rangle = 2\|h\|^2 - \|h-Uh\|^2
\]
follows from expanding $\|h-Uh\|^2 = \|h\|^2 + \|Uh\|^2 - 2\Re\langle h,Uh\rangle$
and using unitarity $\|Uh\|=\|h\|$.  \textbf{Correct.}

%% -------------------------------------------------------------------
\subsection{Remark 3: Support truncation}
%% -------------------------------------------------------------------

If $\supp(g)\subset[\lambda^{-1},\lambda]$ and $a>\lambda^2$, then
$\supp(U_a g) = a\cdot\supp(g) \subset [a\lambda^{-1},a\lambda]$, which is
disjoint from $[\lambda^{-1},\lambda]$ since $a\lambda^{-1}>\lambda$.
Hence $\langle g,U_a g\rangle=0$.  This correctly makes all sums and integrals
finite.  \textbf{Correct.}

%% -------------------------------------------------------------------
\subsection{Lemma 3: Prime energy decomposition}
%% -------------------------------------------------------------------

The proof chains Lemma~1 ($f(p^m)+f(p^{-m})=2\Re\langle g,U_{p^m}g\rangle$)
and Lemma~2 (rewriting $2\Re\langle g,U_{p^m}g\rangle$ as
$2\|g\|_2^2-\|g-U_{p^m}g\|_2^2$) into the definition of $W_p$.  The sign
arithmetic is correct: the negative sign on $W_p$ absorbs the negative sign
from Lemma~2, yielding a \emph{positive} sum of difference norms plus a
constant.  Terms with $p^m>\lambda^2$ vanish by Remark~3.
\textbf{Correct.}

%% -------------------------------------------------------------------
\subsection{Lemma 4: Archimedean energy decomposition}
%% -------------------------------------------------------------------

The change of variables $x=e^t$ transforms $W_\R(f)$ into an integral
over $t\in(0,\infty)$ with weight $w(t)=e^{t/2}/(2\sinh t)$.  The same
chain (Lemma~1 then Lemma~2) is applied.

The integral is split at $t=2L$:
\begin{itemize}[nosep]
\item For $t>2L$: the supports of $\widetilde G$ and $S_t\widetilde G$ are
  disjoint, so
  $\|\widetilde G - S_t\widetilde G\|^2 = \|\widetilde G\|^2+\|S_t\widetilde G\|^2 = 2\|G\|^2$
  (not zero --- the cross-terms vanish).  The integrand becomes
  $2e^{-t/2}\|G\|^2$, and the tail integral converges since
  $w(t)\sim e^{-t/2}/2$ for large $t$, giving $2e^{-t/2}w(t)\sim e^{-t}$.
\item For $t\in[0,2L]$: the ``remainder'' term $2(e^{-t/2}-1)w(t)$ is
  integrable near $t=0$ because $w(t)\sim 1/(2t)$ and
  $e^{-t/2}-1\sim -t/2$, giving an $O(1)$ integrand.
\end{itemize}
All constant contributions are absorbed into the finite constant
$c_\infty(\lambda)$.  \textbf{Correct.}

%% -------------------------------------------------------------------
\subsection{Lemma 6: Markov property}
%% -------------------------------------------------------------------

For a $1$-Lipschitz $\Phi$ with $\Phi(0)=0$, the pointwise bound
$|\Phi(\widetilde G(u))-\Phi(\widetilde G(u-t))|\le|\widetilde G(u)-\widetilde G(u-t)|$
is integrated over $u$, giving
$\|\widetilde{\Phi\circ G}-S_t\widetilde{\Phi\circ G}\|_2^2 \le \|\widetilde G-S_t\widetilde G\|_2^2$.
Integrating/summing against non-negative weights preserves the inequality.  The
condition $\Phi(0)=0$ ensures $\widetilde{\Phi\circ G}=\Phi\circ\widetilde G$
(the extension by zero is respected).  \textbf{Correct.}

%% -------------------------------------------------------------------
\subsection{Lemma 7: Translation-invariance forces null or conull}
%% -------------------------------------------------------------------

The proof uses a standard mollifier argument.  If $\1_B(u)=\1_B(u-t)$ a.e.\
on $I\cap(I+t)$ for all $t\in(0,\varepsilon)$, then for any compact
$J\Subset I$ and $\eta$ small enough, the mollification $f_\eta=\1_B*\rho_\eta$
is smooth on $J_\eta$ and inherits translation invariance there, hence is
constant by connectedness.  Letting $\eta\downarrow 0$ forces $\1_B$ to be
a.e.\ constant on $J$, and since $J$ was arbitrary, on all of~$I$.
\textbf{Correct.}

%% -------------------------------------------------------------------
\subsection{Lemma 8: Indicator-energy criterion}
%% -------------------------------------------------------------------

The key technical detail is the upgrade from ``for a.e.\ $t\in(0,2L)$'' to
``for all $t\in(0,2L)$'': the map
$t\mapsto\|\phi-S_t\phi\|_{L^2(\R)}^2$ is continuous by strong continuity of
the translation group on $L^2(\R)$.  A continuous function that vanishes a.e.\
vanishes everywhere.  Then Lemma~7 applies.  \textbf{Correct.}

%% -------------------------------------------------------------------
\subsection{Fourier analysis (Lemmas 9--14)}
%% -------------------------------------------------------------------

The Plancherel identity for translation differences,
$\|\phi-S_t\phi\|^2 = (2\pi)^{-1}\int 4\sin^2(\xi t/2)\,|\hat\phi(\xi)|^2\,d\xi$,
is standard.  Tonelli's theorem permits interchanging the $\xi$-integral with
the $t$-integration and finite prime sums (all integrands are non-negative).

The lower bound $w(t)\ge c_0/t$ for $t\in(0,t_0]$ uses $\sinh t\le te^t$
(valid since both sides vanish at $t=0$ and the derivative of $\sinh t$ is
$\cosh t\le (1+t)e^t = (te^t)'$).

The logarithmic growth $\psi_\lambda(\xi)\ge c_1\log|\xi|-c_2$ is established
by a standard interval-counting argument: on intervals $J_n$ where
$\sin^2(\xi t/2)\ge 1/2$, the integral of $(1/t)\sin^2(\xi t/2)$ over
$[0,t_0]$ is bounded below by a harmonic-type sum that grows as $\log|\xi|$.
\textbf{Correct.}

%% -------------------------------------------------------------------
\subsection{Closedness and compact embedding (Props.\ 11--12, Prop.\ 17)}
%% -------------------------------------------------------------------

Closedness of $\mathcal E_\lambda^\R$ on $L^2(\R)$ follows from its
identification as a multiplication operator (by $\psi_\lambda$) in Fourier
space; the domain is isometric to a weighted $L^2$ space and hence complete.
Restriction to the closed subspace $H_I$ preserves closedness.

For compact embedding: the set $\mathcal K_M$ (form-norm ball) satisfies
\emph{tightness} automatically (all functions are supported in $\bar I$) and
\emph{translation equicontinuity} by a Fourier splitting argument exploiting
the logarithmic frequency-moment bound (Corollary~15).  The Kolmogorov--Riesz
criterion then gives relative compactness.  \textbf{Correct.}

%% -------------------------------------------------------------------
\subsection{Compact resolvent (Theorem 18)}
%% -------------------------------------------------------------------

If $\{f_n\}$ is bounded in $L^2(I)$ and $u_n=(A_\lambda+1)^{-1}f_n$, then
\[
  \mathcal E_\lambda(u_n)+\|u_n\|_2^2
  = \langle f_n,u_n\rangle
  \le \|f_n\|_2\|u_n\|_2,
\]
giving $\|u_n\|_2\le\|f_n\|_2$ and hence $\|u_n\|_{\mathcal D}^2\le\|f_n\|_2^2$.
Compact embedding (Prop.~17) extracts a convergent subsequence.
\textbf{Correct.}

%% -------------------------------------------------------------------
\subsection{Evenness (Corollary 22)}
%% -------------------------------------------------------------------

Since $A_\lambda$ commutes with $R\colon G(u)\mapsto G(-u)$ (the interval $I$
and all weights are symmetric), $R\psi$ is an eigenfunction for the same lowest
eigenvalue.  Strict positivity of both $\psi$ and $R\psi$ forces
$R\psi=c\psi$ with $c>0$; unitarity of~$R$ (specifically
$\|R\psi\|=\|\psi\|$) yields $c=1$.
\textbf{Correct.}

%% ===================================================================
\section{Summary of the logical flow}
%% ===================================================================

The argument proceeds in five stages.

\subsection*{Stage 1: Energy decomposition (\S\S\,1--4)}

Starting from the two ``input formulas'' $W_p$ (prime local distribution) and
$W_\R$ (archimedean local distribution), taken as given from the theory of the
Weil explicit formula, the paper uses the convolution inner-product identity
(Lemma~1) and the basic unitary identity (Lemma~2) to rewrite
\[
  -\sum_v W_v(g*g^*)
\]
as a \emph{positive} combination of translation-difference energies
$\|\widetilde G - S_t\widetilde G\|_{L^2(\R)}^2$ in logarithmic coordinates,
plus an additive constant multiple of $\|G\|_2^2$ (which merely shifts the
spectrum).  This is assembled into the global quadratic form
$\mathcal E_\lambda(G)$ (Definition~5).

\subsection*{Stage 2: Markov property (\S\,5)}

The difference-energy structure directly gives the normal-contraction inequality
$\mathcal E_\lambda(\Phi\circ G)\le\mathcal E_\lambda(G)$ (Lemma~6), which
ensures the associated semigroup is positivity preserving.

\subsection*{Stage 3: Irreducibility (\S\S\,6--7)}

The archimedean \emph{continuum} of shifts---the integral over
$t\in(0,2L)$ with strictly positive weight $w(t)>0$---is the key ingredient.
If $\mathcal E_\lambda(\1_B)=0$, then $\1_B$ is translation-invariant for all
small~$t$ (upgraded from a.e.\ to everywhere by continuity, Lemma~8).  A
mollifier argument (Lemma~7) then forces $B$ to be null or conull.  Via the
Beurling--Deny/Fukushima theory of symmetric Dirichlet forms (an external
result), this gives irreducibility of the semigroup (Corollary~19).

\subsection*{Stage 4: Operator-theoretic realisation (\S\,7.2)}

The form $\mathcal E_\lambda$ is shown to be closed via its Fourier multiplier
representation (the symbol is $\psi_\lambda(\xi)$).  The logarithmic growth
$\psi_\lambda(\xi)\gtrsim\log|\xi|$ provides enough coercivity for compact
embedding of the form domain into $L^2(I)$ (via the Kolmogorov--Riesz
criterion), hence compact resolvent of $A_\lambda$ (Theorem~18).

\subsection*{Stage 5: Perron--Frobenius conclusion (\S\S\,8--9)}

Combining positivity preservation (Stage~2), irreducibility (Stage~3), and
holomorphy (automatic from self-adjointness) yields \emph{positivity improving}
via an external theorem of Arendt \emph{et~al.}  With compact resolvent, the
Krein--Rutman/Perron--Frobenius theorem gives simplicity and strict positivity
of the ground state (Proposition~20).  Finally, $A_\lambda$ commutes with the
reflection $R\colon G(u)\mapsto G(-u)$ (Proposition~21), which forces the
unique positive eigenfunction to be even (Corollary~22).

%% ===================================================================
\section{Dependency flowchart}
%% ===================================================================

Figure~\ref{fig:flowchart} displays the logical dependencies among all results
in the paper.  Nodes are colour-coded as follows:

\begin{center}
\begin{tikzpicture}[font=\small]
  \fill[red!15] (0,0) rectangle (0.4,0.3);
  \node[anchor=west] at (0.55,0.15) {Input formula (from analytic number theory)};
  \fill[green!15] (0,-0.5) rectangle (0.4,-0.2);
  \node[anchor=west] at (0.55,-0.35) {Result proved in the paper};
  \fill[orange!20] (0,-1.0) rectangle (0.4,-0.7);
  \node[anchor=west] at (0.55,-0.85) {External theorem (cited, not proved)};
  \fill[blue!15] (0,-1.5) rectangle (0.4,-1.2);
  \node[anchor=west] at (0.55,-1.35) {Main conclusion};
\end{tikzpicture}
\end{center}

\begin{figure}[p]
\centering
\resizebox{\textwidth}{!}{%
\begin{tikzpicture}[
    node distance=0.6cm and 0.4cm,
    >=Stealth,
    every node/.style={font=\scriptsize, align=center, text width=2.6cm, minimum height=0.85cm},
    input/.style={rectangle, rounded corners, draw=red!70!black, fill=red!15, thick},
    proved/.style={rectangle, rounded corners, draw=green!60!black, fill=green!12, thick},
    external/.style={rectangle, rounded corners, draw=orange!80!black, fill=orange!18, thick},
    conclusion/.style={rectangle, rounded corners, draw=blue!70!black, fill=blue!15, thick, line width=1.2pt},
    arr/.style={->, thick, >=Stealth, color=gray!70!black}
  ]

  %% Row 0: Inputs
  \node[input] (Wp)  {$W_p(f)$\\Prime distribution\\(Eq.\,2.1)};
  \node[input, right=1.5cm of Wp] (WR)  {$W_\R(f)$\\Archimedean dist.\\(Eq.\,2.2)};

  %% Row 1: Basic lemmas
  \node[proved, below=0.8cm of Wp, xshift=-1.8cm] (L1) {Lem.\,1\\Conv.\ inner-prod.\\$f(a)=\langle g,U_a g\rangle$};
  \node[proved, right=0.3cm of L1] (L2) {Lem.\,2\\Unitary identity\\$2\Re\langle h,Uh\rangle$};
  \node[proved, right=0.3cm of L2] (TRUNC) {Rem.\,3\\Support truncation\\$\langle g,U_a g\rangle=0$, $a>\lambda^2$};
  \node[proved, right=0.3cm of TRUNC] (LOG) {\S\,3\\Log coordinates\\$U_{e^t}\leftrightarrow S_t$};

  %% Row 2: Energy decompositions
  \node[proved, below=0.8cm of L2, xshift=-0.8cm] (PE) {Lem.\,3\\Prime energy\\$-W_p = \Sigma\|...\|^2 + c_p\|G\|^2$};
  \node[proved, right=1.0cm of PE] (AE) {Lem.\,4\\Arch.\ energy\\$-W_\R = \int w(t)\|...\|^2\,dt + c_\infty\|G\|^2$};

  %% Row 3: Definition
  \node[proved, below=0.7cm of PE, xshift=1.3cm] (DEF) {Def.\,5\\$\mathcal E_\lambda(G)$\\Difference-energy form};

  %% Row 4: Markov + Translation inv
  \node[proved, below left=0.8cm and 0.5cm of DEF] (MK) {Lem.\,6\\Markov property\\$\mathcal E(\Phi\circ G)\le\mathcal E(G)$};
  \node[proved, below right=0.8cm and 0.5cm of DEF] (TI) {Lem.\,7\\Translation inv.\\$\Rightarrow$ null or conull};

  %% Row 5: Fourier + Indicator
  \node[proved, below=0.8cm of TI, xshift=-1.2cm] (IE) {Lem.\,8\\Indicator-energy\\criterion};
  \node[proved, below=0.8cm of DEF, xshift=2.8cm] (PLANCH) {Lems.\,9--10\\Plancherel \&\\Fourier repr.};

  %% Row 6: Closedness + coercivity
  \node[proved, below=0.7cm of PLANCH, xshift=-1.0cm] (CLOSED) {Props.\,11--12\\Closedness on\\$L^2(\R)$, $L^2(I)$};
  \node[proved, below=0.7cm of PLANCH, xshift=1.5cm] (PSILOG) {Lems.\,13--14\\$\psi_\lambda(\xi)\gtrsim$\\$\log|\xi|$};

  %% Row 7: Compact embedding
  \node[proved, below=0.7cm of PSILOG] (TAIL) {Cor.\,15\\Log-frequency\\moment bound};
  \node[external, below=0.7cm of TAIL, xshift=-1.0cm] (KR) {Kolmogorov--Riesz\\compactness\\(Lieb--Loss)};
  \node[proved, below=0.7cm of KR, xshift=0.5cm] (COMPACT) {Prop.\,17\\Compact\\embedding};

  %% Operator
  \node[external, below=2.5cm of CLOSED, xshift=-0.5cm] (KATO) {Kato repr.\ thm.\\for closed forms};
  \node[proved, below=0.7cm of KATO, xshift=1.5cm] (OP) {Thm.\,18\\$A_\lambda\ge 0$,\\compact resolvent};

  %% Irreducibility
  \node[external, below=1.0cm of IE, xshift=-1.0cm] (BD) {Beurling--Deny/\\Fukushima\\(Dirichlet forms)};
  \node[proved, below=0.7cm of BD, xshift=1.0cm] (IRRED) {Cor.\,19\\$T(t)$ is\\irreducible};

  %% External PF theorems
  \node[external, below=0.7cm of IRRED, xshift=-1.5cm] (ABHN) {Arendt et al.\\pos.\ + irred.\ + holo.\\$\Rightarrow$ pos.\ improving};
  \node[external, right=0.3cm of ABHN] (KRTM) {Krein--Rutman/\\Perron--Frobenius\\simple eigenvalue};

  %% Ground state
  \node[conclusion, below=0.7cm of ABHN, xshift=1.2cm] (GS) {Prop.\,20\\Simple ground state\\$\psi>0$ a.e.};

  %% Reflection
  \node[proved, right=1.5cm of GS] (REFL) {Prop.\,21\\Reflection symmetry\\$A_\lambda R = R A_\lambda$};

  %% Even
  \node[conclusion, below=0.7cm of GS, xshift=1.0cm] (EVEN) {Cor.\,22\\$\psi(-u)=\psi(u)$\\Even ground state};

  %% ---- Arrows ----

  % To PE
  \draw[arr] (Wp) -- (PE);
  \draw[arr] (L1) -- (PE);
  \draw[arr] (L2) -- (PE);
  \draw[arr] (TRUNC) -- (PE);
  \draw[arr] (LOG) -- (PE);

  % To AE
  \draw[arr] (WR) -- (AE);
  \draw[arr] (L1) -- (AE);
  \draw[arr] (L2) -- (AE);
  \draw[arr] (TRUNC) -- (AE);
  \draw[arr] (LOG) -- (AE);

  % To DEF
  \draw[arr] (PE) -- (DEF);
  \draw[arr] (AE) -- (DEF);

  % To MK
  \draw[arr] (DEF) -- (MK);

  % To IE
  \draw[arr] (DEF) -- (IE);
  \draw[arr] (TI) -- (IE);

  % To PLANCH
  \draw[arr] (DEF) -- (PLANCH);

  % To CLOSED
  \draw[arr] (PLANCH) -- (CLOSED);

  % To PSILOG
  \draw[arr] (PLANCH) -- (PSILOG);

  % To TAIL
  \draw[arr] (PSILOG) -- (TAIL);

  % To COMPACT
  \draw[arr] (TAIL) -- (KR);
  \draw[arr] (KR) -- (COMPACT);

  % To OP
  \draw[arr] (CLOSED) -- (OP);
  \draw[arr] (COMPACT) -- (OP);
  \draw[arr] (KATO) -- (OP);

  % To IRRED
  \draw[arr] (IE) -- (IRRED);
  \draw[arr] (OP) -- (IRRED);
  \draw[arr] (BD) -- (IRRED);

  % To GS
  \draw[arr] (MK) |- (GS);
  \draw[arr] (IRRED) -- (ABHN);
  \draw[arr] (OP) -- (IRRED);
  \draw[arr] (ABHN) -- (GS);
  \draw[arr] (KRTM) -- (GS);
  \draw[arr] (OP) -- (GS);

  % To EVEN
  \draw[arr] (GS) -- (EVEN);
  \draw[arr] (REFL) -- (EVEN);

\end{tikzpicture}
}%
\caption{Dependency flowchart.  
\textcolor{red!70!black}{\textbf{Red}}: input formulas from analytic number theory.
\textcolor{green!60!black}{\textbf{Green}}: results proved in the paper.
\textcolor{orange!80!black}{\textbf{Orange}}: external theorems (cited, not proved).
\textcolor{blue!70!black}{\textbf{Blue}}: main conclusions.
Arrows indicate logical dependence.}
\label{fig:flowchart}
\end{figure}

%% ===================================================================
\section{Catalogue of unproven assumptions and external results}
%% ===================================================================

The paper relies on six external inputs.  All are standard and well-established
in their respective fields.

\subsection{Input formulas from analytic number theory}

\begin{enumerate}[label=(\roman*)]
\item \textbf{Prime local distribution $W_p(f)$} (Eq.\,2.1).  
  The formula $W_p(f)=(\log p)\sum_{m\ge 1}p^{-m/2}(f(p^m)+f(p^{-m}))$
  encodes the contribution of each prime~$p$ to the Weil explicit formula.
  It is taken as a given input from the classical theory.

\item \textbf{Archimedean local distribution $W_\R(f)$} (Eq.\,2.2).  
  The formula involving $(\log4\pi+\gamma)\,f(1)$ plus an integral encodes the
  archimedean (real place) contribution.  Also taken as a given input.
\end{enumerate}

\subsection{External theorems from functional analysis}

\begin{enumerate}[label=(\roman*),resume]
\item \textbf{Kato's representation theorem for closed symmetric forms.}
  Used in Theorem~18 to obtain the self-adjoint operator $A_\lambda$ from
  the closed quadratic form $\mathcal E_\lambda$.  
  \emph{Source}: Kato, \emph{Perturbation Theory for Linear Operators}.

\item \textbf{Kolmogorov--Riesz compactness criterion} (Theorem in \S\,7.2).
  Characterises relatively compact subsets of $L^2(\R)$ via tightness and
  translation equicontinuity.  Used in Proposition~17 (compact embedding).  
  \emph{Source}: Lieb--Loss, \emph{Analysis}.

\item \textbf{Beurling--Deny / Fukushima equivalence} (Proposition~9 / Remark).
  For symmetric Markovian semigroups, the equivalence between
  ``$\mathcal E(\1_B)=0$ implies $B$ null or conull'' and irreducibility of the
  semigroup.  Used in Corollary~19.  
  \emph{Source}: Fukushima, \emph{Dirichlet Forms and Symmetric Markov Processes}.

\item \textbf{Positivity improving from positivity + irreducibility + holomorphy}
  (Theorem~13, attributed to Arendt, Batty, Hieber, Neubrander / Arendt
  et~al.).  
  Used in Proposition~20.  
  \emph{Source}: Arendt et~al., arXiv:1909.12194, Theorem~2.3.

\item \textbf{Simplicity of the principal eigenvalue} (Theorem~14,
  Krein--Rutman / Perron--Frobenius for compact positive operators).
  Under compact resolvent and positivity improving, the bottom of the
  spectrum is a simple eigenvalue with a strictly positive eigenfunction.
  Used in Proposition~20.  
  \emph{Source}: Same reference, Proposition~2.4.
\end{enumerate}

\subsection{Assessment}

All six external results are \textbf{well-established and standard}.  Items
(i)--(ii) are classical objects in analytic number theory.  Items (iii)--(vii)
are core theorems in the theory of closed quadratic forms, Dirichlet forms, and
positive semigroups, available in standard textbooks.  There are \textbf{no
hidden or non-standard assumptions}; the paper is transparent about what is
proved versus what is cited.

%% ===================================================================
\section{Overall assessment}
%% ===================================================================

The paper is \textbf{mathematically correct} throughout.  Every proof is
complete and properly justified.  The logical structure is clean: the
paper builds from concrete input formulas through a sequence of
self-contained lemmas to the final spectral conclusion, with all
external dependencies clearly flagged.

The most notable features of the argument are:
\begin{itemize}[nosep]
\item The energy decomposition (Stage~1) is the paper's core original
  contribution---it is a concrete, verifiable computation.
\item Irreducibility (Stage~3) hinges on the \emph{archimedean continuum}
  of shifts (as opposed to the discrete prime shifts alone), which is what
  forces indicator functions with zero energy to be trivial.
\item The compact-resolvent proof (Stage~4) via logarithmic coercivity of the
  Fourier symbol $\psi_\lambda$ and the Kolmogorov--Riesz criterion is
  self-contained and replaces what was previously an abstract assumption.
\end{itemize}

\end{document}
