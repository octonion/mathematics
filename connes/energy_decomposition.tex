
\documentclass[11pt]{article}
\usepackage[margin=1in]{geometry}
\usepackage{amsmath,amssymb,amsthm,mathtools}
\usepackage[hidelinks]{hyperref}

\newtheorem{theorem}{Theorem}
\newtheorem{lemma}[theorem]{Lemma}
\newtheorem{proposition}[theorem]{Proposition}
\newtheorem{definition}[theorem]{Definition}
\newtheorem{remark}[theorem]{Remark}
\DeclareMathOperator{\supp}{supp}

\newcommand{\R}{\mathbb R}
\newcommand{\C}{\mathbb C}
\newcommand{\Rplus}{\mathbb R_+^*}

\begin{document}

\section*{Energy decomposition (Step 1): a self-contained, rigorous formulation}

\subsection*{1. Basic setup}

Let $\Rplus=(0,\infty)$ with multiplicative Haar measure $d^*x:=dx/x$.
Let $L^2(\Rplus):=L^2(\Rplus,d^*x)$ and $\langle\cdot,\cdot\rangle$ its inner product.
For $a>0$ define the unitary dilation $(U_a g)(x):=g(x/a)$ on $L^2(\Rplus)$.

When $g,h\in L^1(\Rplus,d^*x)$, the multiplicative convolution is
\[
(g*h)(a):=\int_{\Rplus} g(y)\,h(a/y)\,d^*y,
\qquad a>0,
\]
and the involution is $g^*(x):=\overline{g(x^{-1})}$.

\begin{definition}[Correlation function]
\label{def:f}
For $g\in L^2(\Rplus)$ define its correlation function $f:\Rplus\to\C$ by
\[
f(a):=\langle g,U_a g\rangle=\int_{\Rplus} g(x)\,\overline{g(x/a)}\,d^*x.
\]
If in addition $g\in L^1(\Rplus)$, then $f=g*g^*$ almost everywhere and the above
integral coincides with the convolution formula.
\end{definition}

\begin{lemma}[Elementary properties of $f$]
\label{lem:corr}
Let $g\in L^2(\Rplus)$ and $f$ as in Definition~\ref{def:f}. Then for all $a>0$,
\[
f(a^{-1})=\overline{f(a)},\qquad f(1)=\|g\|_2^2,\qquad |f(a)|\le \|g\|_2^2.
\]
In particular $f(a)+f(a^{-1})=2\Re\langle g,U_a g\rangle$.
\end{lemma}

\begin{proof}
The identity $f(a^{-1})=\overline{f(a)}$ follows by substituting $x\mapsto x/a$
and taking complex conjugates. The bound $|f(a)|\le \|g\|_2\|U_ag\|_2=\|g\|_2^2$
is Cauchy--Schwarz and unitarity of $U_a$. Finally $f(1)=\langle g,g\rangle$.
\end{proof}

\begin{lemma}[Completion of squares for unitaries]
\label{lem:unitary}
For any unitary $U$ on a Hilbert space and any vector $h$,
\[
2\Re\langle h,Uh\rangle = 2\|h\|^2-\|h-Uh\|^2.
\]
\end{lemma}

\begin{proof}
Expand $\|h-Uh\|^2=\|h\|^2+\|Uh\|^2-2\Re\langle h,Uh\rangle$ and use $\|Uh\|=\|h\|$.
\end{proof}

\begin{remark}[Support truncation]
\label{rem:truncate}
Fix $\lambda>1$ and assume $\supp(g)\subset[\lambda^{-1},\lambda]$.
Then $f(a)=0$ whenever $a\notin[\lambda^{-2},\lambda^2]$.
Indeed, $\supp(U_ag)=a\cdot\supp(g)\subset[a\lambda^{-1},a\lambda]$.
If $a>\lambda^2$ then $a\lambda^{-1}>\lambda$, hence $\supp(g)\cap\supp(U_ag)=\varnothing$
and $\langle g,U_ag\rangle=0$.
If $0<a<\lambda^{-2}$ then $a\lambda<\lambda^{-1}$ and the supports are again disjoint.
\end{remark}

\subsection*{2. Logarithmic coordinates}

Set $u=\log x$, so that $d^*x=du$.
Write $L:=\log\lambda$ and $I:=(-L,L)$.
Define $G(u):=g(e^u)\in L^2(I)$ and let $\widetilde G$ be its extension by $0$ to $\R$.
Let $S_t$ be translation on $L^2(\R)$: $(S_t\phi)(u):=\phi(u-t)$.

\begin{lemma}[Dilation becomes translation]
\label{lem:diltrans}
For all $t\in\R$,
\[
\|g-U_{e^t}g\|_{L^2(\Rplus)}=\|\widetilde G-S_t\widetilde G\|_{L^2(\R)}.
\]
\end{lemma}

\begin{proof}
By $x=e^u$ and $d^*x=du$,
\[
\|g-U_{e^t}g\|_2^2=\int_{\R} |G(u)-G(u-t)|^2\,du
=\int_{\R} |\widetilde G(u)-(S_t\widetilde G)(u)|^2\,du.
\]
\end{proof}

\subsection*{3. Local distributions}

Assume $\supp(g)\subset[\lambda^{-1},\lambda]$ and let $f$ be as in Definition~\ref{def:f}.

\paragraph{Prime terms.}
For a prime $p$, define
\begin{equation}
\label{eq:Wp}
W_p(f):=(\log p)\sum_{m\ge 1} p^{-m/2}\bigl(f(p^m)+f(p^{-m})\bigr).
\end{equation}
By Remark~\ref{rem:truncate}, the summand vanishes unless $p^m\le \lambda^2$,
so \eqref{eq:Wp} is a finite sum for each fixed $(p,\lambda)$.

\paragraph{Archimedean term.}
Let $\gamma$ be Euler's constant and define the (a priori improper) integral
\begin{equation}
\label{eq:WR}
W_{\R}(f):=(\log 4\pi+\gamma)\,f(1)+\int_1^\infty
\Bigl(f(x)+f(x^{-1})-2x^{-1/2}f(1)\Bigr)\frac{x^{1/2}}{x-x^{-1}}\,d^*x.
\end{equation}
The integrand has a non-integrable weight at $x=1$ unless there is sufficient
cancellation; we will guarantee convergence by an explicit ``energy'' hypothesis below.

Define the strictly positive weight (for $t>0$)
\begin{equation}
\label{eq:w}
w(t):=\frac{e^{t/2}}{e^t-e^{-t}}=\frac{e^{t/2}}{2\sinh t}.
\end{equation}
Note the asymptotics $w(t)\sim (2t)^{-1}$ as $t\downarrow 0$ and $w(t)\sim e^{-t/2}$ as $t\to\infty$.

\subsection*{4. Archimedean energy and its natural form domain}

\begin{definition}[Archimedean difference-energy at scale $\lambda$]
\label{def:Einf}
For $G\in L^2(I)$ (with zero extension $\widetilde G$) define
\[
\mathcal E_{\infty,\lambda}(G):=\int_{0}^{2L} w(t)\,\|\widetilde G-S_t\widetilde G\|_{L^2(\R)}^2\,dt
\ \in [0,\infty].
\]
Let $\mathcal D_{\infty,\lambda}$ be the (maximal) form domain
\[
\mathcal D_{\infty,\lambda}:=\{\,G\in L^2(I):\ \mathcal E_{\infty,\lambda}(G)<\infty\,\}.
\]
Equivalently, $g\in L^2(\Rplus)$ with $\supp(g)\subset[\lambda^{-1},\lambda]$ belongs to the
Archimedean form domain if and only if $\mathcal E_{\infty,\lambda}(G)<\infty$ for $G(u)=g(e^u)$.
\end{definition}

\begin{remark}[A closed form built from a smooth core]
\label{rem:core}
If one prefers a canonical Hilbert space structure, set $\mathcal C_\lambda:=C_c^\infty(I)$ and equip it with
$\|G\|_{\mathcal D}^2:=\|G\|_{L^2(I)}^2+\mathcal E_{\infty,\lambda}(G)$.
Let $\overline{\mathcal D}_{\infty,\lambda}$ be the completion of $\mathcal C_\lambda$ for this norm; then
$\mathcal E_{\infty,\lambda}$ extends by continuity to $\overline{\mathcal D}_{\infty,\lambda}$.
Everything below holds verbatim on $\mathcal D_{\infty,\lambda}$ (maximal domain) and in particular on
$\overline{\mathcal D}_{\infty,\lambda}$.
\end{remark}

\begin{remark}[Concrete sufficient conditions]
\label{rem:sufficient}
If $G$ is, say, $C^1$ on $I$ (or $G\in H^1(I)$), then $\mathcal E_{\infty,\lambda}(G)<\infty$.
Indeed, for small $t$ one has the standard estimate $\|\widetilde G-S_t\widetilde G\|_2\ll |t|^{1/2}(\|G'\|_2+\|G\|_2)$,
and for large $t$ one has $\|\widetilde G-S_t\widetilde G\|_2\le 2\|G\|_2$ while $w(t)$ is integrable on $(1,2L)$.
The point of Definition~\ref{def:Einf} is that no such auxiliary regularity is needed: the finiteness of the energy is the only hypothesis.
\end{remark}

\subsection*{5. Energy decomposition: primes and infinity}

\begin{lemma}[Prime term = discrete difference-energy + constant]
\label{lem:prime}
With the above notation,
\[
-W_p(f)=\sum_{\substack{m\ge 1\\ p^m\le \lambda^2}}(\log p)p^{-m/2}\,
\|\widetilde G-S_{m\log p}\widetilde G\|_{L^2(\R)}^2
\;+\;c_p(\lambda)\,\|G\|_{L^2(I)}^2,
\]
where
\[
c_p(\lambda):=-2(\log p)\sum_{\substack{m\ge 1\\ p^m\le \lambda^2}}p^{-m/2}\in\R.
\]
\end{lemma}

\begin{proof}
By Lemma~\ref{lem:corr},
\[
f(p^m)+f(p^{-m})=2\Re\langle g, U_{p^m}g\rangle.
\]
Hence, by \eqref{eq:Wp},
\[
W_p(f)=(\log p)\sum_{m\ge 1} p^{-m/2}\,2\Re\langle g,U_{p^m}g\rangle.
\]
Apply Lemma~\ref{lem:unitary} with $U=U_{p^m}$:
\[
2\Re\langle g,U_{p^m}g\rangle=2\|g\|_2^2-\|g-U_{p^m}g\|_2^2.
\]
If $p^m>\lambda^2$ then $f(p^m)=0$ by Remark~\ref{rem:truncate}, hence those terms contribute $0$.
Therefore
\[
W_p(f)=\sum_{\substack{m\ge 1\\ p^m\le \lambda^2}}(\log p)p^{-m/2}
\Bigl(2\|g\|_2^2-\|g-U_{p^m}g\|_2^2\Bigr),
\]
which rearranges to the stated formula. Finally $\|g\|_2^2=\|G\|_{L^2(I)}^2$ and
$\|g-U_{p^m}g\|_2=\|\widetilde G-S_{m\log p}\widetilde G\|_2$ by Lemma~\ref{lem:diltrans}.
\end{proof}

\begin{lemma}[Archimedean term = continuum of difference energies + constant]
\label{lem:arch}
Assume $G\in\mathcal D_{\infty,\lambda}$ (Definition~\ref{def:Einf}). Then the improper integral in \eqref{eq:WR} converges,
and one has
\[
-W_{\R}(f)=\int_{0}^{2L} w(t)\,\|\widetilde G-S_t\widetilde G\|_{L^2(\R)}^2\,dt
\;+\;c_\infty(\lambda)\,\|G\|_{L^2(I)}^2,
\]
where
\[
c_\infty(\lambda):=-(\log 4\pi+\gamma)
+\int_{0}^{2L} 2\bigl(e^{-t/2}-1\bigr)w(t)\,dt
+\int_{2L}^{\infty} 2e^{-t/2}w(t)\,dt\in\R.
\]
Both integrals defining $c_\infty(\lambda)$ converge absolutely.
\end{lemma}

\begin{proof}
Start from \eqref{eq:WR} and substitute $x=e^t$ (so $d^*x=dt$). Using \eqref{eq:w},
\[
W_{\R}(f)=(\log 4\pi+\gamma)f(1)
+\int_0^\infty\Bigl(f(e^t)+f(e^{-t})-2e^{-t/2}f(1)\Bigr)w(t)\,dt.
\]
By Lemma~\ref{lem:corr}, $f(1)=\|g\|_2^2$ and
$f(e^t)+f(e^{-t})=2\Re\langle g,U_{e^t}g\rangle$. Thus
\[
W_{\R}(f)=(\log 4\pi+\gamma)\|g\|_2^2
+\int_0^\infty\Bigl(2\Re\langle g,U_{e^t}g\rangle-2e^{-t/2}\|g\|_2^2\Bigr)w(t)\,dt.
\]
Apply Lemma~\ref{lem:unitary} with $U=U_{e^t}$ to rewrite
\[
2\Re\langle g,U_{e^t}g\rangle = 2\|g\|_2^2-\|g-U_{e^t}g\|_2^2.
\]
Hence the integrand becomes
\[
\Bigl(2(1-e^{-t/2})\|g\|_2^2-\|g-U_{e^t}g\|_2^2\Bigr)w(t).
\]

We now split at $t=2L$.
If $t>2L$, then $e^t>\lambda^2$ and also $e^{-t}<\lambda^{-2}$, so by Remark~\ref{rem:truncate}
\[
f(e^t)=f(e^{-t})=0.
\]
Equivalently, $g$ and $U_{e^t}g$ have disjoint supports, hence
\[
\|g-U_{e^t}g\|_2^2=\|g\|_2^2+\|U_{e^t}g\|_2^2=2\|g\|_2^2.
\]
Therefore for $t>2L$ the contribution of the integrand is
\[
\Bigl(2(1-e^{-t/2})\|g\|_2^2-2\|g\|_2^2\Bigr)w(t)=-2e^{-t/2}w(t)\,\|g\|_2^2,
\]
which yields the tail constant $\int_{2L}^\infty 2e^{-t/2}w(t)\,dt$ after moving to $-W_{\R}(f)$.

On $t\in[0,2L]$ we keep the difference-energy term and absorb the remaining
$2(e^{-t/2}-1)w(t)\|g\|_2^2$ into the constant.

Finally, by Lemma~\ref{lem:diltrans},
$\|g-U_{e^t}g\|_2=\|\widetilde G-S_t\widetilde G\|_2$ and $\|g\|_2^2=\|G\|_{L^2(I)}^2$.

Convergence: the tail constant converges since $e^{-t/2}w(t)\sim e^{-t}$ as $t\to\infty$.
On $[0,2L]$, the function $t\mapsto 2(e^{-t/2}-1)w(t)$ is integrable at $0$ because
$e^{-t/2}-1\sim -t/2$ and $w(t)\sim (2t)^{-1}$. The remaining term
$\int_0^{2L} w(t)\|g-U_{e^t}g\|_2^2\,dt$ is finite by the assumption $G\in\mathcal D_{\infty,\lambda}$.
Thus the improper integral in \eqref{eq:WR} converges and the stated identity holds.
\end{proof}

\begin{theorem}[Energy decomposition of the restricted Weil form]
\label{thm:energy}
Let $\lambda>1$ and let $g\in L^2(\Rplus)$ satisfy $\supp(g)\subset[\lambda^{-1},\lambda]$.
Let $f(a)=\langle g,U_a g\rangle$ and define $G(u)=g(e^u)$ on $I=(-\log\lambda,\log\lambda)$, with extension $\widetilde G$ by $0$ to $\R$.
Assume $G\in \mathcal D_{\infty,\lambda}$ (equivalently $\mathcal E_{\infty,\lambda}(G)<\infty$).
Then $W_{\R}(f)$ is well-defined by \eqref{eq:WR} and
\[
-\Bigl(W_{\R}(f)+\sum_{p\ \mathrm{prime}} W_p(f)\Bigr)
=\mathcal E_\lambda(G)+C(\lambda)\,\|G\|_{L^2(I)}^2,
\]
where the \emph{difference-energy form} is
\[
\mathcal E_\lambda(G):=
\int_{0}^{2L} w(t)\,\|\widetilde G-S_t\widetilde G\|_{L^2(\R)}^2\,dt
+\sum_{\substack{p\ \mathrm{prime}\\ p\le \lambda^2}}\ \sum_{\substack{m\ge 1\\ p^m\le \lambda^2}}
(\log p)p^{-m/2}\,\|\widetilde G-S_{m\log p}\widetilde G\|_{L^2(\R)}^2,
\]
and the constant is
\[
C(\lambda):=c_\infty(\lambda)+\sum_{\substack{p\ \mathrm{prime}\\ p\le \lambda^2}} c_p(\lambda)\in\R.
\]
All sums are finite for each fixed $\lambda$.
\end{theorem}

\begin{proof}
Lemma~\ref{lem:arch} gives $-W_{\R}(f)$ as an Archimedean difference-energy plus a constant multiple of $\|G\|_2^2$.
Summing Lemma~\ref{lem:prime} over primes $p\le \lambda^2$ (finitely many) gives the prime contribution.
Adding the identities yields the claim.
\end{proof}

\end{document}