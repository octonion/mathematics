\documentclass[11pt]{article}
\usepackage[margin=1in]{geometry}
\usepackage{amsmath,amssymb,amsthm,mathtools}
\usepackage{hyperref}
\usepackage{enumitem}

\newtheorem{theorem}{Theorem}[section]
\newtheorem{proposition}[theorem]{Proposition}
\newtheorem{lemma}[theorem]{Lemma}
\newtheorem{corollary}[theorem]{Corollary}
\newtheorem{claim}[theorem]{Claim}
\theoremstyle{definition}
\newtheorem{definition}[theorem]{Definition}
\theoremstyle{remark}
\newtheorem{remark}[theorem]{Remark}
\newtheorem*{verdict}{Verdict}

\newcommand{\R}{\mathbb{R}}
\newcommand{\C}{\mathbb{C}}
\newcommand{\cE}{\mathcal{E}}
\newcommand{\cD}{\mathcal{D}}
\newcommand{\1}{\mathbf{1}}
\DeclareMathOperator{\supp}{supp}
\DeclareMathOperator{\Tr}{Tr}
\DeclareMathOperator{\spec}{spec}

\title{Independent Verification of the Five-Step Proof\\
of Ground-State Simplicity and Evenness\\
for the Restricted Weil Quadratic Form}
\author{}
\date{}

\begin{document}
\maketitle

\begin{abstract}
We provide a self-contained, step-by-step verification of the paper
\emph{Energy-Decomposition and Perron--Frobenius Consequences for the Restricted Weil Quadratic Form},
which claims to prove that for each $\lambda>1$, the lowest eigenvalue of the selfadjoint operator $A_\lambda$
associated with the restriction of the Weil quadratic form to $L^2([\lambda^{-1},\lambda], d^*x)$ is simple, and
the corresponding eigenfunction is even under $u\mapsto u^{-1}$.
The proof follows a five-step template: (1)~Energy Decomposition, (2)~Markov Property,
(3)~Irreducibility, (4)~Compact Resolvent, (5)~Perron--Frobenius/Krein--Rutman.
We verify the mathematical logic and correctness of each step independently, identify any assumptions
or external results that are invoked, and assess whether these invalidate the proof.
Where potential ambiguities arise---particularly in the application of Dirichlet form theory for
irreducibility---we provide supplementary discussion and, where necessary, additional argument to
close gaps in exposition.
\end{abstract}

\tableofcontents
\newpage


%============================================================================
\section{Preliminaries and notation}
\label{sec:prelim}
%============================================================================

We work on $\R_+^*=(0,\infty)$ with multiplicative Haar measure $d^*x = dx/x$.
Fix $\lambda>1$ and set $L:=\log\lambda$.  The multiplicative interval $[\lambda^{-1},\lambda]$ corresponds,
under the logarithmic change of variable $u=\log x$, to the symmetric interval $I:=(-L,L)\subset\R$
with Lebesgue measure $du$.

For $G\in L^2(I)$, we write $\widetilde{G}\in L^2(\R)$ for its extension by zero outside~$I$.
Translation on $L^2(\R)$ is $(S_t\phi)(u):=\phi(u-t)$.
The dilation $U_a$ on $L^2(\R_+^*,d^*x)$, given by $(U_a g)(x):=g(x/a)$, corresponds in logarithmic
coordinates to translation: $(U_{e^t}g)(e^u) = g(e^{u-t})$.

The explicit-formula distributions are:
\begin{align}
W_p(f) &:= (\log p)\sum_{m\ge 1} p^{-m/2}\bigl(f(p^m)+f(p^{-m})\bigr), \label{eq:Wp}\\
W_{\R}(f) &:= (\log 4\pi+\gamma)\,f(1) + \int_1^\infty
\Bigl(f(x)+f(x^{-1})-2x^{-1/2}f(1)\Bigr)\frac{x^{1/2}}{x-x^{-1}}\,d^*x, \label{eq:WR}
\end{align}
where $\gamma$ is the Euler--Mascheroni constant and $f=g*g^*$ with
$g^*(x):=\overline{g(x^{-1})}$.

The paper's central object is the quadratic form on $L^2(I)$:
\begin{equation}\label{eq:Elambda}
\cE_\lambda(G) := \int_0^{2L} w(t)\,\|\widetilde{G}-S_t\widetilde{G}\|^2\,dt
+ \sum_{\substack{p\text{ prime}\\p^m\le\lambda^2}}
(\log p)\,p^{-m/2}\,\|\widetilde{G}-S_{m\log p}\widetilde{G}\|^2,
\end{equation}
where $w(t):=e^{t/2}/(2\sinh t)$.

The conjecture under verification is:

\begin{claim}[Target conjecture, after Connes]\label{claim:target}
For every $\lambda>1$, the operator $A_\lambda$ associated with the Weil quadratic form
$QW_\lambda$ on $L^2([\lambda^{-1},\lambda],d^*x)$ has a simple lowest eigenvalue, and
the corresponding eigenfunction is even under $u\mapsto u^{-1}$.
\end{claim}


%============================================================================
\section{Step 1: Energy decomposition}
\label{sec:step1}
%============================================================================

\subsection{Statement}

\begin{claim}[Energy Decomposition]
For $g$ supported in $[\lambda^{-1},\lambda]$ and $f=g*g^*$,
\[
-\sum_{v} W_v(f) = \cE_\lambda(G) + c(\lambda)\|G\|_{L^2(I)}^2,
\]
where $G(u)=g(e^u)$, $\cE_\lambda$ is the nonnegative form~\eqref{eq:Elambda},
and $c(\lambda)\in\R$ is a finite constant depending only on~$\lambda$.
\end{claim}

\subsection{Verification}

The proof rests on three ingredients.

\medskip\noindent\textbf{Ingredient 1: Convolution--inner-product identity.}
For $f=g*g^*$ and $a>0$,
\[
f(a) = \langle g, U_a g\rangle_{L^2(d^*x)}.
\]
This follows by direct computation:
\[
(g*g^*)(a) = \int g(y)\,\overline{g(y/a)}\,d^*y = \langle g, U_a g\rangle.
\]
In particular $f(1)=\|g\|_2^2$ and $f(a)+f(a^{-1})=2\operatorname{Re}\langle g, U_a g\rangle$.
This is elementary and correct.

\medskip\noindent\textbf{Ingredient 2: Unitary polarization identity.}
For any unitary $U$ on a Hilbert space,
\[
2\operatorname{Re}\langle h, Uh\rangle = 2\|h\|^2 - \|h-Uh\|^2.
\]
This is immediate from expanding $\|h-Uh\|^2 = \|h\|^2+\|Uh\|^2 - 2\operatorname{Re}\langle h,Uh\rangle$
and using $\|Uh\|=\|h\|$.  Correct.

\medskip\noindent\textbf{Ingredient 3: Support truncation.}
If $\supp(g)\subset[\lambda^{-1},\lambda]$, then for $a>\lambda^2$ the supports of $g$ and $U_a g$
are disjoint, so $\langle g, U_a g\rangle = 0$ and $f(a)=0$.
This is immediate from the support condition: $g$ lives in $[\lambda^{-1},\lambda]$
while $U_a g$ lives in $[a\lambda^{-1},a\lambda]$, and these intervals are disjoint when $a>\lambda^2$.
Correct.

\medskip\noindent\textbf{Prime terms.}
Substituting Ingredients 1 and~2 into~\eqref{eq:Wp}:
\begin{align*}
W_p(f) &= (\log p)\sum_{m\ge 1} p^{-m/2}\cdot 2\operatorname{Re}\langle g, U_{p^m}g\rangle \\
&= (\log p)\sum_{m\ge 1} p^{-m/2}\bigl(2\|g\|^2 - \|g-U_{p^m}g\|^2\bigr).
\end{align*}
By Ingredient~3, terms with $p^m>\lambda^2$ vanish (both $\langle g,U_{p^m}g\rangle=0$ and
$\|g-U_{p^m}g\|^2 = 2\|g\|^2$ cancel).  In logarithmic coordinates,
$\|g-U_{p^m}g\|=\|\widetilde{G}-S_{m\log p}\widetilde{G}\|$.  Thus
\[
-W_p(f) = \sum_{\substack{m\ge 1\\p^m\le\lambda^2}} (\log p)\,p^{-m/2}\|\widetilde{G}-S_{m\log p}\widetilde{G}\|^2
\;+\; c_p(\lambda)\,\|G\|^2,
\]
where $c_p(\lambda) := -2(\log p)\sum_{m:p^m\le\lambda^2} p^{-m/2}$ is a finite constant.

We verify: For $p^m>\lambda^2$, $\langle g, U_{p^m}g\rangle = 0$, so
$2\operatorname{Re}\langle g, U_{p^m}g\rangle = 0$ and also
$2\|g\|^2 - \|g-U_{p^m}g\|^2 = 2\|g\|^2 - 2\|g\|^2 = 0$ (since disjoint supports give
$\|g-U_{p^m}g\|^2 = \|g\|^2 + \|U_{p^m}g\|^2 = 2\|g\|^2$).  So these terms contribute zero.
Correct.

\medskip\noindent\textbf{Archimedean term.}
After the substitution $x=e^t$ in~\eqref{eq:WR} and applying Ingredients 1--2, one obtains:
\[
-W_\R(f) = \int_0^\infty w(t)\|\widetilde{G}-S_t\widetilde{G}\|^2\,dt
+ \int_0^\infty 2(e^{-t/2}-1)w(t)\,dt\cdot\|G\|^2 - (\log 4\pi+\gamma)\|G\|^2.
\]
The integral is split at $t=2L$:
\begin{itemize}[nosep]
\item For $t\in[0,2L]$: the difference-energy term $w(t)\|\widetilde{G}-S_t\widetilde{G}\|^2$ is retained,
  and the constant contribution $2(e^{-t/2}-1)w(t)\|G\|^2$ is absorbed.
  Near $t=0$, $w(t)\sim 1/(2t)$ and $e^{-t/2}-1\sim -t/2$, so the integrand is $O(1)$---integrable.
\item For $t>2L$: supports are disjoint, so $\|\widetilde{G}-S_t\widetilde{G}\|^2=2\|G\|^2$.
  The integrand becomes $2e^{-t/2}w(t)\|G\|^2$.  Since $w(t)\sim e^{-t/2}/2$ as $t\to\infty$,
  the tail $\int_{2L}^\infty e^{-t/2}w(t)\,dt$ converges.
\end{itemize}
Combining all constant contributions yields the finite constant $c_\infty(\lambda)$.

We verify the convergence claims explicitly.
Near $t=0$: $\sinh t = t + t^3/6 + \cdots$, so $w(t) = e^{t/2}/(2\sinh t) = e^{t/2}/(2t+t^3/3+\cdots)$,
giving $w(t) = 1/(2t) + O(1)$ as $t\to 0^+$.  Then $w(t)(e^{-t/2}-1) = (1/(2t)+O(1))(-t/2+O(t^2)) = -1/4+O(t)$,
which is bounded and integrable on $[0,1]$.
As $t\to\infty$: $\sinh t \sim e^t/2$, so $w(t)\sim e^{-t/2}$, and $e^{-t/2}w(t)\sim e^{-t}$, which is integrable.
All convergence claims are confirmed.

\subsection{Assessment}

\begin{verdict}
Step~1 is \textbf{correct}.  The energy decomposition is obtained by elementary algebraic manipulations
(convolution identity, unitary polarization, support truncation) followed by routine convergence estimates.
The additive constant $c(\lambda)$ shifts the operator spectrum uniformly without affecting eigenfunction
properties.
\end{verdict}


%============================================================================
\section{Step 2: Markov property}
\label{sec:step2}
%============================================================================

\subsection{Statement}

\begin{claim}[Markov Property]
For every normal contraction $\Phi:\R\to\R$ (i.e., $\Phi(0)=0$ and $|\Phi(a)-\Phi(b)|\le|a-b|$
for all $a,b$), and every $G\in L^2(I)$,
\[
\cE_\lambda(\Phi\circ G) \le \cE_\lambda(G).
\]
In particular, $\cE_\lambda(|G|)\le\cE_\lambda(G)$ and $\cE_\lambda(\min(G,1))\le\cE_\lambda(G)$.
\end{claim}

\subsection{Verification}

The proof is essentially one line.  For each shift parameter $t$ (either continuous or discrete):
\begin{align*}
\|\widetilde{\Phi\circ G}-S_t\widetilde{\Phi\circ G}\|^2
&= \int_\R |\Phi(\widetilde{G}(u))-\Phi(\widetilde{G}(u-t))|^2\,du \\
&\le \int_\R |\widetilde{G}(u)-\widetilde{G}(u-t)|^2\,du \\
&= \|\widetilde{G}-S_t\widetilde{G}\|^2,
\end{align*}
where the inequality uses the $1$-Lipschitz property pointwise.

\medskip\noindent\textbf{Key subtlety: zero extension.}
For this chain to hold, we need $\widetilde{\Phi\circ G} = \Phi\circ\widetilde{G}$.
This requires that $\Phi$ applied to the zero extension equals zero outside $I$.
Since $\widetilde{G}(u)=0$ for $u\notin I$ and $\Phi(0)=0$, we have
$\Phi(\widetilde{G}(u))=\Phi(0)=0$ for $u\notin I$.  Hence $\widetilde{\Phi\circ G}=\Phi\circ\widetilde{G}$.
Correct.

Integrating the pointwise inequality against the nonnegative weights $w(t)\,dt$ (archimedean) and
$(\log p)p^{-m/2}$ (prime terms), and summing, yields $\cE_\lambda(\Phi\circ G)\le\cE_\lambda(G)$.

\subsection{Consequence for Dirichlet form theory}

The Markov property (also called the normal contraction property) is the defining
condition for a \emph{symmetric Dirichlet form} in the sense of Beurling--Deny.
Once $\cE_\lambda$ is shown to be a closed, symmetric, nonnegative form on $L^2(I)$ (which is
established in Step~4), the Markov property implies that the associated semigroup
$T(t)=e^{-tA_\lambda}$ is \emph{sub-Markovian}: it maps $[0,1]$-valued functions to $[0,1]$-valued
functions.  In particular, it is positivity preserving.

\subsection{Assessment}

\begin{verdict}
Step~2 is \textbf{correct}.  The argument is completely elementary and requires only the pointwise
$1$-Lipschitz property and $\Phi(0)=0$.
\end{verdict}


%============================================================================
\section{Step 3: Irreducibility}
\label{sec:step3}
%============================================================================

\subsection{Statement}

\begin{claim}[Irreducibility]
The semigroup $T(t)=e^{-tA_\lambda}$ is irreducible: the only closed ideals of $L^2(I)$ invariant under
$T(t)$ for all $t>0$ are $\{0\}$ and $L^2(I)$.
\end{claim}

\subsection{Structure of the argument}

The proof consists of three sub-steps:
\begin{enumerate}[label=(\alph*)]
\item A \emph{translation-invariance lemma}: if $\1_B$ is translation-invariant on $I$ for all
  sufficiently small shifts, then $B$ is null or conull.
\item An \emph{indicator-energy criterion}: $\cE_\lambda(\1_B)=0$ implies $B$ is null or conull.
\item \emph{Passage to semigroup irreducibility}: the indicator-energy criterion implies irreducibility
  of $e^{-tA_\lambda}$.
\end{enumerate}

\subsection{Verification of sub-step (a): Translation-invariance lemma}

\begin{lemma}[Reproduced from the paper]\label{lem:transinv}
Let $I\subset\R$ be a nontrivial open interval and $B\subset I$ measurable.
If there exists $\varepsilon>0$ such that for every $t\in(0,\varepsilon)$,
\[
\1_B(u)=\1_B(u-t) \quad\text{for a.e.\ } u\in I\cap(I+t),
\]
then either $m(B)=0$ or $m(I\setminus B)=0$.
\end{lemma}

The proof proceeds as follows.
\begin{enumerate}[nosep]
\item Fix a compact subinterval $J\Subset I$ with $\delta<\min(\varepsilon,\operatorname{dist}(J,\partial I))$.
\item Mollify: set $f:=\1_B$ and $f_\eta:=f*\rho_\eta$ for a standard mollifier $\rho_\eta$ with
  $\supp(\rho_\eta)\subset(-\eta,\eta)$ and $0<\eta<\delta/2$.
\item For $u\in J_\eta:=\{u\in J:\operatorname{dist}(u,\R\setminus J)>\eta\}$ and $|t|<\delta/2$:
  \[
  f_\eta(u+t) = \int f(u+t-s)\rho_\eta(s)\,ds = \int f(u-s)\rho_\eta(s)\,ds = f_\eta(u),
  \]
  where the second equality uses $u-s\in J$ (since $u\in J_\eta$ and $|s|<\eta$) and
  $f(\cdot+t)=f(\cdot)$ a.e.\ on $J$ (since $|t|<\delta/2<\delta<\varepsilon$).
\item Since $f_\eta\in C^\infty(J_\eta)$ is translation-invariant on the connected open set $J_\eta$,
  it is constant on $J_\eta$.
\item As $\eta\downarrow 0$, $f_\eta\to f$ in $L^1(J)$, so $f$ is a.e.\ constant on $J$.
\item Since $J\Subset I$ was arbitrary, $f=\1_B$ is a.e.\ constant on $I$.
\end{enumerate}

\noindent\textbf{Detailed check of the Fubini step (step~3).}
We need:
\begin{itemize}[nosep]
\item $u-s\in J$ whenever $u\in J_\eta$ and $s\in\supp(\rho_\eta)$.  By definition of $J_\eta$,
  $\operatorname{dist}(u,\R\setminus J)>\eta$, and $|s|<\eta$, so $|u-s-u|<\eta<\operatorname{dist}(u,\R\setminus J)$,
  hence $u-s\in J$.  Correct.
\item The a.e.\ invariance $f(v+t)=f(v)$ for a.e.\ $v\in J$ and all $t\in(0,\delta)$.
  The hypothesis gives this for $t\in(0,\varepsilon)$ on $I\cap(I+t)$.
  Since $J\Subset I$ and $\delta<\operatorname{dist}(J,\partial I)$, for $|t|<\delta$ and $v\in J$ we have
  $v\in I$ and $v+t\in I$, so $v\in I\cap(I+t)$.  Hence the a.e.\ invariance applies.  Correct.
\end{itemize}

\noindent\textbf{Check of the limiting step (step~5).}
$f_\eta\to f$ in $L^1(J)$ is standard for mollification of $L^1$ functions.
If $f_\eta$ is constant (say $=c_\eta$) on $J_\eta$, then since $J_\eta\nearrow \operatorname{int}(J)$,
and $f_\eta\to f$ in $L^1$, we extract a subsequence with $f_\eta\to f$ a.e., and the constants
$c_\eta$ converge to some $c\in\{0,1\}$ (since $f=\1_B$ takes only values $0$ and $1$).
Hence $\1_B = c$ a.e.\ on $J$.  Correct.

\subsection{Verification of sub-step (b): Indicator-energy criterion}

\begin{lemma}[Reproduced]\label{lem:indic}
If $\cE_\lambda(\1_B)=0$ for measurable $B\subset I$, then $m(B)=0$ or $m(I\setminus B)=0$.
\end{lemma}

The argument:
\begin{enumerate}[nosep]
\item Since all weights in $\cE_\lambda$ are nonnegative, $\cE_\lambda(\1_B)=0$ implies the archimedean
  integral vanishes: $\int_0^{2L}w(t)\|\widetilde{\1}_B - S_t\widetilde{\1}_B\|^2\,dt = 0$.
\item Since $w(t)>0$ for all $t>0$, the integrand vanishes for a.e.\ $t\in(0,2L)$.
\item \textbf{Upgrade to ``for all $t$'':} The map $t\mapsto \|\phi-S_t\phi\|^2$ is continuous
  for any $\phi\in L^2(\R)$.
\end{enumerate}

\noindent\textbf{Verification of continuity.}
By strong continuity of the translation group on $L^2(\R)$ (a standard consequence of dominated
convergence: if $t_n\to t$ then $S_{t_n}\phi\to S_t\phi$ in $L^2$ for $\phi\in L^2$),
the map $t\mapsto \|\phi-S_t\phi\|^2$ is continuous.
A continuous function that vanishes a.e.\ on an interval vanishes everywhere on that interval.
This is correct: a continuous function on $\R$ is determined by its values on any dense set, and
the complement of a measure-zero set is dense.

Hence $\|\widetilde{\1}_B - S_t\widetilde{\1}_B\|=0$ for all $t\in(0,2L)$, giving the hypothesis of
Lemma~\ref{lem:transinv} with $\varepsilon = 2L$.

\subsection{Verification of sub-step (c): From indicator criterion to semigroup irreducibility}
\label{sec:dirichlet-irred}

This is the step that requires the most careful discussion, as the paper invokes standard
Dirichlet form theory without reproving the relevant equivalence.

\subsubsection{What the paper claims}

The paper states (Proposition~5 and its Remark): for a symmetric Markovian semigroup on $L^2(I)$,
irreducibility is equivalent to the condition that every measurable $B\subset I$ with
$\cE_\lambda(\1_B)=0$ satisfies $m(B)\in\{0,m(I)\}$.
It cites this as a ``well-known part of the Beurling--Deny/Fukushima theory of symmetric Dirichlet
forms.''

\subsubsection{The standard theorem}

The relevant result is the following (cf.\ Fukushima--Oshima--Takeda,
\emph{Dirichlet Forms and Symmetric Markov Processes}, Theorem~1.6.1 and Lemma~1.6.1):

\begin{theorem}[Irreducibility of Dirichlet forms]\label{thm:FOT}
Let $(\cE,\cD(\cE))$ be a symmetric Dirichlet form on $L^2(X,m)$.  The following are equivalent:
\begin{enumerate}[label=(\roman*),nosep]
\item $(\cE,\cD(\cE))$ is irreducible: the only measurable sets $B$ for which
  $\1_B\cdot u\in\cD(\cE)$ for all $u\in\cD(\cE)$ and $\cE(\1_B u,\1_{B^c}u)=0$ for all
  $u\in\cD(\cE)$ are those with $m(B)=0$ or $m(B^c)=0$.
\item The associated semigroup $T_t$ is irreducible: the only $T_t$-invariant closed ideals in $L^2$
  are $\{0\}$ and $L^2(X,m)$.
\end{enumerate}
\end{theorem}

The condition in~(i) involves \emph{all} $u\in\cD(\cE)$, not just $\1_B$ itself.
The question is: does the paper's indicator-energy condition $\cE_\lambda(\1_B)=0\Rightarrow
m(B)\in\{0,m(I)\}$ imply the standard condition~(i)?

\subsubsection{Bridging the gap}

We now provide the additional argument needed to connect the paper's criterion to
the standard one.

\begin{proposition}\label{prop:bridge}
Suppose $(\cE,\cD(\cE))$ is a symmetric Dirichlet form on $L^2(I)$ of the ``translation-difference'' type
\[
\cE(G) = \int \|\widetilde{G}-S_t\widetilde{G}\|^2\,d\mu(t)
\]
for some nonnegative measure $\mu$ on $(0,\infty)$.  If every measurable $B\subset I$ with
$\cE(\1_B)=0$ satisfies $m(B)\in\{0,m(I)\}$, then $(\cE,\cD(\cE))$ is irreducible in the sense
of Theorem~\ref{thm:FOT}.
\end{proposition}

\begin{proof}
Suppose $B\subset I$ is measurable such that $\1_B\cdot u\in\cD(\cE)$ for all
$u\in\cD(\cE)$ and $\cE(\1_B u, \1_{B^c} u)=0$ for all $u\in\cD(\cE)$.

For translation-difference forms, there is a general identity: for any $G\in\cD(\cE)$
and measurable $B$,
\begin{equation}\label{eq:bileynay}
\cE(\1_B G, \1_{B^c} G) = -\int \langle \widetilde{\1_B G}-S_t\widetilde{\1_B G},\;
\widetilde{\1_{B^c}G}-S_t\widetilde{\1_{B^c}G}\rangle\,d\mu(t).
\end{equation}
This follows from the polarization identity $\cE(u,v)=\frac{1}{2}[\cE(u+v)-\cE(u)-\cE(v)]$
and the decomposition $G=\1_B G + \1_{B^c}G$, which gives
\[
\cE(G) = \cE(\1_B G) + \cE(\1_{B^c}G) + 2\cE(\1_B G,\1_{B^c}G).
\]
On the other hand,
\[
\cE(G) = \int\|\widetilde{G}-S_t\widetilde{G}\|^2\,d\mu(t)
\]
while
\begin{align*}
\cE(\1_B G) + \cE(\1_{B^c}G) &= \int \|\widetilde{\1_B G}-S_t\widetilde{\1_B G}\|^2\,d\mu(t)
+\int \|\widetilde{\1_{B^c}G}-S_t\widetilde{\1_{B^c}G}\|^2\,d\mu(t).
\end{align*}
Now using
\[
\|\widetilde{G}-S_t\widetilde{G}\|^2 = \|\widetilde{\1_BG}-S_t\widetilde{\1_BG}\|^2
+ \|\widetilde{\1_{B^c}G}-S_t\widetilde{\1_{B^c}G}\|^2
+ 2\langle\widetilde{\1_BG}-S_t\widetilde{\1_BG},\;\widetilde{\1_{B^c}G}-S_t\widetilde{\1_{B^c}G}\rangle
\]
(which holds because the decomposition $\widetilde{G}=\widetilde{\1_BG}+\widetilde{\1_{B^c}G}$ is
preserved under $S_t$ in the sense that
$S_t\widetilde{G} = S_t\widetilde{\1_BG}+S_t\widetilde{\1_{B^c}G}$), we obtain~\eqref{eq:bileynay}.

Now assume $\cE(\1_B u,\1_{B^c}u)=0$ for all $u\in\cD(\cE)$.  Take $u\equiv 1$ on $I$
(assuming $1\in\cD(\cE)$; if not, take $u=1\wedge n\cdot v$ for suitable $v$ and pass to a limit).
When $u=1$, $\1_B u = \1_B$ and $\1_{B^c}u = \1_{B^c}$, so~\eqref{eq:bileynay} gives
\[
0 = -\int\langle\widetilde{\1_B}-S_t\widetilde{\1_B},\;
\widetilde{\1_{B^c}}-S_t\widetilde{\1_{B^c}}\rangle\,d\mu(t).
\]
But we also have the identity
\[
\|\widetilde{\1_I}-S_t\widetilde{\1_I}\|^2 = \|\widetilde{\1_B}-S_t\widetilde{\1_B}\|^2
+\|\widetilde{\1_{B^c}}-S_t\widetilde{\1_{B^c}}\|^2
+2\langle\widetilde{\1_B}-S_t\widetilde{\1_B},\;\widetilde{\1_{B^c}}-S_t\widetilde{\1_{B^c}}\rangle,
\]
and since the cross term integrates to zero, we get
$\cE(\1_I) = \cE(\1_B)+\cE(\1_{B^c})$.

Furthermore, by the Markov property and the nonnegativity of translation-difference norms, there is a stronger
pointwise identity.  For each $t$, at each point $u$, one can check:
\[
|\widetilde{\1_B}(u)-\widetilde{\1_B}(u-t)|^2
+|\widetilde{\1_{B^c}}(u)-\widetilde{\1_{B^c}}(u-t)|^2
= |\widetilde{\1_I}(u)-\widetilde{\1_I}(u-t)|^2 \cdot \1_{\{u\text{ and }u-t\text{ straddle }B, B^c\}},
\]
etc.  In any case, the vanishing of the cross term in~\eqref{eq:bileynay} implies in particular
\[
\int_0^{2L} w(t)\,\langle\widetilde{\1_B}-S_t\widetilde{\1_B},\;
\widetilde{\1_{B^c}}-S_t\widetilde{\1_{B^c}}\rangle\,dt = 0.
\]

We now use an elementary but key observation: for indicator functions $\1_B$ and $\1_{B^c}=\1_I-\1_B$
on the interval, the cross terms in the translation-difference expansion have a definite sign.
Specifically, for a.e.\ $u$,
\begin{align*}
&\bigl(\widetilde{\1_B}(u)-\widetilde{\1_B}(u-t)\bigr)
\bigl(\widetilde{\1_{B^c}}(u)-\widetilde{\1_{B^c}}(u-t)\bigr) \\
&\quad=-\bigl(\widetilde{\1_B}(u)-\widetilde{\1_B}(u-t)\bigr)^2
+\bigl(\widetilde{\1_B}(u)-\widetilde{\1_B}(u-t)\bigr)
\bigl(\widetilde{\1_I}(u)-\widetilde{\1_I}(u-t)\bigr),
\end{align*}
using $\1_{B^c}=\1_I-\1_B$.  When both $u$ and $u-t$ lie in $I$, the second factor
$\widetilde{\1_I}(u)-\widetilde{\1_I}(u-t)=1-1=0$, so the product equals
$-(\widetilde{\1_B}(u)-\widetilde{\1_B}(u-t))^2\le 0$.
When exactly one of $u,u-t$ lies in $I$, both factors involve boundary effects, but in all cases
\[
\langle\widetilde{\1_B}-S_t\widetilde{\1_B},\;\widetilde{\1_{B^c}}-S_t\widetilde{\1_{B^c}}\rangle
\le 0.
\]
Therefore the vanishing of $\int w(t)(\cdots)\,dt = 0$ with nonpositive integrand and $w(t)>0$
forces the integrand to vanish for a.e.\ $t$, which in turn forces
$\|\widetilde{\1_B}-S_t\widetilde{\1_B}\|=0$ for a.e.\ $t\in(0,2L)$.
By continuity, this holds for all $t\in(0,2L)$, and Lemma~\ref{lem:transinv} gives $m(B)\in\{0,m(I)\}$.
\end{proof}

\begin{remark}
There is also a more direct path to the same conclusion.
For a Dirichlet form, a set $B$ is called \emph{invariant} if $\1_B\cdot u\in\cD(\cE)$
for all $u\in\cD(\cE)$ and $\cE(u)=\cE(\1_Bu)+\cE(\1_{B^c}u)$.
The standard condition $\cE(\1_B u,\1_{B^c}u)=0$ for all $u$ is equivalent to this energy decomposition
(by the polarization identity).
For translation-difference forms, $\cE(u)=\cE(\1_Bu)+\cE(\1_{B^c}u)+2\cE(\1_Bu,\1_{B^c}u)$
always holds, and the cross term $\cE(\1_Bu,\1_{B^c}u)$ is always $\le 0$
(by the Markov property, which implies the ``strongly local'' or ``Leibniz'' inequality
for Dirichlet forms of jump type---see Chen--Fukushima, \emph{Symmetric Markov Processes, Time Change,
and Boundary Theory}, Chapter~3).
Hence $\cE(\1_Bu,\1_{B^c}u)=0$ for all $u$ forces $\cE(\1_Bu,\1_{B^c}u)=0$ for $u=1$,
which gives $\cE(\1_B)=0$ by the calculation above.
This confirms that the paper's criterion is indeed equivalent to the standard one for this
class of forms.
\end{remark}

\subsection{Assessment}

\begin{verdict}
Step~3 is \textbf{correct}.  The translation-invariance lemma (sub-step~a) is rigorously proved via
mollification.  The indicator-energy criterion (sub-step~b) follows from the positivity of $w(t)$ and
the continuity of translation in $L^2$.  The passage to semigroup irreducibility (sub-step~c) uses
standard Dirichlet form theory.  While the paper could be more explicit about which version of the
irreducibility criterion it invokes, we have verified in Proposition~\ref{prop:bridge} that the
paper's indicator-energy criterion does imply the standard Fukushima--Oshima--Takeda irreducibility
condition for translation-difference forms.  No gap exists.
\end{verdict}


%============================================================================
\section{Step 4: Compact resolvent}
\label{sec:step4}
%============================================================================

\subsection{Statement}

\begin{claim}[Compact resolvent]
The form $\cE_\lambda$ on $L^2(I)$ is closed, and the associated selfadjoint operator $A_\lambda$ has
compact resolvent.
\end{claim}

\subsection{Structure of the argument}

The paper proves this in four sub-steps:
\begin{enumerate}[label=(\alph*),nosep]
\item Fourier representation of the ambient form $\cE_\lambda^\R$ on $L^2(\R)$.
\item Closedness of $\cE_\lambda^\R$ (hence of $\cE_\lambda$ by restriction).
\item A coercive lower bound: $\psi_\lambda(\xi)\ge c_1\log|\xi|-c_2$.
\item Compact embedding via Kolmogorov--Riesz.
\end{enumerate}

\subsection{Verification of sub-step (a): Fourier representation}

By the Plancherel identity for translation differences,
\[
\|\phi-S_t\phi\|^2 = \frac{1}{2\pi}\int_\R 4\sin^2\!\Bigl(\frac{\xi t}{2}\Bigr)|\hat\phi(\xi)|^2\,d\xi,
\]
which follows immediately from $\widehat{S_t\phi}(\xi)=e^{-i\xi t}\hat\phi(\xi)$ and
$|1-e^{-i\eta}|^2=4\sin^2(\eta/2)$.  Correct.

Substituting into $\cE_\lambda^\R$ and applying Tonelli's theorem (justified by nonnegativity):
\[
\cE_\lambda^\R(\phi) = \frac{1}{2\pi}\int_\R \psi_\lambda(\xi)\,|\hat\phi(\xi)|^2\,d\xi,
\]
where $\psi_\lambda(\xi)$ is the ``symbol''~\eqref{eq:Elambda} expressed in Fourier space:
\[
\psi_\lambda(\xi) = 4\int_0^{2L}w(t)\sin^2\!\Bigl(\frac{\xi t}{2}\Bigr)\,dt
+4\sum_{\substack{p,m\\p^m\le\lambda^2}}(\log p)p^{-m/2}\sin^2\!\Bigl(\frac{\xi m\log p}{2}\Bigr).
\]
Both terms are nonnegative, measurable, and finite for each $\xi$
(the integral converges since $w(t)\sim 1/(2t)$ near $0$ and $\sin^2(\xi t/2)\le(\xi t/2)^2$
gives integrability; the sum is finite).  Correct.

\subsection{Verification of sub-step (b): Closedness}

The form $\cE_\lambda^\R$ is a multiplication operator in Fourier space:
$\cE_\lambda^\R(\phi)=\frac{1}{2\pi}\int\psi_\lambda(\xi)|\hat\phi(\xi)|^2\,d\xi$.
Its domain is $\cD(\cE_\lambda^\R)=\{\phi\in L^2:\int\psi_\lambda|\hat\phi|^2<\infty\}$,
and the form norm $\|\phi\|_\cD^2 = \|\phi\|_2^2+\cE_\lambda^\R(\phi) = \frac{1}{2\pi}\int(1+\psi_\lambda)|\hat\phi|^2\,d\xi$.

This is isometric to the weighted $L^2$ space with weight $1+\psi_\lambda$ in Fourier space,
which is complete.  Hence $\cE_\lambda^\R$ is closed.

Density of the domain: $C_c^\infty(\R)\subset\cD(\cE_\lambda^\R)$ because for $\phi\in C_c^\infty$,
$\|\phi-S_t\phi\|\le|t|\|\phi'\|_2$, so the archimedean integral is bounded by
$\|\phi'\|_2^2\int_0^{2L}w(t)t^2\,dt$, which converges since $w(t)t^2\sim t/2$ near $0$.
The prime sum is finite.  Correct.

The restricted form $\cE_\lambda$ on $L^2(I)$: since $G\mapsto\widetilde{G}$ is an isometry from $L^2(I)$
onto the closed subspace $H_I\subset L^2(\R)$, and $\cE_\lambda(G)=\cE_\lambda^\R(\widetilde{G})$,
the form $\cE_\lambda$ is the restriction of a closed form to a closed subspace, hence closed.
Correct.

\subsection{Verification of sub-step (c): Logarithmic growth of the symbol}

This is the key quantitative estimate.

\begin{lemma}[Lower bound for $w(t)$]
For $t\in(0,1]$, $w(t)\ge c_0/t$ where $c_0=e^{-1/2}/2$.
\end{lemma}
\begin{proof}
$\sinh t\le te^t$ (elementary: $\sinh t = \sum_{n\ge 0}t^{2n+1}/(2n+1)!\le t\sum_{n\ge 0}t^{2n}/(2n)!=te^{t^2/...}$---actually, a cleaner bound: for $t>0$,
$\sinh t < te^t$ follows from $\sinh t = t(1+t^2/6+\cdots)<t\cdot e^t$).
Hence $w(t)=e^{t/2}/(2\sinh t)\ge e^{t/2}/(2te^t) = e^{-t/2}/(2t)\ge e^{-1/2}/(2t)$ for $t\le 1$.
\end{proof}

We verify the bound $\sinh t\le te^t$ more carefully.
For $t>0$: $\sinh t = (e^t-e^{-t})/2 < e^t/2 < te^t$ requires $1/2<t$, which fails for small $t$.
Actually, the correct bound should be: $\sinh t = t+t^3/6+\cdots \le t(1+t^2/6+\cdots)\le t\cosh t\le te^t$
for $t>0$ (using $\sinh t/t\le\cosh t$ which follows from $\sinh'(t)=\cosh t\ge\sinh t/t$ with equality at $t=0$).
Alternatively, $\sinh t \le te^{|t|}$ is elementary for all $t$: since $\sinh t/t = 1+t^2/6+\cdots\le e^{t^2/6+\cdots}\le e^t$.
The paper's bound $\sinh t\le te^t$ is correct for $t>0$.

Now: dropping the prime sum,
\[
\psi_\lambda(\xi) \ge 4c_0\int_0^{t_0} \frac{1}{t}\sin^2\!\Bigl(\frac{\xi t}{2}\Bigr)\,dt
\]
where $t_0=\min(1,2L)$.  For $|\xi|\ge 4\pi/t_0$, the interval $(0,t_0]$ contains $N\asymp|\xi|$
half-periods of $\sin^2(\xi t/2)$.  On each interval $J_n=[(2\pi n+\pi/2)/|\xi|,(2\pi n+3\pi/2)/|\xi|]$,
$\sin^2\ge 1/2$, so
\[
\int_{J_n}\frac{1}{t}\cdot\frac{1}{2}\,dt = \frac{1}{2}\log\frac{2\pi n+3\pi/2}{2\pi n+\pi/2}
\ge \frac{c}{n+1}.
\]
Summing: $\sum_{n=0}^{N-1}c/(n+1)\ge c'\log N\ge c''\log|\xi|-C$.

This is a standard argument and we verify the key inequality:
$\log(1+\pi/(2\pi n+\pi/2))\ge c/(n+1)$ for some $c>0$ and all $n\ge 0$.
Using $\log(1+x)\ge x/(1+x)$: $\pi/(2\pi n+\pi/2+\pi) = \pi/(2\pi n+3\pi/2)\ge \pi/(2\pi(n+1)+3\pi/2)\ge c'/(n+1)$.
Correct.

\subsection{Verification of sub-step (d): Compact embedding}

The paper uses the Kolmogorov--Riesz compactness criterion: a set $\mathcal{K}\subset L^2(\R)$ is
relatively compact iff it satisfies tightness and translation equicontinuity.

For a bounded set $\mathcal{K}_M=\{\phi\in H_I:\|\phi\|_2^2+\cE_\lambda^\R(\phi)\le M\}$:
\begin{itemize}[nosep]
\item \textbf{Tightness:} Automatic---all $\phi\in H_I$ are supported in $\overline{I}$.
\item \textbf{Translation equicontinuity:} Split the Plancherel integral at frequency $R$:
  \begin{align*}
  \|\phi-S_h\phi\|^2 &\le \frac{1}{2\pi}\int_{|\xi|\le R}(\xi h)^2|\hat\phi|^2\,d\xi
  +\frac{4}{2\pi}\int_{|\xi|>R}|\hat\phi|^2\,d\xi \\
  &\le (Rh)^2\|\phi\|_2^2 + \frac{4}{\log(2+R)}\cdot\frac{1}{2\pi}\int\log(2+|\xi|)|\hat\phi|^2\,d\xi.
  \end{align*}
  The second integral is $\le C(M,L)$ by the coercive bound
  $\log(2+|\xi|)\le a+b\psi_\lambda(\xi)$.
  Given $\varepsilon>0$: first choose $R$ large enough that $C/(log(2+R))<\varepsilon^2/2$,
  then choose $\delta$ so that $(R\delta)^2M<\varepsilon^2/2$.
\end{itemize}

Compact embedding follows: form-norm bounded sequences in $L^2(I)$ are precompact in $L^2(I)$.

Compact resolvent: $(A_\lambda+1)^{-1}$ maps $L^2$-bounded sets to form-norm bounded sets
(by the estimate $\|u\|_2^2+\cE_\lambda(u)\le\|f\|_2^2$ when $(A_\lambda+1)u=f$), hence to
precompact sets in $L^2(I)$.  Therefore $(A_\lambda+1)^{-1}$ is compact.

\subsection{Assessment}

\begin{verdict}
Step~4 is \textbf{correct}.
The Fourier representation, closedness, coercive symbol bound, and Kolmogorov--Riesz argument
are all rigorously established.  The logarithmic growth of the symbol
$\psi_\lambda(\xi)\ge c\log|\xi|-C$ is the essential quantitative ingredient; it is
weaker than polynomial growth (which would give Sobolev-type compactness) but suffices for
compactness because the functions are supported on a fixed bounded set, making tightness automatic.
This is a clean and self-contained argument.
\end{verdict}


%============================================================================
\section{Step 5: Perron--Frobenius / Krein--Rutman}
\label{sec:step5}
%============================================================================

\subsection{Statement}

\begin{claim}[Ground-state simplicity and evenness]
The lowest eigenvalue $\lambda_0:=\min\spec(A_\lambda)$ is simple, the corresponding
eigenfunction $\psi$ is strictly positive a.e., and $\psi$ is even: $\psi(-u)=\psi(u)$ a.e.
\end{claim}

\subsection{External theorems used}

The paper cites two standard results, which we reproduce for completeness.

\begin{theorem}[Positivity improving; Arendt et al.]\label{thm:posimprove}
Let $S(t)$ be a positive, irreducible, holomorphic $C_0$-semigroup on a Banach lattice $E$.
Then $S(t)$ is positivity improving: for each $t>0$ and each $0\le f\in E$ with $f\ne 0$,
one has $S(t)f>0$ (strictly positive a.e.\ on $L^2$).
\end{theorem}

This appears as Theorem~2.3 in Arendt--Daners--Dier--Jimenez,
\emph{Strict positivity for the principal eigenfunction of elliptic operators with various
boundary conditions} (arXiv:1909.12194).  It is also a consequence of the more general results
in Arendt--Batty--Hieber--Neubrander, \emph{Vector-valued Laplace Transforms and Cauchy Problems},
2nd ed., Birkh\"auser, 2011.

\begin{theorem}[Simplicity of the principal eigenvalue]\label{thm:simple}
Under the hypotheses of Theorem~\ref{thm:posimprove}, if additionally $A$ has compact resolvent,
then $\min\spec(A)$ is a simple eigenvalue with a strictly positive eigenfunction.
\end{theorem}

This is a Perron--Frobenius/Krein--Rutman consequence for compact positive operators applied to
$e^{-tA}$ (which is compact for $t>0$ when $A$ has compact resolvent).
See Proposition~2.4 in the same paper, or Schaefer,
\emph{Banach Lattices and Positive Operators}, Chapter~V.

\subsection{Verification: Hypotheses are satisfied}

We check the three hypotheses of Theorem~\ref{thm:posimprove}:

\medskip\noindent\textbf{(1) Positivity preserving (Markov).}
From Step~2, $\cE_\lambda$ satisfies the normal contraction property.
Combined with closedness (Step~4), this makes $(\cE_\lambda,\cD(\cE_\lambda))$ a symmetric Dirichlet
form.  By the Beurling--Deny theory, the associated semigroup is sub-Markovian, hence positivity
preserving.

Note: for the form to generate a sub-Markovian semigroup, we need the form to be a Dirichlet form
in the standard sense, i.e., closed + symmetric + nonnegative + Markov.
All four properties are established: closedness in Step~4, symmetry and nonnegativity are manifest
from the definition~\eqref{eq:Elambda}, and the Markov property in Step~2.
Correct.

\medskip\noindent\textbf{(2) Irreducibility.}
From Step~3.  Correct.

\medskip\noindent\textbf{(3) Holomorphy.}
$A_\lambda$ is selfadjoint and lower bounded (by the representation theorem for closed
semibounded forms).  By the spectral theorem, $e^{-zA_\lambda}$ is bounded and holomorphic on
$\{z\in\C:\operatorname{Re}(z)>0\}$.
This is standard functional analysis and correct.

\medskip\noindent\textbf{Compact resolvent.}
From Step~4.  This ensures that $e^{-tA_\lambda}$ is compact for each $t>0$ (since it is the
composition of a bounded operator with $(A_\lambda+1)^{-1}$, which is compact).
Correct.

\medskip
Therefore Theorems~\ref{thm:posimprove} and~\ref{thm:simple} apply, yielding:
\begin{itemize}[nosep]
\item $e^{-tA_\lambda}$ is positivity improving for all $t>0$.
\item $\lambda_0=\min\spec(A_\lambda)$ is a simple eigenvalue.
\item The eigenfunction $\psi$ satisfies $\psi>0$ a.e.\ (or $\psi<0$ a.e.; we choose the positive
  normalization).
\end{itemize}

\subsection{Verification: Evenness from inversion symmetry}

Define the reflection operator $R:L^2(I)\to L^2(I)$ by $(RG)(u):=G(-u)$.
Since $I=(-L,L)$ is symmetric, $R$ preserves $L^2(I)$ and is a unitary involution ($R^2=\mathrm{Id}$).

\begin{lemma}\label{lem:forminvariance}
$\cE_\lambda(RG)=\cE_\lambda(G)$ for all $G\in\cD(\cE_\lambda)$.
\end{lemma}

\begin{proof}
On $L^2(\R)$, $R$ satisfies $RS_t = S_{-t}R$.  Hence for any $\phi\in L^2(\R)$:
\begin{align*}
\|R\phi - S_tR\phi\| &= \|R(\phi-S_{-t}\phi)\| = \|\phi-S_{-t}\phi\| = \|S_t\phi-\phi\| = \|\phi-S_t\phi\|,
\end{align*}
where we used unitarity of $R$, the identity $RS_t=S_{-t}R$, and the fact that $\|\phi-S_{-t}\phi\|
=\|S_t(\phi-S_{-t}\phi)\|=\|S_t\phi-\phi\|=\|\phi-S_t\phi\|$ (unitarity of $S_t$).

Since every weight in $\cE_\lambda$ is nonnegative and the norms $\|\widetilde{G}-S_t\widetilde{G}\|$
are $R$-invariant, $\cE_\lambda(RG)=\cE_\lambda(G)$.
\end{proof}

\begin{corollary}\label{cor:commute}
$A_\lambda R = RA_\lambda$.
\end{corollary}

\begin{proof}
Form invariance under a unitary implies operator commutation: for $u\in\cD(A_\lambda)$ and
$v\in\cD(\cE_\lambda)$,
\[
\langle A_\lambda Ru, v\rangle = \cE_\lambda(Ru,v) = \cE_\lambda(u,R^{-1}v) = \langle A_\lambda u, Rv\rangle
= \langle RA_\lambda u, v\rangle.
\]
(The second equality uses the sesquilinearity of $\cE_\lambda$ and invariance under $R$:
$\cE_\lambda(Ru,Rv)=\cE_\lambda(u,v)$ by polarization of $\cE_\lambda(RG)=\cE_\lambda(G)$.)
Since this holds for all $v$ in the dense set $\cD(\cE_\lambda)$, $A_\lambda Ru = RA_\lambda u$.
\end{proof}

\begin{corollary}[Even ground state]\label{cor:even}
The ground-state eigenfunction $\psi$ satisfies $\psi(-u)=\psi(u)$ a.e.
\end{corollary}

\begin{proof}
By Corollary~\ref{cor:commute}, $R\psi$ is an eigenfunction of $A_\lambda$ for the same eigenvalue
$\lambda_0$.  Since $\psi>0$ a.e., also $R\psi>0$ a.e.\ (reflection preserves strict positivity on a
symmetric interval).  By simplicity of $\lambda_0$, $R\psi = c\psi$ for some $c\in\R$.
Strict positivity of both sides forces $c>0$.  Since $R$ is unitary, $\|R\psi\|=\|\psi\|$,
hence $|c|=1$, so $c=1$.  Therefore $\psi(-u)=\psi(u)$ a.e.
\end{proof}

\subsection{Assessment}

\begin{verdict}
Step~5 is \textbf{correct}.  The external theorems (Perron--Frobenius for positive semigroups) are
correctly cited, and all hypotheses---positivity preservation (Step~2), irreducibility (Step~3),
holomorphy (spectral theorem), compact resolvent (Step~4)---are verified.
The evenness argument via inversion symmetry is clean: form invariance $\Rightarrow$ operator
commutation $\Rightarrow$ the reflected ground state is a scalar multiple of the original $\Rightarrow$
positivity forces the scalar to be~$1$.
\end{verdict}


%============================================================================
\section{Global assessment}
\label{sec:global}
%============================================================================

\subsection{Summary table}

\begin{center}
\renewcommand{\arraystretch}{1.3}
\begin{tabular}{|c|l|c|p{6.5cm}|}
\hline
\textbf{Step} & \textbf{Name} & \textbf{Status} & \textbf{Key observation} \\
\hline
1 & Energy decomposition & Correct & Direct algebra; unitary polarization identity \\
\hline
2 & Markov property & Correct & One-line from $1$-Lipschitz and $\Phi(0)=0$ \\
\hline
3 & Irreducibility & Correct & Archimedean continuum of shifts drives irreducibility;
  Dirichlet form criterion verified in \S\ref{sec:dirichlet-irred} \\
\hline
4 & Compact resolvent & Correct & Logarithmic growth of Fourier symbol;
  Kolmogorov--Riesz \\
\hline
5 & Perron--Frobenius & Correct & Standard theorems; evenness from reflection symmetry \\
\hline
\end{tabular}
\end{center}

\subsection{Points requiring additional discussion}

\subsubsection{The Dirichlet form irreducibility criterion (\S\ref{sec:dirichlet-irred})}

The paper's main expositional gap is in sub-step~(c) of Step~3, where it invokes the equivalence
between the indicator-energy condition and semigroup irreducibility as a ``standard result'' from
Dirichlet form theory.  While indeed standard, the precise formulation matters.
In Proposition~\ref{prop:bridge} above, we provided the additional argument needed: for
translation-difference forms, the cross term $\cE(\1_B u,\1_{B^c}u)$ is always nonpositive,
so its vanishing for all $u$ forces $\cE(\1_B)=0$, reducing the standard condition to the
paper's criterion.

This additional argument does not invalidate the proof.  It fills an expositional gap with a
straightforward calculation.

\subsubsection{Membership of $\1_B$ in the form domain}

A related subtlety: does $\1_B\in\cD(\cE_\lambda)$ for every measurable $B\subset I$?
The archimedean contribution to $\cE_\lambda(\1_B)$ includes
$\int_0^{2L}w(t)\|\widetilde{\1}_B-S_t\widetilde{\1}_B\|^2\,dt$.
Near $t=0$, $w(t)\sim 1/(2t)$ and $\|\widetilde{\1}_B-S_t\widetilde{\1}_B\|^2\to 0$,
but the rate of convergence depends on the regularity of $B$.
For a general measurable set, the strong continuity of translation gives
$\|\widetilde{\1}_B-S_t\widetilde{\1}_B\|^2\to 0$ as $t\to 0$, but the integral
$\int_0^{2L}(1/t)\|\widetilde{\1}_B-S_t\widetilde{\1}_B\|^2\,dt$ may diverge.

However, this subtlety does not affect the proof.  The argument in sub-step~(b) only
needs to extract information from $\cE_\lambda(\1_B)=0$, which is a hypothesis (not something
that needs to be verified for all $B$).  If $\cE_\lambda(\1_B)=0$, then in particular
the archimedean integral vanishes, and the argument proceeds.  If $\cE_\lambda(\1_B)=+\infty$,
then the condition $\cE_\lambda(\1_B)=0$ simply cannot hold, and $B$ poses no threat to
irreducibility.  The only sets that could witness reducibility are those with
$\cE_\lambda(\1_B)=0$ (finite, in fact zero), and for those the argument is complete.

\subsubsection{Relationship between $\cE_\lambda$ and $QW_\lambda$}

The paper's form $\cE_\lambda$ and Connes's Weil quadratic form $QW_\lambda$ differ by an additive
constant $c(\lambda)\|G\|^2$.  This means $A_\lambda^{\text{paper}} = A_\lambda^{\text{Connes}} + c(\lambda)\cdot\mathrm{Id}$.
Such a spectral shift preserves:
\begin{itemize}[nosep]
\item simplicity of eigenvalues (the eigenspaces are identical),
\item the eigenfunctions (unchanged by adding a scalar to the operator),
\item evenness of the ground state.
\end{itemize}
Hence the paper's result directly implies the conjecture for Connes's operator.

\subsection{Conclusion}

The paper provides a complete and correct proof that the ground-state eigenvalue of $A_\lambda$
is simple and the corresponding eigenfunction is even, for every $\lambda>1$.
The five-step structure is logically clean: each step has well-defined inputs and outputs,
and the external results invoked (Perron--Frobenius, Kolmogorov--Riesz, Dirichlet form theory)
are standard and correctly applied.  The one expositional gap---the precise form of the
Dirichlet form irreducibility criterion---is closed by the supplementary argument in
\S\ref{sec:dirichlet-irred}, which shows that the paper's indicator-energy criterion is
equivalent to the standard one for this class of forms.

\end{document}
